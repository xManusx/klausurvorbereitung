\section{Vorlesung \Romann{12} --- Soziologie als empirische Wissenschaft}

\subsection{Empirie and stuff}
\begin{itemize}
	\item
		Viele Sozialwissenschaften sind empirisch
	\item
		Empirische Sozialforschung umfasst
		\begin{itemize}
			\item
				systematische und methodisch geleitete
			\item
				Beschaffung, Verarbeitung und Deutung von Informationen
			\item
				"uber die individuelle und soziale Wirklichkeit
			\item
				Und deren Effekte
		\end{itemize}
	\item
		Im Mittelpunkt stehen folgende Ziele:
		\begin{itemize}
			\item
				methodisch geleitete Aufdeckung potentieller Zusammenh"ange zwischen Merkmalen und Variablen (\textit{explorative Untersuchungen}) (Da wo man vorher eigentlich noch gar nichts wei"s)
			\item
				m"oglichst genaue Beschreibung von Sachverhalten, Beziehungen zwischen Variablen und Prozessen (\textit{deskriptive Untersuchungen}) (Daten sammeln um Probleme zu beschreiben)
			\item
				m"oglichst exakte "Uberpr"ufung theoretischer Hypothesen "uber Merkmalszusammenh"ange anhand von empirischen Daten, die m"oglichst gesammelt oder erhoben werden (\textit{theoriegeleitete Untersuchungen}) (nat"urlich nur, wenn es bereits Theorie gibt)
			\item
				systematische und methodisch geleitete Bewertung der Wirksamkeit von praktisch-politischen Ma"snahmen (\textit{Evaluationsstudied}) (meist im Auftrag Dritter)
		\end{itemize}
	\item
		Datenerhebung (Befragungen, Beobachtungen, \dots) und  Datenauswertung (Statistische und interpretative Verfahren) sind Werkzeuge. Methoden und Theorien sind Handwerkszeug eines Sozialwissenschaftleres. Theorielose Studien sind eher schlecht
\end{itemize}

\subsection{Beispielstudie Mobbing}
\begin{itemize}
	\item
		Repr"asentativstudie, deskriptive Untersuchung. Ziel war Beschreibung der Arten und des Umfangs von Mobbinghandlungen am Arbeitsplatz in Deutschland, inklusive Bew"altigungsformen und strategien der Betroffenen, sowie Hintergr"unde, Motive und beg"unstigende Faktoren
	\item
		Zwei telefonische BUS-Befragungen, danach schriftliche Befragung mit standardisiertem FRagebogen unter Mobbingopfern
	\item
		Fragenkomplexe:
		\begin{itemize}
			\item
				Art der Mobbinghandlungen
			\item
				H"aufigkeit und Dauer
			\item
				Merkmale des Mobbers
			\item
				Folgen f"ur die Betroffenen und die Mobber
			\item
				Bew"altigungsformen und strategien der Betroffenen
			\item
				Hintergr"unde, Motive und beg"unstigende Faktoren
		\end{itemize}
	\item
		Datenauswertung: statistisches Analyseverfahren (deskriptive Statistik und multivariate statistische Analysen)
	\item
		Ergebnisse:
		\begin{itemize}
			\item
				In Deutschland j"ahrlich 5.5\% der erwerbst"atigen Bev"lkerung von Mobbing betroffen
			\item
				Eher Frauen betroffen
			\item
				h"oheres Risiko in sozialen Berufen
		\end{itemize}
	\item
		Methodische Kritik:
		\begin{itemize}
			\item
				Ist Stichprobe repr"asentativ? Wenn Aufruf "uber Presse und Internet gemacht wurde?
			\item
				Trifft generalisierte Aussagen "uber \enquote{Mobbing in Deutschland}. Ergebnis aber eigentlich nur f"ur das Jahr 2000
			\item
				Nur Opferperspektive. Motive f"ur Mobbing w"are aber aus T"atersicht sehr interessant
		\end{itemize}
	\item
		Gab am Anfang keine Theorie oder Hypothese --- typische deskriptive Studie
\end{itemize}

\subsection{Qualitative Studien zu Mobbingberatung}
\begin{itemize}
	\item
		Ziel: Erforschung von Mobbinberatung. Wie tats"achlich beraten wird, war nichts bekannt. Soll erforscht werden wie Mobbingberatung abl"auft und wie Berater zur L"osung vorgehen
	\item
		Datenerhebung: Leitfaden Interviews (Gibt Leitfaden mit Fragen, aber haupt\-s"ach\-lich redet Befragte/r)
		\begin{itemize}
			\item
				Wurde gefragt wie man typischen Ablauf bei Beratung vorstellen muss
			\item
				Beschreiben wie zur L"osung eines Mobbingfalls konkret vorgehen
		\end{itemize}
	\item
		Auswertung: Qualitative Inhaltsanalyse und anschlie"sende Typenbildung
	\item
		Auszug aus Fragenkatalog:
		\begin{itemize}
			\item
				WIe muss man sich typischen Ablauf von Beratung vorstellen?
			\item
				Wie machen Sie sich Bild von der Organisation?
			\item
				Wie gehen Sie bei der Mobbing-Problemanalyse vor?
		\end{itemize}
	\item
		Ergebnisse:

		Je nach Diagnose
		\begin{itemize}
			\item
				Konfliktmoderation (Moderation nur am Anfang) oder Mediation (wenn Konflikte schon weiter fortgeschritten sind, Mediator greift auch aktiv ein)
			\item
				Coaching
			\item
				Organisationsentwicklungsma"snahme
		\end{itemize}
		Interessant weil kontingenztheoretischer Ansatz (wichtig in Literatur) weder Coaching noch Organisationsentwicklung empfiehlt

		Diejeingen, die Mobbing als Form von eskaliertem Konflikt interpretierten f"uhrten (nach kontingenztheoretischem Ansatz) eine Moderation oder Mediation durch. Berater, die Coaching oder Organisationsentwicklungsma"snahmen gut finden, interpretieren Mobbing eher als Mehrebenenproblem. Ebene der Intervention muss dann Ebene der Dyade ber"ucksichtigen
	\item
		Ergebnisse lassen sich mit Mehrebenenmodell des Mobbing von Heames und Harvey verkn"upfen: beruht auf Annahme, dass Mobbing negative Konsequenzen auf Ebene
		\begin{itemize}
			\item
				Der betroffenen Individuen (klar)
			\item
				Gruppe (schlechtes Arbeitsklima, nachlassende Gruppenleistung)
			\item
				Organisation (Produktivit"atsverluste, Beratungskosten, Reputation,\dots)
		\end{itemize}
	\item[$\Rightarrow$] Mehrebenenansatz zur Intervention bei Mobbing entwickelt
		\begin{itemize}
			\item
				Wenn negative Folgen auf mehreren  Ebenen vorliegen und behoben werden sollen, dann Interventionsstrategien auf allen Ebenen
			\item
				Ursache des Scheiterns von Interventionen ist, dass L"osung nur auf lokaler Ebene gesucht werden, Konsequenzen auf Gruppen/Organisationsebene ignoriert $\Rightarrow$ jemand anderes wird dann gemobbt
		\end{itemize}
	\item
		3 Hypothesen wurden aufgestellt, die durch zuk"unftige Forschung zu "uberpr"ufen sind:
		\begin{itemize}
			\item
				Mediation ist keine geeignete Interventionsstrategie bei Mobbing am Arbeitsplatz
			\item
				Coaching stellt eine geeignete Interventionsstrategie auf Ebene der Gruppe dar
			\item
				Organisationsentwicklung stellt geeignete Interventionsstrategie auf Organisationsebene dar
		\end{itemize}
\end{itemize}
\subsection{Zusammenfassung beider Studien}
\begin{itemize}
	\item
		Repr"asentativstudie von 2000 beschreibt Arten und Umfang und \dots des Mobbens in Deutschland (Deskriptive Untersuchung)
	\item
		Qualitative Studie von 2005 -- 2007 exploriert wie Mobbinberatung typischerweise abl"auft und wie Berater vorgehen (Explorative Untersuchung)
	\item
		Durch Verkn"upfung der Befunde mit Mehrebenenmodell wird neuer Theorieansatz (Mehrebenenansatz der Intervention bei Mobbing) entwickelt. Dieser Ansatz muss in Zukunft mit theoriegeleiteten Untersuchungen "uberpr"uft werden
	\item
		Empirisch-soziologische Forschung und theoretische Forschung erlauben Generierung von Hypothesen
	\item
		Theoretische Forschung kann nur "uberpr"ufen ob Hypothese im Widerspruch zu anderer Theorie steht. Nur empirische Forschung kann Hypothesen gegen die Wirklichkeit testen
\end{itemize}
\subsection{Empirie und Theorie in der Soziologie}
\begin{itemize}
	\item
		\enquote{Ein empirisch wissenschaftliches System muss an der Erfahrung scheitern k"onnen} (Popper)
	\item
		Wenn Soziologie nur theoretische, aber keine empirische Wissenschaft w"are k"onnten
		\begin{itemize}
			\item
				unbegrenzt inhaltlich falsche (aber theoretisch plausible) Hypothesen "uber Ursachen und Umfang von Mobbing aufgestellt werden --- Falsche Theorien und Hypothesen zu Mobbin k"onnten nicht widerlegt werden
			\item
				Wunschvorstellungen "offentliche und wissenschaftliche Ausseinandersetzung dominieren
			\item
				tats"achlich verbreitete Ma"snahmen gegen Mobbing nicht benannt und Wirkung nicht objektiv festgestellt werden. Auch keine Aufkl"arung "uber unwirksame oder sch"adliche Ma"snahmen
		\end{itemize}
\end{itemize}

