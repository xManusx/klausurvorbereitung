\section{Vorlesung \Romann{10} --- Einf"uhrung in den Bereich Soziologische Theorien}
\subsection{Grundlagen: Theorie allgemein}
\begin{itemize}
	\item
		Was hei"st Theorie?
		\begin{itemize}
			\item
				System von allgemeinen Aussagen "uber die Welt, die Realit"at bzw. eines Ausschnitts davon: Verallgemeinerung und Generalisierung, Unterschied Theorie Empirie, eigene Geltungskriterien (nicht wahr/falsch, sondern Plausibilit"at)
		\end{itemize}
	\item
		Wissenschaftliches Kontinuum von Alexander:
		\begin{itemize}
			\item
				Nicht-empirische (\enquote{metaphysische} Welt)
			\item
				allgemeine Pr"asuppositionen (z.B. \enquote{Die Welt ist systematisch strukturiert} (Luhmann). Sehr sehr allgemeine Aussagen "uber die Welt, quasi die Allgemeinsten die man treffen kann, legen dann aber viel fest)
			\item
				Ideologische Orientierungen (z.B. normative Aussagen: auch wenn man so neutral wie m"oglich analysieren m"ochte, trotzdem immer normative Pr"agung)
			\item
				Modelle
			\item
				Begriffe (z.B. Identit"at, Gesellschaft, \dots, Definitionen unterschiedlich je nach Theorie)
			\item
				Definitionen
			\item
				Klassifikationen (Unterschied Mann$\leftrightarrow$Frau, Klassifikation von Blumen)
			\item
				Gesetze, wiederkerhende Verlaufsmuster
			\item
				Korrelationen zwischen empirischen Daten
			\item
				Methodologische Annahmen (z.B. zur Gewinnung von Daten)
			\item
				Beobachtungen (empirische Daten)
			\item
				Empirische Welt
		\end{itemize}
	\item
		Popper: \enquote{Die Theorie ist das Netz, das wir auswerfen um \enquote{Die Welt} einzufangen -- sie zu rationalisieren, zu erkl"aren und zu beherrschen} $\Rightarrow$ Kann keine theorielosen Beobachtungen geben
	\item
		Kant: \enquote{Gedanken ohne Inhalt sind leer, Anschauungen ohne Begriffe sind blind}
	\item
		Theoretische Abstraktionen bleiben immer farbenloser, flacher als die Empirie
\end{itemize}

\subsection{Charakteristika von Soziologischen Theorien}
\begin{itemize}
	\item
		Was ist Inhalt von soz Theorien?
		\begin{itemize}
			\item
				Theorien des sozialen Handelns
			\item
				Theorien sozialer Ordnung
			\item
				Theorien sozialen Wandels
		\end{itemize}
	\item
		Primat des Sozialen: Beziehungen auf zwischenmenschlich-sozialen Aspekten in Theorien -- nicht reine biologische oder psychologische Aussagen
	\item
		Fragestellungen:
		\begin{itemize}
			\item
				Welcher Logik folgen soziale Handlungen/Interaktionen?
			\item
				Welche Muster/Formen/Strukturen von Vergesellschaftung sind erkennbar?
			\item
				Wie kommen diese zustande? (Primat des Sozialen!)
			\item
				Wie entwickelt sich Gesellschaft (weiter)? Faktoren der Entwicklung?
			\item
				Soziale Wechselwirkungen --- Triebkr"after --- Statik/Dynamik?
			\item
				Moderne? Wandel von Tradition zu Moderne?
		\end{itemize}
	\item
		Unterschiedliche Analyseebenen:
		\begin{itemize}
			\item
				Makro (Gesellschaft): Gesellschaftstheorien, Strukturtheorien
			\item
				Meso (Organisation): Theorien mittlerer Reichweite, Organisationstheorien
			\item
				Mikro (Interaktion): Handlungstheorien
		\end{itemize}
		H"aufig Verbindung v. Handlungs und Strukturebene

	\item
		Entstehungskontext der Soziologie:
		\begin{itemize}
			\item
				Aufkl"arung schafft die Vorstellung transzendenter Kr"afte ab: Empirismus, Skeptizismus, Rationalismus $\Rightarrow$ Entzauberung der Welt
			\item
				Gro"se Transformation 18./19. Jhdt: Fundamentale Reorganisation des wirtschaftlichen lebens, Industrialisierung und Urbanisierung
		\end{itemize}
	\item
		Geburt der modernen Gesellschaft
		\begin{itemize}
			\item
				\enquote{Societas des MAs war eine dauerhafte, als nat"urlich und richtig empfundene Ordnung} (Tenbruck)
			\item
				Nicht mehr Stabilit"at/Ordnung sondern Wandel/Umbruch
			\item
				Entstehung von Societas Civilis: willk"urliche/freie Assoziation von Individuen
			\item
				Hegel: Gesellschaft als Staatsfreier Raum
		\end{itemize}
		$\Rightarrow$ Nachdenken "uber Bedingungen der neuen \enquote{Welt}
	\item

		Theoreme aus der Gr"undungsphase:
		\begin{itemize}
			\item
				Comte: Physique Sociale, Positivismus
			\item
				Marx: Historischer Materialismus, Kapitalismustheorie, Konflikt von Kapital und Arbeit, Dialektische Geschichtsphilosophie
			\item
				Spencer: Soziale Differenzierung (Evolutionstheorie), Soziale Statik/Dynamik
			\item
				Durkheim: Fokus auf Soziale Tatbest"ande (Positivismus), Soziales mit sozialem Erkl"aren
			\item
				Weber: Soziales Handeln verstehen und erkl"aren, Soziologische Grundbegriffe und Idealtypen, Rekonstruktion von Sinn (Motive/Rationalit"aten), Theorem der kulturellen Rationalisierung (Moderne)
		\end{itemize}
\end{itemize}


\subsection{Warum so viele Theorien?}
\begin{itemize}
	\item
		Beispiele:
		\begin{itemize}
			\item
				Klassische Theorien
			\item
				Nationale Theorie-Traditionen
			\item
				Modernisierungstheorien
			\item
				Konflikttheorien
			\item
				Postmoderne Theorien
			\item
				Alte/Neue Kulturtheorien
		\end{itemize}
	\item
		Theoretische Integrationsversuche im 20. Jhdt
		\begin{itemize}
			\item
				Parsons: Structure of Social Action, Social System
			\item
				Coleman: Rational Choice - Theorie
		\end{itemize}
	\item
		Warum Theorienpluralismus:
		\begin{itemize}
			\item
				Vielfalt des Gegenstands (Individuum $\leftrightarrow$ Gesellschaft)
			\item
				Soziologie ist eine \enquote{ewig junge Wissenschaft} (Weber)
			\item
				Vielfalt von Methoden und Methodologien
			\item
				Besonderheit des Gegenstands: Eigensinn des Sozialen. Soziologen interpretieren Menschen, die sich wiederum permanent gegenseitig interpretieren (Giddens)
			\item
				Warum Klassiker? Historisches Bewusstsein, Aktualit"at alter Fragen?
		\end{itemize}
	\item
		Reflexionsdreieck der Soziologie:
		\begin{enumerate}
			\item
				Sozialwissenschaftliche Wissensproduktion
			\item
				Allgemeine Konzepte von gesellschaftlicher Wirklichkeit
			\item
				Soziale Konstruktion gesellschaftlicher Wirklichkeit
		\end{enumerate}
		(Alltagstheorien zwischen Allgemeine und Konstruktion)
\end{itemize}

