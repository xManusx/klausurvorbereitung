\section{Vorlesung \Romann{11} --- Arbeit als soziologische Schl"usselkategorie}
\begin{itemize}
	\item
		Bisher lebte Mensch immer in Gesellschaften, in denen Aufwand betrieben musste um sich zu reproduzieren
	\item
		Arbeit als distinkter Begriff macht nur Sinn, wenn es Dinge gibt, die \enquote{Nicht-Arbeit} sind
	\item
		Leben in modernen Gesellschaften sehr erwerbsorientiert, Arbeitsposition wichtige Einsch"atungskategorie
	\item
		Im Folgenden ist Arbeit nur Erwerbsarbeit (Keine Hausarbeit, DIY, kein Ehrenamt, \dots sehr enge Kategorie!)
	\item
		Zum Begriff der Arbeit
		\begin{itemize}
			\item
				Wird im Verlauf der Geschichte ganz unterschiedlich bewertet und konstruiert
			\item
				Von der Antike (unw"urdig f"ur Vollb"urger), "uber Christlichen Arbeitsbegriff (Aquin: Kontemplation, gottgef"allig Askese, \enquote{ehrliche} Arbeit), Protestantismus (Luther und so, Beruf) zu Arbeit als b"urgerlichem Kampfbegriff
		\end{itemize}
	\item
		Was macht Arbeitsgesellschaft aus?
		\begin{itemize}
			\item
				Erwerbsarbeit dominante Form der Existenssicherung
			\item
				Verkn"upfung von Erwerbsarbeit und gesellschaftlichem Status
			\item
				Existens eines arbeitszentrierten Sozialstaates
			\item
				Erwerbsarbeit als wesentlicher Bezugspunkt individueller Lebensplanung und Identit"atsbildung
		\end{itemize}
	\item
		Krise der Arbeitsgesellschaft: zentrale Themen
		\begin{itemize}
			\item
				Arbeitslosigkeit: Der Gesellschaft geht die Arbeit aus!
			\item
				Krise des arbeitszentrierten Sozialstaats wegen der Aufl"osungstendenzen des Normalarbeitsverh"altnisses
			\item
				Wertewandel und abnehmende Pr"agekraft von Arbeit
		\end{itemize}
	\item
		Geht der Gesellschaft die Erwerbsarbeit aus?
		\begin{itemize}
			\item
				Arbeitsvolumen ist "uber l"angere Zeit bis vor einigen Jahren geschrumpt
			\item
				Aber die Erwerb"atigkeit ist trotzdem gestiegen weil die durchschnittliche Arbeitszeit gesunken ist (Arbeitszeitverk"urzung von Vollzeit, oft Teilzeit)
			\item
				Rekommodifizierung von fr"her nicht warenf"ormig verf"ugbaren T"atigkeiten (Spezialisten f"ur alles und jeden)
		\end{itemize}

	\item
		Ver"anderung des Arbeitskr"afteangebots:
		\begin{itemize}
			\item
				Mittelfristig Abnahme
			\item
				Verschiebung zu h"oherqualifizierten, "alteren Arbeitskr"aften (Weiterbildungen), weiblichen Arbeitskr"aften, Migrationshintergrund
		\end{itemize}
	\item
		Politische Strategien gegen schrumpfendes Arbeitskr"afteangebot:
		\begin{itemize}
			\item
				Einwanderungspolitik
			\item
				(Aus-)Bildungszeiten verk"urzen
			\item
				Wochen/Jahres/Lebens-Arbeitszeiten verl"angern
			\item
				Erh"ohung der weiblichen Erwerbsquote bei gleichzeitiger Reduktion der Teilzeitarbeit zu Gunsten von Vollzeit
		\end{itemize}
	\item
		Zur Subjektivierung von Arbeit:
		\begin{itemize}
			\item
				Dreifacher Bezug auf Arbeit (Was macht Arbeit f"ur mich als Subjekt?): Reproduktion, Expression, Interaktion
				\begin{itemize}
					\item
						Reproduktion: Arbeit als Medium der Existenzsicherung
					\item
						Expression: Arbeit als Medium der Selbstverwirklichung
					\item
						Interaktion: Arbeit als Medium der sozialen Integration (immer auch Kooperation, immer auch soziale Beziehungen auf Arbeit)
				\end{itemize}
		\end{itemize}
	\item
		Zusammenfassung:
		\begin{itemize}
			\item
				Destandardisierung: Krise des Normalarbeitsverh"altnisses, Zunahme atypischer Besch"aftigungsformen
			\item
				Prekarisierung: "Uber 20\% der Vollzeitbesch"aftigten unter der Niedriglohnschwelle,
				Gef"uhlte Prekarisierung bis in die Mittelschichten hinein, 1 Mio Aufstocker
			\item
				Subjektivierung: Neue Anforderungen durch management by objectives trifft auf Interesse von Besch"aftigten an der Selbstgestaltung ihrer Arbeit
			\item
				Intensivierung: Stichwort Burnout
			\item
				Flexibilisierung: der Belegschaft (leiharbeit, Werktvertr"age), von Arbeitszeiten, von Erreichbarkeitsanforderungen
		\end{itemize}
\end{itemize}
