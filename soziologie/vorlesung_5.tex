 %! TEX root = soziologie.tex

\section{Vorlesung \Romann{5} --- Grundbegriffe \Romann{3}: Institution, Organisation, kollektives Handeln}
\subsection{Insitution}

\begin{itemize}
	\item
		Insitutionen: \enquote{Sinneinheit von habitualisierten Formen des Handelns und der sozialen Interatkion, deren Sinn und Rechtfertigung der jeweiligen Kultur entstammen und deren dauerhafte Beachtung die Gesellschaft sichert}

		 Am Beispiel Paar-Werdung: soziale Positionen (wie begr"u"st man sich?, wer macht Kaffee?) werden Institutionalisiert, bilden stabile Muster
	 \item
		 GEsellschaft stabilisiert Verhaltenserwartungen durch Saktionierung von Abweichung
	 \item
		 Was leistet Insitutionalisierung?
		 \begin{itemize}
			 \item
				 Stabilisierung von Verahltenserwartungen und Probleml"osungen (Handlungskoordination)
			 \item
				 Einschr"ankung der Vielfalt von Handlungsm"oglichkeiten
			 \item
				 Stabilisierung von sozialer Ordnung durch Abl"osung der Insitution von der Person (Ehe ist soziale Institution, unabh"angig von konkreten Personen
		 \end{itemize}
		 Folge: \enquote{Das Soziale} erscheint uns fest, selbstverst"andlich. Insitutionen existieren in und durch ihren Vollzug!

	 \item
		 Dimensionen von Institutionen:
		 \begin{itemize}
			 \item
				 Ideell (Trauung und Ehesakramt)
			 \item
				 Personell (Rollen)
			 \item
				 Regeln und Normen (Rituelle Fragen, Ringtausch, \dots)
			 \item
				 Materieller Apparat (Ringe, Kleid, Kirche)
		 \end{itemize}
\end{itemize}

\subsection{Organisation}
\begin{itemize}
	\item
		Unterschied zu Institutionen: absichtlich geschaffen, verfolgen Zwecke, definieren Mitgliedschaften, bilden Hierachien aus, aber wie Institutionen soziale Ph"anomene, keine sachlich fassbaren Dinge
	\item
		Integrieren Mitglieder nur als Funktionen, nicht als ganze Person (im Gegensatz zu z.B. Familie)

		Mitgliedschaft:
		\begin{itemize}
			\item
				Herstellung von Konformit"at (Anerkennung der Regeln, sonst Ausschluss, da Entscheidungsmonopol der Organisation)
			\item
				Erwartungen an Mitglieder formalisiert (z.B. Arbeitsvertrag, Stellenausschreibung) --- aber immer auch nicht formalisierte Erwartungen!
			\item
				Mitglieder werden motiviert durch:
				\begin{itemize}
					\item
						Materielle Anreize (Lohnarbeit)
					\item
						Zwang (Milit"ar)
					\item
						Zeckidentifikation (NGO)
					\item
						Attraktive T"atigkeiten (Verein)
					\item
						Kollegialit"at
				\end{itemize}
		\end{itemize}
	\item
		Moderne Sozialordnungen werden auch als \enquote{Organisationsgesellschaft} bezeichnet, alles ist Organisation (im Gegensatz zu Mittelalter, dort nur einzelne Teilbereiche (Verwaltung, Kirche))
	\item
		Parsons: Organisationen sind wichtigster Mechanismus f"ur hochdifferenzierte Gesellschaft um das System in Gang zu halten, Ziele zu verwirklichen: bauen intern Komplexit"at auf um mit Umwelt umgehen zu k"onnen, hohe funktionale Spezialisierung und Hierarchieaufbau

		$\Rightarrow$ Eigenlogik von Organisationen (z.B. Mittelverwendung in staatlichen Verwaltungen: alles Geld ausgeben, dass da ist um beim n"achsten Mal nicht weniger zu bekommen)

\end{itemize}
