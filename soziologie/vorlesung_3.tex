
\section{Vorlesung \Romann{3} --- Grundbegriffe \Romann{1}: Handeln, Normen}
\subsection{Verhalten, Handeln, soziales Handeln}
\begin{itemize}
	\item
		\enquote{Flirt mit Folgen}
	\item
		Rollen legen Erwartungen an das eigene Verhalten und der jeweils anderen fest
	\item
		Verhaltenskoordination bei Erwartungsunsicherheit schwierig
	\item
		\index{Soziales Handeln}
		Soziales Handeln (Handeln, das auf andere bezogen ist (begrüßen, beobachten, \dots) $\in$ Handeln (das nicht auf andere bezogen ist (den Raum betreten), verfolgt Ziel, setzt intentional Mittel ein, wählt zwischen Alternativen) $\in$ Verhalten (auch unabsichtliche Aktivität, stolpern)
	\item
		Sinn einer Handlung wird von anderen interpretiert, Folgen müssen nicht identisch mit dem intendierten Sinn sein
	\item
		Die Intention von Sozialem Handeln ist auf das Verhalten anderer gerichtet. Sinnorientierungen des Handelns werden gesellschaftlich mitgeprägt (kann sonst von anderen nicht verstanden werden, z.B. Begrüßungsregeln)
	\item
		Gesellschaftlicher Wandel ist über den Wandel der Sinnorientierungen des Handelns erfassbar

	\item
		Sinnorientierungen sozialen Handlens (aka. \enquote{Wo kommen eigentlich die Motive her?} aka. Webersche Handlungstypologie\index{Webersche Handlungstypologie})
		\begin{itemize}
			\item
				Traditional: Sitten und Gebr"auche (Spende aufgrund etablierter Sitten)
			\item
				Affektuell: Emotionen (Spenden aus Mitef"uhl)
			\item
				Wertrational: bewusster Glaube an den ethischen/religi"osen Eigenwert eines Sachverhalts, unabh"angig vom Erfolg (Spende aufgrund spezifischen Ethos, z.B.\ katholische Soziallehre)
			\item
				Zweckrational: orientiert sich an subjektiv als ad"aquat vorgestellten Mitteln f"ur subjektiv eindeutig erfasste Zwecke (Spende um spezifisches Ziel zu verfolgen, z.B.\ Erh"ohung des eigenen Ansehens)
		\end{itemize}
		Keine empirische Motive sondern ideele Typen! In freier Wildbahn treten diese Typen nicht rein auf
	\item
		Welcher Teil sozialen Handels wird bei welchem Handeln kontrolliert? (Nur Reihenfolge merken und dass es absteigend ist, eigentlich auch trivial)

		\begin{tabular}[H]{lllll}
			& Mittel & Zweck & Wert & Folge \\ \hline
			zweckrational & + & + & + & + \\ \hline
			wertrational & + & + & + & - \\ \hline
			affektuell & + & (+) & - & - \\ \hline
			traditional & (+) & - & - & - \\ \hline
		\end{tabular}


	\item
		\begin{enumerate}
			\item Gesellschaft basiert auf individuellem Verhalten 
	\item$\rightarrow$ Verhalten weist Spielr"aume auf 
	\item$\rightarrow$ Konkretes Verhalten ist Folge einer Entscheidunge 
	\item$\rightarrow$ Verhaltensselektion ist sozial mitbestimmt (nur sozial erw"unschtes Verhalten wird gezeigt) 
	\item$\rightarrow$ und von vorne\dots
		\end{enumerate}

	\item
		Kritik an Weberscher Handlungstypologie
		\begin{itemize}
			\item
				Soziale Bestimmungsgr"unde der Sinnorientierung spielen in seiner Typologie keine Rolle, orientiert sich nur an subjektiv vom Handelnden gemeinten Sinn
				\item
					Weiterentwicklung: Talcott Parsons verbindet Handlungstheorie und Rollentheorie
		\end{itemize}
	\item
		Talcott Parsons Pattern Variables (Beziehungsvariablen, Mustervariablen): Verortung von sozialem Handeln in f"unfdimensionalem Raum
		\index{Pattern Variables}\index{Beziehungsvariablen}

		\begin{tabular}[H]{lcl}
			Familie/Gemeinschaft & & Gesellschaft\\\hline
			Affektivit"at (Bsp: Freund) & vs. & Affektive Neutralit"at (Bsp: Lehrer) \\ \hline
			Kollektivbezogenheit (Priester) & vs. & Selbstorientierung (Manager) \\ \hline
			Partikularismus (Gruppen-normen, Eltern) & vs. & Universalismus (allgemeine Normen, Richter) \\ \hline
			Askriptive Zuschreibung (Herkunft, Alter) & vs. & Erwerb durch Leistung (Beruf) \\ \hline
			Diffusit"at (ganze Person, Eltern) & vs. & Spezifit"at (Friseur) \\ \hline
		\end{tabular}

		Beispiel Arzt:
		\begin{itemize}
			\item
				Affektiv neutral: ohne Ansehen der Person behandeln
			\item
				Eher kollektiv bezogen, zumindest laut Rollenerwartung
			\item
				Universalistisch: Auch Schwarzer sollte sich von Wei"sem Arzt behandeln lassen
			\item
				Rollenerwerb Leistungsbezogen: durch Studium und Approbation
			\item
				Spezifisch: Soll sich um Verletzung/Krankheit k"ummern, nicht um ganze Person
		\end{itemize}
\end{itemize}

\subsection{Normen}
\begin{itemize}
	\item
		Was sind Normen? soziale ("uberindividuelle) Festlegungen der Bedeutung von Verhalten in einer Situation: Normen legen unsere Erwartungen an das Verhalten anderer fest.\index{Soziale Norm}

		K"onnen widerspr"uchlich sein: Ehrlichkeit --- Erfolg
	\item
		\enquote{Der Begriff der Norm bezieht sich auf konkretere Vorgaben, mit welchen Mitteln und auf welche Weise erstrebenswerte Ziele erreicht werden sollten und welchen Ma"sst"aben ein Handeln zu gen"ugen hat} (Schwietring)
	\item
		Verhaltenserwartungen: du sollst, du musst, du kannst (nicht)
	\item
		Abweichung wird sanktioniert (positiv/negativ: Gratifikation, Sch"atzung, Gericht, Ausschluss, Antipathie)
	\item
		Norm und Abweichung sind zwei Seiten derselben Medaille
	\item
		Nicht jede Erwartungsentt"auschung f"uhrt zur Sanktion: Korrektiver Austausch
	\item
		Abweichendes Verhalten kann pathologisiert oder kriminalisiert werden

	\item
		Individualisierungschancen nicht in Konformit"at sondern in Abweichung!
	\item
		\textit{Werte} rahmen und legitimieren Normen: kulturell verbreitete Vorstellungen des W"unschbaren, Erstrebenswerten, Wertvollen. Definieren allgemeines Ziel, geben Orientierung, legen kein Handeln fest

		Beispiel: Wert romantische Liebe als Beziehungsideal. Normen: Inszenierung von Romantik, Liebesbekundungen, Heiratsantrag
	\item
		Normen/Werte lernt man durch internalisierung, Rollen durch sozialisation
	\item
		\textit{Handlungsrelevant} werden (internalisierte) Normen in der handlungstheoretischen Perspektive dann in Form von Wissen: Man \enquote{wei"s} wie man flirtet. $\Rightarrow$ Zugang zu Norm

	\item
		\index{Akzeptanz von Werten und Normen}
		Akzeptanz von Werten und Normen nach Robert K. Merton

		\begin{tabular}[H]{lll}
			Werte/Ziele akzeptiert & Normen/Mittel akzeptiert & Anpassungsmuster\\ \hline
			ja & ja & Konformismus \\ \hline
			ja & nein & Innovation \\ \hline
			nein & ja & Ritualismus \\ \hline
			nein & nein & Apathie und R"uckzug \\ \hline
			alternativ & alternativ & Rebellion \\ \hline
		\end{tabular}
\end{itemize}

\subsection{Fazit/Rekap}
\begin{itemize}
	\item
		Handeln: Folge von Abw"agung und daruch versteh- und gestaltbar
	\item
		Handeln hat Folgen und Nebenfolgen; orientiert sich an Normen
	\item
	Normen machen Handeln erwartbar, sie sind im Wissen pr"asent (Wo ist das Soziale zu verorten? Im Wissen handelnder Akteure)
\item
	Norm und Abweichung sind wechselseitig aufeinander bezogene Begriffe
\item
	\enquote{Abweichendes} Verhalten ist wichtige Quelle sozialen Wandels
\item
	Anerkannte Ziele f"ordern auch \enquote{abweichende} Strategien ihres Erreichens, wenn die anerkannten Mittel vorenthalten werden
\item
	Werte wirken nicht nur integrativ sondern auch ausschlie"send
\item
	Pries 2014 Kap. 3 u. 4
\end{itemize}


