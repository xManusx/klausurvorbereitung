\section{Vorlesung \Romann{8} --- Kultur und Kommunikation: Medien, Kultur und Gesellschaft}

\begin{itemize}
	\item
		KUltur 
		\begin{itemize}
			\item
				als das Selbstverst"andliche, fraglos hingenommene Allt"agliche, dem die Handlungen unserer Zeitgenossen und Vorfahren zugrunde liegen
			\item
				als Wissensvorrat, als Ansammlung von materiellen Artefakten und als in einer Gruppe g"ultigen sozialen Formen und Muster
			\item
				als Vergleichsgenerator, im Vergleich mit Fremden und als Bewertungsmaschine
			\item
				Basis von Kultur(en): Interaktionen und Kommunikationen
		\end{itemize}
	\item[$\Rightarrow$] Kultur(en) und Kommunikation(en) als Grundlagen sozialer Ordnung
	\item
		Zugang zur Welt ist nicht direkt und unmittelbar, erfolgt "uber Perzepte und insbesondere durch Kommunikationsformen, also:
		\begin{itemize}
			\item
				Sinnesorgane (Augen, Ohren, \dots): Wahrnehmunsmedien
			\item
				K"orper, Stimme: Ausdrucksmedien
			\item
				Sprache: Verstehensmedien
			\item
				Schrift, Buch, Bildmedien, Computer, \dots: Verbreitungsmedien
			\item
				Wahrheit, Macht, Geld, Liebe: symbolisch generalisierte Kommunikationsmedien
		\end{itemize}
	\item
		Was sind Medien?
		\begin{itemize}
			\item
				\enquote{Medien sind gesellschaftliche Einrichtungen und Technologien, die etwas entweder materiell oder symbolisch \textit{vermitteln} und dabei eine besondere \textit{Probleml"osungsfunktion} "ubernehmen. Sie verf"ugen "uber ein materielles Substrat (und sind deshalb Materialit"aten menschlichen und gesellschaftlichen Seins) welches im Gebrauch oder durch seinen Einsatz Wahrnehmungen, Handlungen, Kommunikationsprozesse, Vergesellschaftung und schlie"slich soziale Ordnung im Generellen erm"oglicht, wie auch formt.}( (Ziemann)\label{Ziemann}
			\item
				Medien als Vermittler, Bote, Kanal, Dazwischen, als Werkzeug der Wahrnehmung und Verst"andigung
			\item
				Medien als eigenst"andiger Bereich der Vergesellschaftung (\enquote{DIE Medien})
			\item
				Medien als Wirklichkeitsproduzenten (z.B. L"ugenpresse)
			\item
				Problem mit dieser Definition: erfasst die zeitliche Dimension der Verstehens- und Verbreitungsmedien nicht
			\item
				Verstehens und Verbreitungsmedien:
				\begin{itemize}
					\item
						zeichnen auf und bewahren selektiv auf (Erz"ahlungen, Texte, Bilder, \dots), k"onnen das wieder in gegenw"artige Situation bringen (auch aufgezeichnetes, erwartetes zuk"unftiges, z.B. Science Fiction, Prophezeiungen)
					\item
						Formen damit die Gegenwart
				\end{itemize}
		\end{itemize}
	\item
		Kultur und Vergesellschaftung:
		\begin{itemize}
			\item
				Vergesellschaftung einerseits Einbindung und Verflechtung von individuellen "Au"serungen und Handlungen zu Kommunikationen und Interaktionen, zu sinnhaften sozialen Formen $\Rightarrow$ Entstehung und Reproduktion von Kultur
				\begin{itemize}
					\item
						Nicht Zustand, sondern Prozess
					\item
						immer zeitlich strukturiert, immer Bez"uge auf Vergangenheit (Erfahrungen) und Zukunft (Erwartungen)
					\item
						findet aber immer nur in Gegenwart statt (trotzdem immer Bez"uge auf past und future)
				\end{itemize}
			\item
				Eigenlogisch soziale Tatsachen: Formen, Muster, Typen, Regeln, Institutionen
		\end{itemize}
	\item
		Medien sind relevant f"ur soziokulturelle Prozesse
		\begin{itemize}
			\item
				ver"andern Reichweite von Kommunikation (zeitlich und r"aumlich)
			\item
				formen kommunizierte Inhalte
			\item
				weil sie Einfluss auf Zeitwahrnehmung und -vorstellung haben
			\item
				Weil sie selbst eigenlogische soziale Tatsachen werden
		\end{itemize}
	\item
		Orale Gesellschaften, die sich nur durch gesprochene Sprache bilden:
		\begin{itemize}
			\item
				Kommunikation nur unter Anwesenden, konkrete Interaktion
			\item
				Lokale Gemeinschaften
			\item
				Erinnerung nur 3 Generationen, dann mythische Zeit
			\item
				Wenig Arbeitsteilung, wenig ausdifferenzierte Funktionen
			\item
				Rituale, Rythmen, Mythen, Erz"ahlungen
			\item
				zyklische Zeitvorstellungen
		\end{itemize}
	\item
		Schrift und Gesellschaftsentwicklung
		\begin{itemize}
			\item
				Kommunikation mit r"aumlich und zeitlich Abwesenden, erlaubt r"aumliche und zeitliche Ausdehnung (Gro"sreiche und Geschichte)
			\item
				symbolische Inhalte einer Kultur werden unabh"angig von konkreter Interaktion
			\item
				Ausdifferenzierung von Spezialisten (Schriftgelehrten), hierarische Differenzierung
			\item
				Entstehung von Zeitrechnung und Geschichtsschreibung
			\item
				Interpretation und Kanonisierung heiliger Schriften
		\end{itemize}
	\item
		Buchdruck und Gesellschaftsentwicklung
		\begin{itemize}
			\item
				Massenproduktion von identischen Texten f"ur anonymes Publikum
			\item
				Institutionalisierung: Autor --- Drucker/Verleger --- Buchh"andler --- Leser
			\item
				schafft Kommunikationsraum (Standardisierung der Druckschriftsprache/Nationalsprachen)
			\item
				Zensur und Urheberrecht
			\item
				(funktionale) Differenzierung der Gesellschaft
		\end{itemize}
	\item
		Buchdruck, Gesellschaftsentwicklung und Wissenschaft
		\begin{itemize}
			\item
				Neuzeitliche Wissenschaft ist aus Buchdruck entstanden, ohne ihn nicht vorstellbar.
			\item
				Sieht man am Beispiel Sterntafeln: Druckerpresse erm"oglicht Herstellung und Verbreitung von ientischen Exemplaren einer Sterntafel, Kritik und Korrektur m"oglich, Durchsetzung gegen kirliche Zensur und popul"are Vorstellungen
			\item
				"Ahnlich bei Soziologie: beruht darauf Erkenntnisse in Textform einem gr"o"seren Fachpublikum vorzustellen, kritisieren, weiterentwicklen
		\end{itemize}
	\item
		Digitalisierung als Mediendifferenzierung und Medienintegration
		\begin{itemize}
			\item
				Medienentwicklung bis Ende 20. Jhds als Mediendifferenzierung (Textmedien, Bildmedien, akustische Medien). Computer: Universalmedium, "Ubertragung vieler analoger Medien auf gemeinsame technische Basis
			\item
				Dezentrale Vernetzung, viel (unsichtbare) Infrastruktur notwendig
			\item
				Geringer Energieaufwand, oft geringe Wissensschwelle f"ur mediale Produktion, Distribution
			\item
				Neue Wissensformen werden sozial relevant
			\item
				Speicher unbegrenzt: Problem der Ordnung und Geltung des global verf"ugbaren Wissens
			\item
				neue Formen der "Offentlichkeit ohne zentralisierte Zugangskontrolle
		\end{itemize}
	\item
		Beispiel Wikipedia: unbezahlt, potentiell f"ur alle beteiligungsoffen, geringe Integration, hohe Fluktuation, demokratischer Anspruch einerseits, hierarchische Positionen (Admin, \dots) andererseits
	\item
		Beispiel Social Network: Medialisierung allt"aglicher Interaktion, allt"agliche Formen der Kommunikation (Klatsch, Flirt, aber auch Verschw"rungstheorien), soziale Beziehungen "uber biographische und r"aumliche Entfernungen hinweg, private "Offentlichkeit/ "offentliche Privatheit, keine Anonymit"at?

	\item
		Medial induzierter soziokultureller Wandel
		\begin{itemize}
			\item
				tendenziell permanente Erreichbarkeit, Ver"anderung der Zeitstukrutren (Rosa: Beschleunigung?)
			\item
				Globalisierung einerseits, Betonung kultureller Eigensinnigkeit andererseits
			\item
				Ausdifferenzierung einer Vielzahl von sozialen Wirklichkeiten: Problem der Integration
		\end{itemize}
	\item
		Folgerung f"ur Soziologie der Medien:
		\begin{itemize}
			\item
				Das mediensoziologische Dreieck (Individuen, Medien, Gesellschaft) (siehe Seite \pageref{Ziemann} um Kulturpol erweitern
			\item
				Mediensoziologie untersucht Rolle von Medien in Prozessen der Kulturentwicklung und Vergesellschaftung
		\end{itemize}
\end{itemize}

