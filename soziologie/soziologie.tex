\documentclass[a4paper, 12pt]{scrartcl}
\usepackage[ngerman]{babel}
\usepackage[utf8]{inputenc}
\usepackage{amsmath}
\usepackage{amssymb}
\usepackage{amsthm}
\usepackage{amstext}
\usepackage{csquotes}
\usepackage{listings}
\lstset{breakatwhitespace=true,
	basicstyle=\footnotesize
}
\usepackage{hyperref}
\usepackage{graphicx}
\usepackage{multirow}
\usepackage{booktabs}


\newcommand{\Romann}[1]{\uppercase\expandafter{\romannumeral#1}}

\begin{document}
\title{Einführung in die Soziologie --- Zusammenfassung}
\author{Magnus Berendes/\href{mailto:magnus.berendes@fau.de}{magnus.berendes@fau.de}}
\maketitle

{ Kein Anspruch auf Vollst"andigkeit oder Korrektheit!}

\tableofcontents
\newpage
\section{Vorlesung \Romann{1}}
\subsection{Was ist Soziologie?}
\begin{itemize}
	\item
\enquote{Die Soziologie hat es mit dem Menschen im Angesicht der ärgerlichen Tatsache der Gesellschaft zu tun}
\item
	Gesellschaft erzwingt und ermöglicht gemeinsames Handeln durch \enquote{Gewebe aus Bedeutung, Erwartung und Verhalten} in \enquote{wechselseitiger Orientierung}
\item
	Kommt durch kulturelle Festlegung von Erwartungen an soziale Positionen zustande (Ärztin, Lehrerin, \dots)
\end{itemize}
\subsection{Soziale Positionen}
\begin{itemize}
	\item
		vermitteln keine individuellen Kenntnisse über Personen (sind auch nicht individuell, sondern sozial/\enquote{überindividuell}, duh)
	\item
		Indem Individuen soziale Rollen \enquote{spielen}, entstehen \textit{Verlässlichkeit, Dauerhaftigkeit} und \textit{Erwartbarkeit} im sozialen Verkehr
	\item
		Rollen stellen Erwartungen, stellen aber auch Ausdrucksmöglichkeiten/Ansprüche für die TrägerInnen bereit.
	\item
		Beispiel: Gastgeber, guter Gastgeber macht Gäster miteinander bekannt
\end{itemize}

\subsection{Wo ist \enquote{Das Soziale} zu verorten? 3 Perspektiven auf Soziologie}

\begin{itemize}
	\item
		Wo ist das soziale zu verorten?
\begin{enumerate}
	\item
		Im Handeln der Individuen $\Rightarrow$ Soziologie ist die Wissenschaft vom Handeln der Individuen

		\enquote{eine Wissenschaft, welche soziales Handeln \textit{deutend verstehen} und dadurch in seinem Ablauf und seinen Wirkungen \textit{ursächlich erklären} will} (Max Weber 1922)

		$\Rightarrow$
		Handlungstheoretische Perspektive
	\item
		In der Wechselwirkung zwischen Individuen $\Rightarrow$ Soziologie ist die Wissenschaft von den sozialen Wechselwirkungen

		Gesellschaft entsteht, \enquote{wo mehrere Individuen in Wechselwirkung treten} (Georg Simmel 1908)

		$\Rightarrow$
		Interaktionistische Perspektive

	\item
		Tritt den Individuen als äußerer Zwang gegenüber $\Rightarrow$ Soziologie ist die Wissenschaft sozialer Institutionen

		\enquote{Wissenchaft von den Institutionen, deren Entstehung und Wirkungsart} (Emile Durkheim 1895)

		$\Rightarrow$
		Institutionalistische Perspektive

\end{enumerate}
\end{itemize}

\subsection{Woher kommt Soziologie: Entwicklung als Wissenschaft}
\begin{itemize}
	\item
		Aufklärung: Entwicklung eines Wissenschaftsverständnis von Experiment und Beobachtung
	\item
		Sozialer Wandel --- Beginn der Moderne:
		\begin{itemize}
			\item
				Soziales Leben wird prekär
			\item
				und kann gestaltet werden
		\end{itemize}
	\item
		Namensgeber: Auguste Comte (1789--1857), als \enquote{soziale Physik}
	\item
		Soziologie heute
		\begin{itemize}
			\item
				Akademische Instituionalisierung im ausgehenden 19. Jhdt
			\item
				Kernfrage: \enquote{Wie ist soziale Ordnung möglich?} (Simmel)
			\item
				Soziologie ist Teil gesellschaftlicher Selbstbeschreibung (Hinterfragung alltäglicher Selbstbeschreibung mittels Wissenschaft)
		\end{itemize}
\end{itemize}

\subsection{Soziologisches Bewusstsein}
\begin{itemize}
	\item
		Misstrauen gegenüber offiziellen \enquote{Problemdefinitionen} (z.B. Verbrechen erklären)
	\item
		Bewusstsein für die Pluralität der sozialen Wirklichkeit (Abweichung kann aus anderer Perspektive Sinn machen)
	\item
		Relativierung des Geltungsanspruchs kultureller Normen und Werte (Entstehungs und Geltungsbedingungen von Werten sind sinnvoller Gegenstand, nicht aber Wahrheit oder Richtigkeit von Normen)
\end{itemize}


\end{document}
