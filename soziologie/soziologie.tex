\documentclass[a4paper, 12pt]{scrartcl}
\usepackage[ngerman]{babel}
\usepackage[utf8]{inputenc}
\usepackage{amsmath}
\usepackage{amssymb}
\usepackage{amsthm}
\usepackage{amstext}
\usepackage{csquotes}
\usepackage{listings}
\lstset{breakatwhitespace=true,
	basicstyle=\footnotesize
}
\usepackage{hyperref}
\usepackage{graphicx}
\usepackage{multirow}
\usepackage{booktabs}
\usepackage{makeidx}


\newcommand{\Romann}[1]{\uppercase\expandafter{\romannumeral#1}}

\makeindex
\begin{document}
\title{Einführung in die Soziologie --- Zusammenfassung}
\author{Magnus Berendes/\href{mailto:magnus.berendes@fau.de}{magnus.berendes@fau.de}}
\maketitle

{ Kein Anspruch auf Vollst"andigkeit oder Korrektheit!}

\tableofcontents
\newpage
\printindex
\newpage
\section{Vorlesung \Romann{2} --- Einf"uhrung: Drei Perspektiven auf soziale Ph"anomene}
\subsection{Was ist Soziologie?}
\begin{itemize}
	\item
\enquote{Die Soziologie hat es mit dem Menschen im Angesicht der ärgerlichen Tatsache der Gesellschaft zu tun}
\item
	Gesellschaft erzwingt und ermöglicht gemeinsames Handeln durch \enquote{Gewebe aus Bedeutung, Erwartung und Verhalten} in \enquote{wechselseitiger Orientierung}
\item
	Kommt durch kulturelle Festlegung von Erwartungen an soziale Positionen zustande (Ärztin, Lehrerin, \dots)
\end{itemize}
\subsection{Soziale Positionen}
\begin{itemize}
	\item
		vermitteln keine individuellen Kenntnisse über Personen (sind auch nicht individuell, sondern sozial/\enquote{überindividuell}, duh)
	\item
		Indem Individuen soziale Rollen \index{Soziale Rollen}\enquote{spielen}, entstehen \textit{Verlässlichkeit, Dauerhaftigkeit} und \textit{Erwartbarkeit} im sozialen Verkehr
	\item
		Rollen stellen Erwartungen, stellen aber auch Ausdrucksmöglichkeiten/Ansprüche für die TrägerInnen bereit.
	\item
		Beispiel: Gastgeber, guter Gastgeber macht Gäster miteinander bekannt
\end{itemize}

\subsection{Wo ist \enquote{Das Soziale} zu verorten? 3 Perspektiven auf Soziologie}

\begin{itemize}
	\item
		Wo ist das soziale zu verorten?
\begin{enumerate}
	\item
		Im Handeln der Individuen $\Rightarrow$ Soziologie ist die Wissenschaft vom Handeln der Individuen

		\enquote{eine Wissenschaft, welche soziales Handeln \textit{deutend verstehen} und dadurch in seinem Ablauf und seinen Wirkungen \textit{ursächlich erklären} will} (Max Weber 1922)

		$\Rightarrow$
		Handlungstheoretische Perspektive
	\item
		In der Wechselwirkung zwischen Individuen $\Rightarrow$ Soziologie ist die Wissenschaft von den sozialen Wechselwirkungen

		Gesellschaft entsteht, \enquote{wo mehrere Individuen in Wechselwirkung treten} (Georg Simmel 1908)

		$\Rightarrow$
		Interaktionistische Perspektive

	\item
		Tritt den Individuen als äußerer Zwang gegenüber $\Rightarrow$ Soziologie ist die Wissenschaft sozialer Institutionen

		\enquote{Wissenchaft von den Institutionen, deren Entstehung und Wirkungsart} (Emile Durkheim 1895)

		$\Rightarrow$
		Institutionalistische Perspektive

\end{enumerate}
\end{itemize}

\subsection{Woher kommt Soziologie: Entwicklung als Wissenschaft}
\begin{itemize}
	\item
		Aufklärung: Entwicklung eines Wissenschaftsverständnis von Experiment und Beobachtung
	\item
		Sozialer Wandel --- Beginn der Moderne:
		\begin{itemize}
			\item
				Soziales Leben wird prekär
			\item
				und kann gestaltet werden
		\end{itemize}
	\item
		Namensgeber: Auguste Comte (1789--1857), als \enquote{soziale Physik}
	\item
		Soziologie heute
		\begin{itemize}
			\item
				Akademische Instituionalisierung im ausgehenden 19. Jhdt
			\item
				Kernfrage: \enquote{Wie ist soziale Ordnung möglich?} (Simmel)
			\item
				Soziologie ist Teil gesellschaftlicher Selbstbeschreibung (Hinterfragung alltäglicher Selbstbeschreibung mittels Wissenschaft)
		\end{itemize}
\end{itemize}

\subsection{Soziologisches Bewusstsein}
\begin{itemize}
	\item
		Misstrauen gegenüber offiziellen \enquote{Problemdefinitionen} (z.B. Verbrechen erklären)
	\item
		Bewusstsein für die Pluralität der sozialen Wirklichkeit (Abweichung kann aus anderer Perspektive Sinn machen)
	\item
		Relativierung des Geltungsanspruchs kultureller Normen und Werte (Entstehungs und Geltungsbedingungen von Werten sind sinnvoller Gegenstand, nicht aber Wahrheit oder Richtigkeit von Normen)
\end{itemize}

\section{Vorlesung \Romann{3} --- Grundbegriffe \Romann{1}: Handeln, Normen}
\subsection{Verhalten, Handeln, soziales Handeln}
\begin{itemize}
	\item
		\enquote{Flirt mit Folgen}
	\item
		Rollen legen Erwartungen an das eigene Verhalten und der jeweils anderen fest
	\item
		Verhaltenskoordination bei Erwartungsunsicherheit schwierig
	\item
		\index{Soziales Handeln}
		Soziales Handeln (Handeln, das auf andere bezogen ist (begrüßen, beobachten, \dots) $\in$ Handeln (das nicht auf andere bezogen ist (den Raum betreten), verfolgt Ziel, setzt intentional Mittel ein, wählt zwischen Alternativen) $\in$ Verhalten (auch unabsichtliche Aktivität, stolpern)
	\item
		Sinn einer Handlung wird von anderen interpretiert, Folgen müssen nicht identisch mit dem intendierten Sinn sein
	\item
		Die Intention von Sozialem Handeln ist auf das Verhalten anderer gerichtet. Sinnorientierungen des Handelns werden gesellschaftlich mitgeprägt (kann sonst von anderen nicht verstanden werden, z.B. Begrüßungsregeln)
	\item
		Gesellschaftlicher Wandel ist über den Wandel der Sinnorientierungen des Handelns erfassbar

	\item
		Sinnorientierungen sozialen Handlens (aka. \enquote{Wo kommen eigentlich die Motive her?} aka. Webersche Handlungstypologie\index{Webersche Handlungstypologie})
		\begin{itemize}
			\item
				Traditional: Sitten und Gebr"auche (Spende aufgrund etablierter Sitten)
			\item
				Affektuell: Emotionen (Spenden aus Mitef"uhl)
			\item
				Wertrational: bewusster Glaube an den ethischen/religi"osen Eigenwert eines Sachverhalts, unabh"angig vom Erfolg (Spende aufgrund spezifischen Ethos, z.B.\ katholische Soziallehre)
			\item
				Zweckrational: orientiert sich an subjektiv als ad"aquat vorgestellten Mitteln f"ur subjektiv eindeutig erfasste Zwecke (Spende um spezifisches Ziel zu verfolgen, z.B.\ Erh"ohung des eigenen Ansehens)
		\end{itemize}
		Keine empirische Motive sondern ideele Typen! In freier Wildbahn treten diese Typen nicht rein auf
	\item
		Welcher Teil sozialen Handels wird bei welchem Handeln kontrolliert? (Nur Reihenfolge merken und dass es absteigend ist, eigentlich auch trivial)

		\begin{tabular}[H]{lllll}
			& Mittel & Zweck & Wert & Folge \\ \hline
			zweckrational & + & + & + & + \\ \hline
			wertrational & + & + & + & - \\ \hline
			affektuell & + & (+) & - & - \\ \hline
			traditional & (+) & - & - & - \\ \hline
		\end{tabular}


	\item
		\begin{enumerate}
			\item Gesellschaft basiert auf individuellem Verhalten 
	\item$\rightarrow$ Verhalten weist Spielr"aume auf 
	\item$\rightarrow$ Konkretes Verhalten ist Folge einer Entscheidunge 
	\item$\rightarrow$ Verhaltensselektion ist sozial mitbestimmt (nur sozial erw"unschtes Verhalten wird gezeigt) 
	\item$\rightarrow$ und von vorne\dots
		\end{enumerate}

	\item
		Kritik an Weberscher Handlungstypologie
		\begin{itemize}
			\item
				Soziale Bestimmungsgr"unde der Sinnorientierung spielen in seiner Typologie keine Rolle, orientiert sich nur an subjektiv vom Handelnden gemeinten Sinn
				\item
					Weiterentwicklung: Talcott Parsons verbindet Handlungstheorie und Rollentheorie
		\end{itemize}
	\item
		Talcott Parsons Pattern Variables (Beziehungsvariablen, Mustervariablen): Verortung von sozialem Handeln in f"unfdimensionalem Raum
		\index{Pattern Variables}\index{Beziehungsvariablen}

		\begin{tabular}[H]{lcl}
			Familie/Gemeinschaft & & Gesellschaft\\\hline
			Affektivit"at (Bsp: Freund) & vs. & Affektive Neutralit"at (Bsp: Lehrer) \\ \hline
			Kollektivbezogenheit (Priester) & vs. & Selbstorientierung (Manager) \\ \hline
			Partikularismus (Gruppen-normen, Eltern) & vs. & Universalismus (allgemeine Normen, Richter) \\ \hline
			Askriptive Zuschreibung (Herkunft, Alter) & vs. & Erwerb durch Leistung (Beruf) \\ \hline
			Diffusit"at (ganze Person, Eltern) & vs. & Spezifit"at (Friseur) \\ \hline
		\end{tabular}

		Beispiel Arzt:
		\begin{itemize}
			\item
				Affektiv neutral: ohne Ansehen der Person behandeln
			\item
				Eher kollektiv bezogen, zumindest laut Rollenerwartung
			\item
				Universalistisch: Auch Schwarzer sollte sich von Wei"sem Arzt behandeln lassen
			\item
				Rollenerwerb Leistungsbezogen: durch Studium und Approbation
			\item
				Spezifisch: Soll sich um Verletzung/Krankheit k"ummern, nicht um ganze Person
		\end{itemize}
\end{itemize}

\subsection{Normen}
\begin{itemize}
	\item
		Was sind Normen? soziale ("uberindividuelle) Festlegungen der Bedeutung von Verhalten in einer Situation: Normen legen unsere Erwartungen an das Verhalten anderer fest.\index{Soziale Norm}

		K"onnen widerspr"uchlich sein: Ehrlichkeit --- Erfolg
	\item
		\enquote{Der Begriff der Norm bezieht sich auf konkretere Vorgaben, mit welchen Mitteln und auf welche Weise erstrebenswerte Ziele erreicht werden sollten und welchen Ma"sst"aben ein Handeln zu gen"ugen hat} (Schwietring)
	\item
		Verhaltenserwartungen: du sollst, du musst, du kannst (nicht)
	\item
		Abweichung wird sanktioniert (positiv/negativ: Gratifikation, Sch"atzung, Gericht/Ausschluss, Antipathie)
	\item
		Norm und Abweichung sind zwei Seiten derselben Medaille
	\item
		Nicht jede Erwartungsentt"auschung f"uhrt zur Sanktion: Korrektiver Austausch
	\item
		Abweichendes Verhalten kann pathologisiert oder kriminalisiert werden

	\item
		Individualisierungschancen nicht in Konformit"at sondern in Abweichung!
	\item
		\textit{Werte} rahmen und legitimieren Normen: kulturell verbreitete Vorstellungen des W"unschbaren, Erstrebenswerten, Wertvollen. Definieren allgemeines Ziel, geben Orientierung, legen kein Handeln fest

		Beispiel: Wert romantische Liebe als Beziehungsideal. Normen: Inszenierung von Romantik, Liebesbekundungen, Heiratsantrag
	\item
		Normen/Werte lernt man durch internalisierung, Rollen durch sozialisation
	\item
		\textit{Handlungsrelevant} werden (internalisierte) Normen in der handlungstheoretischen Perspektive dann in Form von Wissen: Man \enquote{wei"s} wie man flirtet. $\Rightarrow$ Zugang zu Norm

	\item
		\index{Akzeptanz von Werten und Normen}
		Akzeptanz von Werten und Normen nach Robert K. Merton

		\begin{tabular}[H]{lll}
			Werte/Ziele akzeptiert & Normen/Mittel akzeptiert & Anpassungsmuster\\ \hline
			ja & ja & Konformismus \\ \hline
			ja & nein & Innovation \\ \hline
			nein & ja & Ritualismus \\ \hline
			nein & nein & Apathie und R"uckzug \\ \hline
			alternativ & alternativ & Rebellion \\ \hline
		\end{tabular}
\end{itemize}

\subsection{Fazit/Rekap}
\begin{itemize}
	\item
		Handeln: Folge von Abw"agung und daruch versteh- und gestaltbar
	\item
		Handeln hat Folgen und Nebenfolgen; orientiert sich an Normen
	\item
	Normen machen Handeln erwartbar, sie sind im Wissen pr"asent (Wo ist das Soziale zu verorten? Im Wissen handelnder Akteure)
\item
	Norm und Abweichung sind wechselseitig aufeinander bezogene Begriffe
\item
	\enquote{Abweichendes} Verhalten ist wichtige Quelle sozialen Wandels
\item
	Anerkannte Ziele f"ordern auch \enquote{abweichende} Strategien ihres Erreichens, wenn die anerkannten Mittel vorenthalten werden
\item
	Werte wirken nicht nur integrativ sondern auch ausschlie"send
\item
	Pries 2014 Kap. 3 u. 4
\end{itemize}



\section{Vorlesung \Romann{4} --- Grundbegriffe \Romann{2}: Interaktion, Kommunikation}
\subsection{Normen und Interaktionskrisen}
\begin{itemize}
	\item
		Normen sind uns selbstverst"andlich, in \enquote{Fleisch und Blut}
	\item
		Wie bekommt man raus, welche Normen in Situation eine Rolle spielen? Verletzung der Normen und Analyse der Reaktionen
	\item
		\index{Ethnomethodologie}
		Ethnomethodologie: Untersuchung der Methoden, mithilfe der Menschen allt"aglich ihre soziale Welt und deren normative Ordnung hervorbringen und verst"andlich machen. Kritik an der Theorie Parsons, nach der Werte/Normen internalisiert seien.
	\item
		Werden Normen internalisiert? Auf jeden Fall, Schuldbewusstsein. Was folgt daraus f"ur die Regulierung von Interaktionen? Kulturelle Normen werden befolgt, weil sie die Normen "uberwiegend verinnerlicht haben.

		$\Rightarrow$ stimmt nur, wenn wir auf die Normen schauen, nicht wenn wir auf die Interaktionen schauen (TODO)
	\item
		Normen werden von jedem Paar im Prozess der Paarwerdung ausgehandelt. Norm legt nicht fest, wie sie in Paarbeziehung "ubersetzt wird. Das geschieht durch Aushandlungsprozess.
	\item
		Normen werden nicht durch Verinnerlichung handlungsleitend, werden aber in Interaktionssituationen praktisch verfestigt
	\item
		Interaktionen sind Orte der Entstehung, Anwendung und des Wandels von Normen, Werten und Institutionen
\end{itemize}

\subsection{Interaktion}

\begin{itemize}
	\item
		\index{Interaktion}
		Mind. zwei Akteure, die ihr Verhalten wechselseitig aufeinander abstimmen, mittels der symbolischen Bedeutung ihres Tuns (Handwedeln \textit{bedeutet} Gr"u"sen)
	\item
		In Situation gemeinsamer Anwesenheit kann man nicht nicht kommunizieren
	\item
		Verhalten anderer wird in Situation gedeutet $\Rightarrow$ wir definieren die Situation
	\item
		Bedeutungen werden gemeinsam geteilt (hei"st nicht, dass man mit Situationsdefinitionen anderer "ubereinstimmen muss.) Handlungskoordination setzt gemeinsame Situationsdefinition/geteilte Bedeutungen vorraus
	\item
		Interaktionen folgen nicht nur Absichten sondern der Wechselwirkung zwischen Akteuren
	\item
		Symbolischer Interaktionismus (Herbert Blumer) \index{Symbolischer Interaktionismus}
		\begin{itemize}
			\item
				Pr"amisse: Wir handeln gegen"uber Dingen und Anderen aufgrund von Bedeutungen die diese Dinge und Andere f"ur uns haben
			\item
				Bedeutungen sind aus sozialen Interaktionen mit Mitmenschen abgeleitet, in denen sie durch das Handeln definiert werden
			\item
				Bedeutung weder immanente Qualit"at der Dinge noch psychische Leistung der Wahrnehmung (Stuhl kann auch ein Hindernis oder Brennholz sein)
			\item
				Was situativ bedeutsam ist, ist Folge eines aktiven Interpretationsprozesses (Definition der Situation)\index{Definition der Situation}
			\item
				Interaktion: Austausch von Gesten und Symbolen zwischen mehreren Individuen, die miteinander und aufeinander bezogen handeln\index{Interaktion}
			\item
				Symbol: Zeichen dessen Bedeutung geteilt wird (Kommunikative Welt = symbolische Welt)
		\end{itemize}

	\item
		Thomas Theorem: \enquote{if men define situations as real, they are real in their consequences} \index{Thomas-Theorem}

				F"uhrt zu Self fulfilling prohecies: \enquote{Falsche} Situationsdefinition $\rightarrow$ Orientierung des Handelns daran $\rightarrow$ Best"atigung der Annahme/Reale Konsequenzen
\end{itemize}

\subsection{Kommunikation}
\begin{itemize}
	\item
		\index{Kommunikation}\enquote{Kommunikation ist menschliche Verhaltensabstimmung mittels symbolischer Mittel, die in soziale Praktiken eingebettet sind} (Reichertz)
	\item
		Wie ist Kommunikation m"oglich?

		Kommunikation nach Luhmann aus drei Gr"unden unwahrscheinlich
		\begin{itemize}
			\item
				Unwahrscheinlichkeit des Verstehens: Jeder versteht die Welt auf Grundlage seiner individuellen Biografie $\Rightarrow$ jeder unterschiedlich
			\item
				Unwahrscheinlichkeit des Erreichens Abwesender: Kommunikation erreicht vermutlich nicht mehr Personen als in Situation anwesend sind
			\item
				Unwahrscheinlichkeit der Annahme: Dass ein Adressat eine Mitteilung als Pr"amisse seines Verhaltens "ubernimmt ist unwahrscheinlich
		\end{itemize}
\end{itemize}



\end{document}
