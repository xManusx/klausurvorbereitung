\section{Vorlesung \Romann{9} --- Bildung und Lebenslauf: "Uber Liebe und Beziehungsvorstellungen}

\begin{displayquote}
	F"ur die Soziologie gilt, dass Liebe ein Kommunikationscode ist, mit dem die Akteure in der Lage sind, eine romantische Liebesbeziehung zu konstruieren. (Luhmann)
\end{displayquote}
Liebe als Kulturmunster intrepretiert: abh"angig von gesellschaftlichen Verh"altnissen, Orten, Zeiten, in denen sie gelebt wurde und wird

\subsection{Liebe als ein Kulturmuster der Moderne}
\begin{itemize}
	\item
		In Vormoderne:
		\begin{itemize}
			\item
				ganzes Haus zentrale Lebenswelt, mehrere Generationen
			\item
				Produktion geh"ort auch mit dazu
			\item
				Ehe zwar nicht unwichtig, geh"ort aber zu St"andegesellschaft und ist eingebettet in das ganze Haus
			\item
				Liebe keine Rolle bei Eheschlie"sung: stattdessen Zuverl"assigkeit, Achtung des Partners, Gesundheitszustand und Besitz
			\item
				Kaum Raum f"ur Individualit"at und Intimit"at
			\item
				Ab 2. H"alfte 18. Jhdt entsteht in D neues Ehe und Familienbild
		\end{itemize}
	\item
		Dimensionen des Modernisierungsprozesses
		\begin{itemize}
			\item
				Differenzierung, zunehmende Arbeitsteilung (Ort von Produktion und Reproduktion wird getrennt, Privatheit entsteht)
			\item
				Rationalisierung (Welt wird mit Wissenschaft und Technik beherrschbar gemacht)
			\item
				Individualisierung (Herausl"osung des Individuums aus traditionalen Sozialformen und -beziehungen) 
		\end{itemize}
		$\Rightarrow$ Geburtsstunde der \enquote{modernen Familie}
	\item
		Georg Simmel: Durch Individualisierung wird Liebe m"oglich und notwendig. Durch soziale Distanz in komplexer werdender Welt entsteht Bed"urfnis nach N"ahe und Intimit"at
	\item
		Max Weber: sozialstrukturelle Ver"anderung bef"ordern die Bildung von Familie: neues rationales Denken kann keinen Lebenssinn geben, Entzauberung der Welt f"uhrt zu \enquote{Sinn und Orientierungsverlust}. In einer Welt fehlenden Sinns werden Intimbeziehungen, Familie und Liebe notwendig
	\item
		Zwischenfazit: Modernisierungsprozess ist N"ahrboden f"ur das Bed"urfnis nach Intimit"at und N"ahe und das Ph"anomen der \enquote{romantischen Liebe}
	\item
		Romantik (1790 -- 1830) als kulturelle und soziale Bewegung:
		\begin{itemize}
			\item
				beginnt als Literatur- und Kunstkritik
			\item
				Von Jugend begeistert aufgenommen
			\item
				L"asst sich als eine Art Protestbewegung gegen Rationalisierung charakterisieren
		\end{itemize}
	\item
		Zentrale Elemente der romantischen Liebe:
		\begin{itemize}
			\item
				Zuneigung und Sexualit"at geh"oren zusammen
			\item
				Einheit von Liebe und Ehe
			\item
				Integration von Elternschaft
			\item
				Exklusivit"atsanspruch
			\item
				Liebe wird zur wichtigsten Angelegenheit im Leben (Entwertung der Umweltbez"uge)
			\item
				Erwiderte Liebe wird zur eigentlichen Liebe (Symmetrie der Geschlechter)
		\end{itemize}
	\item
		Romantische Liebe zwischen Idee und Praxis:
		\begin{itemize}
			\item
				Unterschied zwischen Diskursebene (Idee der romantischen Liebe wird propagiert) und deren Wirksamwerden in der Beziehungsnorm der RZB
		\end{itemize}
\end{itemize}

\subsection{Liebe und Beziehung in Gegenwart}
\begin{itemize}
	\item
		Einige Thesen:
		\begin{itemize}
			\item
				Niedergangsthese: romantische Liebe als Auslaufmodell
			\item
				das Therapeutische Liebesmodel
			\item
				Liebe gewinnt an Bedeutung und Liebesfragen zu existentiellen Fragen
			\item
				Gegenw"artige Liebesleitbilder sind zugleich romantisch und entromantisiert
		\end{itemize}
	\item
		Grundidealisierung von unbegrenzter Dauer und Treueprinzip gilt immer noch: serielle Monogamie
	\item
		Liebe ist nicht zu haben ohne Kehrseite: Entt"auschung, Erbitterung, Ablehnung, Hass
	\item
		Giddens: romantische Liebe wird durch partnerschaftliche Liebe abgel"o"st (pure relationship)
		\begin{itemize}
			\item
				Beziehung die man um ihrer selbst willen eingeht
			\item
				Gleichheit der Geschlechter
			\item
				Aber: M"anner irgendwie Schl"usselrolle auf dem Weg zur pure relationship
		\end{itemize}
\end{itemize}
