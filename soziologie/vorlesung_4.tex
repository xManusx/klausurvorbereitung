\section{Vorlesung \Romann{4} --- Grundbegriffe \Romann{2}: Interaktion, Kommunikation}
\subsection{Normen und Interaktionskrisen}
\begin{itemize}
	\item
		Normen sind uns selbstverst"andlich, in \enquote{Fleisch und Blut}
	\item
		Wie bekommt man raus, welche Normen in Situation eine Rolle spielen? Verletzung der Normen und Analyse der Reaktionen
	\item
		\index{Ethnomethodologie}
		Ethnomethodologie: Untersuchung der Methoden, mithilfe der Menschen allt"aglich ihre soziale Welt und deren normative Ordnung hervorbringen und verst"andlich machen. Kritik an der Theorie Parsons, nach der Werte/Normen internalisiert seien.
	\item
		Werden Normen internalisiert? Auf jeden Fall, Schuldbewusstsein. Was folgt daraus f"ur die Regulierung von Interaktionen? Kulturelle Normen werden befolgt, weil sie die Normen "uberwiegend verinnerlicht haben.

		$\Rightarrow$ stimmt nur, wenn wir auf die Normen schauen, nicht wenn wir auf die Interaktionen schauen (TODO)
	\item
		Normen werden von jedem Paar im Prozess der Paarwerdung ausgehandelt. Norm legt nicht fest, wie sie in Paarbeziehung "ubersetzt wird. Das geschieht durch Aushandlungsprozess.
	\item
		Normen werden nicht durch Verinnerlichung handlungsleitend, werden aber in Interaktionssituationen praktisch verfestigt
	\item
		Interaktionen sind Orte der Entstehung, Anwendung und des Wandels von Normen, Werten und Institutionen
\end{itemize}

\subsection{Interaktion}

\begin{itemize}
	\item
		\index{Interaktion}
		Mind. zwei Akteure, die ihr Verhalten wechselseitig aufeinander abstimmen, mittels der symbolischen Bedeutung ihres Tuns (Handwedeln \textit{bedeutet} Gr"u"sen)
	\item
		In Situation gemeinsamer Anwesenheit kann man nicht nicht kommunizieren
	\item
		Verhalten anderer wird in Situation gedeutet $\Rightarrow$ wir definieren die Situation
	\item
		Bedeutungen werden gemeinsam geteilt (hei"st nicht, dass man mit Situationsdefinitionen anderer "ubereinstimmen muss.) Handlungskoordination setzt gemeinsame Situationsdefinition/geteilte Bedeutungen vorraus
	\item
		Interaktionen folgen nicht nur Absichten sondern der Wechselwirkung zwischen Akteuren
	\item
		Symbolischer Interaktionismus (Herbert Blumer) \index{Symbolischer Interaktionismus}
		\begin{itemize}
			\item
				Pr"amisse: Wir handeln gegen"uber Dingen und Anderen aufgrund von Bedeutungen die diese Dinge und Andere f"ur uns haben
			\item
				Bedeutungen sind aus sozialen Interaktionen mit Mitmenschen abgeleitet, in denen sie durch das Handeln definiert werden
			\item
				Bedeutung weder immanente Qualit"at der Dinge noch psychische Leistung der Wahrnehmung (Stuhl kann auch ein Hindernis oder Brennholz sein)
			\item
				Was situativ bedeutsam ist, ist Folge eines aktiven Interpretationsprozesses (Definition der Situation)\index{Definition der Situation}
			\item
				Interaktion: Austausch von Gesten und Symbolen zwischen mehreren Individuen, die miteinander und aufeinander bezogen handeln\index{Interaktion}
			\item
				Symbol: Zeichen dessen Bedeutung geteilt wird (Kommunikative Welt = symbolische Welt)
		\end{itemize}

	\item
		Thomas Theorem: \enquote{if men define situations as real, they are real in their consequences} \index{Thomas-Theorem}

				F"uhrt zu Self fulfilling prohecies: \enquote{Falsche} Situationsdefinition $\rightarrow$ Orientierung des Handelns daran $\rightarrow$ Best"atigung der Annahme/Reale Konsequenzen
\end{itemize}

\subsection{Kommunikation}
\begin{itemize}
	\item
		\index{Kommunikation}\enquote{Kommunikation ist menschliche Verhaltensabstimmung mittels symbolischer Mittel, die in soziale Praktiken eingebettet sind} (Reichertz)
	\item
		Wie ist Kommunikation m"oglich?

		Kommunikation nach Luhmann aus drei Gr"unden unwahrscheinlich
		\begin{itemize}
			\item
				Unwahrscheinlichkeit des Verstehens: Jeder versteht die Welt auf Grundlage seiner individuellen Biografie $\Rightarrow$ jeder unterschiedlich
			\item
				Unwahrscheinlichkeit des Erreichens Abwesender: Kommunikation erreicht vermutlich nicht mehr Personen als in Situation anwesend sind
			\item
				Unwahrscheinlichkeit der Annahme: Dass ein Adressat eine Mitteilung als Pr"amisse seines Verhaltens "ubernimmt ist unwahrscheinlich
		\end{itemize}

		Antwort: es gibt Kommunikationsmedien:
		\begin{itemize}
			\item
				Notwendige Bedingung von Kommunikation, machen Kommunikation H"or-, Sicht- oder Sp"urbar
			\item
				Haben materiales Substrat
		\end{itemize}
	\item
		Kommunikation ist Einheit von Information (Was gesagt wird), Mitteilung (Wie es gesagt wird), Verstehen
	\item
		Kommunikation ist auch Interaktion, Interaktion ist allerdings an Kopr"asenz gebunden, Kommunikation nicht (Verbreitungsmedien "uberschreiten r"aumliche und zeitliche Grenzen
	\item
		\index{Erfolgsmedien}Erfolgsmedien (symbolische generalisierte Meiden) steigern Annahmewahrscheinlichkeit, z.B. Geld, Macht, Liebe, Moral
\end{itemize}
