\section{Vorlesung \Romann{7} --- Grundbegriffe \Romann{5} + \Romann{6}: Macht, Herrschaft, funktionale Differenzierung, Integration}

\subsection{Macht, Herrschaft}
\begin{itemize}
	\item
		Beispiel Kreuzfahrtschiff

		Geschichte:
		Wir machen eine Kreuzfahrt
		auf dem Schiff gibt es pools, darum sind liegestuehle gruppiert
		weniger liegestühle als passagiere
		reichen aber trotzdem normalerweise, weil nicht alle gleichzeitig an den pool wollen
		Liegestühle sind für alle frei verfügbar
		Das ist die Ausgangsbedingung

		Kreuzfahrtschiff legt an, Teil der Passagiere wechselt. Paar steigen ein, paar steigen aus
		Neuankömmlinge sind Deutsche, 1. Tat: Liegestühle mit Handtüchern belegen - classic
		Ist nicht durchsetzbar, handtücher werden wieder weggeräumt

		Gruppe der neuen Passagiere setzt jetzt Belegung der Stühle mit Handtüchern durch indem der teil der neuen Gruppe, der gerade am Pool ist, denjenigen die nicht zur Gruppe gehören, indem sie durch Rufen und Gesten das Wegräumen der Handtücher verhindern
		Abschreckungsgesten sind wirkungsvoll, niemand will Gewalt ausüben.
		Folge: Neue Gruppe kann durchsetzen, dass Teil der Liegestühle (mit handtüchern belegt) dauerhaft von ihnen belegt wird, ohne dass jemand drauf liegen würde

		Liegestühle werden zu einem exklusiven Gebrauchsgut $\Rightarrow$ teilt zwei Klassen auf dem Schiff: Neuankömmlinge (blockieren Liegestühle für sich) und Altpassagiere (ohne Möglichkeit Liegestühle frei zu nutzen)

		Negatives Privileg wird etabliert, Altpassagiere sind von Nutzung ausgeschlossen

		Neuankömmlinge fangen an, Liegstühle zu vermieten: zunächst gegen Dienstleistung oder Naturalien (Kaffe holen, Eis,...)
		Mit Einnahmen aus dieser Vermietung finanizert die Gruppe der Liegestuhlbesitzer eine kleine Gruppe von Wächtern, die auf Stühle aufpasst (Besitzer wollen ja auch Kreuzfahren und chillen)

		$\Rightarrow$ Dreigliedrige Struktur:
		\begin{itemize}
			\item Liegestuhlbesitzer (liegen rum, trinken Kaffee)
			\item Wächter (setzen Monopol auf Liegestühle durch)
			\item Große Gruppe von Leuten, die Liegestühle nutzen wollen (müssen jetzt Leistung dafür erbringen)
		\end{itemize}
	\item
		Quellen der Macht: Interessensolidarit"at der Besitzer und Organisationsf"ahigkeit der Besitzerinteressen
	\item \index{Macht}
		Max Weber: Macht bedeutet jede Chance (Sinnbezug, Macht als Potenzial), innerhalb einer sozialen Beziehung (Relation zwischen Akteuren, Macht ist genuin soziales Ph"anomen, Macht ist keine Qualit"at einer Person sondern einer sozialen Beziehung) den eigenen Willen auch gegen Widerstreben (Verhaltens"anderung, aber nicht notwendigerweise Zwang) durchzusetzen, gleichviel worauf diese Chance beruht
	\item
		Machtquellen:
		\begin{itemize}
			\item  k"orperliche "Uberlegenheit, drohender Zwang
			\item
				Pers"onlichkeit: "Uberzeugung, Einfl"usterung, Einfluss
			\item
				Besitz/Eigentum, Kontrolle gesellschaftlich relevanter Ressourcen
			\item
				Organisation, "uberlegene Formen der Kooperation
		\end{itemize}

	\item
		Wirkungsmechanismen von Macht:
		\begin{itemize}
			\item
				Sanktion\index{Sanktion}
			\item
				Kompensation (F"ahigkeit die Unterwerfung zu belohnen)\index{Kompensation}
			\item
				Manipulation (Verschleierung der Unterwerfung), \enquote{hohe Kunst der Macht\-aus\-"ub\-ung}, da sie diese verschleiert\index{Manipulation}
		\end{itemize}

	\item
		Gewalt\index{Gewalt}
		\begin{itemize}
			\item
				spezifische Form von Macht: Verletzungsmacht
			\item
				Macht ist (in soziologischem Sinne, nicht in normativen) konstruktiv, richtet sich auf soziale Beziehung, Gewalt nicht unbedingt
			\item
				Anwendung von Gewalt muss nicht dem Opfer gelten, kann sich auch auf Zuschauer beziehen (Machtdemonstration)
			\item
				Kann soziale Machtbeziehung stabilisieren, Vorraussetzung: Begrenzung der Machtmittel (Warum?)
		\end{itemize}

	\item Institutionalisierung von Macht (=Herrschaft)\index{Herrschaft} nach Popitz:
		Ist Macht auf Dauer so konstituiert, dass der eine m"achtiger ist als der andere $\Rightarrow$ stabile soziale Ordnung $\Rightarrow$ Nicht mehr Macht, sondern Herrschaft
		\begin{enumerate}
			\item
				Abl"osung der Macht von personalen Konstellationen hin zu Funktionen und Positionen
			\item
				Abl"osung der Machtaus"ubung von Willk"ur und Orientierung an feststehenden Verfahrensweisen und Regeln (Formalisierung)
			\item
				Integration der Macht in "ubergreifende Ordnungsgef"uge, Machtbeziehung wird f"ur Au"senstehende sicht- und berechenbar
		\end{enumerate}

	\item
		Stufen der Institionalisierung von Macht nach Popitz:
		\begin{enumerate}
			\item
				Sporadische Macht
			\item
				Normierende Macht (Machthaber kann regelm"a"sig gleichartige Verhaltensweisen fordern), erfordert gewisse Konzentration von Gewaltmitteln im Machtzentrum
			\item
				Positionale Macht: Macht an soziale Position gebunden
			\item
				Positionsgef"uge der Macht (Herrschaftsapparat, Staffelung von Macht und Ohnmacht in Machtpyramide)
				\begin{itemize}
					\item
						Sozialer Mechanismus der Zwei-Fronten-Schicht: die oberen achten, die unteren verachten
					\item
						Interesse der Individuen wird auf Erhalt der eigenen Statusposition umgelenkt (z.B. Beispiel, W"achtergruppe)
				\end{itemize}
		\end{enumerate}
	\item
		\index{Herrschaft}
		Herrschaft nach Max Weber: Herrschaft soll hei"sen die Chance, f"ur einen Befehl bestimmten Inhalts bei angebbaren Personen Gehorsam zu finden
		\begin{itemize}
			\item
				Auf soziale Beziehung gerichtet
			\item
				Muss (wie jedes soziale Ph"anomen) symbolisiert weren
			\item
				Herrschaftsordnungen sind nur stabil, wenn sie legitim gelten
		\end{itemize}

	\item
		3 Typen legitimer Herrschaft nach Max Weber
		\begin{itemize}
			\item
				Traditionale Herrschaft (z.B. K"onig)
				\begin{itemize}
					\item
						Autorit"at legitimiert durch Tradition und Satzung
					\item
						Gehorsam durch Glaube an die Heiligkeit der Traditionen
					\item
						Auch: Eltern-Kinder
				\end{itemize}
			\item
				Charismatische Herrschaft (F"uhrer)
				\begin{itemize}
					\item
						Autorit"at durch F"uhrerprinzip, legitimiert durch Ausstrahlung
					\item
						Gehorsam durch Glaube und Anerkennung der Pflichten der Untergebenen
				\end{itemize}
			\item
				Legale, b"urokratische Herrschaft
				\begin{itemize}
					\item
						Autorit"at durch Amt und Fachwissen
					\item
						Gehorsam: Verfahren und Amtsf"ormigkeit, Expertise
				\end{itemize}
		\end{itemize}
	\item
		Herrschaftsorganisation:
		\begin{itemize}
			\item
				Stabile Herrschaft muss legitim \textit{und} organisiert sein
			\item
				Verwaltungsapparat = Ausdifferenzierung von sozialen Positionen, die mit unterschiedlichen Entscheidungsbefugnissen ausgestattet sind (Minister/Abteilungsleiter/usw.) (Institutionalisierung von W"achtern im Beispiel ist fr"uhe Form von Herrschaftsorganisation)
			\item
				Klassenherrschaft: Klassendifferenzierung nach (Nicht-)Eigentum an Produktionsmitteln

				Herrschaft als institutionalisierte (etwa: Privateigentum an Produktionsmitteln) und organisierte (kapitalistischer Erwerbsbetrieb) Machtaus"ubung einer Klasse "uber eine andere Klasse. Legitimation: Leistung
		\end{itemize}
\end{itemize}

\subsection{Funktionale Differenzierung/Integration}
\begin{itemize}
	\item Funktionale Differenzierung sozialer Ordnung
		\begin{itemize}
			\item
				Nur aus Makroperspektive analysierbar (nicht aus Handlungstheoretischer oder Interaktionistischer Perspektive)
			\item
				Gibt unterschiedliche Benennungen (und Theorien!) funktionaler Differenzierung
		\end{itemize}
	\item \index{Systemtheorie} \index{Luhmann}
		Theorie sozialer Systeme: Niklas Luhmann
		\begin{itemize}
			\item Systeme bestehen aus Elementen und Relationen
			\item Bilden sich durch Schlie"sung (selbstoperierend)
			\item
				Materie und Gedanken sind Umwelt sozialer Systeme
			\item
				Gesellschaft: Umfassendes Sozialsystem aller f"ureinander erreichbarer Kommunikationen
			\item
				TODO
				%TODO
		\end{itemize}
	\item
		\index{Funktionale Differenzierung}
		Funktionale Differenzierung: moderne Gesellschaft ist funktional differenziert: zentrale Funktionen der Gesellschaft werden von einzelnen Teilsystemen ("Okonomie, Politik, Wissenschaft, Kunst, Recht, Religion, u.a.) autonom und nach Eigenlogiken erbracht. 
		\begin{itemize}
			\item
				Ein ver"andertes Handeln mit Blick auf das "Okosystem Erde wird m"oglich durch die jeweiligen Eigenlogiken dieser Funktionssysteme (ohne Wissenschaft keine Klimawandeldiskussion), aber auch beschr"ankt durch diese Eigenlokigen (Wirtschaft kann nur im Modus zahlen/nicht zahlen damit umgehen)
			\item
				Erm"oglicht Komplexit"atssteigerung und Absorption von Konflikten
			\item
				Komplexit"atssteigerung wird mit Steuerungsdefizit erkauft: Gesellschaft hat keinen zentralen Ort der Selbststeuerung mehr
		\end{itemize}
\end{itemize}

