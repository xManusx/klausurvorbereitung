\section{Vorlesung \Romann{6} --- Grundbegriffe \Romann{4}: Soziale Ordnung, sozialer Wandel}
\subsection{Soziale Ordnung}
\begin{itemize}
	\item
		Kernfrage der Soziologie: Wie ist soziale Ordnung m"oglich?
	\item
		Was ist das "uberhaupt?
	\item
		Nach Hobbes (Leviathan): Soziale Ordnung als Produkt zweckrationalen utilitaristischen Handelns von Individuen (w"urden uns sonst alle umbringen, deswegen Staaten und so)
		\begin{itemize}
			\item
				Klassisch b"urgerliche Position: Gewaltmonopol sichert die friedliche Verfolgung der Interessen Freier und Gleicher
			\item
				Nach Hobbes: B"urgerkrieg wird durch Vertrag beendet
			\item
				Problem dabei (Hobbessche Problem): wie kommt es vom (fiktiven) Naturzustand (Krieg aller gegen Alle) zu Ordnungsbildung? Wie k"onnen Akteure wissen, dass Staat gut f"ur sie ist?
			\item
				Parsons: Utilitarismus kann Unterschied zwischen besserem und schlechterem Zustand, und wie es zu "Ubereinstimmung der Akteure kommt nicht erkl"aren
		\end{itemize}
	\item
		Folgerung Parsons: soziale Ordnung basiert auf Werten und Normen
		\begin{itemize}
			\item
				Wert: Selbstzweck (Das W"unschenswerte und Richtige), Norm: stabile wechselseitige Verhaltenserwartung (Handlungsanweisungen und -erwartungen)

				Beispiel. Wert: Individualismus, Norm: Bewahrung k"orperlicher Integrit"at
		\end{itemize}
	\item
		Soziale Ordnung ist individuellem Handeln nicht nachgeordnet, menschliches Leben ist soziales Leben (in sozialen Ordnungen). Indivudalistisch-utilitaristische Modelle sind spezifisch modern und falsch.
	\item
		Was ist soziale Ordnung:
		\begin{itemize}
			\item
				Allgemeinster Begriff: Einheit aller Ordnungsbildungen zu einem gegebenen Zeitpunkt. Niklas Luhmann: Gesellschaft ist \enquote{das umfassende Sozialsystem, das alles Soziale einschlie"st und in Folge dessen keine soziale Umwelt kennt}
			\item
				Gesamtheit der Normen, Werte, Rollen, Institutionen, Organisationen
			\item
				M"ussen legitimiert werden (Leben in Kleinfamilien muss legitimiert werden, man k"onnte ja auch anders zusammenleben), funktioniert "uber Diskreditierung von Abweichung (Heteronormativit"at im 20. Jhdt, aber auch Demokratie, Menschenw"urde heute: Rassismus, Faschismus wird Diskreditiert)
			\item
				Soziale Ordnungsbildung sorgt f"ur Stabilisierung (Stabilit"at = zeitliche Dauerhaftigkeit)der Erwartungen von Individuen an andere, in sozialen Situationen
		\end{itemize}

	\item
		Ordnungsmuster nach Parsons:
		\begin{itemize}
			\item
				Individuum/\enquote{Pers"onlichkeitssystem}:
				\begin{itemize}
					\item
						Identit"at ist wichtig, k"onnten unser Verhalten sonst nicht selbst steuern, legt Verhalten fest, legt Erwartungen fest. (Beispiel \enquote{Verfolgter} von z.B.\ Nazis)
					\item
						Habitus: Identit"at ist Bewusstseinsph"anomen, nicht materiell. Habitus ist Ordnungsmuster auf k"orperlicher Ebene
					\item
						Handlungsleitend durch Internalisierung von Kultur
				\end{itemize}
			\item
				Kultur/Kulturelles System (Ideen, Glaubenssysteme, Werte):
				\begin{itemize}
					\item
						Symbolsysteme, Wertordnungen 
					\item
						Normen im Umgang mit Objekten, \enquote{der inneren Natur}, Anderen (z.B. Erziehungspraktiken, aber nicht auf Ebene von handelnden Akteuren, sondern auf Wertvorstellungen, z.B. Erziehungsratgeber)
					\item
						Pr"agt Sozialstruktur durch Institutionalisierung (z.B. Freiheitsrechte: Institutionalisierung von bestimmten Wertvorstellungen von Freiheit und Gleichheit)
				\end{itemize}
			\item
				Soziales System: (institutionalisierte Rollen, z.B. Arzt $\leftrightarrow$ Patient)
				\begin{itemize}
					\item
						Sozialstruktur: Gef"uge relativ dauerhafter sozialer Positionen, pr"asent in Form von Rollenerwartungen, verweist auf Kultur: Rollen werden symbolisiert (z.B. durch Kleidung)
					\item
						Entwicklung durch Differenzierung, handlungsleitend durch Sozialisation
				\end{itemize}
		\end{itemize}
	\item
		Ordnungstypen:
		\begin{itemize}
			\item
				Moderne Gesellschaften in der Regel polar gebaut:
				\begin{itemize}
					\item
						Gesellschaft $\leftrightarrow$ Gemeinschaft, System $\leftrightarrow$ Lebenswelt, organische Solidarit"at $\leftrightarrow$ mechanische Solidarit"at
				\end{itemize}
		\end{itemize}

\end{itemize}

\subsection{Sozialer Wandel}
\begin{itemize}
	\item
		Jede Ver"anderung von sozialer Ordnung = sozialer Wandel
		\begin{itemize}
			\item
				\enquote{prozessuale Ver"anderung der Sozialstruktur einer Gesellschaft in ihren grundlegenden Institutionen, Kulturmustern, zugeh"origen Handlungen und Bewusstseinsinhalten} (Zapf)
			\item
				Auf verschiedenen Ebenen zu beobachten:  
				\begin{itemize}
					\item auf Makroebene der Sozialstruktur und Kultur:
						\begin{itemize}
							\item
								Wandel der Differenzierungsformen (siehe n"achstes Kapitel TODO verweis)

								Z.B. M"arkte als Steuerungsmedium sozialer Beziehungen, Wandel von Berufsbildern (Industrieges. $\leftrightarrow$ Dienstleistungsges.)
						\end{itemize}

					\item
						Mesoebene der Institutionen, korporativen Akteuren und Gemeinschaften:
						\begin{itemize}
							\item
								Wandel von Wertordnungen, Symbolsystemen, Normen

								z.B. b"urgerliche Gleichheits und Freiheitsideale: kultureller Wandel als historisch langristiger Prozess
						\end{itemize}
					\item
					 Mikroebene der Personen und ihrer Lebensl"aufe:
					 \begin{itemize}
						 \item
							 Wandel von Identit"atszuschreibungen, Biografien, Lebensl"aufen, Habitus

							 Z.B. Jugend als eigenst"andige Lebensphase, Emotionalisierung von Partnerschaft. Salopp: Ver"anderung von Normalbiografien
					 \end{itemize}
				\end{itemize}

				Kunst besteht darin, in der Vielfalt der Wandlungsprozesse  Muster zu beschreiben und zu erkl"aren: Beispiel Zivilisationstheorie von Norbert Elias
		\end{itemize}
	\item
		Normative Perspektive darauf: Modernisierung. Hierarichisiert Sozialordnungen nach ihrem Entwicklungsstand (fr"uher Standard in Sozialwissenschaften, Konsequenz daraus: Entwicklungshilfe)
	\item
		Zivilisationstheorie von Norbert Elias\index{Zivilisationstheorie}
		\begin{itemize}
			\item
				Beobachtung: Scham und Peinlichkeitsschwellen verschieben sich
			\item
				Allgemeine Tendenz: von unmittelbarer wechselseitiger Verhaltenskontrolle zur Selbstkontrolle. Elias: Entwicklung vom Fremd - zum Selbstzwang
			\item
				Korrespondierender kultureller Wandel (z.B. Sittenratgeber, Knigge)
			\item
				Elias: Wandel des Bedeutungsgewebes (Kultur?) ist Folges des Wandels der Sozialstruktur
				\begin{itemize}
					\item
						Herrschaftsr"aume werden gr"o"ser, Macht zentralisierter
					\item
						Zunehmene Arbeitsteilung, Verl"angerung der Interdependenzketten
					\item
						Folge: Man muss Verhalten st"arker kontrollieren (am absolutistischen Hof nicht einfach mit Gewalt, sondern mit Intrige $\rightarrow$ Verhalten selbst kontrollieren)
					\item
						Wandel folgt einer klaren Richtung (Fremd- zu Selbstzwang), aber ist nicht geplant oder gesteuert
				\end{itemize}
		\end{itemize}
	\item
		Wie kommt es zu Wandel der Sozialstruktur?
		\begin{itemize}
			\item
				Ver"anderung kultureller Muster? Ver"anderung individuelles Verhalten?
			\item
				Kausalit"atsfrage schwierig zu beantworten
		\end{itemize}
\end{itemize}
\subsection{Fazit}
\begin{itemize}
	\item
		Soziale Ordnung: Einheit aller Ordnungsbildungen
	\item
		Ordnungsmuster auf unterschiedlichen Ebenen
		\begin{itemize}
			\item
				Individuum, Kultur, Soziales System.
			\item
				und Beziehungen zwischen Ebenen
		\end{itemize}
	\item
		Soziale Ordnung in der Zeit: sozialer Wandel
		\begin{itemize}
			\item
				Ebenen von Wandel und Beziehungen zwischen Ebenen
		\end{itemize}
\end{itemize}
