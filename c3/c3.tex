\documentclass[fleqn,12pt]{scrartcl}
\usepackage{makeidx}
\usepackage[utf8]{inputenc}
\usepackage{color}
\usepackage[ngerman]{babel}
\usepackage{csquotes}
\usepackage{amssymb}
\usepackage{amsthm}
\usepackage{amsmath}
\usepackage{gauss}
\usepackage{braket}
\usepackage{hyperref}
\usepackage{wasysym}
\usepackage{scrpage2}
\usepackage{tikz}
\usetikzlibrary{intersections}
%\pagestyle{scrheadings}
%\clearscrheadfoot
%\ohead{\pagemark}
%\ihead{Magnus Berendes, 21752155}
%\ifoot{\today}
%\ofoot{\blattn}
%\setheadtopline{1pt}
%\setheadsepline{0.4pt}
%\setfootsepline{0.4pt}
\usepackage{enumitem}
%\setenumerate[0]{label=\alph*)}
\newcommand{\id}{\, \mathrm{d}}
\newcommand{\intl}{\int\displaylimits}
% New definition of square root:
% it renames \sqrt as \oldsqrt
\let\oldsqrt\sqrt
% it defines the new \sqrt in terms of the old one
\def\sqrt{\mathpalette\DHLhksqrt}
\def\DHLhksqrt#1#2{%
	\setbox0=\hbox{$#1\oldsqrt{#2\,}$}\dimen0=\ht0
	\advance\dimen0-0.2\ht0
	\setbox2=\hbox{\vrule height\ht0 depth -\dimen0}%
{\box0\lower0.4pt\box2}}

\newcommand{\karos}[2]{
	\begin{tikzpicture}
		\draw[step=0.5cm, color=gray] (0,0) grid (#1 cm , #2 cm);
	\end{tikzpicture}
}
\newcommand{\abs}[1]{
	\left \vert #1 \right \vert
}
\newcommand{\absbb}[1]{
	\left \Vert #1 \right \Vert
}

%TODO
\newcommand{\blattn}{Blatt 5}
\makeindex
\begin{document}

\section{Optimierung}
\subsection{Basics/Wiederholung}
\begin{itemize}
	\item
		Extremalstellen
		\begin{itemize}
			\item
				Eigenwerte\index{Eigenwerte}: EW positiv  $\rightarrow$ Punkt Minimum, alle negativ  $\rightarrow$ Maximum, mindestens ein EW positiv, mindestens ein EW negativ  $\rightarrow$ indefinit, Sattelpunkt
			\item
				Satz von Minimum und Maximum\index{Satz von Minimum und Maximum}: Stetige Funktionen auf kompakten (=abgeschlossen und beschr"ankt)  Mengen (Definitionsbereich) nehmen globale Maxima und Minima an
		\end{itemize}
	\item Integration
		\begin{itemize}
			\item
				\index{Integrationstricks}
				Integration durch Substitution:
				\begin{align*}
					\intl_a^b f(u(x))\cdot u'(x) \id x = \intl_{u(a)}^{u(b)} f(u) \id u\\
				\end{align*}
			\item
				Partielle Integration:
				\begin{align*}
					\intl a'b \id x = ab  - \intl ab' \id x
				\end{align*}
		\end{itemize}
	\item
		\textbf{Mitternachtsformel/L"osungsformel f"ur quadratische Gleichungen:}\index{Mitternachtsformel}
		\begin{align*}
			x_{1,2} = \frac{-b \pm \sqrt{b^2 -4ac}}{2a}
		\end{align*}
	\item
		\textbf{Eigenwerte/-r"aume:}
		\begin{itemize}
			\item
				\index{Eigenwerte}
				Eigenwerte von Matrix $A$ berechnen:
				$\det (A-\lambda E) \overset!= 0$
			\item
				\index{Eigenr"aume}
				Eigenraum zu Eigenwert $\lambda$ berechnen:
				$(A - \lambda E)\cdot \vec v = \vec 0$

				Eigenraum zu Eigenwert $\lambda$ dann $\Set{\alpha \vec v,\, \alpha \in \mathbb{R}}$ ($\vec v$ nicht eindeutig)

				Trick: komplex konjugierte Eigenwerte haben immer auch komplex konjugierte Eigenr"aume
			\item
				Vielfachheit von Eigenwerten:
				\begin{itemize}
					\item Algebraische Vielfachheit: wie oft kommt EW $\lambda$ vor?
					\item
						Geometrische Vielfachheit: wieviele Dimensionen hat der ER zu $\lambda$? (wieviele Nullzeilen bekommt man beim ER berechnen? aus wievielen unabh"angigen Vektoren besteht der ER?)

				\end{itemize}
		\end{itemize}

	\item
		\textbf{Partialbruchzerlegung}\index{Partialbruchzerlegung}

		\begin{enumerate}
			\item
				Nennerpolynom faktorisieren 

				(z.B. $\frac{7x+1}{x^3 + x^2 -x -1} \rightarrow \frac{7x+1}{(x+1)^2(x-1)}$)
			\item
				An den Multiplikationen zwischen den Nennerfaktoren in einzelne Br"uche auseinanderziehen, Z"ahler sind Variablen. Faktoren die mehrfach aufgreten werden in allen Potenzen bis zur vorliegenden angesetzt.

				($\frac{7x+1}{(x+1)^2(x-1)} \rightarrow \frac{A}{x+1} + \frac{B}{(x+1)^2} + \frac{C}{x-1}$)
			\item
				Durch Erweitern auf den (bekannten) Hauptnenner bringen, ausmultiplizieren und nach Potenzen von $x$ sortieren.

				($\frac{A(x^2-1) + B(x^2-1) + C(x-1)}{x^3+x^2 -x -1} \rightarrow \frac{(A+C)x^2 + (B+2C)x + (-A- B+C)}{x^3 + x^2 -x -1}$)

			\item
				Koeffizientenvergleich: Koeffizienten zwischen Ausgangsz"ahler und neuem Z"ahler m"ussen gleich sein 

				($A+C = 0,\, B+2C = 7,\, -A-B+C =1$)

		\end{enumerate}
		
	\item
		\textbf{Inverses einer Matrix} \index{Inverses einer Matrix}
		\begin{align*}
			\begin{pmatrix} a & b \\ c & d \end{pmatrix}^{-1} = 
				\frac{1}{ad-bc} \begin{pmatrix} d & -b \\ -c & a \end{pmatrix}
		\end{align*}
\end{itemize}

\subsection{LaGrange-Formalismus}
		\index{LaGrange-Formalismus}
		\begin{enumerate}
			\item
				Nebenbedingung nach $=0$ umstellen
			\item
				M"ogliche Extrema unter der Nebenbeingung findet man mit $\nabla \mathcal{L}(x, y, \lambda) \overset!= 0$, mit $\mathcal{L}(x,y,\lambda) = f(x) - \lambda \cdot (\text{Nebenbedingung})$
		\end{enumerate}


\subsection{Lokale Aufl"osungsfunktion}
\begin{itemize}
	\item
		\enquote{Auf Existenz pr"ufen} = schauen ob Nenner $\neq 0$
\end{itemize}

\subsection{Bogenl"ange, Kurvenintegral, Oberfl"achenintegral}
\index{Bogenl\"ange}
\begin{itemize}
	\item\index{Kurvenintegral}
		Kurvenintegral
		\begin{itemize}
			\item 1. Art: skalares Linienelement $\id s  = \absbb{\gamma'(t)} \id t$ (=auch f"ur Bogenl"ange)
			\item 2. Art: vektorielles Linienelement $\id \vec s = \gamma'(t) \id t$
		\end{itemize}
	\item\index{Oberfl\"achenintegral}
		Oberfl"achenintegral
		\begin{itemize}
			\item
				1. Art: skalares Oberfl"achenelement $\id o = \absbb{\partial_s\gamma(s,t) \times \partial_t\gamma(s,t)} \id s \id t$
			\item 2. Art:
				vektorielles Oberfl"achenelement $\id \vec o = \partial_s\gamma(s,t) \times \partial_t\gamma(s,t) \id s  \id t$
		\end{itemize}

\end{itemize}

\subsection{Konvexe Mengen und Funktionen}
\begin{itemize}
	\item Differenzierbare Funktionen sind konvex\index{Konvexit"atskriterium}: $f''(x) \geq 0$, bzw. positiv semidefinite Hessematrix (Alle EW $\geq 0$)
	\item
		Eine Menge $M$ ist konvex, wenn $\forall a,b \in M, \lambda \in \mathbb{R}, 0\leq \lambda \leq 1: \lambda a + (1-\lambda)b \in M$
	\item
		Funktion $f$ ist konvex, wenn $f(tx + (1-t)y) \leq tf(x) + (1-t)f(y)$
\end{itemize}

\subsection{Extrema quadratischer Funktionen}
\begin{itemize}
	\item
		Niveaulinien der Funktion sind Ellipsen, Halbachsen in Richtung der Eigenr"aume
	\item
		F"ur L"angen der Halbachsen $a_1,a_2$ gilt: $\frac{a_1}{a_2} = \sqrt{\frac{\lambda_1}{\lambda_2}}$
	\item
		Minimum auf einem Kreis liegt logischerweise in Richtung der langen Halbachse, Maximum in Richtung der k"urzeren
	\item
		$Q(x) = \langle Ax, x \rangle (+ \langle b,x \rangle): \nabla Q(x) = Ax (+ b), \mathcal{H}_Q(x) = A$
\end{itemize}


\subsection{Banach'scher Fixpunktsatz}
Funktion $\Phi: M \rightarrow M$, Fixpnktgleichung $x = \Phi(x)$ \index{Fixpunktgleichung}\index{Banachscher Fixpunktsatz}\index{Fixpunktiteration}
\begin{itemize}
	\item
		Vorraussetzungen:
		\begin{itemize}
			\item
				$M$ muss abgeschlossene Teilmenge des $\mathbb{R}^n$ sein
			\item
				$\Phi$ muss Selbstabildung sein, d.h. $\Phi(M) \subset M$ (f"ur $n = 1$ Pr"ufung indem man globale Extrema von $\Phi$ berechnet)
			\item
				$\Phi$ muss kontrahierend sein, d.h. $\absbb{\Phi(x) - \Phi(y)} \leq k\absbb{x-y}, \forall x,y \in M \text{ und \emph{ein} }k \in (0,1)$ (oder einfacher: $\sup\abs{\Phi'(x)} < 1$, $\sup\abs{\Phi'(x)}$ auch Kontraktionskonstante k)

		\end{itemize}
	\item Dann sagt der Satz:
		\begin{itemize}
			\item
				Fixpunktgleichung $x = \Phi(x)$ hat \emph{genau eine} L"osung: $x_* \in M$
			\item
				Diese kann man mit Fixpunktiteration $x_{n+1} = \Phi(x_n), x_0 \in M$ beliebig bestimmen
			\item Fehlerabsch"atzung der FP-Iteration: $\absbb{x_n - x_*} \leq k^n\cdot \absbb{x_0 - x_*}$ (oder: $\absbb{x_n - x_*} \leq \frac{k}{1-k}\absbb{x_n - x_{n-1}}$
		\end{itemize}
\end{itemize}

\subsection{Jacobi-Verfahren}
\begin{itemize}
	\item
		Konvergiert, wenn A \emph{diagonaldominant}\index{Diagonaldominanz} ist, d.h. Betrag des Diagonalelements ist stets gr"o"ser als Summe der Betr"age der "ubrigen Eintr"age der gleichen Zeile
	\item
		Um Gleichungssysteme $Ax = b$ iterativ zu l"osen. Iterationsalgorithmus:
		\begin{align*}
			x_{n+1} = D^{-1}b-D^{-1}(A-D)x, \, D = \text{diag}\, (a_{i,i})
		\end{align*}
\end{itemize}

\subsection{Lineare Programme}
\index{Lineare Programme}
\begin{itemize}
	\item
		Standardform linearer Programme:\index{Lineare Programme!Standardform}
		\begin{enumerate}
			\item
				$\max$ in $\min$ umwandeln: \\$\max(2x -3y + z) \rightarrow \min(-2x + 3y - z)$
			\item
				$\geq$ in $\leq$ umwandeln: \\$x \geq -10 \rightarrow -x \leq 10$
			\item
				Schlupfvariablen einf"uhren: \\
				$x+3y \leq 1 \Leftrightarrow x + 3y + \tilde x = 1,\, \tilde x \geq 0$
			\item
				Sicherstellen, dass alle Variablen $\geq 0$, falls nicht:\\
				$x =: x^+ - x^-,\, x^+, x^- \geq 0$
			\item
				Elimination redundanter Gleichungen
			\item
				$\vec c = $ Faktoren in zu minimierender Funktion
			\item
				$A = $ Nebenbedingungen in Matrixschreibweise
			\item
				$\vec b = $ rechte Seite der Nebenbedingungen
			\item
				Standardform:
				\begin{align*}
					\text{Suche } &\vec x: f(\vec x) := \langle x, c \rangle \rightarrow \min\\
					\text{s.t. }& A\vec x = \vec b\\
					&\vec x \geq 0
				\end{align*}
		\end{enumerate}
	\item
		Simplex:\index{Simplex}
		\begin{enumerate}
			\item
				LP in Standardform bringen
			\item
				Zul"assige Basisl"osung finden: (Muss L"osung f"ur $Ax = b$ sein, alle Nichtbasis xe m"ussen 0 sein, alle Basis xe $> 0$, oft sind die Schlupfvariablen eine geeignete Basisl"osung)
			\item
				Simplextableau aufstellen
				\begin{align*}
					x_B &= \vec b, x_N = 0 \text{ (per Definition)}\\
					A_B &= \text{Spalten die zur Basis geh"oren}\\
					A_N &= \text{Spalten die nicht zur Basis geh"oren}\\
					c_B^T &= \text{Spalten/Eintr"age von } \vec c \text{, die zur Basis geh"oren}\\
					c_N^T &= \dots\\
					T(B) &= \begin{pmatrix}
						c^T - c^T_BA_B^{-1}A & -f(x)\\
						A_B^{-1}A & x_B\\
					\end{pmatrix}, \, \text{ (allgemein)}\\
						T(B) &= \begin{pmatrix}
							c^T & -f(x)\\
							A & x_B
						\end{pmatrix}, \, \text{ (wenn die Schlupfvariablen die Startbasis bilden)}
				\end{align*}
			\item
				Simplexen bis zu einem Abbruchkriterium:
				\begin{enumerate}
					\item
						W"ahle eine Spalte (au"ser der letzten) mit negativem obersten Eintrag. Gibt es keine solche ist die optimale L"osung gefunden.
					\item
						In dieser Spalte werden alle positiven Eintr"age betrachtet. Gibt es keine positiven Eintr"age in der Spalte ist die Zielfunktion unbeschr"ankt.\\
						Man teilt nun immer den in derselben Zeile ganz rechts stehenden Eintrag durch den betrachteten positiven Eintrag\\
						Das \emph{kleinste} Ergebnis bestimmt, welcher Wert \enquote{eingekastelt} wird
					\item
						Basistausch: Die im ersten Schritt bestimmte Spalte tritt in die Basis ein.
						Jetzt suchen wir in der Zeile mit dem eingekastelten Wert nach der 1, die zu einer Basisspalte geh"ort (erkennt man im allg. daran, dass sie aus genau einer 1 und sonst nur Nullen besteht).\\
						Diese (alte) Basisspalte tritt aus der Basis aus, die neue Basis $B'$ ist dann z.B. $\{1,4,5\}$
					\item
						Um Basistausch vollst"andig zu machen muss eingekastelte Wert zu 1 gemacht werden, restliche Spalte zu 0. (Mit Gauss, dabei nur den eingekastelten Wert benutzen um andere Basisspalten nicht kaputt zu machen)
				\end{enumerate}
			\item
				Auswertung: $x_1$ entspricht 1. Spalte. Da wo hier die 1 steht, steht ganz rechts der Wert f"ur $x_1$, analog f"ur andere Werte in der Basis
		\end{enumerate}
\end{itemize}
\section{Differentialgleichungen/systeme}
\subsection{Klassifizierung}\index{Klassifizierung von DGL}
\begin{itemize}
	\item
		Ordnung: h"ochste vorkommende Ableitung
	\item
		Linearit"at: nur im Bezug auf $y(t)$ (und Ableitungen), nicht auf $t$ allein
	\item
		explizit/implizit: explizit = nach der h"ochsten vorkommenden Ableitung umgestellt, implizit sonst
	\item
		skalar/System: eine Funktion oder mehrere?
	\item
		Autonomie: $t$ nur in Funktionen (autonom) oder auch au"serhalb (nicht autonom)?
\end{itemize}
\subsection{Trennung der Variablen}
\index{Trennung der Variablen}
F"ur separierbare DGLen
\begin{enumerate}
	\item 
		Alle $y$s auf eine (inklusive $\id y$), alle $t$s auf die andere Seite (inklusive $\id t$)
	\item
		Beide Seiten integrieren, nach y umstellen
	\item
		Evt. noch $C$ (vorher nicht vergessen!) bestimmen 
	\item
		Definitionsmenge beachten!
\end{enumerate}
... mit Substitution:
\begin{enumerate}
	\item
		Geeigneten Substanten bestimmen (z.B. $z = \frac{y}{t}$)
	\item
		Nach $y$ umstellen, und $y'$ bestimmen, durch ableiten
	\item
		$y'$ und $y$ jeweils substituieren, wieder nach $z'$ umstellen
	\item
		Anfangswert substituieren (Anfangs-$t$ in $z$ einsetzen)
	\item
		AWP mit $z(t)$ l"osen
	\item
		R"ucksubstituieren (siehe Punkt \enquote{nach $y$ umstellen})
\end{enumerate}

\subsection{Lineare, inhomogene Dgl, Variation der Konstanten}
\index{Variation der Konstanten}
F"ur Gleichungen der Form $y' = f(t)y + g(t)$
\begin{enumerate}
	\item
		Allgemeine L"osung der zugeh"origen homogenen Dgl (Inhomogenit"at/$g(t)$ einfach weglassen) unbedingt $C$ soweit wie m"oglich rausholen!
	\item
		Variation der Konstanten:
		\begin{enumerate}
			\item
				$C$ aus $y_\text{hom}$ zu $C(t)$ machen, $y_\text{hom}$ ableiten
			\item
				In die \emph{inhomogene} Dgl einsetzen, der Term mit $C(t)$ k"urzt sich \emph{immer} weg
			\item
				Gefundenes $C(t)$ in Gleichung aus Schritt a) einsetzen $= y_\text{sp}$
		\end{enumerate}
	\item
		$y(t) = y_\text{hom}(t) + y_\text{sp}(t)$
\end{enumerate}



\subsection{Bernoullische Dgl}
\index{Bernoullische Dgl}
\begin{itemize}
	\item
		Allgemein von der Form $y' = f(t)y + g(t) y^\alpha,\, \alpha \in \mathbb{R}\setminus\{0,1\}$
	\item
		Substitution dann $z(t) := y^{1-\alpha}(t)$
\end{itemize}



\subsection{Lipschitzstetigkeit, Satz von Picard-Lindel"of}
\begin{itemize}
	\item
		Lokale Lipschitzstetigkeit:\index{Lipschitzstetigkeit}
		\begin{itemize}
			\item
				$f$ auf Intervall stetig differenzierbar, so is sie dort auch lokal Lipschitz-stetig
			\item
				Gilt $\lim\limits_{x\rightarrow x_0}{f'(x)} = \pm \infty$, so ist $f$ auf keiner Menge, die $x_0$ enth"alt lokal Lipschitz stetig
		\end{itemize}

	\item
		Satz von Peano\index{Satz von Peano}: das AWP $y' = g(y,t), \, y(t_0) = y_0$ hat immer mindestens eine L"osung auf einer Umgebung von $t_0$, falls die rechte Seite in einer Umgebung von $(t_0, y_0)$ stetig ist
	\item
		Satz von Picard-Lindel"of \index{Satz von Picard-Lindel"of} sagt, dass das AWP in einer Umgebung von $t_0$ eindeutig l"osbar ist,
		falls $g$ in einer Umgebung von $(t_0, y_0)$ stetig ist und dort bzgl $y$ lokal Lipschitz-stetig ist.
\end{itemize}

\subsection{Differentialgleichungssysteme}
\label{sec:DGLS}
\index{Differentialgleichungssysteme}
		$y' = Ay$
		\begin{enumerate}
			\item
				Eigenwerte von $A$ bestimmen, dann Eigenr"aume von $A$ bestimmen
			\item
				Fundamentalsystem $\mathbb{F} = \Set{e^{\lambda_i t} EV_{\lambda_i}, \dots}$
			\item
				Allgemeine L"osung ist dann $y = C_1 e^{\lambda_1} EV_1 + C_2 e^{\lambda_2} EV_2 \dots $
			\item
				Evt. AWP l"osen: Anfangswert einsetzen, Gleichungssystem l"osen
			\item
				Spezialfall: komplexes Fundamentalsystem $\mathbb{F_C}$
				\begin{enumerate}
					\item
						Real- und Imagin"arteil einer komplexen L"osung bestimmen: $e^{it} = \cos(t) + i\sin(t)$ in den Vektor reinmultiplizieren
					\item
						In 4 Vektoren nach $\cos/\sin$ und $i$/kein $i$ aufteilen, $\cos$ und $\sin$ ohne $i$ bilden einen Teil des reelen Fundamentalsystems, mit $i$ einen anderen Teil des Fundamentalsystems (aber ohne $i$!)

						Der andere komplexe Eigenvektor f"allt weg
				\end{enumerate}
			\item
				Spezialfall: Vielfachheit der Eigenwerte passt nicht:
				\begin{enumerate}
					\item
						\index{Hauptvektoren}
				Hat ein Eigenwert eine h"ohere algebraische als geometrische Vielfachheit, braucht man zus"atzlich den Hauptvektor zweiter Stufe $h_2$, der das LGS $(A-\lambda E)h_2 = v$ ($v$ ist zugeh"origer Eigenvektor) l"ost.

						Fundamentall"osungen dann $ve^{\lambda t}, (h_2+tv)e^{\lambda t}$
			\item
				Braucht man noch einen weiteren Hauptvektor, nimmt man den Hauptvektor 3. Stufe $h_2$: $(A-\lambda E)h_3 = h_2$

						Fundamentall"osungen dann: $ve^{\lambda t}, (h_2+tv)e^{\lambda t}, (h_3 + th_2 + \frac{t^2}{\lambda}v)e^{\lambda t}$
					\item
						Hat man mehr als einen Hauptvektor erster Stufe ($v$ und $w$) und braucht noch einen Hauptvektor zweiter Stufe, wei"s man nicht, welcher Hauptvektor auf der Rechten Seite stehen muss. Deswegen nimmt man einen allgemeinen Hauptvektor: $(A-\lambda E)\vec x = \alpha \vec v + \beta \vec w$

						Wichtig bei dem Fundamentalsystem: \enquote{Richtigen} Hauptvektor nehmen, ergibt sich durch $\alpha v + \beta w$
				\end{enumerate}
		\end{enumerate}

		\subsection{Inhomogene Differentialgleichungssysteme}
		\begin{itemize}
			\item
				Allgemeine L"osung $y_\text{H}$ bestimmen, siehe \autoref{sec:DGLS}
			\item
				$y_\text{Sp}(t) = W(t) \intl W^{-1}(t)b(t) \id t$ (komponentenweise integrieren)

				Die Spalten von $W(t)$ sind die linear unabh"angigen L"osungen der homogenen L"osung. Dazu einfach einmal $c_1 = 1, c_2 = 0 \dots$ und einmal $c_1 = 0, c_2 =1 \dots$ setzen, \textbf{$e^t$s dabei nicht vergessen!}

			\item Dann gilt wieder $y(t) = y_\text{Hom} + y_\text{Sp}$

			\item
				Rechnung tendenziell sehr komplex, raten kann unter Umst"anden effektiver sein
		\end{itemize}

		\subsection{Lineare DGLen h"oherer Ordnung}
		\begin{itemize}
			\item
				Standard Ansatz f"ur skalare, lineare DGLen ist $y(t) = e^{\lambda t}$
			\item
				\enquote{Eigenwerte berechnen}: Anzahl der Ableitungsstriche ist die Potenz, Faktor bleibt gleich: $3y''' + 2y'' = 0 \rightarrow 3\lambda^3 + 2 \lambda^2 = 0$
			\item
				L"osung dann $y = c_1e^{\lambda_1 t} + c_2 e^{\lambda_2 t} \dots$
			\item
				Bei \enquote{doppelten Eigenwerten} kommt beim zweiten Auftreten ein $t$ mit dran: $c_1 e^{\lambda_1 t} + tc_2e^{\lambda_1 t}$
			\item
				F"ur die $c$s einfach noch paar mal ableiten, Anfangswerte einsetzen, LGS l"osen

		\end{itemize}

		\subsection{Inhomogne lineare DGLen h"oherer Ordnung}
		\begin{itemize}
			\item
				$y_\text{Hom}$ alles wie gehabt, Inhomogenit"at weglassen
			\item
				\index{Ansatz vom Typ der rechten Seite}
				$y_\text{Sp}$ durch \enquote{Ansatz vom Typ der rechten Seite}, wenn rechte Seite vom Typ $(at + b)e^{ct}$, dann Ansatz $y_\text{Sp} = t^r(At+B)e^{3t}$, $t^r$ wird nur relevant, wenn der Exponent von $e$ eine Nullstelle der charakteristischen Gleichung w"are, dann $r = \text{ Vielfachheit der Nullstelle}$, wenn keine Nullstelle $r = 0$%
			\item
				Ableiten (Produktregel!) und in DGL einsetzen um A und B zu bestimmen
			\item
				$y = y_\text{Hom} + y_\text{Sp}$
		\end{itemize}


		\section{Algebra}
		\subsection{Restklassenringe}
		\index{Restklassenringe}
		\begin{itemize}
			\item
				$[a]_x = [b]_x \Leftrightarrow a\mod x = b \mod x$
			\item
				Rechenregeln:
				\begin{itemize}
					\item
						$[a]_x^i = [a^i]_x$
					\item
						$\left[\sum ab\right]_x = \sum [a]_x [b]_x$
				\end{itemize}
		\end{itemize}


		\subsection{Euklidischer Algorithmus}
		Zur Bestimmung des ggT \index{Euklidischer Algorithmus}\index{ggT}

		Beispiel: $ggT(14, 10)$:
		\begin{align*}
			14 &= 1 \cdot 10 + 4 \\
			10 &= 2 \cdot 4 + 2 \\
			4 &= 2 \cdot \underbrace{2}_{ggT(14,10)} + 0
		\end{align*}

		\subsection{Gruppen, Inverse}
		\begin{itemize}
			\item
				Nach Satz aus VL ist  $(\mathbb{Z}_n \setminus [0]_n, \cdot)$ genau dann eine Gruppe, wenn $n$ prim, dann ist die Gruppe auch abelsch

			\item
				Inverses von $[a]_n$ in $(\mathbb{Z}_n \setminus [9]_n, \cdot)$ mit Euklidischem Algorithmus bzw. Lemma von Bezout besimmen:
				\begin{enumerate}
					\item
						Finde $ggT(n, a)$ mit euklidischem Algorithmus (Beispiel mit $[8]_{25}$):
						\begin{align*}
							25 &= 3 \cdot \underline{8} + 1\\
							8 &= 8 \cdot \underline{1} + 0
						\end{align*}
					\item
						\enquote{R"uckw"artsanwenden:}
						\begin{align*}
							25 = 3 \cdot \underline{8} + 1 \Rightarrow 1 = 25 - 3 \cdot 8 \Rightarrow [1]_{25} = [8]_{25} \cdot [-3]_{25} \Rightarrow [8]_{25}^{-1} = [-3]_{25} = [22]_{25}
						\end{align*}

						Zweites Beispiel: $[43]_{219}^{-1}$
						\begin{align*}
							I:& &219 &= 5 \cdot \underline{43} + 4 \\
							II:& &43 &= 10 \cdot \underline{4} + 3\\
							III:& &4 &= 1 \cdot \underline{3} + 1
						\end{align*}
						\begin{align*}
							\Rightarrow &1 \overset{III}= 4 - 1\cdot 3 \overset{II}=4-(43-10 \cdot 4) =
							\\  &11\cdot 4 - 43 \overset{I}= 11 \cdot (219 - 5 \cdot 43) = 11 \cdot 219 - 56 \cdot 43\\
							\Rightarrow& [43]_{219}^{-1} = [-56]_{219} = [163]_{219}
						\end{align*}

						Oder mit Potenzbildung: solange $[a]_n^{i++}$ rechnen, bis $[1]_n$ rauskommt. $[a]_n^{i-1}$ ist dann $[a]_n^{-1}$
				\end{enumerate}

			\item
				Elemente von $(\mathbb{Z}_{a}^*, \cdot)$ bestimmen: alle nat"urlichen Zahlen $< a$, die teilerfremd zu $a$ sind

			\item
				M"achtigkeit der multiplikativen Gruppe $(\mathbb{Z}^*_n, \cdot)$ ist $\Phi(n)$ \index{Eulersche $\Phi$-Funktion}
				\begin{align*}
					\Phi(p^k) &= p^{k-1} \cdot (p-1),\, &\text{ falls  $p$ prim und $k \in \mathbb{N}$} \\
					\Phi(ab) &= \Phi(a) \cdot \Phi(b), \, &\text{ falls $a,b$ teilerfremd}
				\end{align*}

			\item
				M"achtigkeit einer Untergruppe: Ist $U$ eine Untergruppe einer endlichen Gruppe $G$, so ist die Zahl der Elemente von $U$ ein Teiler der Zahl der Elemente von $G$
		\end{itemize}




		\subsection{Pr"ufsummen, Fehlererkennung}
		Pr"ufsummengleichung der Form $[g_1d_1 + g_2d_2 + \dots + g_id_i]_n = [x]_n$
		\index{Pr"ufsummen}\index{Fehlererkennung}
		\begin{itemize}
			\item
				Einzehlfehler werken erkannt, wenn alle Gewichte teilerfremd $n$ sind

				Gegenbeispiel: Wert von $d_i$ um genau $\frac{n}{g_i}$ ver"andern
			\item
				Nachbarvertauschungsfehler k"onnen erkannt werden, wenn die Differenz zweier aufeinanderfolgender Gewichte teilerfremd zu $n$ ist.

				Gegenbeispiel: trivial

			\item
				Beliebige Vertauschungsfehler k"onnen erkannt werden, wenn die Differenz zweier beliebiger Gewichte teilerfremd zu $n$ ist

				Gegenbeispiel: trivial
		\end{itemize}




\addcontentsline{toc}{section}{Stichwortverzeichnis}
\printindex
\end{document}
