\documentclass[a4paper, 12pt]{scrartcl}
\usepackage[ngerman]{babel}
\usepackage[utf8]{inputenc}
\usepackage{amsmath}
\usepackage{amssymb}
\usepackage{amsthm}
\usepackage{amstext}
\usepackage{bussproofs}
\usepackage{array}
\usepackage{csquotes}
\usepackage{listings}
\lstset{breakatwhitespace=true,
	basicstyle=\footnotesize
}

\usepackage{hyperref}

\begin{document}
\section*{IDB Zusammenfassung}
\tableofcontents

\newpage

\section{Zusammenfassung der ZUsammenfassung}
\subsection{Definitionen}
\begin{description}
	\item[Datenunabh"angigkeit] Datenunabh"angigkeit bezeichnet das Speicheru und Wiedergewinnen (= Auffinden und aush"andigen) von persistenten Daten ohne Kenntnis der Details der Speicherung
	\item[physische Konsistenz] Speicherungsstrukturen sind konsistent, alle Verweise sind korrekt, Indizes sind aktuell
	\item[logische Konsistenz] Korrektheit der Dateninhalte, vollst"andig ausgef"uhrte Transaktionen "uberf"uhren die DB aus einem logisch konsistenten Zustand in einen anderen logisch konsistenten Zustand
	\item
		[Parial Undo/R1-Recovery] Nach Transaktionsfehler, R"ucksetzen auf Zustand vor Beginn der TA, andere TAs werden nicht beinflusst
	\item
		[Partial Redo/R2-Recovery] Nach Systemfeher mit Verlust des Hauptspeichers, Wiederholen aller verlorengegangenen "Anderungen von abgeschlossenen TAs
	\item
		[Global Undo/R3-Recovery] Nach Systemfehler mit Verlust des Hauptspeichers, Zur"ucksetzen aller durch den Ausfall unterbrochenen TAs
	\item
		[Global Redo/R4-Recovery] Nach Ger"atefehler, Backups einspielen
	\item
		[ACID Eigenschaften bei TAs]:
		\begin{itemize}
			\item Atomicity: Alles oder nichts Eigenschaft - entweder alle "Anderungen oder gar keine
			\item Consistency: erfolgreiche TA garantiert, dass alle Konsistenz/Integrit"atsbedingungen eingehalten werden
			\item
				 Isolation: Alle Transaktionen isoliert voneinander, niemand benutzt inkonsistente Zwischenergebnisse anderer TAs
			\item
				 Durability: Alle Ergebnisse einer erfolgreichen TA m"ussen permanent gemacht worden sein, bevor Erfolg an Anwendung gemeldet werden darf
		\end{itemize}

\end{description}

\subsection{Anfrageverarbeitung}
\begin{itemize}
	\item
		Phasen der Anfrageverarbeitung:
		\begin{enumerate}
			\item
				Lexikalische und syntaktische Analyse: Anfragebaum erstellen
			\item
				Semantische Analyse: Existenz/G"ultigkeit der referenzierten Relationen und Attribute, Namensaufl"osung, Konvertierung der Werte in interne Darstellung
			\item
				Zugriffs/Integrit"atskontrolle: Kontrolle von Formaten, ggf. Konvertierung von Datentypen
			\item
				Standardisierung und Vereinfachung: Anfragebaum in Normalform bringen, Elimination von Redundanzen
			\item
				Restrukturierung und Transformation: Algebraische Verbesserungen (heuristische Regeln $\rightarrow$ Restrukturierung) und nicht-algebraische Optimierung (Transformation in g"unstigste  Planalternative)
			\item
				Code-Generierung: erzeugt Zwischencode und ausf"uhrbares Zugriffsmodul
	\end{enumerate}
\end{itemize}
\subsection{Schichtenmodell}
Sch"on auf \href{site:IDB-2015WS-09-Speicherung.pdf}{09/3}
\begin{enumerate}
	\item
		\textbf{Anwendungsstrukturen}:
		\begin{itemize}
			\item
				Aufgabe: Objektverarbeitung
			\item
				Adressierungseinheiten von oben: Anwendungsobjekte
			\item
				Addressierungseinheiten nach unten: Relationen, Sichten, Tupel
			\item
				Schnittstelle nach unten: mengenorientierte Schnittstelle - SQL, Transaktioen
		\end{itemize}
	\item
		\textbf{Logische Datenstrukturen}:
		\begin{itemize}
			\item
				Aufgabe: "Ubersetzung u. Pfadoptimierung
			\item
				Adressierungseinheiten von oben: Relationen, Sichten, Tupel
			\item
				Addressierungseinheiten nach unten: S"atze, B*-B"aume usw.
			\item
				Schnittstelle nach unten: interne Satzschnittstelle - \lstinline$r := SPEICHER <Satz>; LOESCH r$
		\end{itemize}
	\item
		\textbf{Speicherungsstrukturen}:
		\begin{itemize}
			\item
				Aufgabe: Satzverwaltung u. Zugriffspfadverwaltung
			\item
				Adressierungseinheiten von oben: S"atze, B*-B"aume usw.
			\item
				Addressierungseinheiten nach unten: Seiten, Segmente
			\item
				Schnittstelle nach unten: seitenorientierte Pufferschnittstelle - \\ \lstinline$ FIX(Segment, Seite); UNFIX(Seite)$
		\end{itemize}
	\item
		\textbf{Seitenzuordnungsstrukturen}:
		\begin{itemize}
			\item
				Aufgabe: Pufferverwaltung, Einbringstrategien
			\item
				Adressierungseinheiten von oben: Seiten, Segmente
			\item
				Addressierungseinheiten nach unten: Bl"ocke, Dateien
			\item
				Schnittstelle nach unten: blockorientierte Dateischnittstelle - \\ \lstinline$ LIES Datei, Block, k; Schreib k, Datei, Block$
		\end{itemize}
	\item
		\textbf{Speicherzuordnungsstrukturen}:
		\begin{itemize}
			\item
				Aufgabe: Externspeicherverwaltung
			\item
				Adressierungseinheiten von oben: Bl"ocke, Dateien
			\item
				Addressierungseinheiten nach unten: Slots, Spuren, Zylinder, Ger"ate usw.
			\item
				Schnittstelle nach unten: Ger"ateschnittstelle - Kanalkommandos
		\end{itemize}
	\item
		Darunter nur noch physische DB
\end{enumerate}




\newpage
\section{Einführung}
\subsection{Datenunabhängigkeit}
\begin{itemize}
	\item
		dient Speichern und Wiedergewinnung (=Auffinden und Aushändigen) von Daten
	\item
		Keine Details der Speicherung (Adressen $\rightarrow$ Namen, Bytefolgen $\rightarrow$ Typen)
	\item
		Persistente Speicherung
\end{itemize}
\subsection{Schichtenbildung}
\begin{itemize}
	\item Schichten \enquote{benutzen} Ebenen darunter
	\item Schicht bietet Dienste an der Schnittstelle nach oben an, verbirgt darunterliegende Schichten (Datenkapselung)
	\item
		Hierarchieebenen können als abstrakte/virtuelle Maschinen aufgefasst werden
\end{itemize}
\subsection{Datenbanktechnologie}
\begin{itemize}
	\item
		Konzepte, Methoden, Werkzeuge:
		\begin{description}
			\item[dauerhaft] Lebensdauer Daten > Dauer Erzeugungsprozess
			\item[zuverlässig] Integrität, Konsistenz, Verlustsicherheit
			\item[unabhängig] wechselseitige Änderungsimmunität AP $\leftrightarrow$ DBS
			\item[komfortabel] höhere, abstrakte Schnittstelle
			\item[flexibel] Ad-hoc-Zugriffsmöglichkeit
			\item[groß] Datenvolumen $<<$ Hauptspeicher
			\item[integriert] kontrollierte Redundanz von/für mehrere Anwendungen
			\item[mehrfach benutzbar] gleichzeitiger Zugriff
		\end{description}
	\item
		Begriff: System zur Bescreibung, Speicherung und Wiedergewinnung von umfangreichen Datenmengen, die von mehreren Anwendungsprogrammen benutzt werden. Besteht aus:
		\begin{itemize}
			\item
				Datenbank in der die Daten abgelegt werden
			\item
				Datenbank-Managementsystem: Software, die mit den Daten arbeitet (speichert, auffindet, etc.)
		\end{itemize}
\end{itemize}
\section{Dateiverwaltung}
\begin{itemize}
	\item
		Grundlage jedes DBMS
\end{itemize}
\subsection{Logische Speichergeräte}
\begin{itemize}
	\item
		Erhöhung der Störsicherheit, Verdecken von Gerätefehlern, Verdecken von gerätespezifischer IO-Schnittstelle
	\item
		Typische Operationen:
		\begin{itemize}
			\item
				\begin{lstlisting}
				int Device::readBlock (
				int CylinderNo, int TrackNo, int SlotNo, char *BlockBuffer )
				\end{lstlisting}
			\item
				\begin{lstlisting}
				int Device::writeBlock (
				int CylinderNo, int TrackNo, int SlotNo, char *BlockBuffer )
				\end{lstlisting}
		\end{itemize}
		$\Rightarrow$ physische Adressierung der Slots
		\begin{itemize}
			\item
				Vorteile: schnell, maximale Speicherausnutzung
			\item
				Nachteile: kein Schutz, Wissen über Zuordnung zu Anwendungen nur im Programmierer, viele Fehlermöglichkeiten
		\end{itemize}
	\item
		$\Rightarrow$ Dateien
		\begin{itemize}
			\item
				haben Namen, Folge von Blöcken, ansprechbar über Blocknummer, dynamisch erweiterbar
			\item Abstrahieren von eigentlicher Hardware
			\item
				Zugriffsoperationen:
				\begin{itemize}
					\item
						\begin{lstlisting}
						BockFile::BlockFile (char *Filename, char Mode, int *Blocksize )
						\end{lstlisting}
						öffnet die Datei [\dots]
					\item

						\begin{lstlisting}
						int BlockFile::append (int NumberOfBlocks)
						\end{lstlisting}
						erweitert Datei um anzahl Blöcke, Rückgabewert wieviele Blöcke angelegt wurden
					\item

						\begin{lstlisting}
						int BlockFile::write ( int BlockNo, char *BlockBuffer)
						\end{lstlisting}
						Überschreibt angegebenen Block mit neuem Inhalt 
					\item
						\begin{lstlisting}
						int BlockFile::read ( int BlockNo, char *BlockBuffer )
						\end{lstlisting}
						liest Block in Puffer
					\item

						\begin{lstlisting}
						int BlockFile::size()
						\end{lstlisting}
					\item
						\begin{lstlisting}
						void BlockFile::drop (int NumberOfBlocks)
						\end{lstlisting}
						invers zu append
				\end{itemize}
			\item
				Anwendungsprogramme die blockorientierten Dateizugriff verwenden sind unabhängig von Hardware
			\item
				Für Programm ist Datei abstrakte Sicht auf Teile einer Platte $\Rightarrow$ \enquote{logische Platte/virtuelles Speichergerät}
		\end{itemize}
	\item
		Übersicht siehe Foliensatz \href{IDB-2015WS-02-Dateiverwaltung.pdf}{2/17}
\end{itemize}

\section{Sätze}
\begin{itemize}
	\item
		Bisher: Blöcke. Einheit des Transports zwischen Platte und RAM, Größe vom System vorgegeben
	\item
		$\Rightarrow$ neue Abstraktion: Satz
\end{itemize}
\subsection{Grundlegendes}
\begin{itemize}
	\item
		Repräsentieren ein \enquote{Objekt} der Anwendung
	\item
		Größe wird von Anwendung bestimmt und ist variabel (sowohl von Satz zu Satz als auch einzelne Sätze im Laufe der Zeit)
	\item
		Satz hier nur Folge von Bytes, keine innere Struktur
	\item
		Satzdatei: Menge von Sätzen fester oder variabler Länge, Reihenfolge unbestimmt
	\item
		Typischerweise mehr als ein Satz in einem Block
	\item
		Feste satzlänge als Sonderfall möglich, dann auch Ausnutzen (kein Längenfeld), feste Zahl von Sätzen pro Block, ungenutzter Speicherplatz am Ende eines Blockes
	\item
		Variable Satzlänge: Zahl der Sätze unterschiedlich in Blöcken, Sätze umsortieren (sind Menge!), Speicherplatz effizient nutzen
	\item Sequentielle Satzdatei: Folge von Sätzen fester/variabler Länge; Schreibreihenfolge= Abspeicherungsreihenfolge = Lesereihenfolge; sequenzieller Zugriff, kein wahlfreier Zugriff, kein Ändern eines Satzes (da i.Allg. mit Längenänderung)
\end{itemize}
\subsection{Zugriffsoperationen}
\begin{itemize}
	\item Öffnen:

		\begin{lstlisting}
		SeqRecordFile::SeqRecordFile (
		char *Filename, char Mode, int *RecLength )
		\end{lstlisting}
		Satzlänge beim Schreiben Eingabeparam, beim Lesen Ausgabeparam
	\item

		\begin{lstlisting}
		void SeqRecordFile::read-next (
		int *RecordLength, char *RecordBuffer)
		\end{lstlisting}
	\item
		\begin{lstlisting}
		void SeqRecordFile::write-next (
		char *RecordBuffer, int RecordLength )
		\end{lstlisting}
	\item
		Anwendungsprogramm ist Blockunabhängig!
	\item Nachteile: kein Direktzugriff auf Sätze, Gefahr der Fragmentierung der Datei
\end{itemize}
\subsection{Puffer}
\begin{itemize}
	\item
		Block ist weiterhin Einheit des Transfers zwischen  Platte und RAM
	\item
		Immer ganzen Block holen - enthält meistens noch mehr interessante Sätze
	\item
		Block erst auf Platte schreiben wenn voll
\end{itemize}
\subsubsection{Wechselpuffertechnik}
\begin{itemize}
	\item Verwendung von zwei Blockpuffern, die abwechselt genutzt werden können, beim Lesen prefetching im anderen Puffer machen
	\item
		Ist interne Optimierung, unsichtbar für höhere Schichten
\end{itemize}

\subsubsection{Allgemeiner Dateipuffer}
\begin{itemize}
	\item
		Verallgemeinerung: beliebig viele Puffer, evt. auch ganze buffer frames
	\item
		spart Zugriffsarmbewegungen
	\item
		Persistenz wird (zunächst) Problem - nach Schreiben eines Satzes gammelt er erst mal nur im RAM rum
\end{itemize}
\subsubsection{Fazit}
\begin{itemize}
	\item
		primitiv, aber viel Kapselungs- und Optimierungspotential
	\item Übersicht siehe Folie \href{run:IDB-2015WS-03-Saetze.pdf}{3/21}
\end{itemize}
\subsection{Direktzugriff auf Sätze}
\begin{itemize}
	\item
		flexiblerer Ansatz ist gebraucht $\Rightarrow$ Arbeit mit einzelnen Sätzen!
	\item
		Annahmen: variable Satzlänge (allg Fall), Reihenfolge der Abspeicherung $\neq$ Einfügereihenfolge $\Rightarrow$ darf verlorengehen
	\item Satzadresse:
		\begin{itemize}
			\item
				wird beim Einfügen vom DBMS zugeteilt, später zum Zugriff verwenden
			\item
				eindeutig u. unveränderlich (auch dann wenn Sätze gelöscht, verschoben, eingefügt werden)
		\end{itemize}
\end{itemize}
\subsubsection{TID}
\begin{itemize}
	\item
		Hilfsstruktur: Array mit Byte-positionen aller Sätze im Block
	\item
		Satzadresse ist Paar aus Blocknummer und Index in diesem Feld
	\item Löschen: Eintrag im Array wird invalid markiert; andere Sätze können verschoben werden, nur Array muss angepasst werden, Satzadressen bleiben stabil (so wie Nord-Korea)
	\item Ändern: Länge kann sich ändern: Schrumpfen (alles verschieben und Array anpasen) oder zu groß für Block werden (Verschieben des ganzen Satzes in anderen Block$\rightarrow$ überlaufbehandlung)
	\item Überlaufbehandlung: Im alten block bleibt an Stell des Satzes Satzadresse des neuen Blocks, bei weiterer verlagerung wird nur erste Indirektion umgebogen, maximaltiefe=1
\end{itemize}
\subsubsection{Direkte Satzdatei}
\begin{itemize}
	\item
		Zugriff auf beliebige Sätze möglich
	\item
		Operationen:
		\begin{lstlisting}
		RecAdr insert(char *RecBuf, int RecLen)
		void read(RecAdr adr, char *RecBuf, int *RecBuf, int *RecLen)
		void modify(RecAdr adr, char* RecBuf, int RecLen)
		void delete(RecAdr adr)
		\end{lstlisting}
	\item
		Freispeicherverwaltung noch sinnvoll. Entweder am Anfang jedes Blockes \#freeBytes oder Freispeichertabelle am Anfang der Datei (Suche nach Freispeicher sehr schnell, aber Tabelle muss mitgeändert werden (damit mindestens zwei Blöcke bei entsprechenden Operationen), neue Blöcke an Datei anhängen schwierig
	\item
		Sequentielles Lesen immer noch möglich, aber System entscheidet über Abspeicherungsreihenfolge
	\item
		Übersicht siehe Folie \href{run:IDB-2015WS-03-Saetze.pdf}{3/31}
\end{itemize}
\section{Schlüsselzugriff}
\begin{itemize}
	\item
		TIDs sind künstlich, haben nichts mit Anwendung zu tun (wie Telefonnummer)
	\item
		Besser wäre: über Inhalt (bestimmte Felder) zugreifen zu können, assoziativer Zugriff
	\item
		Sequentielle und direkte Satzdatei können das nicht, einzige Möglichkeit: \enquote{Scan}, alles durchsehen $\Rightarrow$ neue Hilfsstrukturen!
\end{itemize}
\subsection{Hashing}
\begin{itemize}
	\item
		Berechnung der Adresse aus Schlüssel dann direktes Lesen des Satzes; Berechnung der Blocknummer genügt, Suche im Block einfach
	\item
		Vorraussetzungen: maximale \#Sätze abschätzbar (zulässige Blocknummern wissen); genug Blöcke von Anfang an (genannt \enquote{Buckets})
	\item
		Hashfunktion:
		\begin{itemize}
			\item
				Ziel: gleichmäßige Verteilung der Sätze, aber Kollisionen bis zu gewissem Grad erwünscht (mehr als ein Satz pro Bucket/Block)
			\item Auswahl der Hashfunktion wichtig, hängt von Verteilung der Schlüssel ab
			\item
				Wenn nicht bekannt: Divisions-Rest-Verfahren beschte! $h(k) = k \mod q$, k Schlüssel, q Anzahl Buckets
			\item
				Überlauf kann trotzdem passieren:
				\begin{itemize}
					\item
						Open Addressing: Ausweichen auf Nachbar-Buckets in festgelegter Reihenfolge: +kein zusätzlicher Speicher, -beim löschen überläufer zurückholen, -zusätzliche Überläufe in Nachbarbuckets
					\item
						Overflow-Buckets (mit \enquote{Seperate Chaining}): Anlegen spezieller Überlaufbuckets mit Verkettung: -Speicher, +kein Mischen von Sätzen, +Nachbarbuckets werden in Ruhe gelassen
				\end{itemize}
			\item
				Hashfunktion nur intern verfügbar, Admin nicht Benutzer wählt aus
			\item
				Admin kann reorganisieren, ohne dass Anwendungen betroffen sind
			\item Operationen:
				\begin{lstlisting}
				void insert ( char *RecBuf, int RecLen, char *KeyVal)
				char *read (char *KeyVal, int *RecLen)
				void modify-key (char *OldKeyVal, char *NewKeyVal)
				\end{lstlisting}
				Rest wie vorher auch.
			\item
				Wenn Schlüssel nicht eindeutig: gefundene Sätze einzeln abrufen (mit Leseposition), mit anderen Feldern richtigen Satz identifizieren
			\item
				Bewertung: +schneller Zugriff über Schlüssel (1-2 Blockzugriffe), -Sätze gemixt, -Speicherplatz im voraus!, -Ordnung nur nach einem Schlüssel
		\end{itemize}
	\item
		Virtuelles Hashing
		\begin{itemize}
			\item
				Fortlaufende Reorganisation der Bucket-Folge während Einfügen und Löschen
			\item
				Gleichzeitig alte Hashfunktion (für kleinere Zahl von Buckets) und neue Hash-Funktion (für größere Zahl) benutzen: neue Funktion muss Buckets die alte Funktion auf ein Bucket abgebildet hat auf zwei Buckets abbilden
			\item
				mit $q$ Buckets beginnen, jeder mit $b$ Sätzen (Kapazität $q\times b$)
			\item
				Belegungsfaktor $\beta = \text{Anzahl gespeicherter Sätze}/(q\times b)$
			\item
				Verwendung eines Schwellwerts $\alpha$ für Belegungsfaktor. Wenn $\beta > \alpha$: Menge Buckets vergrößern
			\item
				Erster Ansatz: VH1
				\begin{itemize}
					\item
						Wenn $\beta > \alpha$ auf einen Schlag q, 2q, 4q neue Blöcke belegen
					\item
						Komplettes Umspeichern der Sätze verzögern: Erst wenn beim Einfügen in Bucket i($<q$) Überlaufbucket benötigt wird Rehash der Sätze dieses Buckets inkl. Überläufer mit h2, Bit i in Bitliste setzen
				\end{itemize}
			\item
				Zweiter Ansatz: Lineares Hashing
				\begin{itemize}
					\item Wenn $\beta > \alpha$ einen neuen Bucket hinten anfügen, Bucket auf dem Positionszeiger p steht aufteilen (auch ohne Überlaufbucket)
					\item
						Immer zuerst h1 anwenden, falls Ergebnis $< p$: h2 verwenden
					\item
						Nachdem Bucket q-1 aufgeteilt wurde: $p=0$, h1 wird nicht mehr benötigt, später h2 u. h3
					\item Wächst Dynamisch, einzige Hilfsstruktur: Positionszeiger (nächstes zu splittendes Bucket)
					\item
						Aber: keine Möglichkeit überlaufsätze zu vermeiden, auch bei vielen Überlaufsätzen ist Datei überbelegt und sollte erweitert werden
				\end{itemize}
		\end{itemize}
\end{itemize}


\subsection{B-Bäume}
\begin{itemize}
	\item
		Ausgangspunkt Binäre Such-Bäume (sind aber unzureichend, zu viele Blockzugriffe bei Abstieg
	\item
		B-Baum: Mehrweg-Baum, jeder Knoten entspricht einem Block
	\item
	Aufbau eines B-Bam-Knotens: $\begin{bmatrix}n & P0 & K1 & D1 & P1 & \dots & K2k&D2k &P2k \end{bmatrix}$\\
		n Anzahl verwendeten Einträge ($k\leq n \leq 2k$, bzw. in der Wurzel $1 \leq n \leq 2k$), (Ki, Di, Pi) sind ein Eintrag:
		\begin{itemize}
			\item Ki: Schlüsselwert
			\item Di: Datensatz
			\item Pi: Zeiger auf den Nachfolgeknoten (=Blocknummer)
		\end{itemize}
	\item
		Alle Schlüsselwerte im Unterbaum auf den P0  zeigt $\leq K1$
	\item
		Alle Schlüsselwerte im Unterbaum auf den Pi zeigt $> Ki \wedge \leq Ki+1$
	\item
		Beispiel siehe Folie \href{run:IDB-2015WS-05-Schluessel-Teil-2.pdf}{5/5}
	\item
		k errechnet sich aus Blockgröße: 2k ist maximale Zahl von Einträgen pro Block
	\item
		h Höhe des Baums, Anzahl der Ebenen
	\item
		Eigenschaften eines B-Baums:
		\begin{itemize}
			\item
				Jeder Pfad vom Wurzelknoten zu Blattknoten hat Länge h-1: Baum ist perfekt Balanciert
			\item
				Jeder Knoten (Ausnahme: Wurzelknoten und Blattknoten) hat mindestens k+1 Nachfolger, jeder Block mindestens halb voll $\Rightarrow$ Speicherplatzausnutzung > 50\%
			\item
				Wurzelknoten ist entweder Blattknoten oder $\geq 2$ Nachfolger
			\item
				Jeder Knoten höchstens 2k+1 Nachfolger (mehr passt ja nicht in Block)
		\end{itemize}
	\item
		Suche im B-Baum:
		\begin{itemize}
			\item
				Beginne im Wurzelknoten, Knoten von links nach rechts durchsuchen:
			\item
				Ki vergleichen
			\item
				$Ki > x$: Suche in Wurzel des an Pi-1 hängenden Unterbaums fortetzen
			\item
				$Ki < x$: Vergleich mit Ki+1 wiederholen
			\item
				$Kn < x$: Suche im Unterbaum von Pn fortsetzen
			\item
				Falls gesuchter Wert in Blattknoten nicht gefunden, Suche abbrechen, Schlüssel nicht vorhanden
		\end{itemize}
	\item
		Einfügen:
		\begin{itemize}
			\item
				Nur in Blattknoten, d.h. zunächst Abstieg durch Baum wie bei Suche, dann im gefunden Blattknoten entsprechend Sortierreihenfolge einfügen
			\item
				Falls Blattknoten schon voll (enthält 2k Sätze): Split
			\item
				die 2k+1 Sätze aufteilen (in sortierordnung)
			\item
				Ersten k Sätze im ersten (linken) Knoten
			\item
				Letzten k Sätze im zweiten (rechten) Knoten
			\item
				mittleren (k+1ten) Satz als neuen Diskriminator, d.h. als Verzweigungsinformation bei der Suche, in den Knoten eine Stufe höher einfügen, der auf den Blatknoten verweist
			\item
				Falls übergeordnete Ebene auch überläuft: Split auf der Ebene wiederholen
			\item
				Split des Wurzelknotens: Erzeugung von zwei neuen Knoten, Baum wächst um 1 (Siehe Folie \href{run:IDB-2015WS-05-Schluessel-Teil-2.pdf}{5/10} und \href{run:IDB-2015WS-05-Schluessel-Teil-2.pdf}{5/11})
		\end{itemize}
	\item
		Dynamische Reorganisation, baum immer balanciert, speicherplatzausnutzung garantiert $\geq 50\%$, bei gleichverteilter Einfügung $\ln 2 \approx 70\%$
	\item
		Löschen:
		\begin{itemize}
			\item
				Siehe \href{run:IDB-2015WS-05-Schluessel-Teil-2.pdf}{5/13} - \href{run:IDB-2015WS-05-Schluessel-Teil-2.pdf}{5/16}
			\item
				Suche zu löschenden Knoten
			\item
				Falls S in Blattknoten, löschen, evt. Unterauf behandeln
			\item Falls S in innerem Knoten, untersuche rechten und linken Unterbaum
			\item
				Betrachte Blattknoten mit direktem Vorgänger S' von S und Blattknoten mit direktem Nachfolger S'' von S (größter Wert, bzw. kleinster Wert in den Unterbäumen)
			\item
				Wähle den aus, der mehr Elemente hat, falls beide gleich viel zufällig
			\item
				Ersetze zu löschenden Schlüssel S durch zuvor gewähltes S' bzw. S''
			\item
				Lösche S' bzw. S'' im gewählten Blattknoten, behandle evt. Unterlauf
			\item
				Unterlaufbehandlung: Mischen des Unterlaufknotens mit seinem Nachbarknoten und dem darüber liegenden Diskriminator (split rückwärts)
			\item
				Evt. nach oben hin fortsetzen
			\item
				Wird die Wurzel erreicht kann der Baum in der Höhe um 1 schrumpfen, evt. kommt es auch zu Überlauf, dann split
		\end{itemize}


\end{itemize}
\subsection{Variante: B*-Bäume}
\begin{itemize}
	\item
		Alle Sätze (Schlüssel + TIDs) werden in Blattknoten abgelegt, innere Knoten nur noch Verzweigungsinformation
	\item
		Sätze auf Blattebene sind sortiert
	\item
		Operationen auf B*-bäumen siehe \href{run:IDB-2015WS-05-Schluessel-Teil-2.pdf}{5/22} - \href{run:IDB-2015WS-05-Schluessel-Teil-2.pdf}{5/27}
	\item Vergleich B und B*-Baum siehe \href{run:IDB-2015WS-05-Schluessel-Teil-2.pdf}{5/28}
	\item
		Die Payload im Baum kann sein: ein Datensatz, eine Liste von Datensätzen (mit gleichem Schlüssel), eine (Liste von) Satzadresse
\end{itemize}
\subsection{Bitmap-Index}
\begin{itemize}
	\item
		B-Bäume und Hashing sinnvoll für Schlüssel mit hoher Selektivität. Faustregel Grenztrefferrate 5\%. Wenn mehr Treffer lohnt sich Aufwand für Indexzugriff nicht mehr
	\item
		Für Schlüssel mit wenig Selektivität (z.B. geschlecht): Bitmap machen.
	\item
		Vorraussetzung: feste Reihenfolge der Sätze
	\item
		Bitwert 1 heißt: der Schlüssel hat im Satz den Wert, zu dem die Liste gehört, 0 heißt: anderer Wert
	\item
		Interessant bis zu ca. 500 verschiedenen Werten
	\item
		Bei kleinen Wertigkeiten nur sinnvoll, wenn Attribut oft in Konjunktion mit anderen indizierten Attributen auftritt (z.B. Geschlecht u. Wohnort), das ist auch Hauptvorteil
	\item
		Indexgr"o"se nicht problematisch, gerade höherwertige Indizes (dann dünn besetzt) gute Kompressionsverfahren verfügbar, z.B. RLE
	\item
		Zugriffsoperatoren bei allen Index-strukturen die selben: Entscheidend ist Zugriff über Schlüssel, nicht die Realisierung
	\item
		Art von Datenunabh"angigkeit: Datenstruktur-Unabhängigkeit  (oder Speicherungsstruktur-Unabhängigkeit)
\end{itemize}

\subsection{Primär- und Sekundär Organisation}
\begin{itemize}
	\item
		Primärorganisation: Speicherung der Sätze selbst - kann sequentiell, direkt oder über Schlüssel sein
	\item
		Sekundärorganisation: verweist auf andere Sätze, die in beliebiger Primärorganisation gespeichert wurden (wenn diese direkten Zugriff auf einzelne Sätze ermöglicht (wenn diese direkten Zugriff auf einzelne Sätze ermöglicht)
	\item
		1 Primärorganisation, beliebig viele sekundär Organisationen
	\item
		eine Datei für Sätze selbst, für jede Sekundär-Organisation jeweils eine Datei
	\item
		Index über Primärschlüssel muss nicht die Primär-Organisation sein
	\item
		Zusamenfassung \href{run:IDB-2015WS-05-Schluessel-Teil-2.pdf}{5/41}
\end{itemize}

\section{Puffer}

\begin{itemize}
	\item
		Im Hauptspeicher Platz für n Blöcke, heutzutage Größenordnung $10^4 - 10^6$
\end{itemize}
\subsection{Blockzugriffe}
\begin{itemize}
	\item
		Sogenannter logischer Zugriff:
		\begin{itemize}
			\item
				Entweder Block bereits im Puffer
			\item
				Oder Einlagern des Blocks, dabei Verdrängung eines anderen Blocks: ist der geändert worden? auf die Platte zurück schreiben. Nicht geändert worden? direkt überschreiben
			\item
				Welcher Block wird verdrängt? nach Alter, Benutzungshäufigkeit, FIFO, LFU?
			\item
				LRU (Least recently used), Alter seit dem letzten Zugriff. Prinzipiell sehr gut, aber auch teuer
			\item
				CLOCK (Second chance)
				\begin{itemize}
					\item LRU-Verhalten mit einfacher Implementierung: Blöcke haben \enquote{Benutzt}-Bit. Bei Verdrängung zyklische Suche mit dem Auswahlzeiger: ist Bit = 1, wird auf 0 gesetzt, nächster Block; ist Bit = 0, Block wird ersetzt: Block \enquote{überlebt} mindestens zwei Zeigerumläufe.
				\end{itemize}
			\item
				Wichtige Restriktion: Blöcke die noch benutzt werden sind nicht ersetzbar (Unterschied zu Paging)
		\end{itemize}

\end{itemize}

\subsection{Schnittstelle einer Pufferverwaltung}
\begin{itemize}
	\item
		Ersetz-/Einbringstrategie vor Benutzer/höheren Schichten verbergen
	\item
		Zugänglich nur über folgende Operationen:
		\begin{lstlisting}
		char *Buffer::fix (
		BlockFile File, int BlockNo, char Mode );

		void Buffer::unfix (char *BufferAddress );
		\end{lstlisting}
		Mode: r/w?
	\item
		Auch leerer Block muss mit fix in den Puffer kommen (natürlich nicht von Platte, aber evt. Verdrängung etc)
\end{itemize}

\subsection{Fehlersituation}
\begin{itemize}
	\item
		Irgendwas ist kaputt: Hardware, Stromversorgung...
	\item
		Hauptspeicher ist mit Puffer weg, manches steht schon auf Platte, manches war noch im Puffer $\Rightarrow$ inkonsistenzen
	\item
		bei der Verdrängung alte Blockinhalte auf Platte nicht überschreiben, sondern in andere Slots schreiben. Wenn \enquote{am Schluss} alle Blöcke auf der Platte sind ununterbrechbar von alt auf neu umschalten
	\item
		Einbringen: Ablegen auf einem nicht-flüchtigen Speicher, so dass es nach Ausfall verwendet werden kann
	\item
		Idee: Trennung in Block und Seite
		\begin{itemize}
			\item
				Seite = Block im Puffer (genauso groß)
			\item
				Anwender arbeit nur noch mit Seiten, mehrere Blöcke für eine Seite (alter und neuer Inhalt jeweils)
		\end{itemize}
	\item Segment:
		\begin{itemize}
			\item
				linearer, logischer, potenziell unendlicher, tatsächlich aber endlicher Adressraum mit sichtbaren Seitengrenzen
			\item
				Entspricht Datei (=Folge von Seiten)
			\item
				Welche Blöcke für eine Seite?
				\begin{itemize}
					\item
						direkt: aufeinanderfolgende Seiten auf aufeinanderfolgende Blöcke einer Datei - keine Hilfsstruktur
					\item
						indirekt: Flexibler, aber Array als Hilfsstruktur: Blocknummer zu jeder Seite
					\item
						Vergleich siehe \href{site:IDB-2015WS-06-Puffer.pdf}{6/17 - 18}
				\end{itemize}
		\end{itemize}
	\item
		Arten von Einbringstrategien:
		\begin{itemize}
			\item
				Bei direkter Seitenzuordnung: bei Verdrängung aus Puffer ersetzt Seite genau den Block aus dem sie ursprünglich gelesen wurde (\enquote{update in place})
				\begin{itemize}
					\item
						Vorteil: Einfach, nur ein Block pro Seite
					\item
						Nachteil: Keine Unterstützung im Fehlerfall (alte Zustand muss \emph{vor} dem Einbringen auf einen sicheren Speicher geschrieben werden (Write-Ahead Log)
				\end{itemize}
			\item
				Bei indirekter Seitenzuordnung: Beim Verdrängen wird Block in irgendeinen freien Block geschrieben. Auch nach Hauptspeicherverlust konsistente Datenbank in alten Blöcken!
				\begin{itemize}
					\item
						Wann werde ich alten Blöcke los?
						\begin{itemize}
							\item
								Schattenspeicher: Inhalte aller Seiten eines Segments werden in einem Sicherungsintervall $\delta t$ in einem konsistenten Zustand unverändert gehalten\\
								Sicherungspunkte sind segmentorientiert (für alle Nutzer des Segments gemeinsam) und bestehen aus: belegten Seiten, Seitentabelle $V_k$, Bitliste $M_j$ auf stabilem Speicher \\
								Im Fehlerfall segmentorientieres Zur"uckgehen auf letzten Sicherungspunkt
							\item
								Twin Slots Prinzip: doppelter Speicherplatzbedarf, Seite braucht Versionsnummer, immer beide Bl"ocke lesen, bei "Anderungen den a"lteren "uberschreiben
						\end{itemize}
				\end{itemize}
		\end{itemize}
	\item
		Zusammenfassung siehe \href{site:IDB-2015WS-06-Puffer.pdf}{6/25 u. 6/26}
\end{itemize}


\section{Zugriff auf DBs aus einem Programm}
\subsection{Relationale Datenbanken}
\begin{itemize}
	\item
		Intern: Satzschnittstelle (virtuelle Objekte Segment/Datei(=Menge von S"atzen) und Satz (=Folge von Bytes, evt. mit Schl"ussel) )
	\item
		OPs: seuqenzielles Lesen aller S"atze (eines Typs)
	\item
		Zugriff "uber Satzadresse/Schl"ussel
	\item
		N"achster Schritt: Unabh"angigkeit von Anwendungen und Organisationsformen (Segmenttypen) und Indexen
		\begin{itemize}
			\item
				Schl"usselzugriff immer noch m"oglich, aber auch m"achtigere Operationen (Boolsche Ausdr"ucke, \enquote{Joins})
		\end{itemize}
	\item
		Weitere Abstraktion: Dateien  $\rightarrow$ Relationen/Klassen; S"atzen $\rightarrow$ Tupeln/Objekten; Feldern $\rightarrow$ Attribute
\end{itemize}

\subsection{Programmzugriff}
\begin{itemize}
	\item
		In C vom Precompiler in Funktionsaufrufe transformiert:
		\begin{itemize}
			\item
				\begin{lstlisting}
				exec sql <SQL-Anweisung>;
				\end{lstlisting}
			\item
				Mehr siehe \href{site:IDB-2015WS-07-Programmzugriff.pdf}{7/5 ff.}
			\item
				Genormt, deswegen portabel
			\item
				Erlaubt Erzeugung von Zugriffsmodulen: Compile-time "Ubergabe an DBS, das kann optimieren. Zur Laufzeit nur noch ausf"uhren, call "uber einen Namen
			\item
				SQL Injektion nicht m"oglich
			\item
				Auch \enquote{Embedded SQL} genannt
		\end{itemize}
	\item
		Auch: Call-level interface/Application-programming interface (CLI/API)
		\begin{itemize}
			\item
				z.B.
				\begin{lstlisting}
				call DBS ("select ... from ... where ... ");
				\end{lstlisting}
			\item
				Einfacher für Hersteller, mehr Arbeit f"ur Anwender
			\item
				Z.B. JDBC f"ur Java
			\item
				Mehr siehe \href{site:IDB-2015WS-07-Programmzugriff.pdf}{07-13 ff}
			\item
				Prepared Statements besser, meiste Verarbeitung passiert bei Erzeugung
			\item
				Stored Procedures noch besser: Analyse und OPtimierung dann nur einmalig
			\item
				Relativ elementar, Optimierung explizit, Gefahr durch SQL-Injection
		\end{itemize}


\end{itemize}


\subsection{O/R-Mapping}
\begin{itemize}
	\item
		F"ur OO-Programmiersprachen
	\item
		Konfigurationsdateien (XML) sagen welche Teile von Objekt in Relation gespeichert werden
	\item
		Transfer der Daten zwischen Datenbank und Anwendung automatisch
\end{itemize}




\section{Transaktionen}
\begin{itemize}
	\item
		Probleme:
		\begin{itemize}
			\item
				Problem: Datenbankanwendungsprogramm st"urzt ab (fehlerhaft oder von au"sen abgebrochen)  $\rightarrow$ Hauptspeicher des Programms ist weg, Daten im Puffer und auf Platte unvollst"andig und inkonsistent
			\item
				Notwendige Ma"snahmen: Puffer wegwerfen, Daten auf Platte von Hand bereinigen
			\item
				Oder: Systemfehler: DBMS oder BS st"urzt ab, Programm kann gar nichts daf"ur
			\item
				War vielleicht sogar schon fertig, aber Daten nur im Puffer $\rightarrow$  Puffer ist weg, alles inkonsistent und bl"od
			\item
				Alle Programme die gerade liefen von Hand nachvollziehen, kl"aren ob Ergebnisse vollst"andig auf Platte und welche wiederholt werden m"ussen
			\item
				er: Ger"atefehler: "Anderungen auf Platte verloren
			\item
				Notwendige Ma"snahmen: Neue Platte, Backup einspielen, alle "Anderungen seit Backup wiederholen

		\end{itemize}
	\item
		Erw"unschte Zust"ande der Daten auf Platte:
		\begin{itemize}
			\item
				Physische Konsistent:
				\begin{itemize}
					\item
						Korrektheit der Speicherungsstrukturen
					\item
						Verweise und Addressen korrekt
					\item
						Indexe sind konsistent mit Prim"ardaten
				\end{itemize}
			\item
				Logische Konsistenz
				\begin{itemize}
					\item
						Korrektheit der Dateninhalte, stellen (Teil der) reale(n) Welt dar
					\item
						Bedingugen des Datenmodells (Prim"arschl"ussel etc) und Assertions erf"ullt
				\end{itemize}
		\end{itemize}
	\item
		Annahmen:
		\begin{itemize}
			\item
				vollst"andig ausgef"uhrte Datenbank-Operationen hinterlassen physisch konsistenten Zustand (DBMS ist korrekt)
			\item
				vollst"andig ausgef"uhrte Anwendungsprogramme hinterlassen logisch konsistenten Zustand (Programmier hat korrekt gearbeitet)
			\item
				Nach Fehlerfall weder logisch noch physisch konsistent
		\end{itemize}
	\item
		Unser Ziel: Systemunterst"tzung zur Recovery nach Fehlerfall
		\begin{itemize}
			\item
				Entweder: Backward Recovery - R"uckgangigmachen der bereits ausgef"uhrten "Anderungen
			\item
				Oder Forward Recovery: Wiederholen verlorengegangener "Anderungen
			\item
				Daf"ur brauchen wir Logging im laufenden Betrieb
		\end{itemize}
\end{itemize}


\subsection{Transaktion}
\begin{itemize}
	\item
		Transaktion(TA): Folge von DB-Operationen, vom Anwender definiert: logisch konsistenter Zustand $\rightarrow$  logisch Konsistenter Zustand
	\item
		Bei Fehler vor Ende TA m"ussen "Anderungen, die eine TA bereits gemacht hat r"uckg"angig gemacht werden, Zustand vor Beginn wieder herstellen. Datenbestand sieht aus als ob TA nie gestartet w"arer
	\item
		Fehler nach Ende TA: durch Fehler verlorengegangene Ergebnisse transparent wiederherstellen
	\item
		Aufgabe des Anwenders dann nur noch Anfang (implizit durch Beginn der DB-Operationen) und Ende (\lstinline$commit$ oder \lstinline$abort$) definieren
	\item
		TA ist logische Arbeitseinheit, gleichzusetzen mit SQL Befehlen (innerhalt TA vor"ubergehend lgisch inkonsistente Zust"ande)
	\item
		Ziel: Transaktionsmanager garantiert entweder vollst"andige Ausf"uhrung von TA oder Wirkungslosigkeit der gesamten TA, TAs sind \emph{atomar}
	\item
		Mehrbenutzerbetrieb: Solange TA l"auft darf niemand sonst von ihr durchgef"uhrte "Anderungen benutzen! TA T1 sieht nur Zustand vor oder nach Transaktion T2
	\item
		TAs m"ussen synchronisiert werden $\rightarrow$  fiktiver Einbenutzerbetrieb f"ur jede Transaktion auch f"ur rein lesende Zugriffe
	\item
		\enquote{ACID}-Eigenschaften:
		\begin{itemize}
			\item
				Atomicity
			\item
				Consistency: erfolgreiche TA $\rightarrow$ Konsistenbedingungen werden eingehalten
			\item
				Isolation: TAs sind isoliert voneinander, benutzen keine inkonsistenten Zwischenergebnisse anderer TAs
			\item
				Durability: Ergebnisse m"ussen persistent sein, bevor Erfolg an Anwendung gemeldet werden darf
		\end{itemize}
\end{itemize}

\subsection{Programmierung von TAs}
\begin{itemize}
	\item
		Wenn Programm nur eine TA ausf"uhrt und scheitert kann es einfach nocheinmal ausgef"uhrt werden
	\item
		Wenn Programm mehrere Tas ausf"uhrt, muss erneute Ausf"uhrung modifiziert werden: erfolgreiche TA darf nicht wiederholt werden!
	\item
		Beispiel \href{site:IDB-2015WS-08-Transaktionen.pdf}{08/25 ff}
\end{itemize}



\section{Speicherung von Tupeln und Relationen}
\begin{itemize}
	\item
		Viele M"oglichkeiten Relationen mit Mitteln der darunterliegenden Schicht abzuspeichern
\end{itemize}

\subsection{Speicherung von Tupeln und S"atzen}
\begin{itemize}
	\item
		Sätze bestehen aus Feldern (mit Namen, Typ, feste/variable L"ange)
	\item
		Systemkatalog: Informationen "uber Felder und Reihenfolge
	\item
		Satztyp: Menge von S"atzen mit gleicher Struktur (z.B. Tupel derselben Relation)
		\begin{itemize}
			\item
				Jeder Satz ein Satztyp
			\item
				Zuordnung zu Segmenten: n:1, manchmal n:m
		\end{itemize}
	\item
		Annahmen: 
		\begin{itemize}
			\item
				Variable L"ange, Satz sollte vollst"andig in einer Seite ablegbar sein, Reihenfolge der Felder unwichtig
		\end{itemize}
	\item
		Anforderungen an Tupelspeicherung
		\begin{itemize}
			\item
				Speicherplatzeffizienz (undefiniete Werte gar nicht speichern, wenige Hilfsstrukturen, variable L"ange)
			\item
				Direkter Zugriff auf Felder
			\item
				Flexibilit"at (Hinzuf"ugen von Feldern bei allen S"atzen, L"oschen eines Feldes aus allen S"atzen)
		\end{itemize}
	\item
		Speicherungsstuktur in S"atzen:
		\begin{itemize}
			\item
				Konkat von feldern fester L"ange (speicher meh)
			\item
				Zeiger im Vorspann (Unflexibel beim hinzuf"ugen von Feldern)
			\item
				Eingebette L"angenfelder (Keine direkte Berechnung der Satzinternen Adresse eines Feldes aus Katalogdaten m"oglich, aber dynamisch erweiterbar)
			\item
				Eingebette L"angenfelder mit Zeigern (Variabel lange Felder ans Ende legen, im festen Strukturteil Felder mit fester L"ange und Zeiger). Schaubild \href{site:IDB-2015WS-09-Speicherung.pdf}{09/9}
			\item
				Gibt noch mehr M"oglichkeiten: Tupel fragmentieren, Spaltenweise abspeichern (gut f"ur Data-Warehousing/analytische Auswertung, nicht so gut f"ur "Anderungen und Verb"unde, aufs Lesen hin optimiert) Mehr zu C-Store in \href{site:IDB-2015WS-09-Speicherung.pdf}{09/12-Ende}
		\end{itemize}
\end{itemize}


\section{Anfrageverarbeitung}
\begin{itemize}
	\item
		Bekommen mengenorientierten Zugriff, m"ussen auf satzorientierte Operatoren und Benutzung von Indexstrukturen Abbilden
	\item
		Schritte:
		\begin{enumerate}
			\item
				Syntaxchecker
			\item
				Zugriffsberechtigung und Integritycheck
			\item
				Optimierung (wichtig!)
			\item
				Ausf"uhrung
		\end{enumerate}
	\item
		Optimalit"at hinsichtlich: Durchsatz? Antwortzeit? Einhalten von Antwortzeitschranken?
	\item
		Aufteilung der Anfrageverarbeitung in:
		\begin{itemize}
			\item
				Anfrageverarbeitung auf logischem DB-Prozessore, liefert Anfrageausf"uhrungsplan
			\item
				Anfrageausf"uhrung auf physischem DB-Prozessor, tats"achliche Ausf"uhrung
		\end{itemize}

	\item
		Phasen der Anfrageverarbeitung:
		\begin{itemize}
			\item
				Lexikalische und syntaktische Analyse, parser, erstellen eines Anfragebaums
			\item
				Semantische Analyse: Existenz und G"ultigkeit der referenzierten Relationen und Attribute, Namensaufl"osung, Konvertierung der Werte vom externen Format in interne Darstellung
			\item
				Zugriffs/Integrit"atskontrolle: Kontrolle von Formaten, ggf Konvertierung von Datentypen
			\item
				Standardisierung und Vereinfachung: Anfragebaum in Normalform bringen, Elimination von Redundanzen
			\item
				Restrukturierung und Transformation: Algebraische Verbesserung (heuristische Regeln $\rightarrow$ globale Verbesserung des Anfragebaums), nicht-algebraische Verbesserung (Transformation, logische Operatoren durch Planoperatoren ersetzen) Auswahl der g"unstigsten Planalternative
			\item
				Code-Generierung: erzeugt zwischencode und ausf"uhrbares Zugriffsmodul

		\end{itemize}

\end{itemize}

\subsection{Interndarstellung einer Abfrage}
\begin{itemize}
	\item
		Relationale Algebra definiert relationale logische Operatoren, mit denen wird Anfragebau/Operatorbaum gebaut
		\begin{itemize}
			\item
				Selektion: Auswahl von Zeilen
				\begin{itemize}
					\item
						SEL(R, pred)
					\item
						entspricht WHERE
				\end{itemize}
			\item
				Projektion: Auswahl von Spalten
				\begin{itemize}
					\item
						Proj(R,L) (mit L Menge von Tupelnamen)
					\item
						Eliminierung von Duplikaten
					\item
						Entspricht SELECT
				\end{itemize}
			\item
				Kreuzprodukt: Konkat jedes Tupels einer Relation mit jedem Tupel einer anderen
				\begin{itemize}
					\item
						Cross(R,S)
					\item
						Wird in FROM definiert (z.B. FROM Personen,Filme)
				\end{itemize}
			\item
				Verbund: Konkat von Tupeln aus zwei Relationen die Bedingung erf"ullen
				\begin{itemize}
					\item
						Join(R,S,pred), pred Pr"adikat "uber Attributen aus beiden Relationen
					\item
						Eigentlich schon Optimierung, da auch durch Kreuzprodukt und Selektion darstellbar
					\item
						Wird in FROM und WHERE Klausel definiert
				\end{itemize}
			\item
				Mengenoperatoren
				\begin{itemize}
					\item
						UNION(R,S)
					\item
						INTERSECT(R,S)
					\item
						EXCEPT(R,S)
					\item
						Auf logischer Ebene auch n-stellige Operatoren, k"onnen auf Sequenz von bin"aren zur"uckgef"uhrt werden
				\end{itemize}
			\item
				Gibt auch Multimengen-orientierte Operatoren
			\item
				Weitere Operatoren, durch SQL dazugekommen:
				\begin{itemize}
					\item
						DUP-ELIM(R)
					\item
						Aggregation
					\item
						GROUP(R,L,agg), L Gruppierungsattribute, agg Aggretion
					\item
						G-PROJ(R,L) mit L = (name1 = expr1, name2= expr2) Liste von Ausdr"ucken zur Berechnung von neuen Attributwerten
					\item
						SORT(R,L)
					\item
						OUTER-JOIN(R,S, pred, left|right|full)
				\end{itemize}
		\end{itemize}
	\item
		Operatorb"aume
		\begin{itemize}
			\item
				Blattknoten "ublicherweise Relationen
			\item
				Knoten Operatoren der Rel.Algebra
			\item
				Gerichtete Kanten Datenfluss
		\end{itemize}
\end{itemize}


\subsection{Standarrdisierung und Vereinfachung}
\begin{itemize}
	\item
		Wahl einer Normalform: Konjunktive (A OR B OR ...) AND .... (..OR..OR) oder Konjunktive (A AND B ANd ..) OR ...(...AND..AND..)
	\item
		Eliminierung von Idempotenzen, leeren Relationen
	\item
		Nutzung von semantischen Integrit"atsbedingungen
\end{itemize}


\subsection{Restrukturierung}
\begin{itemize}
	\item
		Viele Regeln:
		\begin{itemize}
			\item
				n-fachen Verbund durch Folge von bin"aren Verb"unden ersetzen
			\item
				Verbund ist kommutativ und assoziativ
			\item
				Selektionen k"onnen zusammengefasst werden
			\item
				Projektionen k"onnen zusammengefasst werden
			\item
				Selektion und Projektion d"urfen vertauscht werden
			\item
				Selektion und Verbund d"urfen Vertauscht werden
			\item
				Selektion darf mit Vereinigung und Differenz vertauscht werden
			\item
				Selektion und Kreuzprodukt k"onnen zu JOIN zusammengefasst werden
			\item
				Gibt aber noch mehr
		\end{itemize}
	\item
		Zwischenergebnisse m"oglichst klein halten
	\item
		Heuristik zum Restrukturieren:
		\begin{itemize}
			\item
				Komplexe Verb"unde in bin"are zerlegen
			\item
				Selektionen mit mehreren Pr"adikaten zerlegen in Selektionen mit jeweils einem Pr"adikat Term
			\item
				Selektionen so fr"uh wie m"glich ausf"uhren
			\item
				Selektion und Kreuzprodukt zu JOIN zusammenfassen, wenn Pr"adikat Attribute aus beiden Relationen verwendet
			\item
				Einfache Selektionen wieder zusammenfassen (aufeinanderfolgende Selektionen derselben Relation gruppieren)
			\item
				Projektionen auch so fr"uh wie m"oglich ausf"uhren, hinunterschieben zu Bl"attern des Anfragebaums, dabei aber Duplikateliminierung vermeiden
		\end{itemize}
\end{itemize}


\section{Relationale Operatoren}
\subsection{Transformation}
\begin{itemize}
	\item
		Grunds"atzliche Aufgabe: Ersetzen der logischen Operatoren (SEL(), PROJ(), JOIN(), \dots) durch Planoperatoren
	\item
		Teilprobleme:
		\begin{itemize}
			\item
				Gruppierung direkt benachbarter Operatoren zur Auswertung durch einzelnen Planoperator (Verbund mit Selektion und/oder Projektion l"asst sich durch einen speziellen Planoperator gemeinsam ausf"uhren)
			\item
				Bestimmung der Verkn"upfungsreihenfolge bei Verbundoperationen(Minimierung der Zischenergebnisse, kleinsten (Zwischen-)Relationen immer zuerst verkn"upfen
			\item
				Erkennen gemeinsamer Teilb"aume
		\end{itemize}
	\item
		Relationale Planoperatoren:
		\begin{itemize}
			\item
				Unterprogramme/Komponenten des DMBS
			\item
				Parameter: Eingabe-Relationen, Indexstrukturen, Pr"adikate
			\item
				Vorraussetzungen: Vorhandensein bestimmter Speicherungsstrukturen (vor allem Indexe)
			\item
				Ergebnis: ganze Relation, n"achstes Tupel, n"achste n Tupel
			\item
				Haben Kosten (Zeit, Speicher, CPU)
			\item
				Gibt Ein-Variablen-Ausdr"ucke (Auswahl von Elementen aus einer Relation), Zwei-Variablen-Ausdr"ucke (\dots aus zwei Relationen) und k-Variablen-Ausdr"ucke (werden normalerweise in Ein- und Zwei-Variablen-Ausdr"ucke zerlegt
		\end{itemize}

	\item
		Selektion:
		\begin{itemize}
			\item
				Scan-Operators (Wahlweise mit Definition von Start/Stopp-Bedingungen und einfachen Suchargumenten
			\item
				Relationen-Scan (Sequenzielles Lesen aller Tupel einer Relation
			\item
				Index-Scan
		\end{itemize}
	\item
		Projektion
		\begin{itemize}
			\item
				Typischerweise mit im Planoperator von Sortierung, Selektion oder Verbund durchgef"uhrt
		\end{itemize}

	\item
		Sortierung:
		\begin{itemize}
			\item
				Erforderlich f"ur \lstinline$order by$
			\item
				Kann aber auch Joins, Gruppierung und Duplikateliminierung schneller machen
			\item
				Ist blockierend, kein Pipelining m"oglich
			\item
				I.Allg extern, Teilergebnisse m"usse auf Externspeicher ausgelagert werden
		\end{itemize}
	\item
		Join "uber mehrere Relationen
		\begin{itemize}
			\item
				Zerlegung in n-1 Joins
			\item
				$n!$ verschiedene Reihenfolgen m"glichn
		\end{itemize}

	\item
		Join
		\begin{itemize}
			\item
				Teuer und h"aufig
			\item
				Verschiedene Zugriffspfade (Relationen-Scan, Scans "uber Indexe (interessant: Sortierreihenfolge nach den gejointen Attributen
			\item
				Scans "uber zugeh"orige Selektionsattribute ($\neq$ Attributen "uber denen gejoint wird!), wenn daf"ur Index da ist
			\item
				Nested-Loop-Join
				\begin{itemize}
					\item
						Annahme: S"atze in beiden Relationen sind nicht nach Joinattributen geordnet, keine Indexstrukturen "uber Joinattributen
					\item
						Algorithmus f"ur $\Theta$-Verbund
						\begin{lstlisting}
Scan ueber S; //aeussere Schleife
//P(s.SA)=Selektion auf einer Relation
fuer jeden Satz s, fuer den P(s.SA) gilt: 
Scan ueber R; //innere Schleife
fuer jeden Satz r,
fuer den P(r.SA) AND (r.VA OP s.VA) gilt: (OP =, <=, < etc..)
uebernimm kombinierten Satz (r, s) in das Ergebnis;
						\end{lstlisting}
				\end{itemize}
			\item
				Komplexit"at $\mathcal{O}(n^2)$
			\item
				Falls Verbundattribut der zweiten Relation unique ist, kann Suche nat"urlich abgebrochen werden
		\end{itemize}

	\item
		Nested Loop Join mit Indexzugriff
		\begin{itemize}
			\item
				Annahme: Gibt Index "uber Verbundattributen bei beiden Relationen
			\item
				\begin{lstlisting}
Scan ueber S;
fuer jeden Satz s, fuer den P(s.SA) gilt:
ermittle ueber Ir(R.VA) alle Saete mit r.VA = s.VA;
ggf fuer jdes TID: hol Satz r;
fuer jeden Satz r, fuer den P(r.SA) gilt:
uebernimm kombinierten Satz (r,s) in das Ergebnis;
				\end{lstlisting}
			\item
				Eigentlich wird seitenweise vorgegangen um mehrfachen Zugriff auf selbe Seitie zu vermeiden
		\end{itemize}
	\item
		Sort-Merge Verbund
		\begin{itemize}
			\item
				Zweiphasiger Algorithmus: Phase 1 (Sortierung von beiden Relationen nach Verbundattributen, dabei gleich Pr"adikate auf den Relationen ausfuehren), Phase 2(Gleichzeitiger Scan "uber beiden sortieren Relationen
			\item
				$\mathcal{O}(N \log n)$
		\end{itemize}

	\item Hashverbund
		\begin{itemize}
			\item
				Durch immer gr"o"sere Hauptspeicher m"oglich
			\item
				Einfachster Fall (\enquote{Classic Hashing}):
				\begin{itemize}
					\item
						Aufteilen der kleineren Relation in $p$ Abschnitte $R_i(1\leq i \leq p)$, so dass jeder der p Abschnitte in den verf"ugbaren Hauptspeicher passt (evt. Pr"adikat ausf"uhren, wenn vorhanden)
					\item
						Hashtabelle bauen (wird nach jedem Abschnitt wieder gel"oscht/neu aufgebaut)
					\item
						Scan "uber S, hashen nach Verbundattribut, durchf"uhrung des Verbundes
				\end{itemize}
			\item
				$\mathcal{O}(p \times N)$, Idealfall R passt ganz in Hauptspeicher, $p = 1$
			\item
				Nachteil: eine Relation muss p-mal gelesen werden
		\end{itemize}

	\item
		Duplikateliminierung:
		\begin{itemize}
			\item
				Klassisch durch Sortieren und Gruppieren, aber auch durch Hashing m"oglich
		\end{itemize}
	\item
		Grupperiung:
		\begin{itemize}
			\item
				Durch Sortierung und Scan mit Aggregation pro Gruppe, aber auch wieder Hashing m"oglich (Abbildung auf Z"ahler, bisherige Summe, bisheriges Minimum/Maximum)
		\end{itemize}

\end{itemize}
\subsection{Anfrageoptimierung}
\begin{itemize}
	\item
		Zentrales Problem: Globale Optimierung zu aufw"andig  $\rightarrow$ Heuristiken!
	\item
		Ziel:
		\begin{itemize}
			\item
				Maximierung des Outputs (Durchsatzmaximierung? in Transaktionen pro Sekunde?
			\item
				Minimierung der Ressourcennutzung?
		\end{itemize}
	\item
		Kostenarten bei Anfrageoptimierung: (sind nicht unabh"angig voneinander!)
		\begin{itemize}
			\item
				Berechnungskosten
			\item
				I/O-Kosten
			\item
				Speicherungskosten
			\item
				Bei verteilen DBS: Kommunikationskosten
		\end{itemize}
	\item
		Erstellung von Ausf"uhrungspl"anen:
		\begin{itemize}
			\item
				Eingabe: Algbraisch optimierter Anfragebaum, existierende Speicherungsstrukturen und Zugriffspfade, Kostenmodell
			\item
				Ausgabe: m"oglichst optimaler Ausf"uhrungsplan
			\item
				Annahmen: Datenelemente und Attributwerte sind gleichverteilt (im allg. falsch!) und Suchpr"adikate sind unabh"angig (auch falsch!)
			\item
				Vorgehensweise:
				\begin{itemize}
					\item
						Generierung aller \enquote{vern"unftigen} logischen Ausf"uhrungspl"ane
						\begin{itemize}
							\item
								Planoperatoren in verschiedenen Implementierungen
							\item
								Operationsreihenfolge (bei Mehrfachjoins) kann variiert werden
							\item
								Entstehung riesiger Suchr"aume bei komplexen Anfragen ($10^{23}$ m"ogliche Ausf"uhrungspl"ane bei Anfrage mit 15 Joins - $15!$(Reihenfolge)$\cdot3^{15}$ (versch. Planoperatoren) $\cdot 2^{15}$ (Index o. Tablescan)
						\end{itemize}
					\item
						Vervollst"andigen der Ausf"uhrungspl"ane mit Metadaten (Sortierreihenfolge, Zugriffspfadmerkmale, statistische Informationen)
					\item
						Billigsten Ausf"uhrungsplan gem"a"s Kostenmodell ausw"ahlen
				\end{itemize}
			\item
				Unterschiedliche Strategieklassen: 
				\begin{itemize}
					\item
						voll-enumerativ
					\item
						beschr"ankt-enumerativ
					\item
						zufallsgesteuert (genetische Algorithmen und Strategien des \enquote{simulated annealing}
				\end{itemize}
			\item
				Beispiel \href{site:IDB-2015WS-11-RelOperatore.pdf}{11/29 ff.}
		\end{itemize}
	\item
		Berechnung der Kosten f"ur jeden Ausf"uhrungsplan
		\begin{itemize}
			\item
				Kostenformel \#physische-Seitenzugriffe + W $\times$ \#Aufrufe-des-Zugriffssystems (W ist Verh"altnis f"ur Aufruf des Zugriffssystems (interne Satzschnittstelle) zum Aufwand f"ur Seitenzugriff)
		\end{itemize}
	\item
		Wichtige Statistische Kenngr"o"sen:
		\begin{itemize}
			\item
				\dots f"ur Segmente: $M_S$ (Anzahl der Datenseiten des Segments S, $L_S$(Anzahl der leeren Seiten in S)
			\item
				\dots f"ur Relationen: $N_R$ (Anzahl der Tupel in Relation R, $T_{R,S}$ Anzahl der Seiten in S mit Tupeln von R, $C_R$ (Cluster-Faktor, Anzahl der Tupel von R pro Seite)
			\item
				\dots pro Index I auf Attributen A einer Relation R: $j_I$ Anzahl der Attributwerte(Schl"usselwerte) im Index), $B_I$ Anzahl der Blattseiten (beim B*-Baum)
		\end{itemize}
	\item
		Statistische Kenngr"o"sen m"ussten eigentlich gesammelt werden, so viele zus"atzliche Log-Operationen aber aufwendig, stattdessen: Periodische Neubestimmung durch Kommando (\lstinline$analyse table$ bei Oracle)

	\item
		Heuristik zur Absch"atzung des Selektivit"atsfaktors auf \href{site:IDB-2015WS-11-RelOperatore.pdf}{11/36 f.}
	\item
		Grenztrefferrate (ab wo sich ein Indexscan lohnt): 5\% (geringe Trefferrate $\overset{\wedge{}}=$ hoher Selektivit"at)
\end{itemize}


\section{Synchronisation}
\begin{itemize}
	\item
		Arten von Konsistenz:
		\begin{itemize}
			\item
				Datenbankkonistenz: Alle (auf der DB definierten) Konsistenzbedingungen sind erf"ullt
			\item
				Transaktionskonsisten/(operationeelle Ingetrit"at): Der nebenl"aufige Ablauf der Transaktionen ist korrekt
		\end{itemize}
	\item
		Remember: ACID (atomicity, consistency, isolation, durability)
	\item
		L"osung: Serialisierung, Einbenutzerbetrieb. Dann aber m"oglicherweise lange Wartezeiten
	\item
		Stattdessen \enquote{virtuelle} serielle Ausf"uhrung (\enquote{logischer} Einbenutzerbetrieb)
	\item
		A posteriori(Serialisierbarkeitstheorie)/A priori(Sperrverfahren)
	\item
		Serialisierbarer Ablauf: Ein Schedule von TAs ist serialisierbar, wenn er zu irgendeinem seriellen Ablauf der in ihm enthaltenen Transaktionen "aquivalent ist.
	\item
		"Aquivalenz von Abl"aufen: \dots what
\end{itemize}
\subsection{Implementierungsmethoden}
\begin{itemize}
	\item
		Einf"uhren von Locks f"ur Zugriffe auf Datenobjekte f"ur jedes benutzte Datenobjekt zentral in einer Sperrtabelle
	\item Arten von Sperren:
		\begin{itemize}
			\item
				X (exclusive)
			\item
				S (shared)
		\end{itemize}

	\item Wann holt man sich locks?
		\begin{itemize}
			\item
				Statisch: alles zu Beginn holen, man muss alles sperren was man brauchen k"onnte
			\item
				Dynamisch: locks nach Bedarf ziehen, m"oglicherweise deadlocks
		\end{itemize}
	\item
		Locks erst am Ende der TA freigeben
	\item
		Sperrgranularit"at:
		\begin{itemize}
			\item
				Tupel: nicht immer effizient, manche TAs alle Tupel einer Relation, gro"se Sperrtabellen, Phantomproblem (werden nur existierende Tupel gesperrt, andere TA kann Tupel einf"ugen)
			\item
				Hierarchisch:
				\begin{itemize}
					\item Synchronisation langer TAen auf Relationenebene
					\item
						Synchronisation kurzer TAen auf Tupelebene
				\end{itemize}

			\item
				Anwartschaftssperren
				\begin{itemize}
					\item
						Auf h"oherer Ebene werden auch noch \enquote{intention locks} gezogen, um anzuzeigen, dass auf tieferer Ebene locks gezogen sind (Intention-share lock IS/Intention exclusive lock IX). Muss auf allen Vorg"angerknoten im Hierachiebaum passieren, ist dort schon richtiges Lock gezogen kann kein Intention Lock mehr gezogen werden
					\item
						SIX (share and intention exclusive): Sperrt Objekt im S-Modus, auf Ebenen darunter nur noch IX oder X locks. Macht Sinn wenn alle Tupel einer Relation gelesen und nur einige davon ge"andert werden (X auf Relation w"are zu restriktiv, IX auf Relation m"usste jedes Tupel noch extra S locked werden. Passiert hier implizit
					\item
						Kompabilit"atsmatrix siehe \href{site:IDB-2015Ws-12-Synchronisation.pdf}{12/27}
				\end{itemize}
		\end{itemize}
	\item
		Probleme bei Locks:
		\begin{itemize}
			\item
				M"oglicherweise lange Wartezeiten (Hotspots bei wichigen Zugriffspfaden o."a.), m"oglicherweise gro"se Sperrtabellen. 
		\end{itemize}
	\item
		Deadlocks bl"od\\
		M"ogliche L"osungen:
		\begin{itemize}
			\item
				Timeout (Bestimmung des richtigen Werts problematisch)
			\item  Prevention (nicht praktikabel f"ur DBMS)
			\item  Avoidance (Potenzielle Deadlocks im vorraus erkennen und vermeiden - Laufzeit unterst"utzung notwendig
			\item  Detection (Explizites F"uhren eines wait-for graphen
				darin zyklensuche
		\end{itemize}
\end{itemize}


\section{Recovery}
\subsection{Protokollierung und Wiederherstellung allgemein}
\begin{itemize}
	\item
		Sammeln redundanter Infos w"ahrend normalen Betriebs: Logging
	\item
		Mechanismen zur Wiederherstellung des letzten konsistenten Zustands: Recovery
	\item
		Physische Konsistenz: Korrektheit der Speicherungsstrukturen, "Anderungsoperationen sind vollst"andig ausgef"uhrt
	\item
		Logische Konsistenz: Daten sind in sich korrekt. Vollst"andig ausgef"uhrte Transaktionen erhalten die logische Konsistenz
	\item
		Logische Konsistenz setzt physische Konsistenz vorraus
	\item
		Fehlerarten:
		\begin{itemize}
			\item
				Transaktionsfehler (Transaktion aus irgendeinem Grund nicht durchf"uhrbar)
			\item
				Systemfehler: Verlust der Hauptspeicherinhalte/Puffer
			\item
				Ger"atefehler (insb. Medienfehler): Festplatte kaputt!
		\end{itemize}

	\item
		Recovery-Klassen:
		\begin{itemize}
			\item
				Partial Undo (R1-Recovery): Zur"ucksetzen in Zustand vor Transaktion, beinflusst andere TAs nicht
			\item
				Partial Redo (R2-Recovery): Nach Verlust des Hauptspeichers, wiederholen aller verlorengegangenen "Anderungen von abgeschlossenen TAs
			\item
				Global Undo (R3-Recovery): Nach Verlust des Hauptspeichers, zur"ucksetzen aller durch den Ausfall unterbrochenen Transaktionen
			\item
				Global Redo (R4-Recovery): Nach Ger"atefehler, Backup einspielen und alle TAs seit dem Backup nachvollziehen/wiederholen
		\end{itemize}
\end{itemize}


\subsection{Einbringstrategien}
\begin{itemize}
	\item
		Einbringen: G"ultigmachen von Datenobjekten in der Datenbank, so dass sie auch nach Fehlern benutzt werden k"onnen
	\item Einbringstrategien (wann?):
		\begin{itemize}
			\item
				Steal: Bei Verdr"angung aus dem Puffer auch schon vor Ende der TA
			\item
				NoSteal: Fr"uhestens am Ende der erfolgreichen TA, dann kein Undo erforderlich, daf"ur gro"ser Puffer
			\item
				NoForce: Erst bei Verdr"angung aus dem Puffer, i.All. deutlich nach dem Ende der TA
			\item
				Force: Sp"atestens am Ende der erfolgreichen TA, kein Partial Redo erforderlich
		\end{itemize}
	\item
		Einbringstrategien (wie?):
		\begin{itemize}
			\item
				NotAtomic: Direktes Einbringen (update in place), geht nicht atomar
			\item
				Atomar: Indirektes Einbringen, dann einfach umschalten
		\end{itemize}
\end{itemize}


\subsection{Protokolldaten}
\begin{itemize}
	\item
		Protokollinformationen sind nur, was zus"atzlich zur Einbringstrategie ben"otigt wird um nach Systemausfall wieder konsistenten Zustand herzustellen
	\item
		Wann wird in die Logdatei geschrieben?
		\begin{itemize}
			\item
				Undo Informationen: m"ussen geschrieben sein, bevor "Anderungen eingebracht werden - \enquote{Write ahead Log}/WAL
			\item
				Redo Informationen: Muss geschrieben sein bevor der Abschluss der TA an Programm/Benutzer gemeldet wird
		\end{itemize}

	\item
		Physische Protokollierung
		\begin{itemize}
			\item
				Zustandsprotokollierung: Before-Image/BI f"ur Undo, After-Image(AI) f"ur Redo. Einheiten: Seiten oder S"atze/Eintr"age
			\item
				Seitenprotokollierung: vollst"andige Kopie von ganzer Seite in den Log geschrieben (vor und nach "Anderung)
			\item
				Eintragsprotokollierung: nur der tats"achlich ge"anderten Teile einer Seite, Sammlung und Pufferung im Hauptspeicher. Weniger EA-Aufwand, aber komplexere Recovery (Eintrag kann nicht einfach zur"uckkopiert werden, Speicher k"onnte mittlerweile anders genutzt werden  $\rightarrow$ Zur"uckspeichern "ahnlich wie Neueinf"ugen

		\end{itemize}

\end{itemize}
\subsection{Sicherungspunkte}
\begin{itemize}
	\item
		Problem bei Redo: Nach Systemfehler m"ussten alle "Anderungen seit Start des DBS wiederholt werden
	\item
		Direkte Sicherungspunkte: Ausschreiben u. einbringen aller ge"anderten Seiten in die Datenbank
	\item
		Indirekte Sicherungspunkte (\enquote{fuzzy checkpoints}) nicht behandelt
	\item
		Transaction Oriented Checkpoint (TOC): ge"anderten Seiten einer TA nach TA-Ende sofort in DB einbringen, keine Redo-Recovery notwendig aber hohe Belastung im Normalbetrieb
	\item
		Transaction Consistent Checkpoint (TCC): Einbringen aller "Anderungen erfolgreicher TAs, daf"ur Lesesperre auf ganzer DB
	\item
		Action-Consistent Checkpoint (ACC): Zum Zeitpunkt des Checkpoints keine "Anderungsoperationen aktiv, Checkpoint begrenzt nur Redo-Recovery
\end{itemize}

\subsection{Recovery nach Systemfehler}
\begin{itemize}
	\item
		bei direkter Seitenzuordnung:
		\begin{itemize}
			\item
				materialisierte DB chaotisch
			\item
				Block der materialisierten DB ist: aktuell (\checkmark), veraltet ( $\rightarrow$ Redo) oder dirty ( $\rightarrow$ Undo)
		\end{itemize}
	\item
		bei indirekter Seitenzuordnung:
		\begin{itemize}
			\item
				Datenbank ist mindestens physisch konsistent, entspricht zustand des letzten erfolgreichen Einbringens
		\end{itemize}
	\item
		Allgemeine Restart-Prozedur:
		\begin{enumerate}
			\item
				Analyse-Lauf vom letzten Checkpoint bis zum Log-Ende, bestimmen von Gewinner und Verlier TAs
			\item
				R"ucksetzen der Verlierer TAs durch R"uckw"artslesen des Logs
			\item
				Redo Lauf: Vorw"artslesen des Logs (Startpunkt abh"angig vom Checkpoint-Typ), "Anderungen der Gewinner-TAs ggf wiederholen
		\end{enumerate}
\end{itemize}

\subsection{Ger"ate-Recovery}
\begin{itemize}
	\item
		Meist Plattenfehler, versuchen die zu reduzieren (z.B. RAID)
	\item
		Full vs Incremental Backup
\end{itemize}

\section{Programmzugriff}
\subsection{In Java}
\begin{itemize}
	\item
		Zuerst Treiber laden: \lstinline$Class.forName("oracle.jdbc.driver.OracleDriver")$
	\item
		Verbindungsaufbau:\\ \lstinline$Connection con = DriverManager.getConnection(url, "name", "pw")$
	\item
		SQL absetzen: \lstinline$Statement stat = con.createStatement();$\\
		\lstinline$String query = "SELECT Vorname, Nachname FROM Person WHERE Geburtsjahr < 1970";$\\
		\lstinline$ResultSet resSet = stat.executeQuery(query);$
	\item
		\begin{lstlisting}
while(resSet.next()){
	System.out.println(ergebnis.getString(1) + " " + ergebnis.getString(2));
}
		\end{lstlisting}
\end{itemize}







\newpage
\section{Klausur WS 13/14}
\setcounter{section}{0}
\section{Wissensfragen}
\subsection{Definition}
Wenn persistente Speicherung und keine Details der Speicherung (Kemper/Eicker Datenbanksysteme sagt: \enquote{wenn wohldefinierte Schnittstelle darunterliegende Implementierung verdeckt})
\subsection{Richtig oder falsch}
\begin{itemize}
	\item
		Datenbank ist nur f"ur Ablage von Daten da, nicht zum Zugriff:\\
		In Datenbank werden Daten nur abgelegt, f"ur Zugriff darauf ist DBMS zust"andig - das wiederum auf die Daten aus der DB zugreift
	\item
		TID gleich solange Satz existiert:\\
		Jop, falls Satz verschoben werden muss wird Indirektion eingerichtet
	\item
		Seite im Puffer muss selbe Gr"o"se haben wie Seite im BS:\\
		Seite genauso gro"s wie Block im Puffer  $\rightarrow$ Muss nicht, w"are aber sinnvoll (Buffer-Frame = Pageframe im BS?)
	\item
		Mit TID h"ochstens eine Indirektion beim Zugriff auf nicht-fragmentierte Konzepte:\\
		(Sofern direkt "uber TID und nicht "uber Index zugegriffen wird:) Ja, falls Satz ein zweites mal verschoben werden muss, wird einfach erste Indirektion umgebogen
	\item
		DBS kann Blockgr"o"se frei w"ahlen:\\
		Nein, wird von Hardware oder Dateisystem vorgegeben
	\item
		Hashverbund nur wenn ein Hashindex existiert:\\
		Dann zwar besonders sinnvoll, aber Hashindex kann auch \enquote{on the fly} erzeugt werden
	\item
		Prim"arorganisation einer Relation ist nach dem Prim"arschl"ussel organisiert:\\
		Nein, wenn Prim"arorganisation kein B$(^*)$-Baum ist, ist die gar nicht sortiert
\end{itemize}
\subsection{Schichtenmodell}
\subsection{Puffer}
\subsubsection{Indirekte Seitenzuordnung}
\subsubsection{Verdrängung}
\subsection{Recovery}
\subsubsection{Konsistenz}
\subsubsection{Durchf"uhrung}
\begin{enumerate}
	\item
		Analyse der Logdatei: Identifizierung der \emph{Winner} und \emph{Loser}
	\item
		Wiederholung der Historie: (Redo) Alles, was ausgef"uhrt wurde vor absturz, aber noch im Hauptspeicher war wird wiederholt
	\item
		Undo der Loser: Die Loser (die, TAs, die zum Zeitpunkt des Absturzes noch nicht fertig waren) werden in umgekehrter Reihenfolge r"uckg"angig gemacht
\end{enumerate}

\section{Letzte "Ubung}
\subsection{Fragen zur Vorlesung}

\begin{enumerate}
	\item
		\begin{align*}
			r_1(x) \quad w_2(x) \quad r_1(x) &\Rightarrow Non-Repeatable\, Read\\
			r_1(P) \quad w_2(x\, in\, P) \quad r_1(P) &\Rightarrow Phantom\\
			w_1(x) \quad r_2(x) \quad a_1 \quad c_2 &\Rightarrow Dirty\, Read\\
			w_1(x) \quad r_2(x) \quad w_2(x) \quad a_1 \quad c_2 &\Rightarrow Dirty\, Write\\
			r_1(x) \quad r_2(x) \quad w_1(x) \quad w_2(x) &\Rightarrow Lost\, Update
		\end{align*}

	\item
		\begin{description}
			\item[Serialisierbarkeit] Ein Schedule von TAs ist dann serialisierbar, wenn er zu irgendeinem seriellen Ablauf äquivalent ist
			\item[Äquivalenz]Zwei Abläufe sind äquivalent: (Durch vertauschen von zwei benachbarten Operationen, die nicht in Konflikt stehen, kann man einen äquivalenten Ablauf erzeugen
		\end{description}
	\item
		\begin{itemize}
			\item
			 Komplett serialisieren
		 \item
			 Kritische Datenobjekte sperren(Zwei M"oglichkeiten: statisches speren = alle Locks am Anfang der TA holen, die man evt. brauchen \emph{k"onnte} u. dynamisches Sperren Readlocks und Writelocks bei Bedarf)
		\end{itemize}
\end{enumerate}


\subsection{Anomalien}
\begin{itemize}
	\item
		Non-repeatable read T1 Z.15
	\item
		Dirty-Read u. Dirty-Write T1 Z.13 u. Z.19
	\item
		Lost Update T1 Z.24 (mit T4)
\end{itemize}

\subsection{Serialisierbarkeit}
\begin{enumerate}
	\item
		Abhängigkeitsgraph: Pfeil von der Transaktion, die vor anderer Transaktoin ausgefüht werden muss
		\begin{align*}
			T_2  &\rightarrow_B T_3 \text{ T2 muss vor T3 ausgef"uhrt werden}\\
			T_3 &\rightarrow_A T_1 \\
			T_1 &\rightarrow_A T_3
		\end{align*}
		Bei Zyklus im Abhängigkeitsgraph ist nicht serialisierbar, mit sinnvollem dynamischem Locking trotzdem möglich
	\item
		\begin{align*}
			T_3 \rightarrow_A T_1\\
			T_3 \rightarrow_A T_2\\
			T_2 \rightarrow T_1\\
		\end{align*}

\end{enumerate}
	

\subsection{Sperren}
\begin{tabular}[H]{|c|c|c|c|}
	A & & B &  \\
	\hline
	S & X & S & X \\
	\hline
	$T_1, T_3$ & & $T_2$ & $T_2$
\end{tabular}
$T_3$ blockiert bei $r_3(B)$, $T_1$ blockiert  bei $w_1(A)$\newline

$\Rightarrow$ \begin{tabular}[H]{|c|c|c|c|}
	A & & B &  \\
	\hline
	S & X & S & X \\
	\hline
	$T_1, T_3$ & & $$ & $$
\end{tabular} \newline

$\Rightarrow$ \begin{tabular}[H]{|c|c|c|c|}
	A & & B &  \\
	\hline
	S & X & S & X \\
	\hline
	$T_1, T_3$ & & $T_3$ & $$
\end{tabular} $T_3$ blockiert in $W_3(A) \Rightarrow$ deadlock

Zusatzfrage: Nein


%\section{Notizen f"ur Fragestunde}
\begin{itemize}
	\item
		Definiere \enquote{Block}: Gibt Festplattenbl"ocke, Dateisystembl"ocke...
	\item
		Umbenennung (wie in letzter Braindump 6.1) in Anfragegraph?
\end{itemize}

\end{document}
