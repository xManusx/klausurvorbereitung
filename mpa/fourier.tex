\documentclass{scrartcl}

\usepackage{hyperref}
\usepackage{amssymb}
\usepackage{mathtools}
\usepackage{gensymb}

\usepackage[a4paper, left=1cm, right=1cm, top=2cm, bottom=3cm]{geometry}

\DeclarePairedDelimiter\abs{\lvert}{\rvert}%
\DeclarePairedDelimiter\norm{\lVert}{\rVert}%
\DeclarePairedDelimiter\inner{\langle}{\rangle}%
\newcommand{\ffrac}[2]{\ensuremath{\frac{\displaystyle #1}{\displaystyle #2}}}

\begin{document}
\section{Fourier Transform in a Nutshell}
Key Idea: Compare signal with sinusoids of various frequencies and phases. 

\subsection{Continuous Fourier Transform}
\subsubsection*{Analog Signals}
\begin{itemize}
    \item
        Time as well as amplitude are continuous, real-valued parameters
    \item
        Can be modeled as a function $f: \mathbb{R} \rightarrow \mathbb{R}$, which assigns to each time point $t \in \mathbb{R}$ an amplitude value $f(t) \in \mathbb{R}$
\end{itemize}
\subsubsection*{Sinussoidal signal}
\begin{itemize}
    \item
        Function $g: \mathbb{R} \rightarrow \mathbb{R}$ defined by $g(t) = A sin(2\pi(\omega t - \varphi))$
    \item
        $A$ corresponds to the \textbf{amplitude}, $\omega$ to the \textbf{frequency} and $\varphi$ to the \textbf{phase}
    %\item
        %frequency is measured in Hz, phase in normalized radians (1 corresonds to an angle of 360\degree)
    \item
        In Fourier analysis: prototype oscillations that are normalized with regard to their power by setting $A = \sqrt2$
    \item
        Thus for each frequency parameter $\omega$ and phase parameter $\varphi$ we obtain a sinusoid 
        $$cos_{\omega, \varphi}(t) = \sqrt2 cos(2\pi(\omega t - \varphi))$$
    \item
        Phase parameter only has to be considered for $\varphi \in [0, 1)$
\end{itemize}
\subsubsection*{Computing Similarity with Integrals}
\begin{itemize}
    \item
        $\int_{t \in \mathbb{R}} f(t)g(t) dt$
\end{itemize}
\subsubsection*{First Definition of the Fourier Transform}
\begin{itemize}
    \item
        For a fixed frequency $\omega \in \mathbb{R}$, we define
        $$d_\omega = max_{\varphi \in [0,1)} (\int_{t \in \mathbb{R}} f(t) cos_{\omega, \varphi}(t) dt)$$
        $$\varphi_\omega = argmax_{\varphi \in [0,1)} (\int_{t \in \mathbb{R}} f(t) cos_{\omega, \varphi}(t) dt)$$
    \item
        \textbf{Fourier Transform:} "collection" of all coefficients $d_\omega$ and $\varphi_\omega$ for $\omega \in \mathbb{R}$
\end{itemize}
\subsubsection*{Complex Numbers}
\begin{itemize}
    \item
        Extends real numbers by introducing the imaginary number $i = \sqrt{-1}$ with $i^2 = -1$
    \item
        Gets written as $c = a + ib$ 
    \item
        Set of complex numbers $\mathbb{C}$ can be thought of as a two-dimensinal plane, where horizontal dimension corresponds to real part and vertical dimension to imaginary part (so $c = a+ib$ has Cartesian coordinates $(a,b)$)
    \item
        \textbf{Polar coordinates:} Complex number c is described by absolute value $\abs{c}$ (distance from origin) and angle $\gamma$ between the positive horizontal axis and the line from the origin to $c$
    \item
        Deriving polar coordinates:
        $$ \abs{x} = \sqrt{a^2 + b^2} $$
        $$ \gamma = atan2(b, a) $$
    \item
        Regaining complex number:
        $$ exp(i\gamma) = cos(\gamma) + isin(\gamma) $$
        $$ c = \abs{c} exp(i\gamma)$$
\end{itemize}
\subsubsection*{Complex Definition of the Fourier Transform}
\begin{itemize}
    \item
        Idea: encode $d_\omega$ and $\varphi_\omega$ by a single complex number
    \item
        $ c_\omega = \ffrac{d_\omega}{\sqrt{2}} exp(2\pi i(-\varphi_\omega))$
    \item
        $\hat{f}(\omega) = c_\omega$
    \item     
        $\hat{f}$ is referred to as the Fourier transform of f, $\hat{f}(\omega) = c_\omega$ are called the Fourier coefficients
    \item
        Fourier transform can be computed by

        \begin{align*}
            \hat{f}(\omega) = & \int_{t \in \mathbb{R}} f(t) exp(-2\pi i \omega t) dt\\
            & \int_{t \in \mathbb{R}} f(t) cos(-2\pi \omega t) dt + i \int_{t \in \mathbb{R}} f(t) sin(-2 \pi \omega t) dt
        \end{align*}
    \item
        Real part is obtained by comparing $f$ with a cosine function, the imaginary part by comparing $f$ with a sine function
    \item
        $\abs{\hat{f}(\omega)}$ is called the \textbf{magnitude Fourier transform}
    \item
        With $\abs{\hat{f}(\omega)}$ and $\gamma_\omega$ being the polar coordinates of $\hat{f}(\omega)$:
        $$d_\omega = \sqrt{2} \abs{\hat{f}(\omega)}$$
        $$\varphi_\omega = -\ffrac{\gamma_\omega}{2\pi}$$
\end{itemize}
\subsubsection*{Fourier Representation}
\begin{itemize}
    \item
        Signal reconstruction:
        \begin{align*}
            f(t) & = \int_{\omega \in \mathbb{R}_{\geq 0}} d_\omega \sqrt{2} cos(2 \pi (\omega t - \varphi_\omega))d\omega\\
            & = \int_{\omega \in \mathbb{R}} c_\omega exp(2\pi i \omega t)d \omega
        \end{align*}
        
\end{itemize}
\subsection{Discrete Fourier Transform}
\subsubsection*{Equidistant sampling}
\begin{itemize}
    \item
        Digitization: Converting analog signals into finite representations (analog-to-digital conversion) 
    \item
        Given an analog signal $f: \mathbb{R} \rightarrow \mathbb{R}$ and a positive real number $T > 0$, one defines a function $x: \mathbb{Z} \rightarrow \mathbb{Z}$ by setting $x(n) = f(nT)$
    \item
        Since $x$ is only defined on a discrete set of time points, it is also refered to as a \textbf{discrete-time} (DT) signal
    \item
        $x(n)$ is called a sample, taken at time $t = nT$
    \item
        $F_s = 1/T$ is called sampling rate
    \item
        \textbf{Sampling theorem:} Original signal $f$ can be reconstructed perfectly from its sampled version $x$, if $f$ does not contain any frequencies higher than
        $$ \Omega = F_s/2 = 1/(2T) Hz $$
    \item
        We say $f$ is an \textbf{$\Omega$-bandlimited} signal, $\Omega$ is known as the \textbf{Nyquist frequenct}
    \item
        In the case $f$ contains higher frequencies, sampling may cause artifacts referred to as \textbf{aliasing}
\end{itemize}
\subsubsection*{Discrete Fourier Transform}
\begin{itemize}
    \item
        Sample the sinusoudal prototype oscillation in the same fashion as the signal and multiply the two sampled functions in a pointwise fashion (sampled product)
    \item
        Integration in the continuos case becomes summation in the discrete case, where the summands need to be weighted by the sampling period $T$
        $$\sum_{n \in \mathbb{Z}} T f(nT) exp(-2\pi i \omega n T) \approx \hat{f}(\omega)$$
    \item
        "Approximation of integral area via sum of rectangular shapes" (\textbf{Riemann sum})
    \item
        Substituting $T = 1$ and $f(nT) = x(n)$ yields
        $$ \hat{x}(\omega) = \sum_{n \in \mathbb{Z}} x(n) exp(-2\pi i \omega n) $$
    \item
        To recover relation to Fourier transform $\hat{f}$, one needs to know the sampling period $T$
        $$ \hat{x}(\omega) \approx \ffrac{1}{T} \hat{f}(\ffrac{\omega}{T})$$
    \item
        $\omega = 1/2$ for $\hat{x}$ corresponds to the Nyquist frequency $\Omega = 1/(2T)$ of the sampling process
    \item
        Two \textbf{problems} for computation: 
        \begin{enumerate}
            \item
                Infinite number of summands ($\sum_{n \in \mathbb{Z}}$)
            \item
                $\omega$ is a continuos parameter
        \end{enumerate}
    \item
        \textbf{Solutions:}
        \begin{enumerate}
            \item
                Assumption: most relevant information of $f$ is limited to a certain duration in time (e.g. song only lasts a few minutes). This means that we only need to consider a finite amount of samples $x(0),\dots,x(N-1)$ and the sum becomes finite.
            \item
                One computes the Fourier transform only for a finite number of frequencies ("sampling frequency axis" by considering frequencies $\omega = k/M$ for some suitable $M \in \mathbb{N}$ and $k \in [0: M-1]$). Often $N=M$ gets choosen.
        \end{enumerate}
    \item
        \textbf{Discrete Fourier Transform:}
        $$ X(k) = \hat{x}(k/N) = \sum_{n=0}^{N-1} x(n) exp(-2\pi i k n/N)$$
        defined for integers $k \in [0: M-1] = [0 : N-1]$
    \item
        The frequency $\omega$ of $\hat{x}$ corresponds to $\omega/T$ of $\hat{f}$, so $F_{coef}(k) = \ffrac{k}{N \cdot T} = \ffrac{k \cdot F_s}{N}$
    \item
        Upper half of coefficients are redundant, one only needs to consider the coefficients $X(k)$ for $k \in [0: \lfloor N/2 \rfloor]$
    \item
        To compute all coefficients, one needs $N^2$ operations
    \item
        \textbf{Fast Fourier Transform}: Recursive computation, works particularly well in the case of N is a power of two, $Nlog_2N$ operations
\end{itemize}

\subsection{Short-Time Fourier Transform}
\begin{itemize}
    \item
        Instead of considering the entire signal, the main idea of the STFT is to consider only a small section of the signal 
    \item
        One fixes a so-called \textbf{window function}, which is a function that is nonzero for only a short period of time
    \item
        The original signal is multiplied with the window funtion to yield a \textbf{windowed signal}
    \item
        To obtain frequency information at different time instances, one shifts the window function across time and computes a FT for each of the resulting windows
    \item
        STFT reflects not only the properties of original signal but also those of the window function (e.g. rectangular windows typically introduce "ripple" artifacts)
    \item
        Parameters:
        \begin{itemize}
            \item
                Real-valued DT signal $x: \mathbb{Z} \rightarrow \mathbb{R}$ obtained with sampling rate $F_s$
            \item
                Sampled window function $w: [0:N-1] \rightarrow \mathbb{R}$ of length $N \in \mathbb{N}$
            \item
                Hop size $H \in \mathbb{N}$ specified in samples and determining the step size in which the window is to be shifted across the signal (most of the time $H = N/2$)
        \end{itemize}
    \item
        \textbf{Discrete STFT $X$:}
        $$X(m,k) = \sum_{n=0}^{N-1} x(n+mH)w(n)exp(-2\pi i k n/N)$$
        with $m \in \mathbb{Z}$ and $k \in [0:K]$
    \item
        The number $K = N/2$ is the frequency index corresponding to the Nyquist frequency
    \item
        $X(m,k)$ denotes the $k^{th}$ Fourier coefficient for the $m^{th}$ time frame
    \item
        For each fixed time frame $m$, one obtains a spectral vector of size $K+1$. The computation of each such spectral vector amounts to a DFT of size $N$, which can be done efficiently using the FFT
    \item
        $T_{coef}(m) = \ffrac{m \cdot H}{F_s}$ (in seconds) and $F_{coef}(k) = \ffrac{k \cdot F_s}{N}$ (in Hertz)
    \item
        \textbf{Spectrogram}: two-dimensional representation of the squared magnitude of the STFT
        $$Y(m,k) = \abs{X(m,k)}^2$$
\end{itemize}
\end{document}
