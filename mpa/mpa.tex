\documentclass[fleqn,12pt]{scrartcl}
\usepackage{makeidx}
\usepackage[utf8]{inputenc}
\usepackage{color}
\usepackage[ngerman]{babel}
\usepackage{csquotes}
\usepackage{amssymb}
\usepackage{amsthm}
\usepackage{amsmath}
\usepackage{gauss}
\usepackage{braket}
\usepackage{hyperref}
\usepackage{wasysym}
\usepackage{scrpage2}
\usepackage{tikz}
\usetikzlibrary{intersections}
%\pagestyle{scrheadings}
%\clearscrheadfoot
%\ohead{\pagemark}
%\ihead{Magnus Berendes, 21752155}
%\ifoot{\today}
%\ofoot{\blattn}
%\setheadtopline{1pt}
%\setheadsepline{0.4pt}
%\setfootsepline{0.4pt}
\usepackage{enumitem}
%\setenumerate[0]{label=\alph*)}
\newcommand{\id}{\, \mathrm{d}}
\newcommand{\intl}{\int\displaylimits}
% New definition of square root:
% it renames \sqrt as \oldsqrt
\let\oldsqrt\sqrt
% it defines the new \sqrt in terms of the old one
\def\sqrt{\mathpalette\DHLhksqrt}
\def\DHLhksqrt#1#2{%
	\setbox0=\hbox{$#1\oldsqrt{#2\,}$}\dimen0=\ht0
	\advance\dimen0-0.2\ht0
	\setbox2=\hbox{\vrule height\ht0 depth -\dimen0}%
{\box0\lower0.4pt\box2}}

\newcommand{\karos}[2]{
	\begin{tikzpicture}
		\draw[step=0.5cm, color=gray] (0,0) grid (#1 cm , #2 cm);
	\end{tikzpicture}
}
\newcommand{\abs}[1]{
	\left \vert #1 \right \vert
}
\newcommand{\absbb}[1]{
	\left \Vert #1 \right \Vert
}

%TODO
\newcommand{\blattn}{Blatt 5}
\makeindex
\begin{document}
Kein Anspruch auf Vollst"andigkeit oder Korrektheit und so!

\section{DTW}
\subsection{Basic approach}

\begin{itemize}
	\item
		Use cosine distance as cost measurement:
		\begin{equation}
			c(x,y) := 1 - \frac{\langle x | y \rangle}{\absbb{x} \cdot \absbb{y}}
		\end{equation}
	\item
		Warping path: Sequence of tuples
		\begin{itemize}
			\item
				Step size  $\in \Set{(1,1), (1,0), (0,1)}$
			\item
				Starting point $(1,1)$
			\item
				End point $(N,M)$
		\end{itemize}
	\item
		Cost of a warping path ($c_p(X,Y)$) is cost accumulated along the way
	\item
		DTW-Distance (not a real distance!) of two sequences $X$ and $Y$ $DTW(X,Y)$: minimal warping path between them
	\item
		Accumulated cost matrix $D$: $D(N,M) = DTW(N,M)$ can be recursively computed:
		\begin{align}
			&D(n,1) = \sum_{k=1}^n C(k,1) \text{ for } n \in [1 : N]\\
			&D(1,m) = \sum_{k=1}^m C(1,k) \text{ for } m \in [1 : M]\\
			&D(n,m) = C(n,m) + \min \begin{cases}
				D(n-1,m-1)\\
				D(n-1,m)\\
				D(n,m-1)
			\end{cases}
		\end{align}
	\item
		Optimal warping path can be recovered from the accumulated cost matrix via backtracking, starting at $q_1 = (N,M)$
\end{itemize}

\subsection{Variants}
\begin{itemize}
	\item
		New Stepsizes:
		\begin{itemize}
			\item
				$\Sigma = \Set{(2,1),(1,2),(1,1)}$: prohibits horizontal or vertical parts in the warping path, might lead to cells being omitted from either $X$ or $Y$
			\item
				Compliacted warping path only allowing at most 2 vertical or horizontal parts after another
		\end{itemize}
	\item
		Local weights:
		\begin{itemize}
			\item
				Introduce weights for horizontal ($w_h$), vertical ($w_v$) or diagonal ($w_d$) parts
			\item
				Often $w_h = w_d = 1$ and $w_d = 2$, trying to inhibit diagonals since one diagonal step (cost of one cell) corresponds to the combination of one horizontal and one vertical step (cost of two cells)
		\end{itemize}
	\item
		Global constraints: constrain the warping path to a global constraint region
		\begin{itemize}
			\item
				Two famous constraint regions: Sakoe-Chiba band (diagonal) and Itakura parallelogram (parallelogram between $(1,1)$ and $(N,M)$
			\item
				For computation set $C(n,m) = \inf$ if $(n,m)$ lies outside the constraint region --- speeds up computation
		\end{itemize}

	\item
		Multiscale DTW:
		\begin{itemize}
			\item
				Compute DTW at a lower resolution first, restrict later DTWs to the region where the lower resolution optimal path lies
			\item
				Multiscaling is \textit{successfull} if we find the optimal warping path this way
		\end{itemize}

	\item
		Further notes:
		\begin{itemize}
			\item
				Synchronise different recordings
			\item
				Synchronise recordings with sheet music (OMR - Convert to Piano Roll - Convert to Chroma - Synchronise)
			\item
				Create Tempo curves (OMR - Convert to Piano roll, fixed bpm - Convert to Chroma - synchronise - estimate bpm from DTW path)
		\end{itemize}

\end{itemize}

\section{Music structure analysis}
\subsection{Terminology}
\begin{itemize}
	\item Repetition based (identify recurring patterns), novelty based (transitions between contrasts), homogeneity based (consistent passages)
	\item Part (abstract) $\leftrightarrow$ segment (audio) $\leftrightarrow$
		Section (both domains)
	\item
		Strophic form (lots of verses after another), chain form (unrelated parts after another, sometimes repetitions), Rondo form (A1, B, A2, C, A4, \dots), sonata form (I E1 E2 D R C), Pop songs (Intro/Outro, Verse, Chorus, Bridge)
\end{itemize}

\subsection{Self-Similarity Matrices}
\begin{itemize}
	\item
		Self-similarity matrix S: $S(n,m) := s(x_n, x_m)$, high means high similarity, low means low similarity
	\item
		Feature space is $\mathbb{R}^D, D \in \mathbb{N}$, $s(x,y) := \abs{\langle x|y\rangle}$ (normalise features, maximal value of $s$ is 1)
	\item
		Segment is set $\alpha = [s : t]$, specified by starting point and endpoint
	\item Paths:
		\begin{itemize}
			\item
				Path over $\alpha$ of length $L$ is sequence $P = ((n_1,m_1), \dots , (n_L, m_L))$ with $m_1 = s$ and $m_L = t$ and step size is constrained (often $\Sigma = \Set{(2,1), (1,2), (1,1)}$)
			\item
				For a path $P$, associate two segements: $\pi_1(P) := [n_1 : n_L],\, \pi_2(P) := [m_1 : m_L]$
			\item
				From boundary condition: $\pi_2(P) = \alpha$, the other segment $\pi_1(P)$ is called the \textit{induced segment}
			\item
				Score of $P$: $\sigma(P) := \sum_{l=1}^L S(n_l, m_l)$
		\end{itemize}
	\item
		Blocks:
		\begin{itemize}
			\item
				Block over segment $\alpha = [s:t]$: $B = \alpha' \times \alpha \subseteq [1:N] \times [1:N]$
			\item
				$\pi_1$ and $\pi_2$ are defined similar to paths, $\alpha'$ is the induced segment
			\item
				Score $\sigma(B) = \sum_{(n,m) \in B} S(n,m)$
		\end{itemize}
	\item
		Segments are path-similar if there is a path with high score, same for blocks
	\item
		Diagnoal paths can be enhanced by running an averaging filter (or low-pass filter) of length $L$ along the diagonal:
		$$S_L(n,m) := \frac1L \sum_{l = 0}^{L-1} S(n+l,m+l)$$
	\item
		Tempo-invariant smoothing:
		\begin{itemize}
			\item Run the filter along varius directions similar to the diagonal, then take the cell-wise maximum of the resulting matrices. (Robust against tempo variations in different parts)
			\item
				If tempo difference between two segments is $\theta$ (second segment played $\theta$ times slower than the first one), resulting gradient is $(1,\theta)$
			\item
				SSM smoothed in direction $(1,\theta)$:
				$$S_{L,\theta}(n,m) := \frac1L \sum_{l = 0}^{L-1} S(n+l,m+[l\cdot \theta])$$
			\item
				Compute for a set of $\theta$s $\Theta$
			\item
				$S_{L,\Theta}(n,m) := \max_{\theta \in \Theta} S_{L,\theta}(n,m)$
			\item
				Do smoothing forward and backward to prevent fading of paths
		\end{itemize}
	\item
		Account for key changes/transpositions by cyclic shift for all the semi-tones. Combine by cell-wise maximum. Yields transposition-invariant self-similarity matrix $S^{TI}$
	\item
		Store the maximising shift indices in an additional matrix --- get $I$, the transposition index matrix
	\item
		Transposition-invariance increases noise $\Rightarrow$ compute on the basis of smoothed matrices
	\item
		Thresholding to reduce unwanted noise
		\begin{itemize}
			\item
				Simplest form is global Thresholding by parameter $\tau$ $\rightarrow$ set everything smaller to 0
			\item Continue on page 193
		\end{itemize}
\end{itemize}




\end{document}
