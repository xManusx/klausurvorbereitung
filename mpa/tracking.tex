\section{Tempo and Beat Tracking}
\begin{itemize}
	\item
		Measure, Tactum (Beat), Tatum
	\item
		Steps for energy based:
		\begin{enumerate}
			\item
				Take Waveform
			\item
				Amplitude Squaring
			\item
				Windowing (Energy envelope)
			\item
				Differentiation
			\item
				Half wave rectification (throw away bottom part, only energy increases are relevant for onsets)
			\item
				Peak picking
		\end{enumerate}
		Often only works for percussive music
	\item
		Spectral-Based:
		\begin{enumerate}
			\item
				Take Spectrogram
			\item
				Logarithmic Compression ($Y = \log(1+C\cdot \abs{X})$)
			\item
				Differentiation: First-order temporal difference
			\item
				Frame-wise Accumulation  of all positive intensity changes
			\item
				Normalization: Substraction of local average
			\item
				Peak picking
		\end{enumerate}
	\item
		Tempogram: a time-tempo representation that encodes the local tempo of a music signal over time

		STFT based:
		\begin{itemize}
			\item
				Compute STFT of novelty curve
			\item
				Convert frequency axis (given in Hertz) into tempo axis (given in BPM)
			\item
				Magnitude spectrogram indicates local tempo
		\end{itemize}
		Emphasis of tempo harmonics (integer multiples of tempo, eigth, sixteenth, \dots)

		Autocorrelation:
		\begin{enumerate}
			\item
				Compare novelty curve with time-lagged local sections of itself
			\item
				Convert lag-axis (given in seconds) into tempo axis (BPM)
			\item
				Autocorrelogram indicates local tempo
		\end{enumerate}
		Emphasis of tempo subharmonics (integer fractions, measures, sections)
\end{itemize}
