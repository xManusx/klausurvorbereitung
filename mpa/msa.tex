\section{Music structure analysis}
\subsection{Terminology}
\begin{itemize}
	\item Repetition based (identify recurring patterns), novelty based (transitions between contrasts), homogeneity based (consistent passages)
	\item Part (abstract) $\leftrightarrow$ segment (audio) $\leftrightarrow$
		Section (both domains)
	\item
		Strophic form (lots of verses after another), chain form (unrelated parts after another, sometimes repetitions), Rondo form (A1, B, A2, C, A4, \dots), sonata form (I E1 E2 D R C), Pop songs (Intro/Outro, Verse, Chorus, Bridge)
\end{itemize}

\subsection{Self-Similarity Matrices}
\begin{itemize}
	\item
		Self-similarity matrix S: $S(n,m) := s(x_n, x_m)$, high means high similarity, low means low similarity
	\item
		Feature space is $\mathbb{R}^D, D \in \mathbb{N}$, $s(x,y) := \abs{\langle x|y\rangle}$ (normalise features, maximal value of $s$ is 1)
	\item
		Segment is set $\alpha = [s : t]$, specified by starting point and endpoint
	\item Paths:
		\begin{itemize}
			\item
				Path over $\alpha$ of length $L$ is sequence $P = ((n_1,m_1), \dots , (n_L, m_L))$ with $m_1 = s$ and $m_L = t$ and step size is constrained (often $\Sigma = \Set{(2,1), (1,2), (1,1)}$)
			\item
				For a path $P$, associate two segements: $\pi_1(P) := [n_1 : n_L],\, \pi_2(P) := [m_1 : m_L]$
			\item
				From boundary condition: $\pi_2(P) = \alpha$, the other segment $\pi_1(P)$ is called the \textit{induced segment}
			\item
				Score of $P$: $\sigma(P) := \sum_{l=1}^L S(n_l, m_l)$
		\end{itemize}
	\item
		Blocks:
		\begin{itemize}
			\item
				Block over segment $\alpha = [s:t]$: $B = \alpha' \times \alpha \subseteq [1:N] \times [1:N]$
			\item
				$\pi_1$ and $\pi_2$ are defined similar to paths, $\alpha'$ is the induced segment
			\item
				Score $\sigma(B) = \sum_{(n,m) \in B} S(n,m)$
		\end{itemize}
	\item
		Segments are path-similar if there is a path with high score, same for blocks
	\item
		Diagnoal paths can be enhanced by running an averaging filter (or low-pass filter) of length $L$ along the diagonal:
		$$S_L(n,m) := \frac1L \sum_{l = 0}^{L-1} S(n+l,m+l)$$
	\item
		Tempo-invariant smoothing:
		\begin{itemize}
			\item Run the filter along varius directions similar to the diagonal, then take the cell-wise maximum of the resulting matrices. (Robust against tempo variations in different parts)
			\item
				If tempo difference between two segments is $\theta$ (second segment played $\theta$ times slower than the first one), resulting gradient is $(1,\theta)$
			\item
				SSM smoothed in direction $(1,\theta)$:
				$$S_{L,\theta}(n,m) := \frac1L \sum_{l = 0}^{L-1} S(n+l,m+[l\cdot \theta])$$
			\item
				Compute for a set of $\theta$s $\Theta$
			\item
				$S_{L,\Theta}(n,m) := \max_{\theta \in \Theta} S_{L,\theta}(n,m)$
			\item
				Do smoothing forward and backward to prevent fading of paths
		\end{itemize}
	\item
		Account for key changes/transpositions by cyclic shift for all the semi-tones. Combine by cell-wise maximum. Yields transposition-invariant self-similarity matrix $S^{TI}$
	\item
		Store the maximising shift indices in an additional matrix --- get $I$, the transposition index matrix
	\item
		Transposition-invariance increases noise $\Rightarrow$ compute on the basis of smoothed matrices
	\item
		Thresholding to reduce unwanted noise
		\begin{itemize}
			\item
				Simplest form is global Thresholding by parameter $\tau$ $\rightarrow$ set everything smaller to 0
				\item
					Or binarization (everything 0 or 1)
				\item
					Or relative thresholding
				\item
					Or local thresholding column and rowwise

		\end{itemize}
\end{itemize}

\subsection{Audio Thumbnailing}
\begin{itemize}
	\item
		Goal: determine the most prepresentative section for a given audio piece (similar to thumbnails in image viewers)
	\item
		Develop fitness measure that assigns a fitness value to each audio segment. It should capture \textit{how well} a given segment explains other related segments and should indicate \textit{how much} of the overall music recording is covered by these related segments
	\item
		Segment family: $\mathcal{A} := \Set{\alpha_1, \alpha_2, \dots , \alpha_3}$, pairwise disjoint elements
	\item
		Coverage of $\mathcal{A}$: $\gamma(\mathcal{A}) := \sum_{k=1}^K \abs{\alpha_k}$
	\item
		Path family $\mathcal{P} := \Set{P_1, P_2, \dots, P_K}$, induced segments should be disjoint 
		
		$\Set{\pi_1(P_1), \dots, \pi_1(P_K)}$ has to be a segment family
	\item
		Score of Path family: $\sigma(\mathcal{P}):= \sum_{k=1}^K \sigma(P_K)$
	\item
		Optimal path family: $\mathcal{P}^* := \argmax_{\mathcal{P}} \sigma(\mathcal{P})$
	\item
		Following computation of an optimal path family is 
		
		$\mathcal{O}(\text{length of segment} \cdot \text{length of recording})$
	\item
		Recall DTW: align two sequences $X$ and $Y$, step size condition 
		
		$\Sigma = \Set{(2,1), (1,2), (1,1)}$
	\item
		When computing path family over $\alpha$, role of $Y$ is taken over by $\alpha$
	\item
		$\alpha$ can/should be aligned to several non-overlapping subsequences of $X$
	\item
		Instead of finding cost-minimizing warping path, finde score-maximizing path family
	\item
		Only have to look at $S^\alpha$, the submatrix of $S$ only consisting of the columns in $\alpha$
	\item
		Accumulated score matrix $S$, rows are indexed by $[1:N]$, columns are indexed by $[0:M]$, $M:= \abs{\alpha}$
	\item
		$\Phi(n,m) = \Set{(n-i, m-j) | (i,j) \in \Sigma} \cap [1:N] \times [1:M]$, set of predecessors for $(n,m)$, exclude 0
	\item
		$D(n,m) = S^\alpha(n,m) + \max \set{D(i,j) | (i,j) \in \Phi(n,m)}$  accumulated score matrix
	\item
		First column of $D$ ($m= 0$) is important:

		$D(1,0) = 0$

		$D(n,0) = \max{D(n-1,0), D(n-1, M)}$ first term enables the algorithm to move upwards without cost, second term closes paths and ensure that paths do not overlap
	\item
		second column: $D(n,1) = D(n,0) + S^\alpha(n,1)$ makes it possible to start a new path at every row of the \enquote{elevator} column
	\item
		Score of an optimal path family: $\sigma(\mathcal{P}^*) = \max \Set{D(N,0), D(N,M)}$ first term allows final section $X$ to be skipped, second term ensures that in the other case the entire segment is aligned to a suffix of $X$ (whatever)
	\item
		$\mathcal{P}^*$ can be constructed from $D$ using the backtracking from DTW. Only modification: cells of $S^\alpha$ belonging to $m=0$ are to be omitted
	\item
		Fitness measure can not only be score $\sigma$: the segment that covers the entire song perfectly explains the whole song perfectly which is trivial.
	\item
		Solution: normalized score 
		$$\bar\sigma(\alpha) := \frac{\sigma(\mathcal{P}^*) - \abs{\alpha}}{\sum_{k=1}^K L_K}$$

		Subtracting $\abs{\alpha}$: think about the segment $\alpha = [1:N]$

		$L_K := \abs{P_K}$, yielding the average score of the optimal path family

		This answers how well fits the segment
	\item
		To answer how much it covers, introduce normalized coverage measure:
		$$
		\bar\gamma(\alpha) := \frac{\gamma(\mathcal{A}^* - \abs{\alpha}}{N}$$

		$\mathcal{A}^* := \Set{\pi_1(P_1), \dots, \pi_1(P_K)}$

		$\bar\gamma(\alpha)$ expresses the ratio between the union of the induced segments of $\alpha$ and the total length of the original recording (minus proportion for self-explanation)

	\item
		To combine both, introduce fitness $\phi(\alpha)$ using harmonic mean:
		$$ \phi(\alpha) := 2 \cdot \frac{\bar\sigma(\alpha) \cdot \bar\gamma(\alpha)}{\bar\sigma(\alpha) + \bar\gamma(\alpha)}$$

	\item
		$\alpha^* := \argmax_\alpha \phi(\alpha)$

	\item
		There are scape plots, just see book or slides
\end{itemize}



\subsection{Novelty based segmentation}
\begin{itemize}
	\item
		Goals:
		\begin{itemize}
			\item
				Find instances where musical/structural changes occur
			\item
				Find transition between subsequent musical parts
			\item
				Combine global and local aspects within a unifying framework
		\end{itemize}
	\item
		Main idea: use checkerboard-like kernel function to detect corner points on main diagonal of SSM
	\item
		Features to use:
		\begin{itemize}
			\item
				Enhanced SSM
			\item
				Time-lag SSM (morph SSM into parallelogram, such that diagonal is horicontal)
			\item
				Cyclic time-lag SSM (wrap bottom part of parallelogram up, such that the SSM is quadratic again
			\item
				Features are columns of CT-LSSM
		\end{itemize}
\end{itemize}

