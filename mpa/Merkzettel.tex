\documentclass{scrartcl}
\usepackage{hyperref}
\usepackage{amssymb}
\usepackage{amsmath}
\usepackage{amsthm}
\usepackage{braket}
\usepackage{mathtools}
\usepackage{textcomp}
\usepackage{gensymb}

\usepackage[a4paper, left=1cm, right=1cm, top=2cm, bottom=3cm]{geometry}

\DeclarePairedDelimiter\abs{\lvert}{\rvert}%
\DeclarePairedDelimiter\norm{\lVert}{\rVert}%
\DeclarePairedDelimiter\inner{\langle}{\rangle}%
\newcommand{\ffrac}[2]{\ensuremath{\frac{\displaystyle #1}{\displaystyle #2}}}
\DeclareMathOperator*{\argmin}{arg\,min}
\DeclareMathOperator*{\argmax}{arg\,max}

\begin{document}

\section{Fourier stuff}
\begin{itemize}
    \item
        Continous Fourier Transform:
				\marginpar{\Huge ! \hfill}
        $$\hat{f}(\omega) = \int_{t \in \mathbb{R}} f(t) e^{-2 \pi i \omega t} dt$$
    \item
        Polar representation:
        $e^{i \gamma} = \cos(\gamma) + i \sin(\gamma)$, $c = |c| \cdot e^{i \gamma}$\\
    \item
        Sampling:
        $$x(n) = f(nT)$$
        $$\Omega = \frac{F_s}{2} = \frac{1}{2} T$$ 
    \item
        Discrete Fourier:
        $$ \hat{x}(\omega) = \sum_{n \in \mathbb{Z}} T f(nT) e^{-2 \pi i \omega n T} \approx \hat{f}(\omega) $$
        Set $T=1$, assume $n \in [0:N-1]$ and sample frequency axis by fractions of $N$:
        $$ X(k) = \hat{x}(\frac{k}{N}) =  \sum_{n = 0}^{N-1} x(n) e^{-2 \pi i k n/N} \approx \hat{f}(\omega) $$
    \item 
        DFT vs. FFT: $O(N^2)$ and $O(n \, \log_2 \, n)$
    \item
        $F_{coef}(k) = \frac{k \cdot F_s}{N}$, $k=N/2$ $\rightarrow$ Niquist frequency
			\marginpar{\Huge ! \hfill}
    \item
        Discrete STFT:
        $$X(m, k) = \sum_{n=0}^{N-1} x(n+mH) w(n) e^{-2 \pi i k n/N}$$
    \item
        Spectrogram: $ Y(m,k) = |X(m,k)|^2$
    \item
        $(\Gamma_{\gamma} \cdot Y)(m, k) = \log(1 + \gamma \cdot Y(m,k))$, with $\Gamma_{\gamma}(v) = \log(1 + \gamma \cdot v)$ (bringt kleine Werte nach oben)
    \item
        Center frequency $F_{pitch}(p) = 2^{(p-69)/12} \cdot 440$
    \item
        Pitch set $P(p) = \{k: [0:K]: F_{pitch}(p-0.5) \leq F_{coef}(k) < F_{pitch}(p+0.5)\}$
    \item
        Log-frequency spectrogram: $Y_{LF}(m,p) = \sum_{k \in P(p)} |X(m,k)|^2$
    \item
        Chromagram: $C(m,c) = \sum_{\{ p \in [0:127] | p \; mod \; 12 = c\}} Y_{LF}(m,p)$
    \item
        $T_{coef}(m) = \frac{m \cdot H}{F_s}$\\
			\marginpar{\Huge ! \hfill}

    \item
        CFT: aperiodic, cont. signal $\rightarrow$ cont. spectrum
    \item
        DFT: aperiodic, periodically continued, finite, disrecte signal $\rightarrow$ discrete, finite spectrum
\end{itemize}
\section{HPSS}
\begin{itemize}
    \item
        Harmonically and percussively enhanced spectromgrams $\tilde{Y}_p$ and $\tilde{Y}_h$: Median filter along time or frequency axis
    \item
        Binary mask $M_p(m,k)$: 1 if $\tilde{Y}_p \geq \tilde{Y}_h$, else zero (for percussive harmonic mask)
    \item
        Percussive STFT: $X_p = X \cdot M_p$
    \item
        HPSS for chroma enhancement: filter out percussive stuff
    \item
        HPSS for onset detection: filter out harmonic stuff
\end{itemize}

\section{Music Synchronization}
\begin{itemize}
    \item
        General: For a given position in one representation of a piece of music, determine the corresponding position within another representation
    \item
        \textbf{Dynamic Time Warping}: Given two chroma vectors $X$ and $Y$, find non-linear alignment in order to compensate differences in tempo. Corresponds to finding valley of low cost within cost matrix.
    \item
        Local cost measure $c: F \times F \rightarrow \mathbb{R}$
    \item
        Evaluation $c$ for every pair of feature vectors yields cost matrix $C(n,m) = c(x_n, y_m)$
    \item
        $(N-M)$-Warping Path is sequence $P=(p_1, \dots, p_L)$ with $p_l = (n_l, m_l)$:
			\marginpar{\Huge ! \hfill}
        \begin{itemize}
            \item
                Boundary condition: $p_1 = (1,1)$ and $p_L = (N,M)$
            \item
                Monotonicity condition: $n_1 \leq \dots \leq n_L$ and $m_1 \leq \dots \leq m_L$
            \item
                Step size condition: $p_{l+1} - p_l \in \{(1,0), (0,1), (1,1)\}$
        \end{itemize}
    \item
        Total cost of $P$: $c_P(X,Y) = \sum_{l=1}^{L} C(n_l, m_l)$
    \item
			Dynamic Time Warping distance: 
			
			$ DTW(X,Y) = c_{P^*}(X,Y) = \min\Set{c_P (X,Y) \; | \; \text{P is an (N,M)-warping path}}$
\end{itemize}
\section{Music Structure Analysis}
\begin{itemize}
    \item
        General: Divide a music representation into temporal segments that correspond to musical parts and to group these segments into musically meaningful categories
    \item
        Comparing every feature vector with one another via a similarity measure yields a \textbf{Self-Similarity Matrix}
    \item
        Path-like structures: repetitions (e.g. recurring rythmic, harmonic or melodic patterns)
    \item
        Block-like structures: homogenity (e.g. consistent timer, presence of specific instrumentation)
    \item
        Path over segment $\alpha = [s:t]$ is a sequence $P = ((n_1, m_1), \dots, (n_L, m_L))$ with
        \begin{itemize}
            \item
                $m_1 = s, m_L = t$
            \item
                $(n_{l+1}, m_{l+1}) - (n_{l}, m_{l}) \in \{(2,1), (1,2), (1,1)\}$
        \end{itemize}
    \item
        Similar for blocks (subset of $[1:N] \times [1:N]$)
    \item
        Path-similarity, Block-similarity of segments
    \item
        Algorithmic pipeline:
        \begin{itemize}
            \item
                Transform recording into feature sequence
            \item
                Compute SSM
            \item
                Find blocks and paths of overall high score (which each defines a pair of similar segments)
            \item
                Cluster segments into groups
        \end{itemize}
    \item
        Enhancement strategies:
        \begin{itemize}
            \item
                Downsample and smooth chroma features $\rightarrow$ emphasis on rough harmonic content
            \item
                Path smoothing along direction of main diagonal (only works if no relative tempo differences between the segments)
            \item
                Path smoothing alone multiple directions and picking maximum values for final SSM
            \item
                Transposition invariance: Apply cyclic shifting on one sequence before comparing, do for all 12 pitches, pick maximum values for final SSM
            \item
                Global thresholding/binarization
        \end{itemize}
    \item
        Audio Thumbnailing:
        \begin{itemize}
            \item
                Goal: automatically determine the most representative section
            \item
                Path familiy: Set of paths over a segment $\alpha$, which induced segments don't overlap
            \item
                Intoduce fitness of a segment/path family:
                \begin{itemize}
                    \item
                        P = normalize ( coverage(segment) - length(segment)) $\in [0,1]$
                    \item
                        R = normalize ( score(segment) - length(segment)) $\in [0,1]$
                    \item
                        $ F = 2 \cdot P \cdot R / (P + R)$
                \end{itemize}
            \item
                Scape plot:
                \begin{itemize}
                    \item
                        x: center of segment
                    \item
                        y: length of segment
                    \item
                        z: fitness
                \end{itemize}
        \end{itemize}
    \item
        Novelty-Based Segmentation: Shift checkboard kernel along main diagonal, structure features
\end{itemize}

\section{Tempo and Beat Tracking}
\begin{itemize}
    \item
        Onset Detection
    \item
        Beat tracking: Find sequence of perceived pulse positions (phase + period)
    \item
        Tempo estimation: Reciprocal of beat period
\end{itemize}
\textbf{Onset Detection}
\begin{itemize}
    \item
        Pipeline: 1) Convert into feature sequence, 2) Yield novelty function, 3) Peak picking
    \item
        Energy-Based:
        \begin{enumerate}
            \item
                Amplitude squaring
            \item
                Windowing
            \item
                Differentiation
            \item
                Half wave rectification
            \item
                Peak picking
        \end{enumerate}
    \item
        Spectral-Based:
        \begin{enumerate}
            \item
                Apply logarithmic compression onto spectrogram
            \item
                Differentiation (yields gradient spectrogram)
            \item
                Accumulation frame-wise (yields novelty function)
            \item
                Apply averaging window onto function and substract from original (Normalization)
            \item
                Peak picking
        \end{enumerate}
\end{itemize}
\textbf{Tempo Analysis}
\begin{itemize}
    \item
        Assumptions: 1) Beat positions occur at note onset positions 2) Beat positions are equally spaced
    \item
        Tempogram: Indicated for each time instance the local relevance of a specific tempo for a given music recording
    \item Fourier Tempogram:
        \begin{itemize}
            \item
                Compare novelty curve with windowed sinusoids
            \item
                Generally indicates tempo harmonics
        \end{itemize}
    \item
        Autocorrelation Tempogram:
        \begin{itemize}
            \item
                Compare windowed novelty curve with time-shifted version (time lag) of itself
            \item
                Generally indicates tempo subharmonics
        \end{itemize}
\end{itemize}
\textbf{Predominant local pulse (PLP)}
\begin{itemize}
    \item
        For each time position of Fourier tempogram compute maximizing tempo parameter and it's phase
    \item
        Sum up all windowed sinusoids with corresponding phase and frequency (overlap-add)
    \item
        Half wave rectify
    \item
        Local periodicity enhancement of the original novelty function
\end{itemize}
\section{Content-Based Audio Retrival}
\begin{itemize}
    \item
        Audio Identification with fingerprints based on spectral peaks
    \item
        Apply peak picking strategy that identifies time-frequency points that have a higher magnitude than all neighbors ($\rightarrow$ constallation map)
    \item
        Naive approach: Shift query over database documents
    \item
        Indexing: 
        \begin{itemize}
            \item 
                Inverted list $L(h)$ consisting of time stamps $n$ for $(n,h)$ out of $D$

            \item
                $L(h) - n$ contains all the shifts that, when applied to query point $(n,h)$ result in match
            \item
                If $L(h) -n$ contains same shifts for different $h$'s $\rightarrow$ query matches amount of equal shift value
            \item
                Without indexing: Linear in len(Q) $\cdot$ len(D)
            \item
                With indexing: Access and process len(Q) lists of len(D)/numberOfHashes (Assuming hash values are evenly distributed)  $\rightarrow$ gain of numberOfHashed $\rightarrow$ more hashes/shorter inverted lists are better
        \end{itemize} 
    \item
        Shazam:
        \begin{itemize}
            \item
                Idea: Increase number of hashes/Reduce length of hash lists
            \item
                Fix anchor point and define target zone
            \item
                Use every point as anchor point
            \item
                Hash: $(f_1, f_2, \Delta t)$ $\rightarrow$ increased specifity of hashes $\rightarrow$ much smaller hash lists $\rightarrow$ profit
        \end{itemize}
\end{itemize}

\section{Chord Recognition}
\begin{itemize}
    \item
        Goal: Given a audio recording of a piece of music, find which chords are played at which time
\end{itemize}
\subsection{Template-Based Chord Recognition}
\begin{itemize}
    \item
        First step: transform recording into a sequence $X = (x_1, x_2, \dots, x_N)$ of feature vectors
    \item
        Second step: 

        \begin{itemize}
            \item 
                map feature vectors to a chord label via pattern matching
            \item
                Precompute a set of templates (prototypical chroma vectors the represents a specific mucial chord)
            \item
                Similarity measure
        \end{itemize}
\end{itemize}
\subsubsection*{Ambiguities}
\begin{itemize}
    \item
        Chord Ambiguities: Chords may share same notes
    \item
        Acoustic Ambiguities: Harmonics, Major-Minor confusion 
    \item
        Tuning: Instrument not tuned as specified by the center frequency of the equal-tempered scale ($\rightarrow$ notes energy gets spread across neighboring chroma band)
    \item
        Segmentation: Broken chords
\end{itemize}
\subsubsection*{Enhancement Strategies}
\begin{itemize}
    \item
        Templates with harmonics (e.g. energy of $k^{th}$ harmonic is $\alpha^{k-1}$ for some $\alpha \in [0,1]$)
    \item
        Templates from examples (e.g. take average chroma vectors), naturally inherit the musical and acoustic properties. It sufficies to learn $C$ and $Cm$ chord and apply cyclic shifting
    \item
        Spectral enhancement, "smoothing" (e.g. Logarithmic compression)
    \item
        Temporal smoothing, Prefiltering (e.g. shift averaging filter over chroma feature sequence)
\end{itemize}
\subsection{HMM-Based Chord Recognition}
\begin{itemize}
    \item
        Template-based chord recognizer only considers current frame, not frame sequence
    \item
        Some harmonic progressions are much more likeli then other, e.g. $I-IV-V-I$
\end{itemize}
\subsubsection*{Markov Chains and Transition Probabilities}
\begin{itemize}
    \item
        Considered chords are refered to as \textbf{states}, $\mathcal{A} = \{\alpha_1, \alpha_2, \dots, \alpha_I\}$
    \item
        The change from one state to another is specified according to set of probabilities (assumption: probability only depends on current state, not the ones before, "memoryless"), \textbf{state transition probabilities}
    \item 
        \textbf{Initial state probabilities}
\end{itemize}
\subsection*{Hidden Markov Models}
\begin{itemize}
    \item
        With Markov chain we can compute probability for a given observation consisting of a sequence of chords
    \item
        Chord recognition task however we observe a sequence of chroma vectors
    \item
        Each state is equipped with a probability function that expresses the likelihood for a given chord type to output or emit a certain feature vector (\textbf{emission, output probabilities})
    \item
        Discrete HMM: Output space is assumed to be discrete and finite $\rightarrow$ Finite set of output elements $\mathcal{B} = \{\beta_1, \dots, \beta_2\}$
    \item
        Summary: HMM is fully specified by five tuple\\ (set of states, state trans. probs., init state probs. observation syms., emission probs)
\end{itemize}
\subsection*{Evaluation and Model Specification}
\textbf{Evaluation problem}
\begin{itemize}
    \item
        Given a observation sequence $O$, compute probability $P(O|\Theta)$
    \item
        "Score" of how well a given model matches a given observation sequence, interesting when one has to choose among several competing models
    \item
        $P(O|\Theta) = \sum_{S=(s_1, s_2, s_N)} P(0, S|\Theta)$
\end{itemize}
\textbf{Uncovering Problem}
\begin{itemize}
    \item
        Find state sequence that best explains observation sequence 
    \item
        Possible optimization criterion:
        $S^* = \underset{S = (s_1, s_2, \dots, s_N)}{\argmax} P(O,S|\Theta)$
    \item
        Viterbi Algorithm:
        \begin{itemize}
            \item
                Observation sequence $O=(o_1, \dots, o_N)$
            \item
                $O(1:n) = (o_1, \dots, o_n)$
            \item
                $D(i:n) = {\max_S} \; P(O(1:n), (s_1, \dots, s_n=\alpha_i)|\Theta)$, highest probability of $O(1:n)$ with state sequence ending on $\alpha_i$
            \item
                $(I \times N)$ matrix $D$ can be computed recursively\\
            \item
                $ S^* = (\alpha_i, \dots, \alpha_{i_N})$
            \item
                Apply backtracking procedure
            \item
                $i_N = \underset{j \in [1:I]}{argmax} D(j, N)$, ending state of state sequence, which is most probable to generate observations
            \item
                $i_n = \underset{j \in [1:I]}{argmax} \; \; a_{j, i_{n+1}}D(j, n)$, ending state of state sequence of length $n$, which is most probable to create observations $O(1:n)$ weighted with transition probability to previously calculated state $i_{n+1}$\\
            \item
                $O(I^2 N)$ operations much better than brute force $O(I^N)$
        \end{itemize}
\end{itemize}
\textbf{Estimation Problem}
\begin{itemize}
    \item
        Given an observation sequence $O$, improve probability measures of models
    \item
        Baum-Wekch algorithm
\end{itemize}
\subsection*{Application to Chord Recognition}
\textbf{Specification of Emission Probabilities}
\begin{itemize}
    \item
        To make HMM discrete (make observations coming from a finite output space) one can introduce a so-called codebook (set of prototypical vectors). \textbf{Quantization}, clustering
    \item
        Alternative: continuous HMMs, emission probability becomes continuous pdf
\end{itemize}
\textbf{Specification of Transition Probabilities}
\begin{itemize}
    \item
        Can be defined manually by expert
    \item
        Automatic approach:
        \begin{itemize}
            \item
                We assume the training set is represented by a sequence of frames and that each frame is labeled with ont of the states
            \item
                Next, count how often each of the possible chord transitions occurs in the training data
            \item
                $\mu(i,j) = $ number of transitions $\alpha_i \rightarrow \alpha_j$, number of bigrams with first chord as $\alpha_i$ and second chord as $\alpha_j$
            \item
                Transition probability matrix $A = (a_{ij})$ with $a_{ij} = \ffrac{\mu(i,j)}{\sum_k \mu(i,k)}$
            \item
                Dominated by diagonal matrix (self-transition probabilities): Chord durations occuring are much longer than the frame length
            \item
                Other elements, e.g. $\alpha_{i, i+7}$ are also much higher (in this example because change from tonic to dominant more probable) $\rightarrow$ secondary diagonals
            \item
                Indicates that transition probabilities depend on functional relations, independent of underlying key
            \item
                Increase training data:
                \begin{itemize}
                    \item
                        Take each bigram and shift chroma features, such that first chord corresponds to $C$ or $Cm$ $\rightarrow$ C normalized bigram
                    \item
                        Take 12 C normalized bigrams
                    \item
                        Normalize thme to the 12 different chroma values
                    \item
                        Transposition-invariant transition probability matrix
                \end{itemize}

        \end{itemize}
\end{itemize}
\textbf{Evaluation}
\begin{itemize}
    \item
        Context-sensitive smoothing: recognizer tends to stay in the current chord rather than change to another one. Chord changes are only performed when the relatively low transitions probabilities are compensated by a substantial increase of emission probability
    \item
        High self transitions are the main factor
\end{itemize}
\end{document}
