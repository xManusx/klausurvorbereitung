\section{Einstellung und Einstellungsver"anderung, Einstellung und Verhalten}
\subsection{Was sind Einstellungen?}
\begin{itemize}
	\item
		= eine psychische Tendenz, die dadurch zum Ausdruck kommt, dass man ein bestimmtes Objekt mit einem gewissen Grad an Zuneigung oder Abneigung bewertet
	\item
		= Gesamtbewertung eines Einstellungsgegenstandes
	\item
		Drei Komponenten:
		\begin{itemize}
			\item
				Affektive Komponente (wie f"uhle ich mich damit?)
			\item
				Kognitive Komponente (was denke ich dar"uber?)
			\item
				Verhaltesnkomponente (wie gehe ich damit um?)
		\end{itemize}
	\item
		Gibt explizite (= bewusst, leicht benennbar) und implizite (=unwillk"urlich, unkontrollierbar, z.T. unbewusste Bewertung)  Einstellungen
	\item
		Einstellungsmessung, Messung nach... 
		\begin{itemize}
			\item Valenz 

				wird in ein oder zwei Dimensionen gemessen: \\
				%\begin{tabular}{c|c|c|c}
				%\multicolumn{2}{c}{$\qquad$} &  \multicolumn{2}{c}{Positivit"at} \\
				%& & niedrig & hoch\\
				%\hline
				%\multirow{2}{*}{Negativit"at} & niedrig & Indifferenz & Positive Einstellung\\
				%& hoch & Negative Einstellung & Ambivalenz
				%\end{tabular}
				% Please add the following required packages to your document preamble:
				% \usepackage{multirow}
				% Please add the following required packages to your document preamble:
				% \usepackage{multirow}

				% Please add the following required packages to your document preamble:
				% \usepackage{multirow}

				\begin{table}[h]
					\centering
					\label{my-label}
					\begin{tabular}{c|c|c|c|}
						\multicolumn{2}{c|}{\multirow{2}{*}{}}  & \multicolumn{2}{c|}{Positivit"at}           \\ \cline{3-4} 
						\multicolumn{2}{c|}{}                   & niedrig              & hoch                 \\ \hline
						\multirow{2}{*}{Negativit"at} & niedrig & Indifferenz          & Positive Einstellung \\ \cline{2-4} 
																											& hoch    & Negative Einstellung & Ambivalenz           \\ \hline
					\end{tabular}
				\end{table}

			\item und St"arke

				Annahme: besonders starke Einstellungen sind besonders leicht zug"anglich, Messung dann durch Reaktionszeit oder "Au"serungen in freien Antwortformaten
			\item
				Explizite Ma"se:
				\begin{itemize}
					\item
						Meinungs"au"serungen zur Erfassung kognitiver Komponente
					\item
						Gef"uhls"au"serungen f"ur affektive Komponente
					\item
						Selbstbeurteilungsbogen, semantisches Differential
				\end{itemize}
			\item
				Implizite Ma"se:
				\begin{itemize}
					\item
						Verhaltensma"se
					\item
						Messung, wie stark Einstellungsobjekte mit Valenz assoziiert sind, z.B. affektives Priming, impliziter Assoziationstest
				\end{itemize}

		\end{itemize}
	\item
\end{itemize}
\subsection{Einstellungsbildung/-"anderung}
\begin{itemize}
	\item
		Woher kommen Einstellungen?
		\begin{itemize}
			\item
				kognitiv basierte Einstellungen: beruhen auf subjektivem Wissen "uber die Eigenschaften des Objekts und Abw"agung positiver und negativer Charakteristika, z.B. Fishbein 1975: Einstellung = $\sum$(Meinung $\times$ Bewertung)
			\item
				affektiv basierte Einstellungen: beruhen vor allem auf Gef"uhlen und Wertvorstellungen, normalerweise nicht durch rationalit"at beinflussbar. Z.B. Geschmack, politische, religi"ose Einstellungen
				\begin{itemize}
					\item
						Klassische Konditionierung: neutraler Stimulus $\leftrightarrow$ Affektiv besetzter Stimulus pos/neg  $\rightarrow$ Un/Angenehme Gef"uhle $\Rightarrow$ neutraler Stimulus  $\rightarrow$ Un/Angenehme Gef"uhle
					\item
						Operante Konditionierung: Verhalten gg"u. Einstellungsobjekt $\rightarrow$ Belohnung/Bestrafung $\rightarrow$ pos/neg Einstellung gg"u. Einstellungsobjekt
					\item
						Affektives Priming (Box/Smiley/Sadey kurz, Schriftzeichen lang, Bewertng des Schriftzeichens korreliert mit Symbol)

					\item
						mere exposure (Wiederholte, unaufdringliche Darbietung eines vorher neutralen Stimulus f"uhrt zu positiver Bewertung dieses Stimulus)
				\end{itemize}
		\end{itemize}
	\item
		Einstellungs"anderung
		\begin{itemize}
			\item
				... durch kognitive Dissonanz, bei einstellungskontr"arem Verhalten ohne externe Rechtfertigung
			\item
				Persuasive Kommunikation: \emph{Wer} sagt \emph{was} zu \emph{wem}?
				\begin{itemize}
					\item
						Wer/Sendervariablen: Expertise, Glaubw"urdigkeit, Sympathie, Attraktivit"at
					\item
						Was/Merkmale der Botschaft: Qualit"at der Argumente, Extremit"at, emotionaler Gehalt, Warhnehmung des Beinflussungsversuchs, ein- vs. zweiseitige Argumentation, Reihenfolge der Darbietung
					\item
						Wem/Empf"angervariablen: Anf"angliche Einstellung, Aufmerksamkeit, Motivation, intellektuelle F"ahigkeiten, Selbstwertgef"uhl, Alter
				\end{itemize}
			\item
				Sleeper-Effekt: Empf"anger vergisst Einstellung gegen"uber Sender, revalidiert Mitteilung sp"ater
			\item
				Elaboration Likelihood Model:

				2 Wege der Verarbeitung von persuasiver Kommunikation
				\begin{itemize}
					\item
						Zentrale route:
						\begin{itemize}
							\item
								Intensive kognitive Ausseinandersetzung mit Argumenten
							\item
								Aber nur bei hoher Motivation und ausreichenden kognitiven Ressourcen und F"ahigkeit den Argumenten folgen zu k"onnen, ausserdem \enquote{Need for cognition}=\enquote{Ein Pers"onlichkeitsmerkmal, nach dem man Individuen im Hinblick darauf differenzieren kann, wie viel und wie gern sie "uber Themen und Probleme nachdenken}
							\item
								Qualit"at der Argumente entscheidend
							\item
								F"uhrt zu dauerhafter und verhaltensrelevanter Einstellungs"anderung
						\end{itemize}
					\item
						Periphere Route
						\begin{itemize}
							\item
								Eher oberfl"achliche Ausseinandersetzung mit Botschaft, Entscheidung gem"a"s kognitiver Heuristiken
							\item
								Bei geringer Motivation und/oder fehlenden kognitiven Ressourcen
							\item
								Oberfl"achliche Kriterien (Attraktivit"at, Expertenstatus des Sprechers \dots) entscheidend
							\item
								F"uhrt eher nicht zu langfristiger Einstellungs"andernug
						\end{itemize}
				\end{itemize}
			\item
				Persuasive Botschaften am effektivsten, wenn auf Art der Einstellung zugeschnitten, die ver"andert werden soll:
				\begin{itemize}
					\item
						gute Argumente f"ur kognitiv basierte Einstellungen
					\item
						emotionale Appelle zur Ver"anderung affektiv basierter Einstellungen
				\end{itemize}
			\item
				Furchterregende Botschaften kann Motivation erh"ohen Informationsverarbeitung auf zentraler Route durchzuf"uhren, kann aber auch zu Abwehrreaktion f"uhren \\
				Besonders effektiv: M"a"sige Angst erzeugen, gleichzeitig aber Anleitung wie Angst reduziert werden kann
			\item
				Mit guter Stimmung eher periphere Verarbeitung
			\item
				Studie Leventhal, Wattts u. Pagano 1967: Raucher sehen Film "uber Gefahren des Rauchens oder bekamen Instruktionen wie sie mit dem Rauchen aufh"oren k"onnen oder beides. Gruppe mit beidem stabilste Einstellungs/Verhaltens"anderung
			\item
				Einstellungsimpfen: (McGuire 1964) Gegenargumente im Vorraus durchdenken und entkr"aften, kleine Dosen von Gegenargumenten machen immun gegen sp"atere Beeinflussungsversuche, so wie Impfen
			\item
				Bumerang Effekt/Reaktanztheorie:
				\begin{itemize}
					\item
						Psychologische Reaktanz: motivationaler Zustand, der darauf ausgerichtet ist, die bedrohte Freiheit wiederherzustellen
					\item
						Wenn Menschen glauben in Bereich Handlungs/Wahlfreiheit zu haben f"uhrt Bedrohung dieser Freiheit zu psychologischer Reaktanz\\
						$\Rightarrow$ Auf- und Abwertung von Wahlalternativen
				\end{itemize}
		\end{itemize}
\end{itemize}
\subsection{Einstellung und Verhalten}
\begin{itemize}
	\item
		Verhaltensvorhersage durch Einstellungen:
		\begin{itemize}
			\item
				Zusammenhang Verhalten-Einstellung deutlich geringer als man annehmen w"urde (LaPiere 1934: Chinesisches Paar in USA wurden fast "uberall bediehnt obwohl 90\% kategorisch ablehnten
			\item
				Bei spontanen Entscheidungen ist St"arke/Zug"anglichkeit der Einstellungen entscheidend
			\item
				$\Rightarrow$ Theorie des geplanten Verhaltens von Fishbein u. Ajzen 1985
				\begin{itemize}
					\item
						Intention bester Pr"adikator f"ur Verhalten, Intention h"angt von drei Faktoren ab:
						\begin{enumerate}
							\item
								Einstellung gegen"uber dem Verhalten
							\item
								Soziale Normen/was tun/denken die anderen?
							\item
								Erwartung wie einfach oder schwierig die Ausf"uhrung des geplanten Verhaltens wird = Wahrgenommene Verhaltenskontrolle
					\end{enumerate}
				\end{itemize}
		\end{itemize}
\end{itemize}
\subsection{Werbung}
\begin{itemize}
	\item
		Vicary 1957: Cola/Popcorn Werbung subliminal im Kino. Replikation allerdings immer gescheitert
	\item
		Auditive subliminale Botschaften: Gibt keinen Beleg f"ur Wirksamkeit, aber Menschen glauben, es funktioniert
	\item
		Unterschwellige Reize k"onnen zwar Pr"aferenz f"ur mehrdeutige Stimuli beeinflussen aber nur bei vorher neutralen Reizen, unter ganz bestimmten Bediungen. Einfl"usse auf konkretes Verhalten nicht nachgewiesen
\end{itemize}
