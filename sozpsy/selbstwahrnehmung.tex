\section{Das Selbst}
\subsection{Das Selbst: was ist es und welche Funktionen erf"ullt es?}
\begin{itemize}
	\item
		Verschiedene Typen des Selbst:
		\begin{itemize}
			\item
				Physisches Selbst (ich bin gro"s, habe blonde Harre)
			\item
				Soziales Selbst (soziale Rollen: Student, Bruder)
			\item
				Reflexives Selbst (Pers"onlichkeitseigenschaften, z.B. sch"uchtern)
		\end{itemize}
		
	\item
		Selbst = Gesamt an subjektivem Wissen "uber die eigene Person:
		\begin{itemize}
			\item
				F"ahigkeit "uber eigene Person zu reflektieren
			\item
				Betrachtung der eigenen Person von au"sen, \enquote{Selbst als Wissensobjekt}
			\item
				Beinflusst durch soziale Gegebenheiten
		\end{itemize}


	\item
		Komponenten des Selbst:
		\begin{itemize}
			\item
				Selbst als erkenndes Subjekt/I  $\rightarrow$ Selbstregulation
			\item
				Selbst als Objekt der Erkenntnis/Me $\rightarrow$ Selbstkonzept
				\begin{itemize}
					\item
						Selbstkonzept = Theorie "uber das eigene Selbst, organisierte kognitive Struktur, die aus Erfahrungen mit der eigenen Person abgeleitet ist
					\item
						Eigene Person wird als distinkte Einheit erkannt
					\item
						Besteht aus einheitlichem, integriertem Selbst und multiplen Sub-Selbsten/Identit"aten
				\end{itemize}
		\end{itemize}
	\item
		Funktionen des Selbst:
		\begin{itemize}
			\item
				Strukturierende Funktion:
				\begin{itemize}
					\item
						Selbstschemata werden genutzt um Wissen "uber uns selbst zu strukturieren
					\item
						Self-Reference-Effekt: Personen erinnern sich besser an Dinge, die in einem Bezug zu ihrem Selbstkonzept stehen (z.B. an Charaktere, die uns "ahnlich sind nach einem Film, oder nach Veranstaltung mit vielen fremden Person erinnert man sich besser an Personen, mit denen man sich "uber pers"onliche/pers"onlich wichtige Dinge unterhalten hat
				\end{itemize}
			\item
				Ausf"uhrende Funktion: Selbstregulation
				\begin{itemize}
					\item
						Regulation von Handlungen, Entscheidungen, Pl"anen
					\item
						Erm"oglicht uns langfristig zu planen
					\item
						Selbstkontrolle ist begrenzte Ressource (z.B. unter M"udigkeit weniger Selbstkontrolle)
				\end{itemize}
		\end{itemize}
\end{itemize}

\subsection{Kulturelle Unterschiede}
\begin{itemize}
	\item
		Independente (Mutter-Selbst ist abgetrennt von Freund-Selbst, abgetrennt von Geschwister-Selbst, abgegrenzt vom Selbst-Selbst) 
		\begin{itemize}
			\item
				Selbstdefinition st"utzt sich auf eigene Gedanken, Gef"uhle, Handlungen
			\item
				Ursprung eigener Gedanken, Gef"uhle, Handlungen wird in der eigenen Person gesucht
			\item
				Betonung der Unabh"angigkeit und Einzigartigkeit
		\end{itemize}
	\item
		vs. Interdependente (Mutter-Selbst, Freund-Selbst etc. "uberschneiden sich mit Selbst-Selbst)
		\begin{itemize}
			\item
				Selbstdefinition basiert auf eigenen Beziehungen zu anderen und auf sozialen Rollen
			\item
				Wissen, dass das eigene Erleben und Verhalten oft von anderen beinflusst ist
			\item
				Betonung der Verbundenheit/Abh"angigkeit mit/von anderen
		\end{itemize}
		$\Rightarrow$ Japanische Sportler erkl"aren sich Erfolge eher mit Erfahrungen etc. vs. amerikanische eher mit pers"onlichen Charakteristiken (Markus et al. 2006)

		Oder Markus u. Kim 1999: Amerikaner und Asiaten f"ullen am Flughafen Frageb"ogen aus, d"urfen hinterher Stif behalten. VL bietet 5 Stifte an, in zwei verschiedenen Farben (Verh"altnis zw. den Farben 2:3 oder 1:4)  $\rightarrow$ Amerikaner nehmen einzigartigere Farbe
\end{itemize}

\subsection{Quellen der Selbsterkenntnis}
\begin{itemize}
	\item
		Introspektion: Beobachtung der eigenen Gedanken und Gef"uhle
		\begin{itemize}
			\item
				Wir denken aber nur relativ selbst "uber uns selbst nach (ca. 8\% t"aglich)
			\item
				Oft sind wir uns der tats"achlichen Gr"unde f"ur unser Erleben u. Verhalten nicht bewusst, legen uns aber einfach Kausaltheorien zurecht um plausible Gr"unde zu haben
		\end{itemize}
	\item
		Objektive Selbstaufmerksamkeit, Ausl"oser = Spiegel, Kamera, Publikum (Individuum richtet Aufmerksamkeit auf eigenes Verhalten, Stimmung und Standards. Experiment Pryor 1977: VP bewertet sich selbst auf Soziabilit"atsskala, wird anschlie"send in Interaktion von jemand anderstem bewertet. Wenn VP die Skala in einem Raum mit Spiegel ausgef"ullt hat, korrelierten beide Bewertungen signifikant h"oher)
		\begin{itemize}
			\item
				Duval und Wicklung 1972: OSA: Konzentration auf wichtige Aspekte des Selbst, auf innere Werte und Normen. Gleichzeitig Salienz des augenblicklichen Verhaltens  $\rightarrow$ Diskrepanzen zwischen Ist und Soll. Diskrepanzen f"uhren zu negativem Affekt, Motivation zur Reduzierung von OSA
				\begin{itemize}
					\item
						M"oglichkeiten zur Reduzierung: Ausl"oschen bzw. vermindern der Diskrepanz oder Vermeidung von Selbstaufmerksamkeit erregenden Stimuli
					\item
						Folgen: einstellungskonformes Verhalten, Leistungssteigerung oder -minderung, 
					\item
						Messung: verst"arkte Nutzung von \enquote{Ich}-Formulierungen, schnelleres Erkennen ich-bezogener W"orter
					\item
						Praxisrelevanz: Stottern, Pr"ufungsangst, Drogenprobleme, Di"aten
				\end{itemize}
		\end{itemize}
	\item
		Beobachtung eigenen Verhaltens: Selbstwahrnehmungstheorie (\enquote{Sind Sie defensiver oder offensiver Autofahrer?}  $\rightarrow$ zur"uckerinnern an eigenes Verhalten in Vergangenheit um Frage zu beantworten). Annahme: Menschen schlie"sen auf ihre inneren Zust"ande in der gleichen Weise, wie sie dies bei anderen Personen tun
		\begin{itemize}
			\item
				Wesentlich f"ur Attributionen auf eigenes Selbst sind Handlungsfreiheit und geringe externe Rechtfertigung
			\item
				Korrumpierungseffekt: Person nimmt wahr, dass sie f"ur T"atigkeit, die sie bisher gern ausge"ubt hat eine Belohnung erh"alt, daraus schlie"st die Person, dass sie die T"atigkeit doch nicht so gern tut
		\end{itemize}
	\item
		Andere Menschen als Quelle der Selbsterkenntnis
		\begin{itemize}
			\item
				Selbsterkenntnis durch Beobachtung von Reaktionen anderer (\enquote{looking glass Self}
			\item
				Soziale Vergleiche, Festinger 1954: Theorie des sozialen Vergleichs
				\begin{itemize}
					\item
						Vergleichen und mit anderen, insbesondere wenn kein objektiver Ma"sstab da ist
					\item
						Meist Vergleich mit Peers, da h"ochster Informationsgehalt
					\item
						jedoch auch Abw"arts(erh"ohen Selbstwert, aber negativ auf Anstrengung/Motivation. Besondere Form = Vergleich mit fr"uherem Selbst) bzw. Aufw"arts(um eigene M"oglichkeiten auszuloten, kann motivierend wirken, evt. aber auch negativ auf Selbstwert) gerichteter Vergleich
				\end{itemize}
		\end{itemize}
\end{itemize}

\subsection{Objektive Selbstaufmerksamkeit}
Siehe oben (Foliengliederung wtf is that?)
\begin{itemize}
	\item
		Eigene Emotionen verstehen: Zwei-Faktoren Theorie von Schachter 1964:
		\begin{itemize}
			\item
				Emotionales Erleben l"auft in Zwei Stufen ab:
				\begin{enumerate}
					\item
						Wahrnehmung physiologischer Erregung
					\item
						Interpretation derselben
			\end{enumerate}
		\item
			Versuch mit attraktiver Frau und m"annlichen Versuchspersonen auf H"angebr"ucke
		\end{itemize}
\end{itemize}

\subsection{Selbstdarstellung und Impression-Management}
\begin{itemize}
	\item
		Versuch sich als den Menschen darzustellen, als der man wahrgenommen weden m"ochte
	\item
		Strategien: Einschmeichelungstaktik und self-handicapping (externe Erkl"arungen f"ur eigene Misserfolge werden vorbereitet)
\end{itemize}

\section{Selbstwertgef"uhl, Selbstregulation}
\begin{itemize}
	\item
		Hoher Selbstwert korreliert mit Neurotizismus, gewissenhaftigkeit, extraversiv, vertr"aglichkeit und Offenheit
	\item
		Menschen sind grunds"atzlich motiviert ihr Selbstwertgef"uhl zu sch"utzen bzw. zu erhalten
	\item
		Streben nach stabilem und positiven Selbstbild
\end{itemize}

\subsection{Theorie der kognitiven Dissonanz}
\begin{itemize}
	\item
		Formuliert von Leon Festinger 1957
	\item
		Kognitive Dissonanz = unangenehmer Spannungszustand, der durch widerspr"uchliche Kognitionen verursacht wird, bzw. insbesondere durch dem Selbsbild widersprechende Handlungen hervorgerufen wird
	\item
		Urspr"ungliche Formulierung:
		\begin{itemize}
			\item
				Kognitionen k"onnen in relevanter oder irrelevanter Beziehung zueinander stehen. (Ob ich Atomkraft gut oder schlecht finde hat nichts damit zu tun, ob ich gerne Nudeln esse, also irrelevant)
			\item
				relevante Beziehungen k"onnen konsonant oder dissonant sein
			\item
				St"arke der Dissonanz abh"angig vom Anteil dissonanter Relationen und Wichtigkeit der beteiligten Kognitionen
			\item
				Dissonanz erzeugt Motivation zur Dissonanzreduktion
		\end{itemize}
	\item
		Studie Jones u. Kohler 1959: Befragung zur Rassentrennung pro/contra. Bef"urworter haben glaubw"urdige pro und nichtglaubw"urdige contra Argumente besser erinnert (und andersherum)
	\item
		Anwendung der Dissonanztheorie:
		\begin{itemize}
			\item
				Postdecisional dissonance \end{itemize}
			\item
				Justification of effort: Attraktivit"at von freiwilliger Aufgabe steigt mit H"ohe der Anstrengung
			\item
				Forced Compliance:
				\begin{itemize}
					\item
						Studenten langeweilige Aufgabe, sollten sp"ater anderen erz"ahlen, wie spannend die Aufgabe war. Manche bekamen Belohnung, je geringer Belohnung desto gr"o"ser Einstellungs"anderung (Aufgabe wurde als weniger langweilig empfunden)
					\item
						Geringe Strafandrohung eher Einstellungs"anderung als hohe Strafandrohung (forbidden-toy Paradigma)
					\item
						Auch wieder kulturelle Unterschiede, in kollektivistischen Gesellschaften gibt es Dissonanzerleben auf Gruppenebene
				\end{itemize}
			\item
				Ist Dissonanz ein Erregungszustand?
				\begin{itemize}
					\item
						Zanna und Cooper 1974
					\item
						VP kriegen Placebo mit Info kein Effekt/erregend/entspannend, schreiben dann Einstellungskontr"aren Aufsatz
					\item
						H"ochste Einstellungs"anderung bei entspannend, dann bei kein Effekt
					\item
						\enquote{Wieso bin ich erregt? Ach ja, wegen der Pille, nicht wegen kognitiver Dissonanz!} vs. \enquote{ich sollte doch entspannt sein, wieso bin ich erregt? Vermutlich, wegen kognitiver Dissonanz, weil ich diesen Aufsatz schreibe!}
				\end{itemize}
\end{itemize}
