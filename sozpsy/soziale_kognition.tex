\section{Soziale Kognition - Soziale Informationsverarbeitung}
\subsection{Soziale Kognition}
\begin{description}
	\item[Kognition] mentale Prozesse und Strukturen eines Individuums
	\item[Soziale Kognition] Art und Weise wie der Mensch "uber sich selbst und seine soziale Umwelt nachdenkt; genauer gesagt die Art und Weise, in der er soziale Information ausw"ahlt, interpretiert, abspeichert und abruft um Urteile zu f"allen und Entscheidungen zu treffen (Aronson)
\end{description}
\begin{itemize}
	\item
		Informationsverarbeitung = alle Aktivit"aten des psychischen Systems (Informationserhalt, -auswahl, -transformation, -organisation
	\item
		bezieht sich darauf, wie das psychische System Repr"asentationen der Wirklichkeit konstruiert und Wissen aufbaut
	\item
		Soziale Informationsverarbeitung: Informationsverarbeitung ist sozial weil:
		\begin{itemize}
			\item
				ihre Objekte sozial sind, auf soziale Sacheverhalte bezogen
			\item
				ihr Ursprung sozialer Natur ist (aus sozialer Interaktion entstanden)
			\item
				sie sozial geteilt ist, d.h. "ahnliche Kultur $\Rightarrow$ "ahnliche soziale Repr"asentationen
		\end{itemize}
	\item
		In Prozessstufen organisiert: Information  $\rightarrow$ Wahrnehmung  $\rightarrow$ Aufmerksamkeit $\rightarrow$ Kategorisierung $\rightarrow$ 
		Integration  $\rightarrow$ Verhalten, dazwischen Feedbackprozesse. Wahrnehmung + Aufmerksamkeit = Selektion, Kategorisierung + Integration = Inferenz
	\item
		Informationsverarbeitung/Speicherung:
		\begin{itemize}
			\item
				3 Stufen: Enkodierung, Speicherung, Abruf
			\item
				Reize $\rightarrow$ Sensorischer Speicher $\rightarrow$ Arbeitsspeicher/Kurzzeitged"achtnis $\rightarrow$ Langzeitged"achtnis, "uberall auch Verlust von unwichtigen Informationen
		\end{itemize}
	\item
		Selektion:
		\begin{itemize}
			\item
				Zwei Richtungen der Verarbeitungen
				\begin{itemize}
					\item
						Top-down: in allen Phasen der Informationsverarbeitung ist ein Schema aktiviert, Personeninformation kann im Lichte des Schemas verarbeitet werden = Erleichterung
					\item
						Bottom-up: in allen Phasen der Informationsverarbeitung kann kein Schema angewandet werden, Personeninformation muss Attribut f"ur Attribut verarbeitet werden
				\end{itemize}
			\item
				Auswahl von Information:
				\begin{itemize}
					\item
						nach Aufmerksamkeit
					\item
						Salienz (=pers"onliche Wichtigkeit)
					\item
						Lebhaftigkeit
					\item
						pers"onliche Relevanz
					\item
						nach vergangenen Erfahrungen und Wahrnehmungshypothesen
					\item
						nach gerade aktiviertem Wissen
				\end{itemize}
		\end{itemize}
	\item
		Inferenz:
		\begin{itemize}
			\item
				Kategorisierung: Zuordnung eines Reizinputs zu einer bedeutungsvollen Kategorie: Prototype (abstrakte Infos, Durchschnitt, Ideale) vs. Exemplarbasierte Repr"asentationen (Beispiel f"ur Kategorie), der Reizinput wird mit reiz-unabh"angigem Wissen "uber die Kategorie angereichert
			\item
				Integration: Der kategorisierte Reizinput wird zusammengefasst und mit weiterem im Ged"achtnis gespeichertem Material integriert: ein Urteil kommt zustande
		\end{itemize}
	\item
		Soziale Kognition: wird nach automatischem und kontrolliertem Denken unterschieden
\end{itemize}

\subsection{Automatisches Denken in Schemata}
\begin{itemize}
	\item
		Definition: \enquote{mentale Strukturen, mit denen der Mensch sein Wissen "uber die soziale Welt in Themenbereiche und Kategorien einordnet; sie beeinflussen die Informationen, die er wahrnimmt, "uber die er nachdenkt und die er abspeichert} (Aronson)
	\item
		= allgemeine Wissensstrukturen, welche die wichtigsten Merkmale eines Gegenstandsbereiches, sowie die Beziehungen zwischen den Merkmalen wiedergeben
	\item
		auf h"oherem Abstraktionsniveau als konkrete Erinnerung
	\item
		Inhalte:
		\begin{itemize}
			\item
				Personenschemata: Wissen "uber andere Menschen
			\item
				Selbstschemata: Wissen "uber eigene Person
			\item
				Rollenschemata: Wissen "uber soziale Rollen
			\item
				Ereignissschemata: Wissen "uber Alltagsaktivit"aten
			\item
				Inhaltsfreie Schemata: Allgemeine Denkregeln
		\end{itemize}
	\item
		Schubladenanalogie: Schema entspricht einer Schublade in einem Schrank, die mit einem bestimmten Etikett markiert sind, in dem die zugeh"origen Dinge gesammelt werden
	\item
		H"aufig benutzte Schemata sind leichter zug"anglich (frequency u. recency)  $\rightarrow$ Priming
	\item
		Funktion von Schemata: Umwelt organisieren und Sinn geben, Mustererkennung, Bedeutungsverleihung, Erinnerungshilfe
	\item
		Studie Kelley 1950: Bild von Mann und \enquote{Bekannten beschreiben ihn...}, zwei verschiedene Beschreibungen
	\item
		Studie Carli (1999): Geschichte, entweder Heiratsantrag oder Vergewaltigung, hinterher mehr falsch erinnerte Details, die zu dem entsprechenden Schema passen
	\item
		Aktivierte Schemata beeinflussen Interpretation von Informationen: nicht eindeutige Informationen werden im Licht des Schemas interpretiert (Assimilation)
	\item
		Perseveranz: auch wenn Schema falsch ist, wird daran festgehalten (Ross, Lepper und Hubbard 1975: Test, Versagens/Erfolgs-Feedback, selbsteinsch"atzung)
	\item
		Mit Perseveranz verkn"upft: Self-fulfilling prophecy: Schema von anderer Person, beinflusst Verhalten dieser Person gegen"uber, andere Person verh"alt sich erwartungskonform, best"atigt Erwartungen/Schema (Rosenthal u. Jacobosn 1968: Lehrer bekommt Info "uber angeblich gute Sch"uler, diese werden tats"achlich besser)
	\item
		Priming:
		\begin{itemize}
			\item
				Studie Higgins, Roles u. Jones 1977: Wortliste wiederholen, dann Personenbeschreibung Donald (Abenteurer), dann Donald beurteilen. Wenn in Wortliste positive W"orter, dann Beurteilung auch positiver
			\item
				Studie Bargh, Chen, Burrows 1996: W"orter zu grammatikalisch korrekten S"atzen ordnen. UV: W"orter die sich auf Stereotyp alter Menschen beziehen (oder eben nicht), AV: Geschindigkeit beim Verlassen des Labors  $\rightarrow$ auf alt geprimte VP gehen langsamer
		\end{itemize}
	\item
		Mentale Strategien und Abk"urzungen:
		\begin{itemize}
			\item Verf"ugbarkeitsheuristik beruht auf Leichtigkeit mit der bestimmter Ged"achtnisinhalt abrufbar ist. Studie Schwarz 1991: Beispiele f"ur selbst(un)sicheres Verhalten aufschreiben, entweder 6 oder 12. Hinterher Selbstsicherheit sch"atzen. 12 Beispiele schwieriger verf"ugbar als nur 6, einsch"atzung positiver (\enquote{War gerade schwierig so viele (12) Beispiele f"ur unsicheres Verhalten zu finden, dann bin ich wohl selbstsicher})
			\item
				Repr"asentativit"atsheuristik: Zuordnungsurteil beruhyt darauf wie "ahnlich Sachverhalt einem bestimmten Prototypen ist (Studie Kahnemann 1983: Linda ist Bankangestellte und Feministin... Wahrscheinlichkeit sch"atzen)
			\item
				Ankereffekt: Urteil/Sch"atzung wird bestimmtem Ausgangswert angepasst. Beispiel Tversky u. Kahnemann 1974: Afrikanische Staaten in der UNO, vorher Gl"ucksrad $\rightarrow$ gesch"atzter Anteil liegt nahe bei Gl"ucksrad
		\end{itemize}

\end{itemize}

\subsection{Kontrollierte soziale Kognition: Aufw"andiges Denken}
\begin{itemize}
	\item
		ist bewusst, zielgerichtet, willentlich und mit Aufwand, h"angt von Kapazit"at und Motivation ab
	\item
		Zur Kontrolle automatischer Prozesse
	\item
		Gilbert 1991: Theorie der automatischen Akzeptanz
		\begin{itemize}
			\item
				Menschen akzeptieren/glauben erst mal alles (automatisch)
			\item
				"Uberpr"ufen dann den Wahrheitsgehalt (kontrolliet)
			\item
				De-Akzeptieren wenn notwendig (kontrolliet)
		\end{itemize}
\end{itemize}

\subsection{Das Zusammenspiel automatischer und kontrollierter Prozesse}
\begin{itemize}
	\item
		Kontrafaktisches Denken: Vergleich mit dem was fast oder unter bestimmten Umst"anden eingetreten w"are
		\begin{itemize}
			\item
				Aufw"artsgerichtet (bessere Konstruktion): je leichter vorstellbar, dass negatives Ereignis nicht eingetroffen w"are, desto mehr negative Reaktionen 
			\item oder abw"artsgerichtet (schlechter Konstruktion): Je leichter vostellbar, dass noch negativeres Ereignis h"atte eintreten k"onnen, desto mehr positive Reaktionen
		\end{itemize}
		Kognitive Konsequenzen: lernen, zuk"unftig anders zu handeln

	\item
		Gedankenunterdr"uckung
		\begin{itemize}
			\item
				Operator (bewusst Ablenkung) und Monitor (automatische Suche nach unerw"unschten Gedanken, die bewusst werden sollen) f"ur Erfolg von Unterdr"uckung
	\item
		Ironische Prozesse: Versuch der Gedankenunterdr"uckung, diese treten aber gerade deswegen in den Vordergrund
		\end{itemize}
	\item
		Was ist wichtiger, automatisches oder kontrolliertes Denken?
		\begin{itemize}
			\item
				Verschiedene Modellvorstellungen "uber sozialen Denker:
				\begin{itemize}
					\item
						Kognitiver Geizkragen (=viel automatisch denken, kostet weniger)
					\item
						Motivierter Denker (wenn viel Zeit und Gelegenheit f"ur tiefere Verarbeitung vorhanden)
					\item
						Pragmatiker
				\end{itemize}
			\item
				Verbesserung von Denkprozessen
				\begin{itemize}
					\item
						Barriere der subjektiven Sicherheit (meisten Menschen setzen zu gro"ses Vertraue in eigenes Wissen und Richtigkeit eigener Urteile) durchbrechen
				\end{itemize}
		\end{itemize}
\end{itemize}
