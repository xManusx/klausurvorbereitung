
\section{Was ist Sozialpsychologie? Im Kanon anderer psychologischer Teildisziplinen; Historische und theoretische Perspektiven der Sozialpsychologie}
\subsection{Grundlegendes}
\begin{description}
	\item[Unabh"angige Variable] Einflussgr"o"se die systematisch variiert wird
	\item[Abh"angige Variable]
		Variable, deren Auspr"agung gemessen wird und "uber deren Ver"anderung in Abh"angigkeit von der UV Hypothesen aufgestellt werden
	\item[Kontrollvariable]
		Variable, die erhoben wird um m"ogliche St"oreinfl"usse zu kontrollieren
	\item[Interne Validit"at]
		gibt an, inwieweit in Untersuchung sichergestellt ist, dass Ver"anderungen in der AV nur durch die UV beinflusst werden
	\item[Externe Validit"at]
		Gibt an, inwieweit empirische Ergebnisse auf andere Situationen oder Personen verallgemeinerbar sind
\end{description}
\subsection{Was ist Sozialpsychologie?}
\begin{itemize}
	\item
		Psychologie allgemein: Lehre vom Denken, F"uhlen und Verhalten von Menschen
	\item
		Sozialpsychologie: Wie wird Denken, F"uhlen und Verhalten von Menschen durch die tats"achlich, vorgestellte oder implizite Anwesenheit von anderen beeinflusst
	\item
		Zentrale Fragestellungen der Vergangenheit:
		\begin{itemize}
			\item
				Wie vermittelt eine Generation ihre Kultur an die N"achcste?
			\item
				Was passiert mit dem Geistesleben einer Person, wenn sie Kontakt mit anderen hat?
			\item
				Sind Menschen einzigartig oder im Wesentlichen gleich?
			\item
				Ist das Individuum eine Funktion der Gesellschaft oder die Gesellschaft eine Funktion des Individuums
			\item
				Ist die \enquote{Natur} des Menschen egoistisch oder sozial?
			\item
				Kann der Mensch frei handeln oder ist er determiniert

		\end{itemize}

	\item
		Griechische Sozialphilosophen
		\begin{itemize}
			\item
				Platon (427 -- 347 v.Chr.)
				\begin{itemize}
					\item
						Gesellschaften und Staatsgebilde entstehen, weil Menschen sich nicht selbst gen"ugen und Hilfe von Anderen brauchen  $\rightarrow$  utilitaristische Sichtweise
					\item
						Ideale Staatsform: Aristokratie $\rightarrow$ Regeln der Philosophen bestimmen Zusammenleben
				\end{itemize}
			\item
				Aristoteles (384 -- 322 v.Chr.)
				\begin{itemize}
					\item
						Motiv Gesellschaften zu gr"Unden nicht utilitaristisch begr"undet, sondern naturgegeben
					\item
						Mensch ist ein \enquote{Zoon politikon}
					\item
						Ideale Staatsform: Demokratie
				\end{itemize}
		\end{itemize}

	\item
		Simple und sourver"ane Theorien im 19. Jahrhundert
		\begin{itemize}
			\item
				Wollten simple, einheitliche Erkl"arungen finden mit denen sich dann alles erkl"aren l"asst
			\item
				Z.B.:
				\begin{itemize}
					\item
						Freude vs. Schmerz: Hedonismus
						\begin{itemize}
							\item
								Von Bentham (1748 -- 1832): Menschen wollen positive Empfindungen erlangen und negative vermeiden
							\item
								\enquote{Homo oeconomicus}, der hedonistisches Kalk"ul vornimmt
							\item
								Im 20. Jhdt. von Freud (Lustprinzip) und Skinner (reinforcement, Konditionierung) vertreten
						\end{itemize}
					\item
						Egoismus/Macht
						\begin{itemize}
							\item
								Von Hobbes (1588 -- 1679) - Krieg aller gegen aller, Leviathan
							\item
								im 20. Jhdt Alfred Adler: Unterlegenheitsgef"uhl f"uhrt zu starkem Machtmotiv
						\end{itemize}
					\item
						Sympathie (Liebe), Soziabilit"at
						\begin{itemize}
							\item
								Adam Smith (1723 -- 1790): Mensch interessiert sich f"ur Schicksal von anderen. Zwei Formen: schnell und reflexiv (Mitf"uhlen) vs. "uberlegt und reflektiv (Einf"uhlen)
							\item
								Heute Bedeutung der Empathie f"ur prozosiales Handeln, z.B. Batson
						\end{itemize}
					\item
						Imitation/Suggestion (sozialer Einfluss)
						\begin{itemize}
							\item
								Imitation z.B. Tarde (1843 -- 1904): \enquote{Gesellschaft ist Imitation}, im 20. Jhdt Bandura, Modell-Lernen
							\item
								Suggestion z.B. Le Bon (1895): Einfluss der \enquote{Masse}, im 20. Jhdt z.B. Cialdini: sozialer Einfluss als Kernthema der Sozialpsychologie
						\end{itemize}
				\end{itemize}
			\item
				Mensch ist aber nicht durch einen Faktor erkl"arbar, gennannten sind aber alles wichtige Komponenten
			\item
				F"ur Sozialpsychologie besonders wichtig: Sozialer Einfluss (vgl. Definition Allport von oben)
		\end{itemize}

\end{itemize}


\subsection{Anf"ange der Sozialpsychologie im 19. Jhdt}
\begin{itemize}
	\item
		Evolutionstheoretiker, z.B. Darwin: Arbeiten zum Emotionsadruck und dessen Entwicklung
	\item
		Anthropologie/V"olkerpsychologie: Betonung der menschlichen Gemeinschaft und des \enquote{Volksgeistes}
	\item
		Massenpsychologie: Individum beherrscht, rational $\leftrightarrow$ Masse impulsiv, irrational, primitiv. Wie k"onnen Massen beeinflusst, kontrolliert, mobilisiert werden? (z.B. LeBon \enquote{Suggestion})
	\item
		1898 erstes sozialpsychologisches Experiment (Triplett): Leistungssteigerung durhc Publikum/Wettbewerb
	\item
		1908 ersten beiden sozialpsychologischen Lehrb"ucher Ross u. McDougall
\end{itemize}
\subsection{Sozialpsychologie im 20. Jhdt}
\begin{itemize}
	\item
		Konsolidierungsphase: Selbstdefinition, Bem"uhen wissenschaftliche Standards zu erreichen. Themen: sozialer Einfluss auf Wahrnehmung und Ged"achtnis, Gruppenprozesse, Einstellungen und Einstellungsmessung
	\item

		Boom nach dem zweiten Weltkrieg (Emmigration vieler europ"aischer Forscher in die USA (z.B. Lewin, Heider)
	\item
		Kognitive Wende: Behaviorismus  $\rightarrow$ kognitive Orientierung
	\item
		Wechsel von Theorien mit breitem zu solchen mit engem Geltungsbereich
	\item
		Ziel: praktischen Beitrag zur L"osung wichtiger gesellschaftlicher Probleme leisten
	\item
		In 60er/70er Jahre Forschungsboom aber auch Aufweichung des Faches und Kritik an experimentellen Methoden
	\item
		Wichtige Theorien:
		\begin{itemize}
			\item
				Soziale Vergleichstheorie (Festinger): Vergleich mit Minschen hat 3 m"ogliche Funktionen\footnote{vgl. \href{https://de.wikipedia.org/wiki/Theorie_des_sozialen_Vergleichs}{Wikipedia zu Theorie des sozialen Vergleichs}}
				\begin{itemize}
					\item
						Wer realistische Informationen "uber sein gegenw"artiges Selbst ben"otigt vergleicht sich mit "Ahnlichen, Gleichgestellten, Peers
					\item
						Wer sein Selbstwertegef"uhl sch"utzen oder verbessern will, vergleicht sich mit Menschen, die im interessierenden Merkmal unterlegen sind: der abw"arts gerichtete Vergleich
					\item
						Wer wissen will, welche M"oglichkeiten er hat, welche Verbesserungen m"oglich sindk vergleicht sich mit Menschen die im interessierenden Merkmal "uberlegen sind: der aufw"arts gerichtete Vergleich
				\end{itemize}
			\item
				Dissonanztheorie (Festinger): Zwei Hypothesen
				\begin{enumerate}
					\item
						Das Vorhandensein von Dissonanz wird als unangenehmer Spannungszustand erlebt und "ubt auf den Menschen Druck aus, diesen zu beseitigen oder zu reduzieren. Die St"arke des Drucks zur Dissonanzreduktion ergibt sich aus der St"arke der Dissonanz
					\item
						Die Dissonanz zwischen zwei kognitiven Elementen kann nicht gr"o"ser sein, als notwendig ist, um das weniger widerstandsf"ahige der beiden Elemente zu "andern. Der Grund ist, dass sich zum Zeitpunkt gr"o"stm"oglicher Dissonanz das weniger widerstandsf"ahige Element "andern w"urde - die Dissonanz w"are beseitigt
				\end{enumerate}
			\item
				Kognitive Lerntheorien (z.B. Bandura): Zwei Phasen mit jeweils zwei Prozessen
				\begin{enumerate}
					\item
						Aneignungsphase (Kompetenz, Akquisition): Aufmerksamkeitsprozesse (Verhalten des Vorbilds wird beobachtet, wichige Bestandteile des Verhaltens ausgew"ahlt) und Ged"achtnisprozesse (Akkomodation u. Assimilation)
					\item
						Ausf"urhungsphase(Performanz): Motorische Reproduktionsprozesse (Lerner erinnert sich und versucht das Beobachtete zu reproduzieren) und Verst"arkungs und Motivationsprozesse (Lernen h"angt von Motivation ab, Motivation ist eng mit Aussicht auf Bekr"aftigung verbunden)
				\end{enumerate}
			\item
				Attributionstheorien (z.B. Kelley):  \\
				\includegraphics[scale=0.4]{Attributionstheorie_nach_Harold_Kelly.png}
			\item
				Theorien sozialer Identit"at und Gruppenzugeh"origkeit (z.B. Tajfel):
		\end{itemize}
	\item
		In Europa geht Sozialpsychologisch gesehen nicht so viel
\end{itemize}

\section{Theoretische Perspektiven der Sozialpsychologie}
\begin{itemize}
	\item
		Macht des sozialen Einflusses wird meist untersch"atzt (erkl"aren Verhalten von anderen Menschen meist aus ihren Eigenschaften)
		\begin{itemize}
			\item
				Studie Ross\&Samuels 1993: Wall-Street vs. Gemeinschaftsspiel, anderer Name, gleiche Regeln. Gemeinschaftsspiel VP deutlich kooperativer, unabh"angig davon, ob sie vorher als konkurrenzorientiert oder kooperativ beschrieben wurden
			\item
				Subjektive Wahrnehmung der Situation ist wichtig (vs. Behaviorismus: Gedanken und Gef"uhle egal, nur Belohnungen und Strafen im Umfeld wichtig)
			\item
				Zwei Bed"urfnisse bei der subjektiven Interpretation der Welt:
				\begin{enumerate}
					\item
						Bed"urfnis realistisch zu sein: Ansatz der sozialen Kognition, Mensch als \enquote{Amateurdetektiv}  $\rightarrow$  Attributionstheorien, z.B. Theorie der Kausalit"at von Heider
					\item
						Bed"urfnis mit sich selbst zufrieden zu sein (der \enquote{Wunschdenker} oder \enquote{Konsistenzsucher})  $\rightarrow$ Konsistenztheorien, z.B. Dissonanztheorie von Festinger
				\end{enumerate}
		\end{itemize}

	\item
		Sozialpsychologie betrachtet 4 Ebenen:
		\begin{itemize}
			\item
				Intrapersonale Perspektive: z.B. soziale Informationsverarbeitung:
			\item
				Interpersonale Perspektive: Interaktion zwischen Personen, z.B.  Hilfeverhalten, Freundschaft
			\item
				Intragruppale Perspektive: Interaktion in Gruppen, z.B. Konformit"at und Abweichung
			\item
				Intergruppale Perspektive: Interaktion zwischen Gruppen, z.B. Diskriminierung von Fremdgruppen
		\end{itemize}
	\item
		Kann in verschiedenen Zeitperspektiven betrachtet werden:
		\begin{itemize}
			\item
				Situativer Einfluss (hier und jetzt)
			\item
				Entwicklungs und Sozialisationseinfluss (Ontogenese)
			\item
				Evolutions- und Kultureinfluss (Phylogenese):
				\begin{itemize}
					\item
						Alle Organismen sind komplexe Systeme mit adaptivem Design, Nebenprodukten und Zufallsrauschen
					\item
						Evolution"are Psychologie: metatheoretischer Ansatz, baut auf Evolutionstheorie auf
				\end{itemize}
		\end{itemize}
\end{itemize}
