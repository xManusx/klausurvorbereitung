\section{Denken und F"uhlen}
\subsection{Gef"uhle}
\begin{itemize}
	\item
		Abgrenzungen zwischen Gef"uhlen, Affekten, Emotionen und Stimmungen:
		\begin{itemize}
			\item
				Stimmungen: Gef"uhle, die weniger intensiv sind als Emotionen und nicht unbedingt ein Bezugsobjekt haben, h"aufig unbekannte Ursachen und l"anger andauernd
			\item
				Emotionen: starke Gef"uhle, die auf Person oder Gegenstand gerichtet sind
			\item
				Affekt: bezieht sich auf Valenz (= positive und negative Aspekte) von Dingen, Oberbegriff f"ur breites Spektrum an Gef"uhlen. Sowohl Emotionen als auch Stimmungen
			\item
				Gef"uhl: umgangssprachlich f"ur Vielzahl affektiver und nicht-affektiver Gef"uhle (z.B. Anstrengung f"ur nicht-affektiv)
		\end{itemize}
		$\Rightarrow$ Gef"uhle teilen sich auf in nicht affektive Gef"uhle und Affekte, diese spalten sich auf in Emotionen und Stimmungen

	\item
		Funktionen von Emotionen:
		\begin{itemize}
			\item
				Informative Funktion (Situation gut/schlecht?)
			\item
				Motivationale Funktion (Angst? Lauf weg!)
			\item
				Kommunikative Funktion (Zeig gegen"uber, dass du unzufrieden bist)
		\end{itemize}

	\item Somatic marker Hypothesis (Damasio 1994)
			\begin{itemize}
				\item
					Im Pr"afrontalen Cortex (vorne im Kopf) sitzen \enquote{Somatische Marker}, die bei Entscheidungen aktiviert werden k"onnten
				\item
					Somatische Marker bestehen aus Erfahrungen, die Alternativen und Konsequenzen einer Handlung mit positiven oder negativen Emotionen belegt.
				\item
					K"onnen diese nicht mehr benutzt werden (Verletzung im Kopf oder so), wird schnelles, adaptives Entscheiden (was ziehe ich heute an?) unm"oglich.
				\item
					Im Gegensatz dazu auch \enquote{High reason}: Abw"agen, nachdenken, dauert lange. Somatic Marker sind schnell
			\end{itemize}
\end{itemize}

\subsection{Stimmung und Denken}
\begin{itemize}
	\item
		Stimmungen entstehen durch abklingende Emotionen, Erinnerungen, tats"achliche Ereignisse, physiologische Zust"ande, \enquote{facial feedback} (eigener Gesichtsaudruck)
	\item
		Funktionen von Stimmungen:
		\begin{itemize}
			\item
				Stimmung und Verarbeitungsstil
				\begin{itemize}
					\item
						Negative Stimmungen signalisieren etwas ist nicht i.O. $\rightarrow$ detailorientierte bottom-up Verarbeitung
					\item
						Positive Stimmungen signalisieren i.O. $\rightarrow$ globale top-down Verarbeitungsstrategie
				\end{itemize}
 $\rightarrow$ Isen et.al. 1987 Kerzenaufgabe: Kerze an Korkwand befestigen, dass nichts auf Boden tropft. Muss out-of-the-box denken, um auf L"osung zu kommen. In neutraler Stimmung schaffen nur 20\% die Aufgabe, in lustiger Stimmung 75\%
			\item
				Stimmung und Wahrnehmung
				\begin{itemize}
					\item
						Spezifische Stimmungen wirken wie Netzwerkknoten, Verf"ugbarkeit f"ur emotionskongruentes Material wird erh"oht
					\item
						Niedenthal 1997: lexikalische Entscheidungsaufabe, w"ahrenddessen traurige/fr"ohliche Musik, W"orter die zu aktueller Stimmung passen (\enquote{fun} bei fr"ohlicher Musik) werden schneller erkannt
				\end{itemize}
			\item
				Stimmung und Erinnerung
				\begin{itemize}
					\item
						Verf"ugbarkeit von gespeichertem Material h"angt von mood congruity/Stimmungskongruenz ab
					\item
						In einem bestimmten emotionalen Zustand Gelerntes ist sp"ater in demselben Zustand leichter abrufbar als in einem anderen
					\item
						Bowser 1981: Wortlisten in positiver/negativer Stimmung lernen, sp"ater in positiver/negativer Stimmung Pr"ufung. Bei gleicher Stimmung besseres Ergebnis
				\end{itemize}
			\item
				Stimmung und Urteilsbildung
				\begin{itemize}
					\item
						Stimmungskongruenzhypothese von Mayer et.al. 1992
					\item
						How do i feeel about Heuristik - \enquote{Ist die Empfindung angenehm ist auch der Urteilsgegenstand gut!} (vgl. Nach Kino, ob Film gut war: Film verantwortlich f"ur Stimmung, brauchbares Urteil, vs. Telefoninterview nach Wetter: Wetter verantwortlich f"ur Stimmung)
					\item
						Abwertungsprinzip: Wenn externale Erkl"arungen f"ur momentane Befindlichkeit besteht, verliert Stimmung ihren Einfluss (Telefoninterview Frage nach Wetter)
					\item
						Abele u. Gendolla 1999: pos. bzw. neg. Passage in Liebesfilm anschauen, hinterher Fragen beantworten. Starker Kontrasteffekt bei Beziehung zum eigenen Partner (pos. Film, neg. Bewertung), bei anderen Lebensbereichen signifikanter Assimlationseffekt 
					\item
						Embodiement von Gef"uhlen, unauff"alliges Facial Feedback: VP sollen Stife testen, zwischen Z"ahnen, zwischen Lippen oder in der Hand. Finden Cartoons lustiger mit Stifen zwischen Z"ahnen (weil Lachen)
				\end{itemize}
		\end{itemize}
\end{itemize}
