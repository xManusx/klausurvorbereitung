\section{Sprache und Kommunikation}
\subsection{Kommunikation und Sprache}

\begin{itemize}
	\item
		"Ubertragung("uber einen Kommunikationskanal/medium) bedeutungsvoller Information(=eine Sendung) von einer Person(Sender) zur anderen (Empf"anger)
	\item
		Sprache f"ur Menschen wichtigstes Kommunikationsmedium
	\item
		Sprache: Ein system von T"onen die aufgrund grammatischer und semantischer Regeln Bedeutung vermitteln. Kann sein:
		\begin{itemize}
			\item
				Ein Laut (z.B. \enquote{Psst!})
			\item
				Lokation: Worte in einer Sequenz, z.b. \enquote{es ist hei"s hier}
			\item
				Illokation: Sequenz und Kontext, z.B. \enquote{es ist hei"s hier} kann Feststellung, Kritik oder Aufforderung sein
			\item
				Kulturelle Regeln/unterschiede bei Benutzung von Sprache: z.B. High Context (Starker Kontextbezug, viele Sachen werden nicht direkt ausgesprochen) vs Low Context (Schwacher Kontextbezug, alles wird direkt beim Namen genannt, pr"azise Angaben)
		\end{itemize}
	\item
		Searle 1975 unterscheide 5 Arten von Bedeutung die Sprache vermitteln kann:
		\begin{itemize}
			\item
				Feststellung/Beschreibung
			\item
				Aufforderung
			\item
				Befindlichkeiten ausdr"ucken
			\item
				Eine Verpflichtung eingehen
			\item
				Etwas direkt in Angriff nehmen
		\end{itemize}
	\item
		Sprache und...

		\begin{itemize}
			\item
				\dots Denken: Sapir u. Whorf 1956 Hypothese der Linguistischen Relativit"at: die Sprache bestimmt das Denken, Menschen die verschiedene Sprachen sprechen sehen die Welt anders. Heutige relativierte Form: Sprache bestimmt nicht das denken, aber das Vorhandensein von Worten (=linguistischen Kategorien) "uber Sachverhalte erm"oglicht leichteres Denken und Kommunizieren dar"uber
			\item
				\dots Sprechen: Parasprache bezeichnet nicht-linguistischen Komponenten der Sprache (=Prosodie). Sprachstil ist nicht was, sondern \emph{wie} etwas gesagt wird -- besonders wichtig bei sozialpsychologischen Forschungen
			\item
				\dots Identit"at: Menschen beziehen ihr Selbstwertgef"uhl auch aus Mitgliedschaft in Gruppe (= Theorie der sozialen Identit"at), diese Soziale Identit"at wird auch "uber Sprache ausgedr"uckt (z.B. t"urkisch in t"urkischer Community in DE)
			\item
				\dots Altersgruppen: Babysprache j"ungerer gegen"uber sehr alten Menschen, Generationen unterscheiden sich in Sprachnutzung
		\end{itemize}
	\item
		Sprache verschiedener Gruppen, z.B. Geschlecht:
		\begin{itemize}
			\item
				Stereotype "uber M"anner und Frauensprache sind bekannt
			\item
				Befunde: Frauen reden nicht mehr als M"anner, aber sprechen weniger \enquote{m"achtige} Sprache (mehr Verst"arker, mehr Absicherungen, mehr Best"atigungsfragen)
		\end{itemize}
\end{itemize}



\subsection{Nonverbale Kommunikation}
\begin{itemize}
	\item
		= die "ubertragung bedeutungsvoller Informationen von einer auf die andere Person mit anderen Mitteln als der geschriebenen oder gesprochenen Sprache
	\item
		Frauen sind h"aufig besser in Dekodierung nonverbaler Kommunikation als M"anner

	\item
		Bei Vetrauten Menschen funktioniert Dekodierung meist besser als bei Fremden
	\item
		Blick und Blickkontakt:
		\begin{itemize}
			\item
				In 2-Personen Gespr"achen ca. 61\% der Zeit Anschauen, etwa 31\% Blickkontakt
			\item
				Mehr Blickkontakt ist mehr Intimit"at, Fehlen von Blickkontakt immer st"orend
			\item
				Mehr Schauen beim Zuh"oren als beim Sprechen
			\item
				Personen mit niedrigerem Status schauen Statush"ohere l"anger an, aber h"ohere halten Blickkontakt l"anger aufrecht
			\item
				Blick kann sowohl Inimit"at als auch Dominanz ausdr"ucken
		\end{itemize}
	\item
		Gestik und K"orperhaltung
		\begin{itemize}
			\item
				H"aufig Hand- und Armgesten, sprachbegleitend
			\item
				Embleme = spezifische Gesten, die eine genaue Bedeutung haben, kulturspezifisch
			\item
				Ber"uhrungen signalisieren wieder sowohl Intimit"at als auch Status (M"anner ber"uhren Frauen h"aufiger als umgekehert)
			\item
				Proxemik = Untersuchung der interpersonellen Distanz. Intime Distanz bis 0.5m, Pers"onliche Distanz 0.5 -- 1.25m, Soziale Distanz 1.25--4m, "Offentliche Distanz $>$ 4m
		\end{itemize}

	\item
		Nonverbale Kommunikation kann auch eingesetzt werden um zu t"auschen oder bestimmten Eindruck zu vermitteln
	\item
		L"ugner...
		\begin{itemize}
			\item
				bleiben sprachlich relativ ungenau
			\item
				Sprachmelodie leicht erh"oht
			\item
				mehr Selbstber"uhrungen
			\item
				mehr Selbstber"uhrungen

		\end{itemize}
\end{itemize}


