
\section{Soziale Wahrnehmung, Eindrucksbildung, Attribution}
\subsection{Eindrucksbildung und Attribution}
\begin{itemize}
	\item
		Stadien der Personenwahrnehmung:
		\begin{enumerate}
			\item
				Erster, "au"serer Eindruck/snap judgements
				\begin{itemize}
					\item
						Physische Erscheinung
					\item
						verbale Merkmale
					\item
						Verhalten
					\item
						nonverbale(mimik, gestik) und paraverbale(Stimmlage, Lautst"arke, Artikulation)  Merkmale
						\begin{itemize}
							\item
								Aktivit"atsreize/Reaktionsbereitschaft: Aktivit"at des Gesichts, Mimik, Gestik, Stimme, Bein- und Fu"sbewegungen
							\item
								Entspanntheitsreize (soziale Kontrolle: Neigung des K"orpers, Armhaltung, Stellung der Beine
							\item
								Unmittelbarkeitsreize (Sympathie): Blickkontakt, K"orperorientierung, Interpersonale Distanz, Ber"uhrungen
						\end{itemize}
				\end{itemize}
			\item
				Verhaltensbeobachtung  $\rightarrow$ Eigenschaftsimplikationen
			\item
				Eindrucksbildung "uber die Pers"onlichkeit
			\item
				Vorhersage zuk"unftigen Verhaltens
				\begin{itemize}
					\item
						Erkl"arung von beobachtetem Verhalten von zentraler Bedeutung
					\item
						Attributionstheorien: Wie ziehen wir Schl"usse "uber die Gr"unde f"ur das Verhalten anderer?
					\item
						Kausalattribution = Zuschreibung von Ursachen, zwei grundlegende Attributionsarten:
						\begin{itemize}
							\item
								internale Attribution (bevorzugt)
							\item
								externale Attribution (Umwelt/Situation)
						\end{itemize}
				\end{itemize}
		\end{enumerate}
	\item
		Kelley (1967): Kovariationsprinzip: - Frage nach:
		\begin{itemize}
			\item Konsensus: Wie verhalten sich andere Menschen gegen"uber demselben Stimulus? (Niedriger Konsensus = Andere Menschen verhalten sich ganz anders)
			\item
				Distinktheit: Wie verh"alt sich die Person gegen"uber "ahnlichen Stimuli? (Niedrige Distinktheit = verh"alt sich eigentlich immer so in "ahnlichen Situationen)
			\item
				Konsistenz: wird das spezifische Verhalten nur einmal gezeigt oder h"aufiger? (Niedrige Konsistenz = wird nur einmal gezeigt)
		\end{itemize}
		Sp"atere Studien zeigen v.a. Konsensus und Distinktheit entscheidend, Konsistenz wird h"aufig vernachl"assigt
\end{itemize}

\subsection{Typische Attributionsfehler}
\begin{itemize}
	\item
		Der fundamentale Attributionsfehler
		\begin{itemize}
			\item
				Auch correspondence bias/Korrespondenzverzerrungt
			\item
				Menschen neigen dazu Verhalten internal zu attribuieren
			\item
				Einfluss externaler Faktoren wird untersch"atzt
			\item
				Studie von Ross 1977:
				\begin{itemize}
					\item
						3 VP Quizmaster, Kandidat und Beobachter
					\item
						Quizmaster denkt sich Fragen aus, Kandidat kann sie meistens nicht beantworten
					\item
						Hinterher bewertet jeder Quizmaster und Kandidat: Nur Quizmaster bewertet Kandidat nicht schlecht und Quizmaster gut, alle anderen schon, obwohl jeder wei"s, dass Quizmaster die Fragen ausdenkt (\enquote{Quizmaster wei"s etwas, was Kandidat nicht wei"s, also muss QM schlauer sein!})
				\end{itemize}
			\item
				Jones u. Harris 1967: Aufsatz "uber Castros Cuba, entweder Pro/Contra + Information, dass Position freiwillig gew"ahlt wurde vs. zugeordnet
				\begin{itemize}
					\item
						Hinterher vermutete Einstellung des Verfassers
					\item
						freiw/pro 60, zugeordnet/pro 45, freiwillig/anti 15, zugeordnet/anti 25
				\end{itemize}
			\item
				M"ogliche Gr"unde:
				\begin{itemize}
					\item
						Aufmerksamkeit: Wollen Ursache f"ur Verhalten ergr"unden, damit ist Aufmerksamkeit mehr auf Person gerichtet, nicht auf Situation
					\item
						Perzeptuelle Salienz: (Taylor u. Fiske 1975) Saliente und auff"allige Objekte gro"ser Einfluss auf Ursachenzuschreibung. Zwei diskutierende Akteure, VP kann entweder einem, dem anderen oder beiden ins Gesicht sehen. Wer nur einen Akteur sieht denkt, derjenige bestimmt die Diskussion
					\item
						Kultureinfluss: Menschen aus kollektivistischen Kulturen (Asien) machen Attributionsfehler seltener (Sportartikel aus Hongkong und USA, USA mehr dispositionale Attributionen, aus Hongkong mehr situative Attribution
					\item
						Zwei-Stufen-Prozess
						\begin{enumerate}
							\item
								personale Attribution: automatisch, schnell, wenig Anstrengung
							\item
								Situation wird mit einbezogen: bewusst, nur wenn Motivation und gen"ugend freie Ressourcen da sind, selbst dann Ankereffekt
					\end{enumerate}
				\end{itemize}

		\end{itemize}

	\item
		Akteur-Beobachter Divergenz
		\begin{itemize}
			\item
				Tendenz das Handeln vor allem internal zu attribuieren, bei eigener Person aber externale Aspekte st"arker zu ber"ucksichtigen (Studie "ahnlich wie Taylor zur perzeptuellen Salienz aber mit Videos, hinterher noch aus anderer Perspektive
		\end{itemize}
	\item
		Selbstwertdienliche Attribution
		\begin{itemize}
			\item
				Erfolge werden internal stabil und variabel attribuiert (Stabil vs variabel: intern: F"ahigkeit vs. Anstrengung oder extern: Schwierigkeit vs. Zufall)
			\item
				Misserfolge werden external (oder internal variabel) attribuiert
		\end{itemize}
	\item
		Falscher Konsensus Effekt (egocentric bias)
		\begin{itemize}
			\item
				"Ahnlichkeit der Meinungen anderer mit der eigenen Meinung wird "ubersch"atzt
			\item
				Krueger u. Clement 1994: Sch"atzung Prozentsatz Zustimmung bzw. Ablehnung bei Anderen von S"atzen, denen man selbst zustimmt oder ablehnt (\enquote{ Ich lese gerne Liebesgeschichten, also glaube ich dass 53\% der anderen auch gerne liebesgeschichten lesen!} vs. 47\% bei nicht gern Liebesgeschichtenlesern
		\end{itemize}
\end{itemize}

\subsection{Integrationsregeln der Eindrucksbildung}
\begin{itemize}
	\item
		\begin{itemize}
			\item
				Primacy Effect: Der erste Eindruck ist entscheidend
			\item
				Recency Effect: der neueste (most recent) Endruck ist entscheidend
			\item
				Negativit"atseffekt: Negative Informationen st"arkeres Gewicht als positive
			\item
				Exremit"atseffekt: Extreme Infos st"arkere Gewichtung als mittlere
			\item
				Kontexteffekt: Konzept der \enquote{zentralen Eigenschaft}, h"aufig kontextabh"angig
		\end{itemize}
	\item
		Beispiel Studie Aisch (1946):
		\begin{itemize}
			\item
				VPN bekommen Liste mit 7 W"ortern, zwei Gruppen, einziger Unterschied warmherzig/kalt
			\item
				AV: Personenbeschreibung
			\item
				vgl. Kelley 1950
		\end{itemize}
\end{itemize}

\subsection{Implizite Pers"onlichkeitstheorien}
\begin{itemize}
	\item
		Entwickelt jeder Mensch im Laufe der Sozialisation individuell
	\item
		Welche beobachtbaren Merkmale sehe ich bei anderen Person? Welche R"uckschl"usse kann ich daraus ziehen? Wie bewerte ich diese R"uckschl"usse?
	\item
		Durch Perseveranzeffekt nachhaltige Beeinflussung
	\item
		z.B. Halo-Effekt: von bekannten Eigenschaften wird auf unbekannte Eigenschaften geschlossen, der erzeugte positive oder negative Eindruck "uberstrahlt die weitere Wahrnehmung der Person
	\item
		Confirmation bias: Tendenz Informationen zu suchen, bzw. zu interpretieren, dass bereits bestehende Meinungen best"atigt werden
		\begin{itemize}
			\item
				Perseveranzeffekt spielt mit rein
			\item
				M"ogliche Gegenma"snahmen: Unsicherheit "uber vorgefasste Meinung oder hohe Motivation, richtigen Eindruck zu bilden
		\end{itemize}
\end{itemize}
