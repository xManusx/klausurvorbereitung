\section*{1. Introduction}
\textbf{TODO: Performance Evaluation}
\subsection*{Patter Recognition Pipeline}
\begin{align*}
    \textbf{recording} \rightarrow \textbf{preprocessing} \rightarrow \textbf{feature extraction} \rightarrow \textbf{classif}& \textbf{ication} \rightarrow\\
    & \uparrow\\
    \textbf{training set} \rightarrow \textbf{trai} & \textbf{ning} 
\end{align*}

\subsection*{Posulates of Pattern Recognition}
\begin{enumerate}
    \item
        Availability of a representative sample $\omega$ of patterns $^if(x)$ for a given field of problems $\Omega$
        $$\omega = \{^1f(x),\dots,^Nf(x)\}\subseteq \Omega$$
    \item
        A (simple) pattern has features, which characterize its membership in a certain class $\Omega_{\kappa}$
    \item
        Compact domain in the feature space of features of the same class;\\ 
        domains of different classes are (reasonably) seperable.
        \begin{itemize}
            \item
                small intra-class distance
            \item
                high inter-class distance
        \end{itemize}
    \item
        A (complex) pattern consists of simpler consituents, which have certain relations to each other. A pattern may be decomposed into the constituents.
    \item
        A (complex) pattern $f(x) \in \Omega$ has a certain structure. Not any arrangement of simple constituents is a valid pattern. Many patterns may be represented with relatively few constituents.
    \item
        Two patterns are similar if their features or simpler constituents differ only lightly.
\end{enumerate}

\subsection*{Performance Evaluation}
    \textbf{TODO}
