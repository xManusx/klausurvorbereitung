\section{Homogene und heterogene Multikern-Architekturen}
\subsection{Motivation Multikern Architekturen}
\begin{itemize}
	\item
		Warum multicore?
		\begin{itemize}
			\item
				Bis 2002 leistungssteigerung primär durch höhere Taktzahlen
			\item
				Erhöhung der Taktfrequenz stößt an Grenzen (Wärme) \Ra{} Verbesserungen durch intelligentere Architekturen, aber dynamische Sprungvorhersage bereits 95\% Trefferquote, geht jetzt auch nicht mehr viel
		\end{itemize}
	\item
		Mehrere Prozessorkerne auf einem Chip
	\item
		Pollacks Regel:
		\begin{itemize}
			\item
				Rechenleistungszuwachs {\raise.17ex\hbox{$\scriptstyle\sim$}} $\sqrt{\text{Anstieg Komplexität}}$
			\item
				Verdopplung der Logik in Prozessor \ra 40\% mehr Leistung
		\end{itemize}
	\item
		Weitere Vorteile: Einzelkerne an und ausschaltbar, immer optimale Versorgungsspannung und Frequenz, Rechenlast gleichmäßiger Verteilen
	\item
		Umgekehrte  Anwendung der Regel von Pollack: im oberen beherrschabren Energieverbrauchs Spektrum bleiben und Anzahl der Kerne erhöhen (statt $1\times1000000000$ lieber $100\times10000000$ Transistoren)
		\begin{itemize}
			\item
				Leistung nimmt invers quadratisch ab: auf halber Fläche 70\% der Leistung des größeren Systems möglich
			\item
				Leistungsverbrauch pro Kern nimmt linear ab/zu
			\item
				Durchsatz steigt annähernd linear mit größerer Anzahl Kerne
		\end{itemize}
	\item
		Nach Amdahls Law ist Geschwindigkeitszuwachs durch parallelisierung limitiert: Gilt jedoch nur für eine auf allen Kernen laufende Applikation. Häufig jedoch viele Applikationen
	\item
		Roofline Modell:
		\begin{itemize}
			\item
				Wichtige Größe: Operationelle Intensität -- Anzahl der Operationen pro geholtem Byte [Flops/Byte], messgröße für Verkeher zwischen DRAM und Cache
		\end{itemize}
\end{itemize}
\subsection{Leistungsanalyse und -bewertung homogener und heterogener Multikern Prozessoren}
\subsection{Beispiel homogene Multikern-Architekturen}
\subsubsection{Intel Skylake, Haswell, Sandy Bridge}
\subsubsection{ AMD Ryzen}
