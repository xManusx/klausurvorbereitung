
\section{Entwicklung von Wahrnehmung und Motorik}
\subsection{Wahrnehmung}
\begin{itemize}
	\item
		Empfindung: Verarbeitung basaler Information aus der "au"seren Welt in Sinnesorganen und neuronalen Verb"anden im Gehirn
	\item
		Wahrnehmung: Prozess der Strukturierung und Interpretation von Sinnesinformation
\end{itemize}

\subsubsection{Sehen}
\begin{itemize}
	\item
		Schon wenige Minuten nach Geburt fangen fangen Neugeborene damit an Welt visuell zu erkunden
	\item
		Blickpräferenz: VErfahren zur Untersuchung von visueller Aufmerksamkeit bei S"auglingen, zwei Muster/Objekte vorgelegt, wird untersucht welches bevorzugt wird
		\begin{itemize}
			\item
				Kinder bevorzugen Muster "uber unstrukturierter Oberfl"ache
			\item
				grobes Zebramuster neben feinem Zebramuster, Kinder schauen dahin wo sie noch ein Muster erkennen k"onnen und nicht nur eine graue Fl"ache\\
				$\Rightarrow$ 
				Bereits mit 8 Monaten Sehverm"ogen "ahnlich wie Erwachsene

			\item
				Lieber Viel Kontrast, da Kleinkinder nur geringes Kontrastverm"ogen: Nur 2\% des Lichts das in die Augen f"allt erreicht Zapfen (Erwachsene bis zu 65\%)
			\item
				Mit 1 Monat Fokus auf Kontur/Au"senkanten, mit 2 Monaten Fokus sowohl Gesamtform als auch Details im Inneren verarbeitbar
		\end{itemize}
	\item
		Wahrnehmungskonstanz (Objekt wird kleiner $\rightarrow$ geht weiter weg, ver"andert gr"o"se nicht) schon bei Neugeborenen, Objekttrennung auch bei S"auglingen (daf"ur gemeinsame Bewegung der Objekte wichtig)
	\item
		Schon fr"uh empf"anglich f"ur Objektausdehnung (Objekt wird gr"o"ser $\rightarrow$ kommt auch mich zu)
\end{itemize}

\subsubsection{Akustische Wahrnehmung}
\begin{itemize}
	\item
		Akutisches System bei Geburt gut entwickelt, Neugeborene aber etwas schwerh"orig
	\item
		Akutische Lokalisation schon 10 Minuten nach Geburt
	\item
		M"oglicherweise angeborene, biologische Grundlage f"ur Musikwahrnehmung (Musik und Sprache "ahnliche Lernmechanismen und Gehiranktivit"aten), S"auglinge pr"aferenz f"ur konsonante (vs. dissonante) Melodien, Schon 5 Monate alte Kinder nehmen Melodie gleich war, egal wie hoch oder tief gespielt, als nicht gleich, wenn T"one vertauscht sind
\end{itemize}

\subsubsection{Ber"uhrung}
\begin{itemize}
	\item
		Anfangs viel mit Mund/Zunge
	\item
		Ab ca. 4 Monaten mehr Arm/Hand
\end{itemize}

\subsubsection{Intermodale Wahrnehmung}
\begin{itemize}
	\item
		= Kombination von Informationen aus zwei oder mehr Sinnessystemen
	\item
		Zusammenfassung Siegler 258
\end{itemize}

\subsection{Motorik}
\begin{itemize}
	\item
		Babys erst mal nur Reflexe: Greifreflex, Suchreflex (Ber"uhrung Wange $\rightarrow$ Kopfdrehung und Mund"offnen), Saugreflex
\end{itemize}
\subsubsection{Meilensteine der Motorik}
\begin{itemize}
	\item
		Siehe Siegler 261
\end{itemize}


\subsubsection{Aktuelle Perspektiven}
\begin{itemize}
	\item
		Fr"uher: neuronale Reifung des Gehirns $\rightarrow$ motorische Entwicklung
	\item
		Heute eher: Dynamische Systeme, Zusammenspiel zahlreicher Faktoren (K"orperkraft, Kontrolle "uber K"orperhaltung, Balance, Wahrnehmung...)
	\item
		Zusammenfassung Siegler 271 u 291 ff.
\end{itemize}
