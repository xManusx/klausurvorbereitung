\section{Bindungsentwicklung}
\subsection{Die Bindung zwischen Kindern und ihren Bezugspersonen}
\subsubsection{Bowlbys Bindungstheorie}
\begin{itemize}
	\item
		postuliert biologische Veranlagung von Kindern, Bindungen zu Versorgungspersonen zu entwickeln um eigene "Uberlebenschancen zu erh"ohen
	\item
		Kompetenzmotiviertes Kleinkind, dass engste Betruungsperson als \enquote{sichere Basis} (= Anwesenheit einer Bindungsperson bietet Gef"uhl von Sicherheit, dass es erm"oglicht die Umwelt zu erforschen) nutzt
	\item
		Bindung findet in vier Phasen statt:
		\begin{enumerate}
			\item
				Vorphase der Bindung (Geburt bis 6 Wochen): Kind zeigt angeborene Signale, meistens Schreien, ruftt dadurch andere zu sich, f"uhlt sich getr"stet
			\item
				Entstehende Bindung (6 Wochen bis 6--8 Monate): 
				\begin{itemize}
					\item
						beginnen auf vertraute Personen zu reagieren (l"acheln, plappern h"aufiger wenn Bezugspersonen da sind)
					\item
						Entwickeln Erwartungen, wie F"ursorger auf Bed"urfnisse reagieren
					\item
						Entwickeln Gef"uhl, wie sehr sie Bezugspersonen vertrauen k"onnen
				\end{itemize}
			\item
				Ausgepr"agte Bindung (6--8 Monate bis 18--24 Monate):
				\begin{itemize}
					\item
						Suchen aktiv Kontakt zu Bezugspersonen
					\item
						begr"u"sen Mutter bei ihrem Erscheinen freudig, zeigen Unbehagen wenn sie weg geht
					\item
						Meistens Mutter nun sichere Basis
				\end{itemize}
			\item
				Reziprope Beziehungen (ab 18--24 Monaten):
				\begin{itemize}
					\item
						Ansteigenden kognitiven und sprachlichen F"ahigkeiten erm"oglichen Gef"uhle, Ziele und Motive der Eltern zu verstehen
					\item
						Nutzen dieses Verst"andnis um ihre Anstreungen darauf auszurichten in N"ahe der Eltern zu kommen
					\item
						Trennungsstress geht zur"uck, mehr wechselseitig geregelte Beziehung entsteht
				\end{itemize}
		\end{enumerate}
	\item
		Ergebnis der Phasen ist andauernde emotionale Verkn"upfung und ein \enquote{inneres Arbeitsmodell von Bindung}(mentale Repr"asentation des Selbst, der Bindungsperson und der Beziehungen im Allgemeinen, leitet die Interatkion mit den Versorgern und anderen Personen in Kindheit und sp"ater)
	\item
		Theorie sp"ater von Ainsworths empirisch bewiesen und erweitert
	\item
		Messung mit \enquote{Fremde Situation}
	\item
		Entdeckung von drei Bindungskategorien:
		\begin{enumerate}
			\item
				Sichere Bindung: Mehrzahl der Kinder (ca 65\%)
				\begin{itemize}
					\item
						Eltern als sichere Basis
					\item
						Qualitativ hochwertige, relativ eindeutige Beziehung zu Bindungsperson
					\item
						Weint vielleicht wenn Eltern weggehen, freuen sich aber wenn sie zur"uckkommen, erholen sich schnell von Unbehagen
				\end{itemize}
			\item
				Unsicher-vermeidend: 20\%
				\begin{itemize}
					\item
						weniger positive Bindung zu Bezugsperson als sicher gebundene Kinder
					\item
						meiden Eltern in Fremder Situation
					\item
						Begr"u"sen Bezugsperson beim Wiedersehen nicht einmal, ignorieren sie
				\end{itemize}
			\item
				Unsicher-amivalent: 15\%
				\begin{itemize}
					\item
						Kinder/S"auglinge klammern, bleiben nahe bei Bezugsperson statt Umwelt zu erkunden
					\item
						"angstlich in Fremder Situation, k"onnen von Fremden nicht leicht beruhigt werden
					\item
						Wenn Bezugsperson zur"uckkommt lassen sie sich nur schwer beruhigen, suchen einerseits Trost, andererseits widersetzen sie sich getr"ostet zu werden
				\end{itemize}
			\item
				Desorganisiert-Desorientiert: (im Anschluss an Ainsworths Forschungen hinzugekommen, weniger als 5\%)
				\begin{itemize}
					\item
						Keine konsistente Stressbew"altigungsstrategie in Fremder Situation, Verhalten oft konfus, widerspr"uchlich
					\item
						wollen sich Elternteil n"ahern, sehen ihn aber auch als Quelle der Angst
				\end{itemize}
		\end{enumerate}
	\item
		Bei Kindern mit unsicherer Bindung in bindungsrelevanten Situation Cortisolreaktion (Stressreaktion)
	\item
		H"angt mit Bindungsmodellen der Eltern zusammen
	\item
		"Ahnlich in allen Kulturen, ein paar Unterschiede gibt es jedoch:
		\begin{itemize}
			\item
				Alle japanischen unsicher gebundenen Kinder wurden als Unsicher-ambivalent klassifiziert
		\end{itemize}
\end{itemize}


\subsubsection{Einflussfaktoren auf die kindliche Bindungssicherheit}
\begin{itemize}
	\item
		Einf"uhlungsverm"ogen der Eltern (Einf"uhlsame Eltern produzieren sicher gebundene Kinder)
	\item
		Temperament des Kindes spielt nur vergleichsweise geringe Rolle (eher unintuitiv)
\end{itemize}

\subsubsection{Langzeitwirkungen der Bindungssicherheit}
\begin{itemize}
	\item
		Kinder die sensible, unterst"utzende ERziehung erleben (wie sie mit sicherer Bindung einhergeht) lernen, dass es akzeptabel ist, Emotionen in angemessener Weise auszudr"ucken und dass emotionale Kommunikation wichtig ist
	\item
		Sichere Kinder ungef"ahr in allem (Freundschaften, kontaktfreudigkeit, Schule) besser
	\item
		Vermutlich fr"Uhe Bindung lang anhaltende Wirkung.
	\item
		Allerdings auch Belege, dass sich Bindungssicherheit mit Ver"anderungen der Umwelt ver"andert (Belastungen und Konflikten in der Famlie, beispielsweise)
	\item
		Zusammenfassung Siegler 601 f.
\end{itemize}


\subsection{Konzeptionen des Selbs}
\begin{itemize}
	\item
		Selbst = Konzeptsystem, das aus den Gedanken und Einstellungen "uber sich selbst besteht
\end{itemize}
\subsubsection{Entwicklung der Vorstellungen vom Selbst}
\begin{itemize}
	\item
		Schon in ersten Lebensmonaten rudiment"are Vorstellung vom Selbst: Vorstellung von ihrer F"ahigkeit, Objekte au"serhalb ihres Selbst zu kontrollieren (Rassel z.B.)
	\item
		Zwischen 18 und 20 Monaten sich im Spiegel erkennen (Rouge-Test)
	\item
		W"ahrend drittem Lebensjahr Selbs-Bewusstsein deutlich durch Verlgenheit und Scham, aber auch Trotz
	\item
		Zwischen drei und vier Jahren Bezug auf konkrete, beobachtbare Eigenschaften (k"operliche Attribute, soziale Beziehungen, psychische Zust"ande), aber unrealistisch positiv (sie denken sie sind, was sie sein wollen)
	\item
		In Grundschule viele soziale Vergleiche, achten bei Aufgaben auf Diskrepanz zwischen eigener und anderer Leistung
	\item
		In Adoleszenz Sorgen "uber soziale Kompetenz und soziale Akzeptanz verst"arkt
	\item
		Selbstkonzept von Jugendlichen kann mehr als ein Selbst umfassen (z.B. in verschiedenen Situation, gegen"uber verschiedenen Personen(gruppen))
	\item
		Von \enquote{Pers"onlicher Fabel} gekennzeichnet: sagenhafte Selbstbeschreibung, die den Glauben an die Einzigartigkeit ihrer Gef"uhle und ihre Unsterblichkeit beinhaltet
	\item
		Oft Gedanken was andere "uber sie denken: Imagni"ares Publikum: vom Egozentrismus der Jugendlichen begr"undete Vorstellung, dass jeder andere Mensch seine Aufmerksamkeit auf Erscheinung und Verhalten des Jugendlichen richtet
	\item
		In sp"ater Adoleszenz wird Vorstellung des Individuums vom selbst st"arker integriert und weniger durch die Gedanken/Bewertungen anderer bestimmt: Widerspr"uchlichkeiten in verschiedenen Kontexten/Zeitpunkten werden in Selbst integriert (im Gegensatz zu fr"uher Adoleszenz)
\end{itemize}

\subsubsection{Eriksons Theorie der Identit"atsbildung}
\begin{itemize}
	\item
		L"osung der Identit"atsfragen als zentrale Entwicklungsaufgabe der Adoleszenz
	\item
		Krise von Identit"at vs. Rollendiffusion: psychosoziale Entwicklungsphase w"ahrend der Adoleszenz - Jugendliche/junge Erwachsene entwickeln entweder eigene Identit"at oder erfahren unvollst"andiges und manchmal unkoh"arentes Selbstgef"uhl(Rollendiffusion)
	\item
		Erfolgreiche L"sung der Krise impliziert Konstruktion koh"arenter Identit"at
	\item
		M"ogliches Resultat einer misslungen Identit"atssuche ist Rollendiffusion: Jugendliche in diesem Zustand h"aufig verloren, isoliert, deprimiert 
	\item
		Rollendiffusion sehr h"aufig w"ahrend Adoleszenz, i.Allg. aber nur kurz andauernd
	\item
		"Ubernommene Identit"at: voreilig auf Identit"at festlegen, ohne auszuloten welche M"oglichkeiten bestehen w"urden
	\item
		Negative Identit"at: Gegenteil von dem was das Umfeld des Jugendlichen schätzt (z.B. Pfarrerstochter mit h"aufig wechselnden Geschlechtspartnern). Erikson nimmt an, dass dies ein Weg ist Aufmerksamkeit zu bekommen, wenn andere Versuche fehlgeschlagen sind
	\item
		Erikson schl"agt psychosoziales Moratiorum vor: Auszeit in der von Jugendlichen nicht erwartet wird Erwachsenenrolle zu "ubernehmen, stattdessen Selbsterfahrung (allerdings nur in wenigen Kulturen akzeptabel und auch dann eher Privileg. in traditionellen Gesellschaften unbekannt und unn"otig: Rollenauswahl ist beschr"ankt, von Kindheit an vorbestimmt)
	\item
		Marcia definiert 4 Kategorien:
		\begin{enumerate}
			\item
				Identit"atsdiffusion: keine stabile Festlegung, keine Fortschritte
			\item
				"Ubernommene Identit"at: Individuum hat nichts ausprobiert sondern berufliche und ideologische Identit"at entwickelt die auf Auswahl oder Werten anderer beruht
			\item
				Moratorium: erkundet verschiedene berufliche und ideologische Wahlm"oglichkeiten, aber nicht festgelegt
			\item
				Erarbeitete Identit"at: koh"arente und gefestigte Identit"at erreicht, beruht auf pers"nlichen Entscheidungen. Glaub dass diese Entscheidungen eigenh"andig getroffen wurden, f"uhlt sich ihnen verpflichtet
		\end{enumerate}
		Im Idealfall Verlauf zur Erarbeiteten Identit"at
	\item
		Zusammenfassung Sigler 618
\end{itemize}
