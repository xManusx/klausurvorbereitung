
\section{Soziale und emotionale Entwicklung}
\begin{itemize}
	\item
		Emotionale Intelligenz: F"ahigkeiten zur Kompetenz im sozialen und emotionalen Bereich. Sich selbst motivieren, trotz Frustration, Kontrollimpulsen und Belohnungsaufschub hartn"ackig zu bleiben. Eigene/die anderer Gef"uhle erkenne und verstehen. Eigene Stimmungen und den Gef"uhlsausdruck in sozialen Interaktionen zu regulieren und sich in Emotionen von anderen hineinzuversetzen
	\item
		EQ sagt viel besser als IQ vorraus wie gut Leute im Leben zurecht kommen
\end{itemize}

\subsection{Entwicklung von Emotionen in der Kindheit}
\begin{itemize}
	\item
		Emotionen bestehen aus verschiedenen Komponenten: (Angst am Beispiel eines knurrenden Hundes)
		\begin{enumerate}
			\item
				Wunsch etwas zu tun, einschlie"slich des Wunsches, Menschen oder Dingen in der Umwelt zu entfliehen, sich ihnen zu n"ahern oder sie zu ver"andern  (weglaufen)
			\item
				Physiologische Korrelate wie Herz- oder Atemfrequenz, Hormonspiegel\dots (Erregung, Adrenalin)
			\item
				Subjektive Gef"uhle (sp"uren subjektiv Angst)
			\item
				Kognitionen, die Emotionen hervorrufen oder begleiten k"onnen ("uberlegen wie sie der Hund verletzen k"onnte)
	\end{enumerate}
\end{itemize}

\subsubsection{Theorien "uber Wesen und Entstehung von Emotionen}
\begin{itemize}
	\item
		Theorie der diskreten Emotionen/Basis-Emotionen (Izard 1991, Tomkins 1962): Emotionen angeboren, jede Emotion spezifischer und unverwechselbarer Satz an k"orperlichen und mimischen Reaktionen, diese abgrenzbaren Emotionen seit fr"uhester Kindheit vorhanden
	\item
		Affektsysteme (Sroufe 1979 u 1995): Drei grundlegende Affektsysteme - Freude/Vergn"ugen, Wut/Frustration, Misstrauen/Angst entwickeln sich in ersten Lebensjahren von primitiver Form zu fortgeschrittener
	\item
		Funktionalistischer Ansatz (Campos 1994): Grundfunktion von Emotionstheorie zielgerichtete Handlungen zu f"ordern. Emotionen sind nicht abgegrenzt und variieren je nach sozialer Umwelt in bestimmtem Ausma"s
	\item
		Alle durch empirische Studien gest"utzt, aber keine "uberlegen
\end{itemize}

\subsubsection{Entstehung von Emotionen im Entwicklungsverlauf}
\begin{itemize}
	\item
		Positive Emotionen:
		\begin{itemize}
			\item
				W"ahrend dritter Lebensmonat: Soziales L"acheln (L"acheln, das an Menschen gerichtet ist)
			\item
				Ab 2 Monaten Freude, wenn Ereignis kontrolliert werden kann (z.B. Rassel)
		\end{itemize}
	\item
		Negative Emotionen:
		\begin{itemize}
			\item
				Manchmal (in einigen Kontexten) Wut/Traurigkeit gegen Schmerz/Unbehagen abgrenzbar bei 2 Monate alten Babys
			\item
				S"auglinge manchmal negative Emotionen, die nicht zur Situation passen
			\item
				Nur undifferenziertes Missbehagen, keine Unterscheidung zwischen Wut Schmerz/Missbehagen vor 6 Monaten
			\item
				Danach insbesondere Angst vor Fremden (spiegelt vermutlich wachsende Bindung an Eltern wider)
			\item
				Ab ca. 8 Monaten: Trennungsangst
			\item
			 Ab 12 Monate: Wut, wenn nicht in der Lage Umwelt zu kontrollieren
		\end{itemize}
	\item
		selbst-bewussten Emotionen: Verlegenheit, Stolz, Schuld und Scham
		\begin{itemize}
			\item
				Nicht vor zweitem Lebensjahr, weil daf"ur Verst"andnis gebraucht wird, selbst eine von anderen abgrenzbare Person zu sein. Dies entsteht erst nach und nach w"ahrend erster Lebensjahre
			\item
				Ab 15-24 Monaten Verlegenheit
			\item
				Ob Kinder Scham oder Schuld erleben h"angt zum Teil von Erziehungspraktiken ab: Eher Schuld wenn Eltern Schlechtigkeit des Verhaltens und nicht des Kindes betonen
			\item
				Stolz nachdem Leistung erbracht wurde, h"angt aber von Kultur ab (Zuni-Indianer \enquote{nicht auffallen}, deswegen kein Stolz nach besonderer Leistung)
		\end{itemize}

	\item
		Zusammenfassung Siegler 543 f
\end{itemize}

\subsection{Die Regulierung von Emotionen}
\begin{itemize}
	\item
		Emotionale Selbst-Regulation: Prozess der Initiierung, Hemmung oder Modulierung innerer Gef"uhlszust"ande, emotionsbezogener physiologischer Prozesse und Kognitionen, sowie des Verhaltens im Dienste der Zielerreichung
	\item
		Kinder brauchen Jahre um F"ahigkeit zur Emotionsregulation zu erlernen

\end{itemize}


\subsubsection{Entwicklung der Emotionsregulierung}
\begin{itemize}
	\item
		3 allgemeine, altersbezogne Ver"anderungsmuster:
		\begin{enumerate}
			\item
				"Ubergang der Kinder von fast v"ollig auf andere Personen zu verlassen bei der Regulation von Emotionen zu helfen $\rightarrow$ wachsende F"ahigkeit zur Selbst-Regulierung
			\item
				$\rightarrow$ kognitive Strategien zur Kontrolle negativer Emotionen
			\item
				$\rightarrow$ 
				Auswahl von geeigneten Regulierungsstrategien
		\end{enumerate}
	\item
		Von Regulierung durch andere zur Selbst-Regulation
		\begin{itemize}
			\item
				Erst nur durch Eltern (tr"osten, ablenken, \dots)
			\item
				Ab ca. 6 Monaten teilweise selbst beruhigen
			\item
				Mit dem Alter bessere Selbst-regulation
			\item
				Ver"anderte Erwartungen der Eltern, erwarten dass Kinder mit emotionaler Erregung umgehen k"onnen
		\end{itemize}
	\item
		Der Gebrauch kognitiver Strategien zur Steuerung negativer Emotionen
		\begin{itemize}
			\item
				Kleinere Kinder regulieren durch Verhaltensstrategien (z.B. Ablenkung)
			\item
				Sp"ater kognitive Strategien, z.B. Bedeutung herunterspielen
		\end{itemize}

	\item
		Auswahl geeigneter Regulierungsstrategien
		\begin{itemize}
			\item
				F"ahigkeit geeignete Strategien auszuw"ahlen wird dadurch gesteigert, dass Kinder lernen zwischen kontrollierbaren (Hausaufgaben) und unkontrollierbaren (Medizinische Untersuchung) zu unterscheiden
		\end{itemize}
\end{itemize}

\subsubsection{Zusammenhang zwischen Emotionsregulierung und sozialer Kompetenz}
\begin{itemize}
	\item
		Kinder die konstruktiv mit stressvoller Situation umgehen k"onnen sind insgesamt sozial kompetenter als Kinder, die v"ollig vermeiden sich mit stressvollen Situationen auseinanderzusetzen
	\item
		Zusammenfassung Siegler S. 549
\end{itemize}



\subsection{Individuelle Unterschiede bei Emotionen und ihrer Regulierung}
\subsubsection{Temperament}
\begin{itemize}
	\item
		= veranlagunsbedingte individuelle Unterschiede in der emotionalen, motorischen und aufmerksamkeitsbezogenen Reagibilit"at und in der Selbstregulierung, "uber Situationen hinweg konsistent, sowie "uber die Zeit stabil
	\item
		New Yorker Langzeitstudie Thomas \& Chess 1977 wichtig (Tabelle "uber Temperamentsdimensionen auf verschiedenen Altersstufen Siegler 551 f.)
		\begin{itemize}
			\item
				Aktivit"atsniveau (hoch/niedrig)
			\item
				Rhythmus (Regelm"a"sig/Unregelm"a"sig)
			\item
				Ablenkbarkeit (ablenkbar/nicht ablenkbar)
			\item
				Annh"aherung/R"uckzug (positiv/negativ)
			\item
				Anpassungsf"ahigkeit (Anpassungf"ahig/Nicht anpassungf"ahig)
			\item
				Aufmerksamkeitsdauer (lang/kurz)
			\item
				Reakiontsintensit"at (Stark/Schwach)
			\item
				Reaktionschwelle (niedrig/hoch)
			\item
				Stimmungsqualit"at (positiv/negativ)
		\end{itemize}
	\item
		3 Gruppen von Babys:
		\begin{itemize}
			\item
				Einfache Babys
			\item
				Schwierige Babys
			\item
				Langsam auftauende Babys
		\end{itemize}
	\item
		Neuere Studien nur 6 Dimensionen:
		\begin{itemize}
			\item
				Angstvolles Unbehagen
			\item
				Reizbares Unbehagen
			\item
				Aufmerksamkeitsspanne und Ausdauer
			\item
				Aktivit"atsniveau
			\item
				Positiver Affekt
			\item
				Rhythmus
		\end{itemize}
	\item
		Temperament weitgehend stabil, teilweise sogar pr"a und postnatal
	\item
		Unterschiede im Temperament korrelieren mit Unterschieden in sozialer Kompetenz und Anpassungf"ahigkeit
	\item
		Anpassung"ute = Ausma"s in dem Temperament eines Individuums mit Anforderungen und Erwartungen des sozialen Umfelds "ubereinstimmt
	\item
		Zusammenfassung Siegler S 560
\end{itemize}

\subsection{Die emotionale Entwicklung von Kindern in der Familie}
\begin{itemize}
	\item
		Gene betr"achtlicher Anteil an Temperament
	\item
		Eltern-Kind-Beziehung beeinflusst emotionale Entwicklung: sichere Bindung $\rightarrow$ mehr positive Emotionen und weniger "Angstlichkeit
\end{itemize}

\subsubsection{Die elterliche Sozialisation der emotionalen Reaktion von Kindern}
\begin{itemize}
	\item
		Eltern sozialisieren emotionale Entwicklung ihrer Kinder durch
		\begin{enumerate}
			\item
				ihren Ausdruck von Emotionen gegen"uber ihren Kindern und anderen Personen
				\begin{itemize}
					\item
						Emotionen die zu Hause gezeigt werden k"onnen Sicht der Kinder auf sich selbst und andere beeinflussen
					\item
						Emotionen der Eltern dienen Kindern als Modell, wann und wie man Emotionen ausdr"uckt
				\end{itemize}
			\item
				ihre Reaktionen auf den kindlichen Ausdruck von Emotionen
			\item
				die Diskussion, die sie mit ihren Kindern "uber Emotionen und emotionale Regulierung f"uhren
	\end{enumerate}
\item
	\dots
\item
	Zusammenfassung Siegler 579 ff
\end{itemize}
