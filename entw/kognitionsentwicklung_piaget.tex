\section{Entwicklung von Kognition \& Denken: Piaget und Informationsverarbeitungstheorien}
\subsection{Die Sicht auf das Wesen des Kindes}
\begin{itemize}
	\item
		Piagets Ansatz oft als konstruktivistisch bezeichnet: Kinder konstruieren sich ihr Wissen (als Reaktion auf ihre Erfahrungen) selbst
	\item
		\enquote{Kind als Wissenchafter}: Aufstellen von Hypothesen, Durchf"uhren von Experimenten, Ziehen von Schlussfolgerungen
	\item
		Kinder lernen viele wichtige Dinge selbst, sind nicht auf Instruktionen von Erwachsenen angewiesen
	\item
		Kinder von sich aus (intrinsisch) motiviert zu lernen, brauchen daf"ur keine Belohnung
\end{itemize}

\subsection{Zentrale Entwicklungsfragen}
\subsubsection{Quellen der Kontinuit"at}
\begin{itemize}
	\item
		3 Prozesse:
		\begin{enumerate}
			\item
				Assimilation: Prozess durch den Menschen eintreffende Informationen in eine Form "uberf"uhren, die sie verstehen k"onnen
			\item
				Akkomodation: Prozess durch den Menschen vorhandene Wissensstrukturen als Reaktion auf neue Erfahrungen anpassen
			\item
				"Aquilibration: Prozess durch den Menschen Assimilation und Akkomodation ausbalancieren um stabiles Verstehen zu schaffen, besteht aus drei Phasen:
				\begin{enumerate}
					\item
						"Aquilibrium: Kinder sind mit Verst"andnis eines Ph"anomens zufrieden
					\item
						Dis"aquilibrium: Kinder merken, dass ihr Verst"andnis unzureichend ist, sind aber noch nicht in der Lage Alternative zu entwickeln
					\item
						Stabileres "Aquilibrium: differenziertes Verst"andnis, dass das Unzul"anglichkeiten der bisherigen Verstehensstrukturen "uberwindet
				\end{enumerate}
		\end{enumerate}
\end{itemize}

\subsubsection{Quellen der Diskontinuit"at}
\begin{itemize}
	\item
		Stufentheorie, Vier Stufen:
		\begin{enumerate}
			\item
				Sensumotorisches Stadium:
				\begin{itemize}
					\item
						Von Geburt bis 2 Jahre
					\item
						Intelligenz kommt durch sensorische und motorische F"ahigkeiten zum Ausdruck
					\item
						Objektpermanenz ab 4 bis 8 Monaten
					\item
						Ab 1 Jahr erste \enquote{wissenschaftliche Experimente}
					\item
						Ab 18 Monaten zeitlich verz"ogerte Nachahmung
				\end{itemize}
			\item
				Pr"a-operatorisches Stadium:
				\begin{itemize}
					\item
						von 2 bis 7 Jahren
					\item
						Kinder lernen ihre Erfahrungen in Form von Sprache, geistigen Vorstellungen und symbolischen Denken zu repr"asentieren
					\item
						Kinder k"onnen noch keine mentalen Operationen (reversible geistige Aktivit"aten, z.B. vorstellen das Wasser wieder ins andere Glas zu sch"utten) ausf"uhren
					\item
						Entwicklung symbolischer Repräsentation
					\item
						Egozentrismus (3-Berge-Versuch)
					\item
						Zentrierung: konzentrieren sich auf einzelne, auff"allige Aspekte/Dimensionen eines Ereignisses oder Problems
					\item
						Fokus auf statische Zust"ande, nicht auf Ver"anderung
				\end{itemize}
			\item
				konkret-operatorisches Stadium:
				\begin{itemize}
					\item
						von 7 bis 12 Jahre
					\item
						Kinder k"onnen "uber konkrete Gegenst"ande und Ereignisse logisch nachdenken
					\item
						f"allt ihnen aber immer noch schwer in rein abstrakten Begriffen zu denken und Informationen systematisch zu kombinieren
					\item
						Konzept er Erhaltung: bei Ver"anderung der Erscheinung bleiben Schl"usseleigenschaften von Objekten trotzdem erhalten (Knetmasse in anderer Form wird nicht mehr oder weniger)
				\end{itemize}
			\item
				Formal-operatorisches Stadium
				\begin{itemize}
					\item
						ab 12 Jahren
					\item
						Kinder k"onnen "uber Abstraktionen, M"oglichkeiten und hypothetische Situationen nachdenken
				\end{itemize}
		\end{enumerate}
	\item
		Kritik an Piagets Stufentheorie
		\begin{itemize}
			\item
				Stellt Denken von Kindern konsistenter da als es ist
			\item
				S"auglinge und Kleinkinder sind kognitiv kompetenter als Piaget dachte (z.B. Objektpermanenz wohl schon ab 3 Monaten)
			\item
				sch"atzt den Betrag der sozialen Welt zur kognitiven Entwicklung zu gering
			\item
				Unscharf hinsichtlich der kognitiven Prozesse, die das Denken des Kindes verursachen und der Mechanismen, die kognitives Wachstum hervorrufen
		\end{itemize}
\end{itemize}


\subsection{Robbie Case}
\begin{itemize}
	\item
		Entwickelt eigenes Stufensystem, das auf Piaget beruht:
		\begin{enumerate}
			\item
				Sensumotorische Hauptstufe ( Geburt -- 18 Monate):
				\begin{itemize}
					\item
						Repr"asentationen bestehen aus sensorischem Input und motorischen Aktionen -- keine kognitive Verarbeitung von Information
				\end{itemize}
			\item
				Relationale Hauptstufe (18 Monate -- 5 Jahre):
				\begin{itemize}
					\item
						Konkrete Vorstellungen (=Repr"asentationen) m"oglich
					\item
						Beziehen sich nur auf Relationen zwischen Objekten, Ereginissen und Personen, welche entdeckt und koordiniert werden k"onnen
				\end{itemize}
			\item
				Dimensionale Hauptstufe(5 -- 11 Jahre):
				\begin{itemize}
					\item
						Abstrakte Repr"asentationen von Stimuli
					\item
						Einfache Transformationen m"oglich
					\item
						Logische Gestzm"a"sigkeiten von Beziehungen werden erfasst (z.B. Transitivit"at bei $<$)
				\end{itemize}
			\item
				Vektorielle Hauptstufe(wird zwischen 11 und 19 Jahren erreicht):
				\begin{itemize}
					\item
						Nun auch komplexe Transformationen m"oglich
					\item
						Vektoriell = mehrere Dimensionen einer Repr"asentation in abstrakter Form darstellbar
				\end{itemize}
		\end{enumerate}
	\item
		3 Entwicklungsmechanismen:
		\begin{itemize}
			\item
				Central conceptual structures: Veränderung der Wissenstrukturen
				\begin{itemize}
					\item
						Bei Piaget logische Strukturen, bei Case semantische Strukturen.
					\item
						\enquote{Zentrale begriffliche Strukturen} = semantische Netzwerke bzw. Wissensknoten, bilden das stadientypische Basiswissen des Kindes
				\end{itemize}
			\item
				Automatisierung: Steigerung der Verarbeitungseffizienz
			\item
				Biologische Reifung: v.a. Myelinisierung der Nervenbahnen (Myelinisierung = Neuriten/Nervenfasern werden umh"ullt, dadurch schnellere Erregungsleitung m"oglich, vor allem sensorische oder motorische neuronale Verbindungen)
		\end{itemize}
\end{itemize}
