\section{Moralentwicklung}
\subsection{Piagets Theorie des moralichen Urteils}
\begin{itemize}
	\item
		Moralisches Denken in Kindern wandelt sich vom starren übernehmen der gebote und regeln von Autoritätspersonen zu moralische Regeln sind ein Produkt sozialer Interaktion und veränderbar
\end{itemize}
\subsubsection{Heteronome Moral}
\begin{itemize}
	\item
		Vor konkret Operationalen Phase, also ca. unter 7 oder 8 Jahren
	\item
		Autoritäten geben vor was gut und böse ist, Folgen wichtiger als Motive
	\item
		Regeln sind absolut
\end{itemize}
\subsubsection{Übergangsphase}
\begin{itemize}
	\item
		Zwischen 7/8 und 10 Jahren, Interaktion mit Peers
	\item
		Lernen dass Regeln von Gruppen aufstellt werden und verändert werden können
\end{itemize}
\subsubsection{Stadium der Autononem Moral}
\begin{itemize}
	\item
		Gerechtigkeit und Gleichberechtigung wichtig, Bestrafung angemessen, Motive müssen Berücksichtigt werden, nicht nur Ergebnis
\end{itemize}
\subsubsection{Bewertung}
\begin{itemize}
	\item
		Interaktion mit Peers regt Moralentwicklung von sich aus nicht an, Qualität der Interaktion wichtig
	\item
		Kinder erkennen schon im Kindergartenalter, dass schlimme Absichten schlimmer sind als gutartige Absichte
\end{itemize}

\subsection{Kohlbergs Theorie des moralischen Urteils}
\begin{itemize}
	\item
		Stufenmodell, siehe Siegler et.al. S 762
	\item
		Drei Ebenen des moralischen Urteils:
		\begin{enumerate}
			\item
				Präkonventionelles moralisches Denken: selbstbezogen, Belohnung bekommen - Strafe vermeiden
				\begin{enumerate}
					\item
						Orientierung an Strafe und Gehorsam
					\item
						Orientierung an Kostne-Nutzen und Reziprozität
				\end{enumerate}
			\item
				Konventionelles moralisches Denken: Übereinstimmung mit sozialen Pflichten und Gesetzen
				\begin{enumerate}
					\item
						Orientierung an wechselseitigen zwischenmenschlichen Erwartungen, Beziehungen und zwischenmenschlicher Übereinstimmung (\enquote{gutes Mädchen, guter Junge})
					\item
						Orientierung am sozialan System und am Gewissen (\enquote{Recht und Ordnung})
			\end{enumerate}
			\item
				Postkonventionelles moralisches Denken: auf Ideale ausgerichtet, moralische Prinzipien
				\begin{enumerate}
					\item
						Orientierung am sozialen Vertrag oder an individuellen Rechten
					\item
						Orientierung an universellen ethischen Prinzipien
			\end{enumerate}
		\end{enumerate}
	\item
		Jede Ebene in 2 Stufen unterteilt
	\item
		Fähigkeit zur Perspektivübernahme wichtig
	\item
		Moral Kombination aus kognitiver, sozialer und emotionaler Entwicklung
	\item
		Kritik an Kohlberg:
		\begin{itemize}
			\item
				Westlich-zentrierte Moralvorstellung (andere Gesellschaften weniger Individuelle Freiheiten)
			\item
				Entgegen Kohlberg moralische Entwicklung kontinuierlich, auch Rückschritte sind möglich, auch mehrere Stufen gleichzeitig
		\end{itemize}
\end{itemize}



\subsection{Prosoziales moralisches Urteilsvermögen}
\begin{itemize}
	\item
		Prosoziale moralische Dilemmata: jemandem helfen oder eigenen Bedürfnissen nachgehen?
	\item
		Eisenbergs fünf Stufen des prosozialen moralischen Denkens:
		\begin{enumerate}
			\item
				Hedonistische, selbstbezogene Orientierung
			\item
				Orientierung an Bedürfnissen
			\item
				Orientierung an Anerkennung und/oder Stereotyp
			\item
				\begin{enumerate}
					\item
						Selbstreflexive empathische Orientierung
					\item
						Übergangsniveau
			\end{enumerate}
		\item
			Stark internalisiertes Stadium

	\end{enumerate}
\end{itemize}

\subsection{Bereiche sozialer Urteile}
\begin{itemize}
	\item
		Moralische Urteile betreffen Fragen von Richtig und Falsch, Fairness und Gerechtigkeit
	\item
		Sozial-konventionale Urteile beziehen sich auf Sitten und Regeln zur sozialen Koordination und Organisation
	\item
		Persönliche Urteile Entscheidungen über Handlungen bei denen persönliche Präferenz berücksichtigt werden
	\item
		Kinder unterscheiden zwischen moralische Verfehlungen schlimmer sind als gegen soziale Konventionen zu verstoßen
	\item
		Zusammenfassung Siegler et.al. 773
\end{itemize}

\section{Entwicklung von Prosozialem und Aggressiven Verhalten}
\begin{itemize}
	\item
		Gewissen: Ein innerer Regulationsmechanismus, der die Fähigkeit eines Individuums erhöht Verhaltensstandards seiner Kultur zu entsprechen
	\item

\end{itemize}
\subsection{Faktoren, welche die Gewissensentwicklung beeinflussen}
\begin{itemize}
	\item
		Ab zwei Jahren fangen Kinder an Verständnis f"ur NOrmen und Regeln zu zeigen, erste Anzeichen von Schuldgef"uhlen
	\item
		"Ubernehmen vermutlich Werte der Eltern
	\item
		Bei ängstlichen Kindern \enquote{behutsame} Disziplinierungspraktiken für Gewissensentwicklung
	\item
		Bei furchtlosen Kindern besser positive Eltern-Kind-Beziehung, Kinder wollen mehr der Mutter gefallen als sich vor ihr f"urchten
\end{itemize}

\section{Prosoziales Verhalten}
\begin{itemize}
	\item
		Gibt Entwicklungskonsistenz von Kindern, die sich auf prosoziale Verhaltensweisen einlassen
\end{itemize}
\subsection{Die Entwicklung des prosozialen Verhaltens}
\begin{itemize}
	\item
		Urspr"unge f"ur altruistisches prosoziales Verhalten liegen in Empathie und Mitleid begr"undet. (Empathie = F"ahigkeit sich in andere hineinzuversetzen, emotionale Reaktion auf Zustand des anderen, die Gef"uhlslage des anderen widerspiegelt
	\item
		Unterschied Mitleid Empathie: Mitleid zus"atzlich Sorge und Anteilnahme, helfen wollen
	\item
		Wichtiger Faktor f"ur Mitleid: F"ahigkeit Perspektive des anderen einzunehmen (entgegen Piaget ("nicht vor 6 oder 7 Jahren") schon viel fr"uher)
		Im zweiten und 3. Lebensjahr erh"ohen sich H"aufigkeit und Vielfalt prosozialer Verhaltensweisen: Nicht nur tr"osten, sondern auch helfen
	\item
		Aber auch aggresives Verhalten, Sticheln oder Gleichgültigkeit gegen"uber Unbehagen und Bed"urfnisse anderer
	\item
		Prosoziales Verhalten wird mit Alter h"aufiger
\end{itemize}
\subsection{Urspr"unge individueller Unterschiede beim prosozialen Verhalten}
\subsubsection{Biologische Faktoren}
\begin{itemize}
	\item
		Genetische Faktoren scheinen moderate Rolle zu spielen (kleinder bei Kindern als bei Erwachsenen): Eineiige Zweillinge im Hinblick auf Empathie und prosoziales Verhalten "ahnlicher als Zweieiige. 
	\item
		Temperamentsunterschiede spielen Rolle: Kinder, die Gef"uhle bewusst erleben k"onnen ohne "uberw"altigt zu sein besonders h"aufig empathisch
\end{itemize}
\subsubsection{Die Sozialisation prosozialen Verhaltens}
\begin{itemize}
	\item
		3 Erziehungsarten mit denen Eltern soziales Verhalten von Kindern fördern:
		\begin{enumerate}
			\item
				Vorbild sein und prosoziales Verhalten beibringen
				\begin{itemize}
					\item
						Kinder und Eltern häufig ähnliches Niveau an dAnteilnahme und prosozialem Verhalten
					\item
						Studie mit Rettern von Juden und Zuschauern Oliner \& Oliner 1988: Fairness/Gerechtigkeit bei beidne Gruppen gleich: Sorge für andere (/als univeselles Prinzip) bei Retter häufiger
					\item
						Wirksames Mittel: an Mitgefühl appelieren (\enquote{die armen Kinder...}), danach Kinder eher bereit zu spenden als wenn einfach nur \enquote{Helfer sind gut}
				\end{itemize}

			\item
				arrangieren Gelegenheiten, bei denen sich kinder prosozial verhalten können
				\begin{itemize}
					\item
						Z.B. Haushaltspflicht, freiwillige soziales Dienste 
					\item
						Gelegenheit für Hilfeleistunhgen emotionale Belohnung zu erfahren, lernen sich in andere hineinzuversetzen und Vertrauen in eigene Fähigkeit zu Hlefen zu steigern
				\end{itemize}
			\item
				Erziehung und Disziplinierung zu prosozialem Verhalten
				\begin{itemize}
					\item
						autoritäterer Erziehungsansatz (Bestrafungen, Drohungen) häufig Kinder mit Mangel an Mitgefühl und prosozialem Verhalten
					\item
						Wenn Kinder bestraft/materiell belohnt werden anderen zu helfen, nehmen sie an, dass sie nur um der Belohnung/Bestrafung willen helfen. Ist das nicht mehr gegeben fällt Anreiz zu helfen weg
					\item
						Rationale, vernünftige Argumente helfen Kindern Folgen ihres Verhaltens zu verstehen und geben Gründe mit, an denen Verhalten orientiert werden kann (Schon bei 1/2-jährigen)
					\item
						Kinder meistens sozialer, wenn Eltern nicht nur Wärme und Unterstützung bieten sondern bei Erziehung auch prosoziales Verhalten vorleben
				\end{itemize}
	\end{enumerate}
\item
	Fernsehen zeigt antisoziales Verhalten (Gewalt etc.), Gefahr?
	\begin{itemize}
		\item
			Manche Inhalte zeigen prosoziales Verhalten, Kinder die so etwas (Sesamstraße...) sehen, neigen sofort danach und auch manchmal noch später zu prosozialem Verhalten, meistens jedoch kein lang anhaltender Effekt
	\end{itemize}
\item
	Zusammenfassung Siegler et.al. 786
\end{itemize}

\section{Antisoziales Verhalten}
\begin{itemize}
	\item
		Aggression: Verhalten das darauf abzielt andere zu schädigen oder zu verletzen
\end{itemize}
\subsection{Die Entwicklung von Aggression und anderer anti-sozialer Verhaltensweisen}
\begin{itemize}
	\item
		Schon zwischen 12 und 18 Monaten Konflikte, meistens ohne Aggression
	\item
		Ab 18 Monaten körperliche Aggression wie Schlagen oder Stoßen, steigen bis 2 Jahren
	\item
		Danach sinkt die Häufigkeit körperlicher Aggression, mit den sprachlichen Fähigkeiten kommt verbale Aggression
	\item
		Viel Streit um Eigentumsverhältnisse, dies ist Beispiel für instrumentelle Aggression: Aggression mit Wunsch etwas zu erreichen
	\item
		Manchmal auch Beziehungsaggression um Peers zu beherrschen oder verletzen (Peer-Beziehungen schädigen: aus Gruppe ausschließen, negative Gerüchte)
	\item
		Aggressives Verhalten bei jüngeren Kindern normalerweise durch wunsch motiviert, bei Grundschulkindern häufig durch Feindschaft, Wunsch anderen zu verletzen oder Bedürfnis sich gegen wahrgenommene Bedrohung des Selbstwerts zu schützen
	\item
		In Adoleszenz sinkt Häufigkeit offener Aggressionen, schwere Gewaltanwendung steigt aber
\end{itemize}
\subsection{Die Beständigkeit aggressiven und antisozialen Verhaltens}
\begin{itemize}
	\item
		Kinder die mit 8 Jahren von Peers als aggressiv eingeschätzt wurden mit 30 Jahren mehr kriminelle Vorstrafen und häufiger Aggressiv als andere (Eron et.al. 1987)
	\item
		Größtes Risiko für Kinder die sowohl aggressiv sind, als auch anderes antisoziales Verhalten (lügen, stehlen) zeigen, Aggression aber kein notwendiger Bestandteil zukünftiger Verhaltensprobleme
	\item
		Viele Kinder die bereits früh Aggressionen zeigen haben neurologische Defizite
\end{itemize}


\subsection{Kennzeichen aggressiver und/oder antisozialer Kinder und Jugendlicher}
\subsubsection{Temperament und persönlichkeit}
\begin{itemize}
	\item
		Intensive negative Emotionen im Kleinkindalter neigen später zu mehr Problemverhalten (z.B. Aggression) (Bates et.al 1991)
	\item
		Vor Schuleintritt wenig Selbstkontrolle, Impulsivität und hohes Aktivierunsniveau und reizbar und ablenkbar im Alter zwischen 9 und 15 Jahren mehr Schlägereien, Kriminalität etc.
	\item
		Kombination aus Impulsivität, Aufmerksamkeitsprobleme und Verlogenheit in Kindheit gute Vorhersage für antisoziales Verhalten  in Adoleszenz
\end{itemize}
\subsubsection{Soziale Kognition}
\begin{itemize}
	\item
		Auch soziale Kognition ist wichtig, weil sie sich darauf auswirkt, wie Kinder ihre Interaktionen mit anderen interpretieren und wie sie auf diese reagieren
	\item
		Aggressive Kinder betrachten Welt durch \enquote{aggressive Linse}
	\item
		Aggressive Kinder häufig negative, feindselige Ziele (Bedrohung von Peers, anderen heimzahlen...)
	\item
		Aggressive Kinder bewerten aggressive Reaktionen positiver als andere Kinder, prosoziale Reaktionen weniger günstig
	\item
		Haben mehr Vertrauen in eigene Fähigkeiten körperlich und verbale Aggressionen auszuführen
	\item
		Kinder mit reaktiver Aggression (emotionsgesteuerte, feindselige Aggression) nehmen Motive anderer häufig als feindselig war, reagieren auf Provokationen aggressiv
	\item
		Kinder mit proaktiver Aggression (nicht emotional begründet, geht darum Bedürfnissen nachzukommen) erwarten positive soziale Folgen von Aggression
\end{itemize}


\subsection{Ursprünge der Aggression}
\subsubsection{Biologische Faktoren}
\begin{itemize}
	\item
		Genetik spielt Rolle, vor allem wenn Verhalten schon in Kindheit und nicht erst in Adoleszenz
	\item
		Testosteronspiegel wird manchmal mit aggressivem Verhalten in Verbindung gebracht
	\item
		Genetische, neurologische oder hormonelle Eigenschaften sind Risikofaktoren, entscheiden aber nicht, ob Kind tatsächlich aggressiv wird
\end{itemize}

\subsubsection{Die Sozialisation von Aggression und antisozialem Verhalten}
\begin{itemize}
	\item
		Ausmaß in dem schlechte Erziehung für antisoziales Verhalten verantwortlich ist nicht bekannt
	\item
		Elterliche Bestrafung
		\begin{itemize}
			\item
				Kinder neigen zu Problemverhalten wenn Eltern generell kaltherzig und strafend erziehen, streng aber nicht misshandelnd körperlich bestrafen
			\item
				Hohe Wahrscheinlichkeit, dass antisoziale Verhaltensweisen auf misshandelnde Bestrafung folgen
			\item
				Kinder mit hoher Ausprägung von antisozialem Verhalten und wenig Selbstregulation häufig mit strengen Erziehungsmaßnahmen konfrontiert $\Rightarrow$ Teufelskreis
			\item
				Genetik und Sozialisation schwer zu trennen bei diesen Untersuchungen (aggressive gene bei eltern $\Rightarrow$ auch bei Kindern, die damit sowohl aggressives verhalten zeigen, als auch streng/strafend erzogen werden
		\end{itemize}
	\item
		Unwirksame Erziehungsmaßnahmen
		\begin{itemize}
			\item
				Inkonsequente Bestrafungen führen häufig zu aggressiven und kriminellen Kindern
			\item
				Wenn Eltern Kinder überwachen/beaufsichtigen generell weniger Verhaltensprobleme. Möglicher Grund: Kinder hängen dann nicht mit antisozialen Peers rum
			\item
				Wenn Wutausbrüchen und Wünschen von Jungen nachgegeben wird, wird Aggression des Kindes verstärkt. Gibt Gründe zur Annahme, dass Mädchen anders reagieren
		\end{itemize}
	\item
		Konflikte zwischen den Eltern
		\begin{itemize}
			\item
				Kinder, die Zeuge verbaler und physischer Gewalt zwischen Eltern werden neigen dazu aggressiver und antisozialer zu sein. Grund: streitende Eltern als Modelle für aggressives Verhalten
			\item
				Scheidung/Wiederheirat führen tendenziell auch zu antisozialem Verhalten
			\item
				Mütter unterstützen Kinder nach Scheidung erst mal weniger und sind weniger konsequent bei Kontrolle und Beaufsichtigung
		\end{itemize}
	\item
		Kinder aus einkommensschwachen Familien in der Regel antisozialer und aggressiver
		\begin{itemize}
			\item
				Mehr Stressoren (Krankeit/Gewalt/Scheidung/Kriminalität in der Familie, Gewalt in der Wohngegend) als reiche Kinder
			\item
				Möglicherweise alleinerziehendes Elternteil oder sehr frühe Eltern (beides mit aggressivem Verhalten verknüpft)
			\item
				Von Armut gestresste Eltern häúfig Vorbilder für Aggression und \enquote{schlechte Erziehung}
		\end{itemize}
\end{itemize}


\subsubsection{Der Einfluss der Peers}
\begin{itemize}
	\item
		Aggressive Kinder tun sich gern zusasmmen
	\item
		Mäßig aggressive Kinder werden aggressiver, wenn sie mit aggressiven Freunden rumhängen
	\item
		Kriminelle Peers die verdeckte(Diebstahl, Drogenhandel)/offene(Gewalt, Waffengebrauch) antisoziale Verhaltensweisen zeigen erhöhen Wahrscheinlichkeit für krminelles Verhalten um 400/300 Prozent
	\item
		Gangs sind schlecht für Kinder
	\item
		Gewalt im Fernsehen wirkt sich auf Kinder aus
		\begin{itemize}
			\item
				Gewalt im Fernsehen kausaler Faktor für spätere Aggression
		\end{itemize}
	\item
		Zusammenfassung Siegler 804 ff.
\end{itemize}

