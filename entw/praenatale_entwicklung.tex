
\section{Pränatale Entwicklung, Neugeborenenalter}
\begin{itemize}
	\item
		Epigenese: Vorstellung, dass sich neue Strukturen und Funktionen entwickeln und nicht von Anfang an da sind und nur wachsen
	\item
		Meiose Zellteilung bei der die entstehenden Zellen nur die Hälfte der Chromosomen besitzt
\end{itemize}

\subsection{Pränatale Entwicklung}
\begin{itemize}
	\item
		Zellmigration: Wanderung neu gebildeter Zellen von ihrem Ausgangspunkt an andere Stelle im Embryo (z.b. Neuronen im Cortex)
	\item
		Apoptose: programmierter Zelltod
\end{itemize}

\subsubsection{Früheste Entwicklung}
\begin{itemize}
	\item
		Am vierten Tag nach Befruchtung formen sich die Zellen zu einer Hohlkugel (Blastozyste) in der sich ein Zellhaufen (die \enquote{innere Zellmasse}) befindet
	\item
		In diesem Stadium entstehen Eineiige Zwillinge am häufigsten
	\item
		Ende der ersten Woche Einnistung in Gebärmutterschleimhaut (so weit kommt es nur in weniger als der Hälfte der entstandenen Zygoten)
	\item
		Gastrulation: Ausdifferenzierung - aus innerer Zellmasse wird Embryo aus Rest Unterstützungssystem
	\item
		Innere Zellmasse anfangs nur eine Schicht, faltet sich später zu 3 Schichten:
		\begin{itemize}
			\item
				Obere Schicht wird Nervensystem, Nägel, Zähne, Innenohr, Augenlinse und äußere Oberfläche der Haut
			\item
				mittlere Schicht wird Muskeln, Knochen, Blutkreislauf, innere Schichten der Haut, andere inneren Organe
			\item
				untere Schicht wird zum Verdauungssystem, den Lungen, Harnorganen und Drüsen
		\end{itemize}
	\item
		Paar Tage später bildet sich vom Zentrum eine U-förmige Furche nach unten, Neuralohr entsteht: Ein Ende davon wird zum Gehirn, andere Hälfte zum Rückenmark
	\item
		Unterstützungssystem wird u.A. zur Plazenta und Fruchtblase
\end{itemize}

\subsubsection{Verhalten des Fetus}
\begin{itemize}
	\item
		Ab 5. oder 6. Woche spontane Bewegungen, mit 7 Wochen Schluckauf
	\item
		Fetus schluckt Fruchtwasser (wichtig für Entwicklung des Gaumens)
	\item
		Schon vor 10. Woche \enquote{atmet} es zeitweise
	\item
		Zweiten Schwangerschaftshälfte bewegt sich Fetus nur 10 - 30 Prozent der Zeit
	\item
		Werden längere Verhaltensmuster sichtbar (z.B. frühen Morgen ruhig, Abends aktiver)
\end{itemize}

\subsubsection{Das Erleben des Fetus}
\begin{itemize}
	\item
		Fetus erfährt taktile Reize: Daumen lutschen, Nabelschnur umfassen, später auch gegen Gebärmutterwand stoßen
	\item
		Feten haben Geschmackspräferenz für süßes Fruchtwasser
	\item
		Fruchtwasser kann Gerüche von dem annehmen, was die Mutter gegessen hat, Fetus hat olfaktorische Erfahrungen
	\item
		Spätestens ab sechstem Monat reagiert Fetus auf Geräusche, rufen Veränderungen in Bewegung und Pulsfrequenz des Fetus hervor
	\item
		Sehen spielt vermutlich keine Rolle
\end{itemize}

\subsubsection{Das Lernen des Fetus}
\begin{itemize}
	\item
		Feten zeigen  ab (frühestens) 32. Woche Aumferksamkeits und Habituationsreaktionen gegenüber vielen Lauten und Geräuchen (z.B. verschiedene Silben unterscheiden)
	\item
		Neugeborene zeigen Präferenz für den Geruch des eigenen Fruchtwassers und für von der Mutter ausgiebig konsumierte Geschmäcke (Viel Karottensaft in Schwangerschaft z.B.)
	\item
		Bevorzugt Stimme von Mutter
\end{itemize}

\subsection{Risiken der pränatalen Entwicklung}
\begin{itemize}
	\item
		Ungefähr 45\% der Schwangerschaften spontan beendet, bevor die Frau überhaupt etwas mitbekommen hat
	\item
		Etwa 15 - 20 \% der Frauen die Schwangerschaft bemerken haben Fehlgeburt
	\item
		Über 90\% der tatsächlich geborenen Kinder aber völlig gesund
	\item
		Teratogene: Umwelteinflüsse die potentiell Schädigungen während Schwangerschaft hervorrufen können
	\item
		Teratogene oft aber nur gefährlich, wenn sie während \textbf{sensibler Phase} auftreten (z.B. Contergan nur schädlich zwischen vierter und sechster Woche, Zeit in der sich Gliedmaßen bilden)
	\item
		Teratogene auch individuell mehr oder weniger schlimm, Schädigung kann manchmal erst im Erwachsenenalter der Kinder festgestellt werden

\end{itemize}

\subsubsection{Legale Drogen}
\begin{itemize}
	\item
		Manche Medikamente können gefährlich sein - nur unter ärztlicher Aufsicht einnehmen
	\item
		Rauchen $\Rightarrow$ verlangsamtes Wachstum und geringes Geburtsgewicht
		\begin{itemize}
			\item
				Rauchen trotz Schwangerschaft varriert nach Schicht: 24\% Oberschicht, 15\% Mittelschicht, 40\% Unterschicht
		\end{itemize}
	\item
		Alkoholkonzentration in Mutter und Fetus gleichen sich an, Fetus kann aber schlechter abbauen $\Rightarrow$ länger im System des Fetus
		\begin{itemize}
			\item
				Babys alkoholkranker Frauen häufig Alkoholembryopathie: bei Geburt deformierte Gesichtszüge, geistige Behinderung, Aufmerksamkeitsprobleme, Hyperaktivität, Organschäden
			\item
				Auch mäßiges Trinken (weniger als ein Getränk pro Tag) kurz- und langfristige negative Auswirkungen
			\item
				Vollrausch besonders schädlich
			\item
				Vermutlich weil Alkohol Zellen im fetalen Gehirn absterben lässt
		\end{itemize}
\end{itemize}

\subsubsection{Illegale Drogen}
\begin{itemize}
	\item
		Schwierig zu untersuchen, da zusätzlich häufig Alkohol oder Tabak konsumiert wird
	\item
		Marihuana im Verdacht negative Auswirkungen auf Entwicklung zu haben, aber noch keine Beweise
	\item
		Kokain $\Rightarrow$ verzögertes Wachstum im Uterus, Frühgeburt, kleiner Kopfumfang. Bei Neugeborenen und älteren Kindern Fähigkeit Erregung und Afumerksamkeit zu steuern beeintr"achtigt
\end{itemize}

\subsubsection{Umweltverschmutzung}
\begin{itemize}
	\item
		Schlecht für Kinder
\end{itemize}


\subsubsection{Mütterseitige Faktoren}
\begin{itemize}
	\item
		Wenn Mutter über 15 und unter 35 größte Wahrscheinlichkeit für gesundes Kind
	\item
		ältere Eizellen schlechter
	\item
		Unangemessene Ernährung $\Rightarrow$ Mangelerscheinungen bei Kind, besonders Gehirnwachstum beeintr"achtigt
	\item
		Unterern"ahrung später in Schwangerschaft nur untergewichtige Babys mit kleinem Kopfumfang. Unerern"ahrung am Anfang der Schwangerschaft f"uhrt zu schweren k"orperlichen Sch"adigungen
	\item
		STDs sehr gef"ahrlich f"ur Fetus
	\item
		Zusammenhang zwischen Fr"uhgeburt und gerigenm Geburtsgewicht und Stress in Schwangerschaft
	\item
		Zusammenfassung Siegler 92 f.
\end{itemize}


\subsection{Die Geburtserfahrung}
\begin{itemize}
	\item
		Geburt ist anstrengend und schmerzhaft
	\item
		F"ur Baby vermutlich weniger schlimm als f"ur Mutter
\end{itemize}

\subsection{Das Neugeborene}
\subsubsection{Aktivierungszust"ande}
\begin{itemize}
	\item
		Aktivierungszustand ist Kontinuum von Erregungsniveaus - von Tiefschlaf bis intensiver Aktivit"at
	\item
		Große Unterschiede zwischen Babys, wieviel wach, wieviel schreien, wieviel aktiv, wieviel schlafen...
\end{itemize}

\subsubsection{Schlafen}
\begin{itemize}
	\item
		Babys viel (bis 50\%) im REM Schlaf. Vermutung dass dies zur Entwicklung des visuellen Systems beitr"agt
	\item
		H"aufiger Wechsel zwischen Schlafen und Wach sein über den Tag
\end{itemize}
\subsubsection{Schreien}
\begin{itemize}
	\item
		In den ersten 3 Monaten durchschnittlich 2 Stunden am Tag, 1 Stunde im restlichen ersten Lebensjahr
	\item
		Schreien Ausdruck von Unbehagen - Schmerz, Hunger, K"alte, "Uberreizung, allerdings auch Frustration
	\item
		Wird nach und nach zu kommuniktivem Akt, versuchen Betreuungsperson etwas mitzuteilen
	\item

\end{itemize}


\subsection{Ung"unstige Geburtsausg"ange}
\subsubsection{S"auglingssterblichkeit}
\begin{itemize}
	\item
		definiert als Tod innerhalb des ersten Lebensjahres
	\item
		relativ hoch in den USA - schlechter Zugang zu Gesundheitsf"ursorge
\end{itemize}

\subsubsection{Untergewicht}
\begin{itemize}
	\item
		Durchschnittsgewicht in USA 3400 Gramm
	\item
		Weniger als 2500 Gramm ungergewichtigvor
	\item
		Eng mit Armut verkn"upft
	\item
		Durchschnittlich mehr Entwicklungsprobleme als normalgewichtige, je geringer Gewicht desto schlimmer
	\item
		Oft ablenkbar, hyperaktiv, Lernschwierigkeiten
	\item
		Meisten untergewichtigen Kinder entwickeln sich trotzdem gut
	\item
		Babymassagen helfen untergewichtigen Kindern
\end{itemize}


\subsubsection{Modell multipler Risiken}
\begin{itemize}
	\item
		Risiken treten häufig zusammen auf, je mehr desto schlechter
	\item
		Armut vereint viele Risiken
	\item
		Entwicklungsresilienz ist erfolgreiche Entwicklung trotz mehrfacher Entwicklungsrisiken
		\begin{itemize}
			\item
				Oft Zwei Faktoren: wohlwollende F"ursorge von einer Person und bestimmte Pers"onlichkeitseigenschaften (Intelligenz, Empathie)
		\end{itemize}

	\item
		Zusammenfassung Siegler 111 ff.
\end{itemize}

