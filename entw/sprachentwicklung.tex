\section{Sprachentwicklung}
Symbole: Systeme mit denen unsere Gedanken, Gefühle und Wissensbestände repräsentiert und anderen Menschen mitgeteilt werden können.
\subsection{Die Komponenten der Sprache}
\begin{itemize}
	\item Generativität: aus endlicher Menge von Wörtern wird unendliche Menge von Sätzen
	\item Phoneme: elementare lautliche Einheiten
	\item Morpheme: Mehrere Phoneme zusammengefasst, kleinste bedeutungstragende Einheit(ich, Hund jeweils ein Morphem, Hund\textbf{e} aber zwei)
	\item Syntax einer Sprache: Regeln die spezifizieren wie Wörter zusammengefügt werden können
	\item Metalinguistisches Wissen: Metawissen über Sprache, Eigenschaften, Funktion..
	\item Schritte auf Erwerb von Sprache:
		\begin{enumerate}
			\item Phonologische Entwicklung: Erwerb von Wissen über das Lautsystem
			\item
				semantische Entwicklung: Erlernen des Systems mit dem Bedeutung ausgedrückt wird, einschließlich Lernen von Wörtern
			\item
				syntaktische Entwicklung: Erlernen der Syntax
			\item Pragmatische Entwicklung: wie Sprache angewendet wird

		\end{enumerate}
\end{itemize}
\subsection{Vorraussetzungen des Spracherwerbs}
\subsubsection{ Das menschliche Gehirn}
\begin{itemize}
	\item Tiere können nur eingeschränkt Sprache lernen, selbst nach intensivem Unterricht nur Grundzüge $\leftrightarrow$ Menschen lernen Sprache "'von selbst"'
	\item Beziehung zwischen Sprache und Gehirn:
		\begin{itemize}
			\item
				Sprachverarbeitet funktional lokalisiert: Bei den 90\% rechtshändern ist Sprache vorwiegend im linken Neocortex (Paul Broca 1861, nach Untersuchen an Unfallopfern)
			\item Broca-Aphasie: Schädigung des Broca-Areals, kurze Wortketten, keine Syntax, zögernd
			\item
				Wernicke-Aphasie: Schädigung des Wernicke-Areals/auditiver Cortex, Produzieren Sprache, aber ohne Sinn, auch Verständnis beeinträchtigt.
			\item
				Auch bei Gebärdensprachebenutzern: Verletzung linke Hemisphäre $\Rightarrow$ Aphasie: 
				Linke Späre spezialisiert auf analytische, serielle Verarbeitung von Sprache
		\end{itemize}
	\item
		Frühen Lebensjahre kritische Phase für Sprachererwerb, danach Spracherwerb deutlich schwieriger
		\begin{itemize}
			\item
				\enquote{Wolfskinder} liefern Indizien, aber keine guten (andere Gründe für keine Sprache?)
			\item
				Menschen die mit 4+-Jahren Englisch als Zweitsprache lernen: weniger linkshemisphärische Lokalisation von Englischverarbeitung (unabhängig von wie lange schon Englischsprachig bei Untersuchung)
			\item
				Selbes Prinzip bei Gehörlosen/Gebärdensprache: Je früher anfangen, desto besser als Erwachsene
			\item
				Möglicheweise aufgrund begrenzter kognitiver Fähigkeiten von Kindern: Können nur kleinere Sprachsamples verarbeiten, an denen lassen sich Struktur leichter erkennen als an großen Samples
		\end{itemize}
\end{itemize}
\subsubsection{Eine menschliche Umwelt}
\begin{itemize}
	\item
		Um Sprache zu lernen ist es wichtig mit anderen sprechenden Menschen in Kontakt zu kommen
	\item
		Spezieller Redestil, der an Kinder gerichtet ist (\enquote{Ammensprache})
		\begin{itemize}
			\item
				Emotionaler Tonfall
			\item
				Übertreibung
			\item
				Höhere Stimme, Intonationsmuster schwankt stark (abrupter Wechsel zwischen hohen und tiefen Tönen)
			\item
				Langsamer und deutlicher
			\item
				Übertriebene Mimik
			\item
				Viele dieser Kennzeichnungen in vielen verschiedenen Sprachen, auch Gebärdensprache
		\end{itemize}
	\item
		Kinder nutzen Tonfall ihrer Mütter um Bedeutung zu interpretieren ("NEIN!" vs. "jaaa :)")
		\begin{itemize}
			\item
				Nachgewiesen in Experiment von Anne Fernald (1989): Kindern wurde Spielzeug gegeben mit "yes, good boy"/"no, don't touch" in positivem/negativem Tonfall (gemixt). Tonfall war wichtiger als tatsächlich gesagtes
		\end{itemize}
	\item
		Kinder präferieren kindliche Sprache, auch wenn an anderes Kind gerichtet und in fremder Sprache ist
		\begin{itemize}
			\item
				Kinder hören kindlicher Sprache länger zu als erwachsener Sprache (auch in fremder Sprache)
			\item
			 	Kleinkinder u. Erwachsene lernen schneller Fremdsprachenwörter, wenn in kindlicher Sprache
		\end{itemize}
	\item
		Kindliche Sprache aber nicht universell:
		\begin{itemize}
			\item
				Verschiedene Inselvölker glauben das Kindern die Fähigkeit zum Sprachverstehen fehlt
			\item
				Sprechen nicht mit Kindern
			\item
				Wenn Kinder mit Sprechen beginnen gibt es \enquote{Sprachtraining}, keine kindliche Sprache
	\item
		Ob Eltern direkt zu Kindern sprechen oder nicht beeinflusst Geschwindigkeit des frühen Sprachlernens, nicht aber letzendliches Sprachniveau
		\end{itemize}
\end{itemize}

\subsection{Der Prozess des Spracherwerbs}
Bei Spracherwerb sowohl Zuhören als auch Sprechen beteiligt
\subsubsection{Sprachwahrnehmung}
\begin{itemize}
	\item
		Schon im Mutterleib Vorliebe für Sprache/Stimme der Mutter: Grund ist Prosodie(Charakteristischer Rythmus, Tempo, Tonfall, Melodie, Intonation...etc. einer Sprache)
	\item
		Prosodie allein reicht nicht fürs lernen, sprachlichen Laute müssen unterschieden werden können - können Kinder aber bereits
	\item
		Wahrnehmung von sprachlichen Lauten als diskrete Klassen: kategoriale Wahrnehmung
		\begin{itemize}
			\item
				Sprachsynthesizer produziert Kontinuum aus /b/s, die langsam in /p/s umgewandelt wurden (Laut grundsätzlich gleich, nur VOT - Voice onset time, Zeitpunkt(Vibration der Stimmbänder)-Zeitpunkt(Freilassen des Luftstroms durch die Lippen)). Bei /b/ (15 ms) viel kürzer als /p/ (150ms)
			\item
				Erwachsene können kontinuierlichen Übergang nicht wahrnehmen, hören viele bs, dann abrupter Wechsel zu ps (VOT$<$25ms $\Rightarrow$ b, VOT$>$25ms $\Rightarrow$ p), nehmen zwei diskrete Kategorien war
			\item
				Babys können auch Unterscheiden: Saugen an Schnuller $\Rightarrow$ Laut wird von Computer abgespielt. Nach mehrmals dem selben Laut saugen Babys weniger begeistert. Neuer Laut $\Rightarrow$ Saugreaktion steigt an, können Ton unterscheiden. Zwei Klassen von Babys, bei beiden erst Habituation auf b:
				\begin{itemize}
					\item
						Wechsel auf Ton der (von Erwachsenen) als p wahrgenommen wird
					\item
						Wechsel auf Ton der (von Erachsenen) immer noch als b wahrgenommen wird
				\end{itemize}
				VOT Unterschiede aber bei beiden gleich.\\
				Bei p (neue Lautklasse) erhöhte Saugrate, bei b weiter Habituation (Eimas, Siqueland, Jusczyk, Vigorito, 1971)
		\end{itemize}
	\item
		Kinder unterscheiden aber mehr als Erwachsene: Sprache verwenden nur Teilmenge aller Phonemklassen (r und l in Deutsch unterschiedlich, nicht in Japanisch)
	\item
		Diese Fähgikeit ist angeboren (bereits bei Geburt vorhanden) und unabhängig von Erfahrung (Kinder unterscheiden Laute, die sie nie zuvor gehört haben)
	\item
		Auch manche Tiere können kategorial Laute Wahrnehmen: keine menschliche Fähgikeit, keine Fähigkeit, die ausschließlich auf Sprache spezialisiert ist.
	\item
		Sprachwahrnehmung verändert sich mit Entwicklung:
		\begin{itemize}
			\item
				Fähigkeit Phoneme zu unterscheiden geht verloren, am Ende des ersten Lebensjahres Sprachwahrnehmung ähnlich wie Eltern
			\item
		\end{itemize}
	\item
	 9 Monate alte Babys können Unterschiede in Betonungsmustern wahrnehmen
 \item
	 Verteilungscharakteristika (Wahrscheinlichkeit für bestimmte Lautkombinationen) wird wahrgenommen
 \item
	 Säuglinge strengen sich an um Muster in Lauten zu identifizieren
\end{itemize}
\subsubsection{Vorbereitung für die Sprachproduktion}
\begin{itemize}
	\item
		6-8 Wochen erste sprachliche Laute ("'oooh"'"'aaah"'"'gu"'), gennannt \enquote{cooing}
	\item
		Zwischen 6. und 10. Monat, ca. 7 Monate tritt \enquote{babbling}, plappern ein: Konsonant+Vokal wird immer wieder wiederholt.
	\item
		Wichtig dabei: Rückmeldung zu erhalten. Gehörlose Babys bis 5/6 Monate  ähnliche Laute, plappern aber erst sehr spät und begrenzt
	\item
		Erste Anzeichen von Kommunikation ist turn-taking (Guck-guck, nimm-und-gib-spiele), Wechsel zwischen aktiv und passiv (wie in gespräch)
	\item
		Wichtig auch: Intersubjektivität, gemeinsames Aufmerksamkeitszentrum (vorher geteilte Aufmerksamkeit: Eltern schauen da hin wo baby hin schaut): Mit Finger auf etwas Zeigen: Vor 9 Monaten schauen auf finger, danach schauen auf gezeigtes
\end{itemize}
\subsubsection{Ersten Worte}
\begin{itemize}
	\item
		Frühe Worterkennung: Bereits mit 4.5 Monaten kann eigener Name erkannt werden, Wörter müssen erst erkannt werden, Verstehen erst danach
	\item
		Mit ca. 6 Monaten erste Referenzen verstanden (Mama und Papa)
	\item
		Mit 10 Monaten Verstehenswortschatz zwischen 11 und 145 Wörtern
	\item
		Frühe Wortprodutkion: zwischen 10 und 15 Monaten ersten Worte. Erstes Wort = spezifische Äußerung um konsistent etwas zu bezeichnen (\enquote{woof} für Hund auch Wort)
	\item
		Am Anfang Problem Wörter auszusprechen: schwierige Teile werden Weggelassen oder vertauscht. Banane $\rightarrow$ nane, Spaghetti $\rightarrow$ pasketti
	\item
		Holographische Phase: Kinder sprechen nur Wort für Wort
	\item
		Entwicklung des Wortschatzes proportional zu Menge an Sprache die Kind hört
	\item
		Überdehnung: Verwendung von Wort in breiterem Kontext: (Hund für jedes 4beinige Tier) (Kind schaut auf zwei Bilder von Schaf und Hund, sagt zu beidem Hund. Soll es auf Schaf zeigen, kann es das tun, das Wort fehlt nur in seinem produktivem Wortschatz)
	\item
		Mit 18 Monaten produktiver wortschatz von ca 50 wörtern, lernen danach sehr schnell (durchschnittlich 5-10 Wörter pro Tag)
	\item
		Kinder nutzen Prinzipien um wörter zu lernen:
		\begin{itemize}
			\item für das ganze Objekt, nicht Teil (Ganzheitsconstraint)
			\item  nur ein Name für Sachverhalt(Disjunktionsconstraint)
			\item  pragmatische Hinweise (Aufmerksamkeit Erwachsener)
			\item grammatische Klasse (bereits ab 2 und 3 Jahren)
			\item	Objekte desselben Typs eher mit demselben Wort bezeichnet als thematisch verbundene Objekte(taxonomieconstraint, z.B. Hund f"ur verschiedene Hunderassen, aber nicht f"ur Knochen, Halsband, Leine)
		\end{itemize}
\end{itemize}
\subsubsection{Das Zusammenfügen von Wörtern}
\begin{itemize}
	\item
		Gegen Ende 2. Lebensjahr erste Sätze im Telegrammstil, 2wortsätze
	\item
		Mit ca. 2.5 Jahren vierwortsätze, dann komplexere Sätze mit mehr als einer Phrase (nebensätze etc)
	\item Kinder übern Sprache selbst, allein (abends im bett z.B.)
	\item
		Kinder lernen grammatikalische Regeln, nachgewiesen durch Übergeneralisierung (man $\rightarrow$ mans, statt man  $\rightarrow$  men), gelingt es kindern nicht korrekte unregelmäßige Form abzurufen wird die Standardform verwendet
	\item
		Rückmeldung der Eltern (Korrektur von Fehlern) spielt nur sehr gerine Rolle
\end{itemize}

\subsubsection{Gesprächsfähigkeit}
\begin{itemize}
	\item
		Kinder sprechen zu sich selbst um Handlungen zu organisieren, später dann als Denken internalisiert
	\item
		Neigen dazu egozentrisch zu sprechen, kollektive Monologe entstehen, keine Gespräche, von 21 - 36 Monaten steigt Anteil von sinnvollen Antworte nvon 20 auf mehr als 40 prozent
	\item
		3jährige kaum über Vergangenes, fünfjährige schon.  
\end{itemize}

Zusammenfassung Siegler et.al. 351 ff.
