\documentclass[a4paper]{scrartcl}
\usepackage[ngerman]{babel}
\usepackage[utf8]{inputenc}
\usepackage{amsmath}
\usepackage{amssymb}
\usepackage{csquotes}
\usepackage{hyperref}

\title{Entwicklungs-Psychologie Mitschrift WS14/15}
\begin{document}
\maketitle
\tableofcontents
\newpage
\section{Einführung/16.10.}
\begin{itemize}
	\item
		Passwort für studon: "'EP1516"'
	\item
		Was ist Entwicklungspsychologie?
		\begin{itemize}
			\item
				Psychologie: "Lehre vom Verhalten und Erleben des Menschen"
				\begin{itemize}
					\item Beschreiben (wie?)
					\item Erklären (warum?)
					\item Vorhersagen (Gutachter z.B.: "wahrscheinlich geht es Kind bei Elternteil A besser als bei Elternteil B), Vorhersagen nur wenn wir Erklären können, Erklären nur wenn wir beschreiben können.

				\end{itemize}
			\item
				EP: Veränderungen des Verhaltens und Erlebens des Menschen über das Alter hinweg
			\item Entwicklungspsychologie der Lebensspanne.\\
				Früher: Entwicklungspsychologie als Kinder- und JugendEP, keine Entwicklung im Erwachsenenalter. \\
				Mittlerweile: Veränderungen von Konzeption bis zum Tod)
			\item
				Veränderungen innerhalb spezifischer Verhaltensbereiche, bzw. psychischer Funktionen
			\item
				Qualitative vs. quantitative Veränderungen 
				\begin{itemize}
					\item
						Qualitative: z.B. motorische Entwicklung (Kopfheben, krabbeln, stehen, gehen)
					\item
						Quantitative: z.B. Intelligenzquotient/Intelligenz wird "mehr", Entwicklung des Wortschatzes
				\end{itemize}
			\item Allgemeine vs. individuelle Veränderungen
				\begin{itemize}
					\item z.B. manche Kinder werden schneller "'intelligent"' als andere
				\end{itemize}
			\item Grundlagen der Entwicklung
				\begin{itemize}
					\item Reifungsprozesse, soziale oder kulturelle Einflüsse
					\item "Lesen" entsteht weil wir das von Kindern verlangen, Laufen lernt jedes Kind
				\end{itemize}
			\item
				Ursachen individueller Unterschiede
				\begin{itemize}
					\item
						genetische Unterschiede?
					\item
						Umweltunterschiede? (Viel mit Kind gesprochen/ viel gefördert?)
				\end{itemize}
		\end{itemize}
	\item
		Unterschiede zwischen ... und EP
		\begin{itemize}
					\item
						Behaltensleistung von an die Wand projeziierten Bildern Nimmt ab über Zeit nach sehen.
				\begin{itemize}
			\item Allgemeine Psychologie: allgemein nimmt ab, Unterschiede nur "Störfaktor"
					\item Differentielle: Person A merkt sich besser als Person B
						\item
							Sozialpsychologie: In Situation A kann man sich Dinge besser als in Situation B merken, sozialer Kontext spielt Rolle
							\item Pädagogische/Klinische Psychologie: Anwedungsdisziplinen. Personen zeigen Abweichungen, Unterschiede zwischen Soll/Ist
								\item Entwicklungspschologie: 

				\end{itemize}
		\end{itemize}
\end{itemize}
\section{Theorien, Modelle und Methoden der Entwicklungspsychologie/23.10./30.10.}
\subsection{Theorien, Modelle...}
\subsection{Methoden der Entwicklungspsychologie}
Viel Schmarrn unterwegs, deswegen hier gute Vorgehensmodelle und Methoden:
\begin{itemize}
	\item
		Veränderung, Entwicklungsverläufe, Stabilität
		\begin{itemize}
			\item
				Qaulitative Veränderungen
			\item
				Quantitative Veränderungen
			\item
				Veränderung im Sinne komplexer Aufbauprozesse
				\begin{itemize}
					\item
						Leisungsmotivation erst mit ca. 3 Jahren
					\item
						Heckhausen (Nobelpreisträger), "Gewinnen" beim Turmbauen erst mit ca. 3 Jahren
					\item
						In den Jahren vorher aber vorläufer: "Selbermachen", Trotz
				\end{itemize}
		\end{itemize}
		\item
			Allgemeine Entwicklung
			\begin{itemize}
				\item
					Entwicklungsfunktion
			\end{itemize}
	\item
		Entwicklungspsychologische Designs
	\item
		Datengewinnungsmethoden
	\item
		Forschungsstrategien
\end{itemize}
\section{Biologische Grundlagen/6.11.}

\section{Pränatale Entwicklung, Neugeborenenalter}
\begin{itemize}
	\item
		Epigenese: Vorstellung, dass sich neue Strukturen und Funktionen entwickeln und nicht von Anfang an da sind und nur wachsen
	\item
		Meiose Zellteilung bei der die entstehenden Zellen nur die Hälfte der Chromosomen besitzt
\end{itemize}

\subsection{Pränatale Entwicklung}
\begin{itemize}
	\item
		Zellmigration: Wanderung neu gebildeter Zellen von ihrem Ausgangspunkt an andere Stelle im Embryo (z.b. Neuronen im Cortex)
	\item
		Apoptose: programmierter Zelltod
\end{itemize}

\subsubsection{Früheste Entwicklung}
\begin{itemize}
	\item
		Am vierten Tag nach Befruchtung formen sich die Zellen zu einer Hohlkugel (Blastozyste) in der sich ein Zellhaufen (die \enquote{innere Zellmasse}) befindet
	\item
		In diesem Stadium entstehen Eineiige Zwillinge am häufigsten
	\item
		Ende der ersten Woche Einnistung in Gebärmutterschleimhaut (so weit kommt es nur in weniger als der Hälfte der entstandenen Zygoten)
	\item
		Gastrulation: Ausdifferenzierung - aus innerer Zellmasse wird Embryo aus Rest Unterstützungssystem
	\item
		Innere Zellmasse anfangs nur eine Schicht, faltet sich später zu 3 Schichten:
		\begin{itemize}
			\item
				Obere Schicht wird Nervensystem, Nägel, Zähne, Innenohr, Augenlinse und äußere Oberfläche der Haut
			\item
				mittlere Schicht wird Muskeln, Knochen, Blutkreislauf, innere Schichten der Haut, andere inneren Organe
			\item
				untere Schicht wird zum Verdauungssystem, den Lungen, Harnorganen und Drüsen
		\end{itemize}
	\item
		Paar Tage später bildet sich vom Zentrum eine U-förmige Furche nach unten, Neuralohr entsteht: Ein Ende davon wird zum Gehirn, andere Hälfte zum Rückenmark
	\item
		Unterstützungssystem wird u.A. zur Plazenta und Fruchtblase
\end{itemize}

\subsubsection{Verhalten des Fetus}
\begin{itemize}
	\item
		Ab 5. oder 6. Woche spontane Bewegungen, mit 7 Wochen Schluckauf
	\item
		Fetus schluckt Fruchtwasser (wichtig für Entwicklung des Gaumens)
	\item
		Schon vor 10. Woche \enquote{atmet} es zeitweise
	\item
		Zweiten Schwangerschaftshälfte bewegt sich Fetus nur 10 - 30 Prozent der Zeit
	\item
		Werden längere Verhaltensmuster sichtbar (z.B. frühen Morgen ruhig, Abends aktiver)
\end{itemize}

\subsubsection{Das Erleben des Fetus}
\begin{itemize}
	\item
		Fetus erfährt taktile Reize: Daumen lutschen, Nabelschnur umfassen, später auch gegen Gebärmutterwand stoßen
	\item
		Feten haben Geschmackspräferenz für süßes Fruchtwasser
	\item
		Fruchtwasser kann Gerüche von dem annehmen, was die Mutter gegessen hat, Fetus hat olfaktorische Erfahrungen
	\item
		Spätestens ab sechstem Monat reagiert Fetus auf Geräusche, rufen Veränderungen in Bewegung und Pulsfrequenz des Fetus hervor
	\item
		Sehen spielt vermutlich keine Rolle
\end{itemize}

\subsubsection{Das Lernen des Fetus}
\begin{itemize}
	\item
		Feten zeigen  ab (frühestens) 32. Woche Aumferksamkeits und Habituationsreaktionen gegenüber vielen Lauten und Geräuchen (z.B. verschiedene Silben unterscheiden)
	\item
		Neugeborene zeigen Präferenz für den Geruch des eigenen Fruchtwassers und für von der Mutter ausgiebig konsumierte Geschmäcke (Viel Karottensaft in Schwangerschaft z.B.)
	\item
		Bevorzugt Stimme von Mutter
\end{itemize}

\subsection{Risiken der pränatalen Entwicklung}
\begin{itemize}
	\item
		Ungefähr 45\% der Schwangerschaften spontan beendet, bevor die Frau überhaupt etwas mitbekommen hat
	\item
		Etwa 15 - 20 \% der Frauen die Schwangerschaft bemerken haben Fehlgeburt
	\item
		Über 90\% der tatsächlich geborenen Kinder aber völlig gesund
	\item
		Teratogene: Umwelteinflüsse die potentiell Schädigungen während Schwangerschaft hervorrufen können
	\item
		Teratogene oft aber nur gefährlich, wenn sie während \textbf{sensibler Phase} auftreten (z.B. Contergan nur schädlich zwischen vierter und sechster Woche, Zeit in der sich Gliedmaßen bilden)
	\item
		Teratogene auch individuell mehr oder weniger schlimm, Schädigung kann manchmal erst im Erwachsenenalter der Kinder festgestellt werden

\end{itemize}

\subsubsection{Legale Drogen}
\begin{itemize}
	\item
		Manche Medikamente können gefährlich sein - nur unter ärztlicher Aufsicht einnehmen
	\item
		Rauchen $\Rightarrow$ verlangsamtes Wachstum und geringes Geburtsgewicht
		\begin{itemize}
			\item
				Rauchen trotz Schwangerschaft varriert nach Schicht: 24\% Oberschicht, 15\% Mittelschicht, 40\% Unterschicht
		\end{itemize}
	\item
		Alkoholkonzentration in Mutter und Fetus gleichen sich an, Fetus kann aber schlechter abbauen $\Rightarrow$ länger im System des Fetus
		\begin{itemize}
			\item
				Babys alkoholkranker Frauen häufig Alkoholembryopathie: bei Geburt deformierte Gesichtszüge, geistige Behinderung, Aufmerksamkeitsprobleme, Hyperaktivität, Organschäden
			\item
				Auch mäßiges Trinken (weniger als ein Getränk pro Tag) kurz- und langfristige negative Auswirkungen
			\item
				Vollrausch besonders schädlich
			\item
				Vermutlich weil Alkohol Zellen im fetalen Gehirn absterben lässt
		\end{itemize}
\end{itemize}

\subsubsection{Illegale Drogen}
\begin{itemize}
	\item
		Schwierig zu untersuchen, da zusätzlich häufig Alkohol oder Tabak konsumiert wird
	\item
		Marihuana im Verdacht negative Auswirkungen auf Entwicklung zu haben, aber noch keine Beweise
	\item
		Kokain $\Rightarrow$ verzögertes Wachstum im Uterus, Frühgeburt, kleiner Kopfumfang. Bei Neugeborenen und älteren Kindern Fähigkeit Erregung und Afumerksamkeit zu steuern beeintr"achtigt
\end{itemize}

\subsubsection{Umweltverschmutzung}
\begin{itemize}
	\item
		Schlecht für Kinder
\end{itemize}


\subsubsection{Mütterseitige Faktoren}
\begin{itemize}
	\item
		Wenn Mutter über 15 und unter 35 größte Wahrscheinlichkeit für gesundes Kind
	\item
		ältere Eizellen schlechter
	\item
		Unangemessene Ernährung $\Rightarrow$ Mangelerscheinungen bei Kind, besonders Gehirnwachstum beeintr"achtigt
	\item
		Unterern"ahrung später in Schwangerschaft nur untergewichtige Babys mit kleinem Kopfumfang. Unerern"ahrung am Anfang der Schwangerschaft f"uhrt zu schweren k"orperlichen Sch"adigungen
	\item
		STDs sehr gef"ahrlich f"ur Fetus
	\item
		Zusammenhang zwischen Fr"uhgeburt und gerigenm Geburtsgewicht und Stress in Schwangerschaft
	\item
		Zusammenfassung Siegler 92 f.
\end{itemize}


\subsection{Die Geburtserfahrung}
\begin{itemize}
	\item
		Geburt ist anstrengend und schmerzhaft
	\item
		F"ur Baby vermutlich weniger schlimm als f"ur Mutter
\end{itemize}

\subsection{Das Neugeborene}
\subsubsection{Aktivierungszust"ande}
\begin{itemize}
	\item
		Aktivierungszustand ist Kontinuum von Erregungsniveaus - von Tiefschlaf bis intensiver Aktivit"at
	\item
		Große Unterschiede zwischen Babys, wieviel wach, wieviel schreien, wieviel aktiv, wieviel schlafen...
\end{itemize}

\subsubsection{Schlafen}
\begin{itemize}
	\item
		Babys viel (bis 50\%) im REM Schlaf. Vermutung dass dies zur Entwicklung des visuellen Systems beitr"agt
	\item
		H"aufiger Wechsel zwischen Schlafen und Wach sein über den Tag
\end{itemize}
\subsubsection{Schreien}
\begin{itemize}
	\item
		In den ersten 3 Monaten durchschnittlich 2 Stunden am Tag, 1 Stunde im restlichen ersten Lebensjahr
	\item
		Schreien Ausdruck von Unbehagen - Schmerz, Hunger, K"alte, "Uberreizung, allerdings auch Frustration
	\item
		Wird nach und nach zu kommuniktivem Akt, versuchen Betreuungsperson etwas mitzuteilen
	\item

\end{itemize}


\subsection{Ung"unstige Geburtsausg"ange}
\subsubsection{S"auglingssterblichkeit}
\begin{itemize}
	\item
		definiert als Tod innerhalb des ersten Lebensjahres
	\item
		relativ hoch in den USA - schlechter Zugang zu Gesundheitsf"ursorge
\end{itemize}

\subsubsection{Untergewicht}
\begin{itemize}
	\item
		Durchschnittsgewicht in USA 3400 Gramm
	\item
		Weniger als 2500 Gramm ungergewichtigvor
	\item
		Eng mit Armut verkn"upft
	\item
		Durchschnittlich mehr Entwicklungsprobleme als normalgewichtige, je geringer Gewicht desto schlimmer
	\item
		Oft ablenkbar, hyperaktiv, Lernschwierigkeiten
	\item
		Meisten untergewichtigen Kinder entwickeln sich trotzdem gut
	\item
		Babymassagen helfen untergewichtigen Kindern
\end{itemize}


\subsubsection{Modell multipler Risiken}
\begin{itemize}
	\item
		Risiken treten häufig zusammen auf, je mehr desto schlechter
	\item
		Armut vereint viele Risiken
	\item
		Entwicklungsresilienz ist erfolgreiche Entwicklung trotz mehrfacher Entwicklungsrisiken
		\begin{itemize}
			\item
				Oft Zwei Faktoren: wohlwollende F"ursorge von einer Person und bestimmte Pers"onlichkeitseigenschaften (Intelligenz, Empathie)
		\end{itemize}

	\item
		Zusammenfassung Siegler 111 ff.
\end{itemize}



\section{Entwicklung von Wahrnehmung und Motorik}
\subsection{Wahrnehmung}
\begin{itemize}
	\item
		Empfindung: Verarbeitung basaler Information aus der "au"seren Welt in Sinnesorganen und neuronalen Verb"anden im Gehirn
	\item
		Wahrnehmung: Prozess der Strukturierung und Interpretation von Sinnesinformation
\end{itemize}

\subsubsection{Sehen}
\begin{itemize}
	\item
		Schon wenige Minuten nach Geburt fangen fangen Neugeborene damit an Welt visuell zu erkunden
	\item
		Blickpräferenz: VErfahren zur Untersuchung von visueller Aufmerksamkeit bei S"auglingen, zwei Muster/Objekte vorgelegt, wird untersucht welches bevorzugt wird
		\begin{itemize}
			\item
				Kinder bevorzugen Muster "uber unstrukturierter Oberfl"ache
			\item
				grobes Zebramuster neben feinem Zebramuster, Kinder schauen dahin wo sie noch ein Muster erkennen k"onnen und nicht nur eine graue Fl"ache\\
				$\Rightarrow$ 
				Bereits mit 8 Monaten Sehverm"ogen "ahnlich wie Erwachsene

			\item
				Lieber Viel Kontrast, da Kleinkinder nur geringes Kontrastverm"ogen: Nur 2\% des Lichts das in die Augen f"allt erreicht Zapfen (Erwachsene bis zu 65\%)
			\item
				Mit 1 Monat Fokus auf Kontur/Au"senkanten, mit 2 Monaten Fokus sowohl Gesamtform als auch Details im Inneren verarbeitbar
		\end{itemize}
	\item
		Wahrnehmungskonstanz (Objekt wird kleiner $\rightarrow$ geht weiter weg, ver"andert gr"o"se nicht) schon bei Neugeborenen, Objekttrennung auch bei S"auglingen (daf"ur gemeinsame Bewegung der Objekte wichtig)
	\item
		Schon fr"uh empf"anglich f"ur Objektausdehnung (Objekt wird gr"o"ser $\rightarrow$ kommt auch mich zu)
\end{itemize}

\subsubsection{Akustische Wahrnehmung}
\begin{itemize}
	\item
		Akutisches System bei Geburt gut entwickelt, Neugeborene aber etwas schwerh"orig
	\item
		Akutische Lokalisation schon 10 Minuten nach Geburt
	\item
		M"oglicherweise angeborene, biologische Grundlage f"ur Musikwahrnehmung (Musik und Sprache "ahnliche Lernmechanismen und Gehiranktivit"aten), S"auglinge pr"aferenz f"ur konsonante (vs. dissonante) Melodien, Schon 5 Monate alte Kinder nehmen Melodie gleich war, egal wie hoch oder tief gespielt, als nicht gleich, wenn T"one vertauscht sind
\end{itemize}

\subsubsection{Ber"uhrung}
\begin{itemize}
	\item
		Anfangs viel mit Mund/Zunge
	\item
		Ab ca. 4 Monaten mehr Arm/Hand
\end{itemize}

\subsubsection{Intermodale Wahrnehmung}
\begin{itemize}
	\item
		= Kombination von Informationen aus zwei oder mehr Sinnessystemen
	\item
		Zusammenfassung Siegler 258
\end{itemize}

\subsection{Motorik}
\begin{itemize}
	\item
		Babys erst mal nur Reflexe: Greifreflex, Suchreflex (Ber"uhrung Wange $\rightarrow$ Kopfdrehung und Mund"offnen), Saugreflex
\end{itemize}
\subsubsection{Meilensteine der Motorik}
\begin{itemize}
	\item
		Siehe Siegler 261
\end{itemize}


\subsubsection{Aktuelle Perspektiven}
\begin{itemize}
	\item
		Fr"uher: neuronale Reifung des Gehirns $\rightarrow$ motorische Entwicklung
	\item
		Heute eher: Dynamische Systeme, Zusammenspiel zahlreicher Faktoren (K"orperkraft, Kontrolle "uber K"orperhaltung, Balance, Wahrnehmung...)
	\item
		Zusammenfassung Siegler 271 u 291 ff.
\end{itemize}


\section{Soziale und emotionale Entwicklung}
\begin{itemize}
	\item
		Emotionale Intelligenz: F"ahigkeiten zur Kompetenz im sozialen und emotionalen Bereich. Sich selbst motivieren, trotz Frustration, Kontrollimpulsen und Belohnungsaufschub hartn"ackig zu bleiben. Eigene/die anderer Gef"uhle erkenne und verstehen. Eigene Stimmungen und den Gef"uhlsausdruck in sozialen Interaktionen zu regulieren und sich in Emotionen von anderen hineinzuversetzen
	\item
		EQ sagt viel besser als IQ vorraus wie gut Leute im Leben zurecht kommen
\end{itemize}

\subsection{Entwicklung von Emotionen in der Kindheit}
\begin{itemize}
	\item
		Emotionen bestehen aus verschiedenen Komponenten: (Angst am Beispiel eines knurrenden Hundes)
		\begin{enumerate}
			\item
				Wunsch etwas zu tun, einschlie"slich des Wunsches, Menschen oder Dingen in der Umwelt zu entfliehen, sich ihnen zu n"ahern oder sie zu ver"andern  (weglaufen)
			\item
				Physiologische Korrelate wie Herz- oder Atemfrequenz, Hormonspiegel\dots (Erregung, Adrenalin)
			\item
				Subjektive Gef"uhle (sp"uren subjektiv Angst)
			\item
				Kognitionen, die Emotionen hervorrufen oder begleiten k"onnen ("uberlegen wie sie der Hund verletzen k"onnte)
	\end{enumerate}
\end{itemize}

\subsubsection{Theorien "uber Wesen und Entstehung von Emotionen}
\begin{itemize}
	\item
		Theorie der diskreten Emotionen/Basis-Emotionen (Izard 1991, Tomkins 1962): Emotionen angeboren, jede Emotion spezifischer und unverwechselbarer Satz an k"orperlichen und mimischen Reaktionen, diese abgrenzbaren Emotionen seit fr"uhester Kindheit vorhanden
	\item
		Affektsysteme (Sroufe 1979 u 1995): Drei grundlegende Affektsysteme - Freude/Vergn"ugen, Wut/Frustration, Misstrauen/Angst entwickeln sich in ersten Lebensjahren von primitiver Form zu fortgeschrittener
	\item
		Funktionalistischer Ansatz (Campos 1994): Grundfunktion von Emotionstheorie zielgerichtete Handlungen zu f"ordern. Emotionen sind nicht abgegrenzt und variieren je nach sozialer Umwelt in bestimmtem Ausma"s
	\item
		Alle durch empirische Studien gest"utzt, aber keine "uberlegen
\end{itemize}

\subsubsection{Entstehung von Emotionen im Entwicklungsverlauf}
\begin{itemize}
	\item
		Positive Emotionen:
		\begin{itemize}
			\item
				W"ahrend dritter Lebensmonat: Soziales L"acheln (L"acheln, das an Menschen gerichtet ist)
			\item
				Ab 2 Monaten Freude, wenn Ereignis kontrolliert werden kann (z.B. Rassel)
		\end{itemize}
	\item
		Negative Emotionen:
		\begin{itemize}
			\item
				Manchmal (in einigen Kontexten) Wut/Traurigkeit gegen Schmerz/Unbehagen abgrenzbar bei 2 Monate alten Babys
			\item
				S"auglinge manchmal negative Emotionen, die nicht zur Situation passen
			\item
				Nur undifferenziertes Missbehagen, keine Unterscheidung zwischen Wut Schmerz/Missbehagen vor 6 Monaten
			\item
				Danach insbesondere Angst vor Fremden (spiegelt vermutlich wachsende Bindung an Eltern wider)
			\item
				Ab ca. 8 Monaten: Trennungsangst
			\item
			 Ab 12 Monate: Wut, wenn nicht in der Lage Umwelt zu kontrollieren
		\end{itemize}
	\item
		selbst-bewussten Emotionen: Verlegenheit, Stolz, Schuld und Scham
		\begin{itemize}
			\item
				Nicht vor zweitem Lebensjahr, weil daf"ur Verst"andnis gebraucht wird, selbst eine von anderen abgrenzbare Person zu sein. Dies entsteht erst nach und nach w"ahrend erster Lebensjahre
			\item
				Ab 15-24 Monaten Verlegenheit
			\item
				Ob Kinder Scham oder Schuld erleben h"angt zum Teil von Erziehungspraktiken ab: Eher Schuld wenn Eltern Schlechtigkeit des Verhaltens und nicht des Kindes betonen
			\item
				Stolz nachdem Leistung erbracht wurde, h"angt aber von Kultur ab (Zuni-Indianer \enquote{nicht auffallen}, deswegen kein Stolz nach besonderer Leistung)
		\end{itemize}

	\item
		Zusammenfassung Siegler 543 f
\end{itemize}

\subsection{Die Regulierung von Emotionen}
\begin{itemize}
	\item
		Emotionale Selbst-Regulation: Prozess der Initiierung, Hemmung oder Modulierung innerer Gef"uhlszust"ande, emotionsbezogener physiologischer Prozesse und Kognitionen, sowie des Verhaltens im Dienste der Zielerreichung
	\item
		Kinder brauchen Jahre um F"ahigkeit zur Emotionsregulation zu erlernen

\end{itemize}


\subsubsection{Entwicklung der Emotionsregulierung}
\begin{itemize}
	\item
		3 allgemeine, altersbezogne Ver"anderungsmuster:
		\begin{enumerate}
			\item
				"Ubergang der Kinder von fast v"ollig auf andere Personen zu verlassen bei der Regulation von Emotionen zu helfen $\rightarrow$ wachsende F"ahigkeit zur Selbst-Regulierung
			\item
				$\rightarrow$ kognitive Strategien zur Kontrolle negativer Emotionen
			\item
				$\rightarrow$ 
				Auswahl von geeigneten Regulierungsstrategien
		\end{enumerate}
	\item
		Von Regulierung durch andere zur Selbst-Regulation
		\begin{itemize}
			\item
				Erst nur durch Eltern (tr"osten, ablenken, \dots)
			\item
				Ab ca. 6 Monaten teilweise selbst beruhigen
			\item
				Mit dem Alter bessere Selbst-regulation
			\item
				Ver"anderte Erwartungen der Eltern, erwarten dass Kinder mit emotionaler Erregung umgehen k"onnen
		\end{itemize}
	\item
		Der Gebrauch kognitiver Strategien zur Steuerung negativer Emotionen
		\begin{itemize}
			\item
				Kleinere Kinder regulieren durch Verhaltensstrategien (z.B. Ablenkung)
			\item
				Sp"ater kognitive Strategien, z.B. Bedeutung herunterspielen
		\end{itemize}

	\item
		Auswahl geeigneter Regulierungsstrategien
		\begin{itemize}
			\item
				F"ahigkeit geeignete Strategien auszuw"ahlen wird dadurch gesteigert, dass Kinder lernen zwischen kontrollierbaren (Hausaufgaben) und unkontrollierbaren (Medizinische Untersuchung) zu unterscheiden
		\end{itemize}
\end{itemize}

\subsubsection{Zusammenhang zwischen Emotionsregulierung und sozialer Kompetenz}
\begin{itemize}
	\item
		Kinder die konstruktiv mit stressvoller Situation umgehen k"onnen sind insgesamt sozial kompetenter als Kinder, die v"ollig vermeiden sich mit stressvollen Situationen auseinanderzusetzen
	\item
		Zusammenfassung Siegler S. 549
\end{itemize}



\subsection{Individuelle Unterschiede bei Emotionen und ihrer Regulierung}
\subsubsection{Temperament}
\begin{itemize}
	\item
		= veranlagunsbedingte individuelle Unterschiede in der emotionalen, motorischen und aufmerksamkeitsbezogenen Reagibilit"at und in der Selbstregulierung, "uber Situationen hinweg konsistent, sowie "uber die Zeit stabil
	\item
		New Yorker Langzeitstudie Thomas \& Chess 1977 wichtig (Tabelle "uber Temperamentsdimensionen auf verschiedenen Altersstufen Siegler 551 f.)
		\begin{itemize}
			\item
				Aktivit"atsniveau (hoch/niedrig)
			\item
				Rhythmus (Regelm"a"sig/Unregelm"a"sig)
			\item
				Ablenkbarkeit (ablenkbar/nicht ablenkbar)
			\item
				Annh"aherung/R"uckzug (positiv/negativ)
			\item
				Anpassungsf"ahigkeit (Anpassungf"ahig/Nicht anpassungf"ahig)
			\item
				Aufmerksamkeitsdauer (lang/kurz)
			\item
				Reakiontsintensit"at (Stark/Schwach)
			\item
				Reaktionschwelle (niedrig/hoch)
			\item
				Stimmungsqualit"at (positiv/negativ)
		\end{itemize}
	\item
		3 Gruppen von Babys:
		\begin{itemize}
			\item
				Einfache Babys
			\item
				Schwierige Babys
			\item
				Langsam auftauende Babys
		\end{itemize}
	\item
		Neuere Studien nur 6 Dimensionen:
		\begin{itemize}
			\item
				Angstvolles Unbehagen
			\item
				Reizbares Unbehagen
			\item
				Aufmerksamkeitsspanne und Ausdauer
			\item
				Aktivit"atsniveau
			\item
				Positiver Affekt
			\item
				Rhythmus
		\end{itemize}
	\item
		Temperament weitgehend stabil, teilweise sogar pr"a und postnatal
	\item
		Unterschiede im Temperament korrelieren mit Unterschieden in sozialer Kompetenz und Anpassungf"ahigkeit
	\item
		Anpassung"ute = Ausma"s in dem Temperament eines Individuums mit Anforderungen und Erwartungen des sozialen Umfelds "ubereinstimmt
	\item
		Zusammenfassung Siegler S 560
\end{itemize}

\subsection{Die emotionale Entwicklung von Kindern in der Familie}
\begin{itemize}
	\item
		Gene betr"achtlicher Anteil an Temperament
	\item
		Eltern-Kind-Beziehung beeinflusst emotionale Entwicklung: sichere Bindung $\rightarrow$ mehr positive Emotionen und weniger "Angstlichkeit
\end{itemize}

\subsubsection{Die elterliche Sozialisation der emotionalen Reaktion von Kindern}
\begin{itemize}
	\item
		Eltern sozialisieren emotionale Entwicklung ihrer Kinder durch
		\begin{enumerate}
			\item
				ihren Ausdruck von Emotionen gegen"uber ihren Kindern und anderen Personen
				\begin{itemize}
					\item
						Emotionen die zu Hause gezeigt werden k"onnen Sicht der Kinder auf sich selbst und andere beeinflussen
					\item
						Emotionen der Eltern dienen Kindern als Modell, wann und wie man Emotionen ausdr"uckt
				\end{itemize}
			\item
				ihre Reaktionen auf den kindlichen Ausdruck von Emotionen
			\item
				die Diskussion, die sie mit ihren Kindern "uber Emotionen und emotionale Regulierung f"uhren
	\end{enumerate}
\item
	\dots
\item
	Zusammenfassung Siegler 579 ff
\end{itemize}

\section{Bindungsentwicklung}
\subsection{Die Bindung zwischen Kindern und ihren Bezugspersonen}
\subsubsection{Bowlbys Bindungstheorie}
\begin{itemize}
	\item
		postuliert biologische Veranlagung von Kindern, Bindungen zu Versorgungspersonen zu entwickeln um eigene "Uberlebenschancen zu erh"ohen
	\item
		Kompetenzmotiviertes Kleinkind, dass engste Betruungsperson als \enquote{sichere Basis} (= Anwesenheit einer Bindungsperson bietet Gef"uhl von Sicherheit, dass es erm"oglicht die Umwelt zu erforschen) nutzt
	\item
		Bindung findet in vier Phasen statt:
		\begin{enumerate}
			\item
				Vorphase der Bindung (Geburt bis 6 Wochen): Kind zeigt angeborene Signale, meistens Schreien, ruftt dadurch andere zu sich, f"uhlt sich getr"stet
			\item
				Entstehende Bindung (6 Wochen bis 6--8 Monate): 
				\begin{itemize}
					\item
						beginnen auf vertraute Personen zu reagieren (l"acheln, plappern h"aufiger wenn Bezugspersonen da sind)
					\item
						Entwickeln Erwartungen, wie F"ursorger auf Bed"urfnisse reagieren
					\item
						Entwickeln Gef"uhl, wie sehr sie Bezugspersonen vertrauen k"onnen
				\end{itemize}
			\item
				Ausgepr"agte Bindung (6--8 Monate bis 18--24 Monate):
				\begin{itemize}
					\item
						Suchen aktiv Kontakt zu Bezugspersonen
					\item
						begr"u"sen Mutter bei ihrem Erscheinen freudig, zeigen Unbehagen wenn sie weg geht
					\item
						Meistens Mutter nun sichere Basis
				\end{itemize}
			\item
				Reziprope Beziehungen (ab 18--24 Monaten):
				\begin{itemize}
					\item
						Ansteigenden kognitiven und sprachlichen F"ahigkeiten erm"oglichen Gef"uhle, Ziele und Motive der Eltern zu verstehen
					\item
						Nutzen dieses Verst"andnis um ihre Anstreungen darauf auszurichten in N"ahe der Eltern zu kommen
					\item
						Trennungsstress geht zur"uck, mehr wechselseitig geregelte Beziehung entsteht
				\end{itemize}
		\end{enumerate}
	\item
		Ergebnis der Phasen ist andauernde emotionale Verkn"upfung und ein \enquote{inneres Arbeitsmodell von Bindung}(mentale Repr"asentation des Selbst, der Bindungsperson und der Beziehungen im Allgemeinen, leitet die Interatkion mit den Versorgern und anderen Personen in Kindheit und sp"ater)
	\item
		Theorie sp"ater von Ainsworths empirisch bewiesen und erweitert
	\item
		Messung mit \enquote{Fremde Situation}
	\item
		Entdeckung von drei Bindungskategorien:
		\begin{enumerate}
			\item
				Sichere Bindung: Mehrzahl der Kinder (ca 65\%)
				\begin{itemize}
					\item
						Eltern als sichere Basis
					\item
						Qualitativ hochwertige, relativ eindeutige Beziehung zu Bindungsperson
					\item
						Weint vielleicht wenn Eltern weggehen, freuen sich aber wenn sie zur"uckkommen, erholen sich schnell von Unbehagen
				\end{itemize}
			\item
				Unsicher-vermeidend: 20\%
				\begin{itemize}
					\item
						weniger positive Bindung zu Bezugsperson als sicher gebundene Kinder
					\item
						meiden Eltern in Fremder Situation
					\item
						Begr"u"sen Bezugsperson beim Wiedersehen nicht einmal, ignorieren sie
				\end{itemize}
			\item
				Unsicher-amivalent: 15\%
				\begin{itemize}
					\item
						Kinder/S"auglinge klammern, bleiben nahe bei Bezugsperson statt Umwelt zu erkunden
					\item
						"angstlich in Fremder Situation, k"onnen von Fremden nicht leicht beruhigt werden
					\item
						Wenn Bezugsperson zur"uckkommt lassen sie sich nur schwer beruhigen, suchen einerseits Trost, andererseits widersetzen sie sich getr"ostet zu werden
				\end{itemize}
			\item
				Desorganisiert-Desorientiert: (im Anschluss an Ainsworths Forschungen hinzugekommen, weniger als 5\%)
				\begin{itemize}
					\item
						Keine konsistente Stressbew"altigungsstrategie in Fremder Situation, Verhalten oft konfus, widerspr"uchlich
					\item
						wollen sich Elternteil n"ahern, sehen ihn aber auch als Quelle der Angst
				\end{itemize}
		\end{enumerate}
	\item
		Bei Kindern mit unsicherer Bindung in bindungsrelevanten Situation Cortisolreaktion (Stressreaktion)
	\item
		H"angt mit Bindungsmodellen der Eltern zusammen
	\item
		"Ahnlich in allen Kulturen, ein paar Unterschiede gibt es jedoch:
		\begin{itemize}
			\item
				Alle japanischen unsicher gebundenen Kinder wurden als Unsicher-ambivalent klassifiziert
		\end{itemize}
\end{itemize}


\subsubsection{Einflussfaktoren auf die kindliche Bindungssicherheit}
\begin{itemize}
	\item
		Einf"uhlungsverm"ogen der Eltern (Einf"uhlsame Eltern produzieren sicher gebundene Kinder)
	\item
		Temperament des Kindes spielt nur vergleichsweise geringe Rolle (eher unintuitiv)
\end{itemize}

\subsubsection{Langzeitwirkungen der Bindungssicherheit}
\begin{itemize}
	\item
		Kinder die sensible, unterst"utzende ERziehung erleben (wie sie mit sicherer Bindung einhergeht) lernen, dass es akzeptabel ist, Emotionen in angemessener Weise auszudr"ucken und dass emotionale Kommunikation wichtig ist
	\item
		Sichere Kinder ungef"ahr in allem (Freundschaften, kontaktfreudigkeit, Schule) besser
	\item
		Vermutlich fr"Uhe Bindung lang anhaltende Wirkung.
	\item
		Allerdings auch Belege, dass sich Bindungssicherheit mit Ver"anderungen der Umwelt ver"andert (Belastungen und Konflikten in der Famlie, beispielsweise)
	\item
		Zusammenfassung Siegler 601 f.
\end{itemize}


\subsection{Konzeptionen des Selbs}
\begin{itemize}
	\item
		Selbst = Konzeptsystem, das aus den Gedanken und Einstellungen "uber sich selbst besteht
\end{itemize}
\subsubsection{Entwicklung der Vorstellungen vom Selbst}
\begin{itemize}
	\item
		Schon in ersten Lebensmonaten rudiment"are Vorstellung vom Selbst: Vorstellung von ihrer F"ahigkeit, Objekte au"serhalb ihres Selbst zu kontrollieren (Rassel z.B.)
	\item
		Zwischen 18 und 20 Monaten sich im Spiegel erkennen (Rouge-Test)
	\item
		W"ahrend drittem Lebensjahr Selbs-Bewusstsein deutlich durch Verlgenheit und Scham, aber auch Trotz
	\item
		Zwischen drei und vier Jahren Bezug auf konkrete, beobachtbare Eigenschaften (k"operliche Attribute, soziale Beziehungen, psychische Zust"ande), aber unrealistisch positiv (sie denken sie sind, was sie sein wollen)
	\item
		In Grundschule viele soziale Vergleiche, achten bei Aufgaben auf Diskrepanz zwischen eigener und anderer Leistung
	\item
		In Adoleszenz Sorgen "uber soziale Kompetenz und soziale Akzeptanz verst"arkt
	\item
		Selbstkonzept von Jugendlichen kann mehr als ein Selbst umfassen (z.B. in verschiedenen Situation, gegen"uber verschiedenen Personen(gruppen))
	\item
		Von \enquote{Pers"onlicher Fabel} gekennzeichnet: sagenhafte Selbstbeschreibung, die den Glauben an die Einzigartigkeit ihrer Gef"uhle und ihre Unsterblichkeit beinhaltet
	\item
		Oft Gedanken was andere "uber sie denken: Imagni"ares Publikum: vom Egozentrismus der Jugendlichen begr"undete Vorstellung, dass jeder andere Mensch seine Aufmerksamkeit auf Erscheinung und Verhalten des Jugendlichen richtet
	\item
		In sp"ater Adoleszenz wird Vorstellung des Individuums vom selbst st"arker integriert und weniger durch die Gedanken/Bewertungen anderer bestimmt: Widerspr"uchlichkeiten in verschiedenen Kontexten/Zeitpunkten werden in Selbst integriert (im Gegensatz zu fr"uher Adoleszenz)
\end{itemize}

\subsubsection{Eriksons Theorie der Identit"atsbildung}
\begin{itemize}
	\item
		L"osung der Identit"atsfragen als zentrale Entwicklungsaufgabe der Adoleszenz
	\item
		Krise von Identit"at vs. Rollendiffusion: psychosoziale Entwicklungsphase w"ahrend der Adoleszenz - Jugendliche/junge Erwachsene entwickeln entweder eigene Identit"at oder erfahren unvollst"andiges und manchmal unkoh"arentes Selbstgef"uhl(Rollendiffusion)
	\item
		Erfolgreiche L"sung der Krise impliziert Konstruktion koh"arenter Identit"at
	\item
		M"ogliches Resultat einer misslungen Identit"atssuche ist Rollendiffusion: Jugendliche in diesem Zustand h"aufig verloren, isoliert, deprimiert 
	\item
		Rollendiffusion sehr h"aufig w"ahrend Adoleszenz, i.Allg. aber nur kurz andauernd
	\item
		"Ubernommene Identit"at: voreilig auf Identit"at festlegen, ohne auszuloten welche M"oglichkeiten bestehen w"urden
	\item
		Negative Identit"at: Gegenteil von dem was das Umfeld des Jugendlichen schätzt (z.B. Pfarrerstochter mit h"aufig wechselnden Geschlechtspartnern). Erikson nimmt an, dass dies ein Weg ist Aufmerksamkeit zu bekommen, wenn andere Versuche fehlgeschlagen sind
	\item
		Erikson schl"agt psychosoziales Moratiorum vor: Auszeit in der von Jugendlichen nicht erwartet wird Erwachsenenrolle zu "ubernehmen, stattdessen Selbsterfahrung (allerdings nur in wenigen Kulturen akzeptabel und auch dann eher Privileg. in traditionellen Gesellschaften unbekannt und unn"otig: Rollenauswahl ist beschr"ankt, von Kindheit an vorbestimmt)
	\item
		Marcia definiert 4 Kategorien:
		\begin{enumerate}
			\item
				Identit"atsdiffusion: keine stabile Festlegung, keine Fortschritte
			\item
				"Ubernommene Identit"at: Individuum hat nichts ausprobiert sondern berufliche und ideologische Identit"at entwickelt die auf Auswahl oder Werten anderer beruht
			\item
				Moratorium: erkundet verschiedene berufliche und ideologische Wahlm"oglichkeiten, aber nicht festgelegt
			\item
				Erarbeitete Identit"at: koh"arente und gefestigte Identit"at erreicht, beruht auf pers"nlichen Entscheidungen. Glaub dass diese Entscheidungen eigenh"andig getroffen wurden, f"uhlt sich ihnen verpflichtet
		\end{enumerate}
		Im Idealfall Verlauf zur Erarbeiteten Identit"at
	\item
		Zusammenfassung Sigler 618
\end{itemize}

\section{Entwicklung von Kognition \& Denken: Piaget und Informationsverarbeitungstheorien}
\subsection{Die Sicht auf das Wesen des Kindes}
\begin{itemize}
	\item
		Piagets Ansatz oft als konstruktivistisch bezeichnet: Kinder konstruieren sich ihr Wissen (als Reaktion auf ihre Erfahrungen) selbst
	\item
		\enquote{Kind als Wissenchafter}: Aufstellen von Hypothesen, Durchf"uhren von Experimenten, Ziehen von Schlussfolgerungen
	\item
		Kinder lernen viele wichtige Dinge selbst, sind nicht auf Instruktionen von Erwachsenen angewiesen
	\item
		Kinder von sich aus (intrinsisch) motiviert zu lernen, brauchen daf"ur keine Belohnung
\end{itemize}

\subsection{Zentrale Entwicklungsfragen}
\subsubsection{Quellen der Kontinuit"at}
\begin{itemize}
	\item
		3 Prozesse:
		\begin{enumerate}
			\item
				Assimilation: Prozess durch den Menschen eintreffende Informationen in eine Form "uberf"uhren, die sie verstehen k"onnen
			\item
				Akkomodation: Prozess durch den Menschen vorhandene Wissensstrukturen als Reaktion auf neue Erfahrungen anpassen
			\item
				"Aquilibration: Prozess durch den Menschen Assimilation und Akkomodation ausbalancieren um stabiles Verstehen zu schaffen, besteht aus drei Phasen:
				\begin{enumerate}
					\item
						"Aquilibrium: Kinder sind mit Verst"andnis eines Ph"anomens zufrieden
					\item
						Dis"aquilibrium: Kinder merken, dass ihr Verst"andnis unzureichend ist, sind aber noch nicht in der Lage Alternative zu entwickeln
					\item
						Stabileres "Aquilibrium: differenziertes Verst"andnis, dass das Unzul"anglichkeiten der bisherigen Verstehensstrukturen "uberwindet
				\end{enumerate}
		\end{enumerate}
\end{itemize}

\subsubsection{Quellen der Diskontinuit"at}
\begin{itemize}
	\item
		Stufentheorie, Vier Stufen:
		\begin{enumerate}
			\item
				Sensumotorisches Stadium:
				\begin{itemize}
					\item
						Von Geburt bis 2 Jahre
					\item
						Intelligenz kommt durch sensorische und motorische F"ahigkeiten zum Ausdruck
					\item
						Objektpermanenz ab 4 bis 8 Monaten
					\item
						Ab 1 Jahr erste \enquote{wissenschaftliche Experimente}
					\item
						Ab 18 Monaten zeitlich verz"ogerte Nachahmung
				\end{itemize}
			\item
				Pr"a-operatorisches Stadium:
				\begin{itemize}
					\item
						von 2 bis 7 Jahren
					\item
						Kinder lernen ihre Erfahrungen in Form von Sprache, geistigen Vorstellungen und symbolischen Denken zu repr"asentieren
					\item
						Kinder k"onnen noch keine mentalen Operationen (reversible geistige Aktivit"aten, z.B. vorstellen das Wasser wieder ins andere Glas zu sch"utten) ausf"uhren
					\item
						Entwicklung symbolischer Repräsentation
					\item
						Egozentrismus (3-Berge-Versuch)
					\item
						Zentrierung: konzentrieren sich auf einzelne, auff"allige Aspekte/Dimensionen eines Ereignisses oder Problems
					\item
						Fokus auf statische Zust"ande, nicht auf Ver"anderung
				\end{itemize}
			\item
				konkret-operatorisches Stadium:
				\begin{itemize}
					\item
						von 7 bis 12 Jahre
					\item
						Kinder k"onnen "uber konkrete Gegenst"ande und Ereignisse logisch nachdenken
					\item
						f"allt ihnen aber immer noch schwer in rein abstrakten Begriffen zu denken und Informationen systematisch zu kombinieren
					\item
						Konzept er Erhaltung: bei Ver"anderung der Erscheinung bleiben Schl"usseleigenschaften von Objekten trotzdem erhalten (Knetmasse in anderer Form wird nicht mehr oder weniger)
				\end{itemize}
			\item
				Formal-operatorisches Stadium
				\begin{itemize}
					\item
						ab 12 Jahren
					\item
						Kinder k"onnen "uber Abstraktionen, M"oglichkeiten und hypothetische Situationen nachdenken
				\end{itemize}
		\end{enumerate}
	\item
		Kritik an Piagets Stufentheorie
		\begin{itemize}
			\item
				Stellt Denken von Kindern konsistenter da als es ist
			\item
				S"auglinge und Kleinkinder sind kognitiv kompetenter als Piaget dachte (z.B. Objektpermanenz wohl schon ab 3 Monaten)
			\item
				sch"atzt den Betrag der sozialen Welt zur kognitiven Entwicklung zu gering
			\item
				Unscharf hinsichtlich der kognitiven Prozesse, die das Denken des Kindes verursachen und der Mechanismen, die kognitives Wachstum hervorrufen
		\end{itemize}
\end{itemize}


\subsection{Robbie Case}
\begin{itemize}
	\item
		Entwickelt eigenes Stufensystem, das auf Piaget beruht:
		\begin{enumerate}
			\item
				Sensumotorische Hauptstufe ( Geburt -- 18 Monate):
				\begin{itemize}
					\item
						Repr"asentationen bestehen aus sensorischem Input und motorischen Aktionen -- keine kognitive Verarbeitung von Information
				\end{itemize}
			\item
				Relationale Hauptstufe (18 Monate -- 5 Jahre):
				\begin{itemize}
					\item
						Konkrete Vorstellungen (=Repr"asentationen) m"oglich
					\item
						Beziehen sich nur auf Relationen zwischen Objekten, Ereginissen und Personen, welche entdeckt und koordiniert werden k"onnen
				\end{itemize}
			\item
				Dimensionale Hauptstufe(5 -- 11 Jahre):
				\begin{itemize}
					\item
						Abstrakte Repr"asentationen von Stimuli
					\item
						Einfache Transformationen m"oglich
					\item
						Logische Gestzm"a"sigkeiten von Beziehungen werden erfasst (z.B. Transitivit"at bei $<$)
				\end{itemize}
			\item
				Vektorielle Hauptstufe(wird zwischen 11 und 19 Jahren erreicht):
				\begin{itemize}
					\item
						Nun auch komplexe Transformationen m"oglich
					\item
						Vektoriell = mehrere Dimensionen einer Repr"asentation in abstrakter Form darstellbar
				\end{itemize}
		\end{enumerate}
	\item
		3 Entwicklungsmechanismen:
		\begin{itemize}
			\item
				Central conceptual structures: Veränderung der Wissenstrukturen
				\begin{itemize}
					\item
						Bei Piaget logische Strukturen, bei Case semantische Strukturen.
					\item
						\enquote{Zentrale begriffliche Strukturen} = semantische Netzwerke bzw. Wissensknoten, bilden das stadientypische Basiswissen des Kindes
				\end{itemize}
			\item
				Automatisierung: Steigerung der Verarbeitungseffizienz
			\item
				Biologische Reifung: v.a. Myelinisierung der Nervenbahnen (Myelinisierung = Neuriten/Nervenfasern werden umh"ullt, dadurch schnellere Erregungsleitung m"oglich, vor allem sensorische oder motorische neuronale Verbindungen)
		\end{itemize}
\end{itemize}

\section{Entwicklung von Kognition \& Denken II: Vor- und Nicht-Sprachliche Kognitionen, soziale Kognition/15.1.}
\section{Gedächtnisentwicklung/22.1.}
\section{Selbst- und Identitätsentwicklung/29.1.}
\section{Zusammenfassung, Diskussion, Vorschau/5.2.}
\section{Sprachentwicklung}
Symbole: Systeme mit denen unsere Gedanken, Gefühle und Wissensbestände repräsentiert und anderen Menschen mitgeteilt werden können.
\subsection{Die Komponenten der Sprache}
\begin{itemize}
	\item Generativität: aus endlicher Menge von Wörtern wird unendliche Menge von Sätzen
	\item Phoneme: elementare lautliche Einheiten
	\item Morpheme: Mehrere Phoneme zusammengefasst, kleinste bedeutungstragende Einheit(ich, Hund jeweils ein Morphem, Hund\textbf{e} aber zwei)
	\item Syntax einer Sprache: Regeln die spezifizieren wie Wörter zusammengefügt werden können
	\item Metalinguistisches Wissen: Metawissen über Sprache, Eigenschaften, Funktion..
	\item Schritte auf Erwerb von Sprache:
		\begin{enumerate}
			\item Phonologische Entwicklung: Erwerb von Wissen über das Lautsystem
			\item
				semantische Entwicklung: Erlernen des Systems mit dem Bedeutung ausgedrückt wird, einschließlich Lernen von Wörtern
			\item
				syntaktische Entwicklung: Erlernen der Syntax
			\item Pragmatische Entwicklung: wie Sprache angewendet wird

		\end{enumerate}
\end{itemize}
\subsection{Vorraussetzungen des Spracherwerbs}
\subsubsection{ Das menschliche Gehirn}
\begin{itemize}
	\item Tiere können nur eingeschränkt Sprache lernen, selbst nach intensivem Unterricht nur Grundzüge $\leftrightarrow$ Menschen lernen Sprache "'von selbst"'
	\item Beziehung zwischen Sprache und Gehirn:
		\begin{itemize}
			\item
				Sprachverarbeitet funktional lokalisiert: Bei den 90\% rechtshändern ist Sprache vorwiegend im linken Neocortex (Paul Broca 1861, nach Untersuchen an Unfallopfern)
			\item Broca-Aphasie: Schädigung des Broca-Areals, kurze Wortketten, keine Syntax, zögernd
			\item
				Wernicke-Aphasie: Schädigung des Wernicke-Areals/auditiver Cortex, Produzieren Sprache, aber ohne Sinn, auch Verständnis beeinträchtigt.
			\item
				Auch bei Gebärdensprachebenutzern: Verletzung linke Hemisphäre $\Rightarrow$ Aphasie: 
				Linke Späre spezialisiert auf analytische, serielle Verarbeitung von Sprache
		\end{itemize}
	\item
		Frühen Lebensjahre kritische Phase für Sprachererwerb, danach Spracherwerb deutlich schwieriger
		\begin{itemize}
			\item
				\enquote{Wolfskinder} liefern Indizien, aber keine guten (andere Gründe für keine Sprache?)
			\item
				Menschen die mit 4+-Jahren Englisch als Zweitsprache lernen: weniger linkshemisphärische Lokalisation von Englischverarbeitung (unabhängig von wie lange schon Englischsprachig bei Untersuchung)
			\item
				Selbes Prinzip bei Gehörlosen/Gebärdensprache: Je früher anfangen, desto besser als Erwachsene
			\item
				Möglicheweise aufgrund begrenzter kognitiver Fähigkeiten von Kindern: Können nur kleinere Sprachsamples verarbeiten, an denen lassen sich Struktur leichter erkennen als an großen Samples
		\end{itemize}
\end{itemize}
\subsubsection{Eine menschliche Umwelt}
\begin{itemize}
	\item
		Um Sprache zu lernen ist es wichtig mit anderen sprechenden Menschen in Kontakt zu kommen
	\item
		Spezieller Redestil, der an Kinder gerichtet ist (\enquote{Ammensprache})
		\begin{itemize}
			\item
				Emotionaler Tonfall
			\item
				Übertreibung
			\item
				Höhere Stimme, Intonationsmuster schwankt stark (abrupter Wechsel zwischen hohen und tiefen Tönen)
			\item
				Langsamer und deutlicher
			\item
				Übertriebene Mimik
			\item
				Viele dieser Kennzeichnungen in vielen verschiedenen Sprachen, auch Gebärdensprache
		\end{itemize}
	\item
		Kinder nutzen Tonfall ihrer Mütter um Bedeutung zu interpretieren ("NEIN!" vs. "jaaa :)")
		\begin{itemize}
			\item
				Nachgewiesen in Experiment von Anne Fernald (1989): Kindern wurde Spielzeug gegeben mit "yes, good boy"/"no, don't touch" in positivem/negativem Tonfall (gemixt). Tonfall war wichtiger als tatsächlich gesagtes
		\end{itemize}
	\item
		Kinder präferieren kindliche Sprache, auch wenn an anderes Kind gerichtet und in fremder Sprache ist
		\begin{itemize}
			\item
				Kinder hören kindlicher Sprache länger zu als erwachsener Sprache (auch in fremder Sprache)
			\item
			 	Kleinkinder u. Erwachsene lernen schneller Fremdsprachenwörter, wenn in kindlicher Sprache
		\end{itemize}
	\item
		Kindliche Sprache aber nicht universell:
		\begin{itemize}
			\item
				Verschiedene Inselvölker glauben das Kindern die Fähigkeit zum Sprachverstehen fehlt
			\item
				Sprechen nicht mit Kindern
			\item
				Wenn Kinder mit Sprechen beginnen gibt es \enquote{Sprachtraining}, keine kindliche Sprache
	\item
		Ob Eltern direkt zu Kindern sprechen oder nicht beeinflusst Geschwindigkeit des frühen Sprachlernens, nicht aber letzendliches Sprachniveau
		\end{itemize}
\end{itemize}

\subsection{Der Prozess des Spracherwerbs}
Bei Spracherwerb sowohl Zuhören als auch Sprechen beteiligt
\subsubsection{Sprachwahrnehmung}
\begin{itemize}
	\item
		Schon im Mutterleib Vorliebe für Sprache/Stimme der Mutter: Grund ist Prosodie(Charakteristischer Rythmus, Tempo, Tonfall, Melodie, Intonation...etc. einer Sprache)
	\item
		Prosodie allein reicht nicht fürs lernen, sprachlichen Laute müssen unterschieden werden können - können Kinder aber bereits
	\item
		Wahrnehmung von sprachlichen Lauten als diskrete Klassen: kategoriale Wahrnehmung
		\begin{itemize}
			\item
				Sprachsynthesizer produziert Kontinuum aus /b/s, die langsam in /p/s umgewandelt wurden (Laut grundsätzlich gleich, nur VOT - Voice onset time, Zeitpunkt(Vibration der Stimmbänder)-Zeitpunkt(Freilassen des Luftstroms durch die Lippen)). Bei /b/ (15 ms) viel kürzer als /p/ (150ms)
			\item
				Erwachsene können kontinuierlichen Übergang nicht wahrnehmen, hören viele bs, dann abrupter Wechsel zu ps (VOT$<$25ms $\Rightarrow$ b, VOT$>$25ms $\Rightarrow$ p), nehmen zwei diskrete Kategorien war
			\item
				Babys können auch Unterscheiden: Saugen an Schnuller $\Rightarrow$ Laut wird von Computer abgespielt. Nach mehrmals dem selben Laut saugen Babys weniger begeistert. Neuer Laut $\Rightarrow$ Saugreaktion steigt an, können Ton unterscheiden. Zwei Klassen von Babys, bei beiden erst Habituation auf b:
				\begin{itemize}
					\item
						Wechsel auf Ton der (von Erwachsenen) als p wahrgenommen wird
					\item
						Wechsel auf Ton der (von Erachsenen) immer noch als b wahrgenommen wird
				\end{itemize}
				VOT Unterschiede aber bei beiden gleich.\\
				Bei p (neue Lautklasse) erhöhte Saugrate, bei b weiter Habituation (Eimas, Siqueland, Jusczyk, Vigorito, 1971)
		\end{itemize}
	\item
		Kinder unterscheiden aber mehr als Erwachsene: Sprache verwenden nur Teilmenge aller Phonemklassen (r und l in Deutsch unterschiedlich, nicht in Japanisch)
	\item
		Diese Fähgikeit ist angeboren (bereits bei Geburt vorhanden) und unabhängig von Erfahrung (Kinder unterscheiden Laute, die sie nie zuvor gehört haben)
	\item
		Auch manche Tiere können kategorial Laute Wahrnehmen: keine menschliche Fähgikeit, keine Fähigkeit, die ausschließlich auf Sprache spezialisiert ist.
	\item
		Sprachwahrnehmung verändert sich mit Entwicklung:
		\begin{itemize}
			\item
				Fähigkeit Phoneme zu unterscheiden geht verloren, am Ende des ersten Lebensjahres Sprachwahrnehmung ähnlich wie Eltern
			\item
		\end{itemize}
	\item
	 9 Monate alte Babys können Unterschiede in Betonungsmustern wahrnehmen
 \item
	 Verteilungscharakteristika (Wahrscheinlichkeit für bestimmte Lautkombinationen) wird wahrgenommen
 \item
	 Säuglinge strengen sich an um Muster in Lauten zu identifizieren
\end{itemize}
\subsubsection{Vorbereitung für die Sprachproduktion}
\begin{itemize}
	\item
		6-8 Wochen erste sprachliche Laute ("'oooh"'"'aaah"'"'gu"'), gennannt \enquote{cooing}
	\item
		Zwischen 6. und 10. Monat, ca. 7 Monate tritt \enquote{babbling}, plappern ein: Konsonant+Vokal wird immer wieder wiederholt.
	\item
		Wichtig dabei: Rückmeldung zu erhalten. Gehörlose Babys bis 5/6 Monate  ähnliche Laute, plappern aber erst sehr spät und begrenzt
	\item
		Erste Anzeichen von Kommunikation ist turn-taking (Guck-guck, nimm-und-gib-spiele), Wechsel zwischen aktiv und passiv (wie in gespräch)
	\item
		Wichtig auch: Intersubjektivität, gemeinsames Aufmerksamkeitszentrum (vorher geteilte Aufmerksamkeit: Eltern schauen da hin wo baby hin schaut): Mit Finger auf etwas Zeigen: Vor 9 Monaten schauen auf finger, danach schauen auf gezeigtes
\end{itemize}
\subsubsection{Ersten Worte}
\begin{itemize}
	\item
		Frühe Worterkennung: Bereits mit 4.5 Monaten kann eigener Name erkannt werden, Wörter müssen erst erkannt werden, Verstehen erst danach
	\item
		Mit ca. 6 Monaten erste Referenzen verstanden (Mama und Papa)
	\item
		Mit 10 Monaten Verstehenswortschatz zwischen 11 und 145 Wörtern
	\item
		Frühe Wortprodutkion: zwischen 10 und 15 Monaten ersten Worte. Erstes Wort = spezifische Äußerung um konsistent etwas zu bezeichnen (\enquote{woof} für Hund auch Wort)
	\item
		Am Anfang Problem Wörter auszusprechen: schwierige Teile werden Weggelassen oder vertauscht. Banane $\rightarrow$ nane, Spaghetti $\rightarrow$ pasketti
	\item
		Holographische Phase: Kinder sprechen nur Wort für Wort
	\item
		Entwicklung des Wortschatzes proportional zu Menge an Sprache die Kind hört
	\item
		Überdehnung: Verwendung von Wort in breiterem Kontext: (Hund für jedes 4beinige Tier) (Kind schaut auf zwei Bilder von Schaf und Hund, sagt zu beidem Hund. Soll es auf Schaf zeigen, kann es das tun, das Wort fehlt nur in seinem produktivem Wortschatz)
	\item
		Mit 18 Monaten produktiver wortschatz von ca 50 wörtern, lernen danach sehr schnell (durchschnittlich 5-10 Wörter pro Tag)
	\item
		Kinder nutzen Prinzipien um wörter zu lernen:
		\begin{itemize}
			\item für das ganze Objekt, nicht Teil (Ganzheitsconstraint)
			\item  nur ein Name für Sachverhalt(Disjunktionsconstraint)
			\item  pragmatische Hinweise (Aufmerksamkeit Erwachsener)
			\item grammatische Klasse (bereits ab 2 und 3 Jahren)
			\item	Objekte desselben Typs eher mit demselben Wort bezeichnet als thematisch verbundene Objekte(taxonomieconstraint, z.B. Hund f"ur verschiedene Hunderassen, aber nicht f"ur Knochen, Halsband, Leine)
		\end{itemize}
\end{itemize}
\subsubsection{Das Zusammenfügen von Wörtern}
\begin{itemize}
	\item
		Gegen Ende 2. Lebensjahr erste Sätze im Telegrammstil, 2wortsätze
	\item
		Mit ca. 2.5 Jahren vierwortsätze, dann komplexere Sätze mit mehr als einer Phrase (nebensätze etc)
	\item Kinder übern Sprache selbst, allein (abends im bett z.B.)
	\item
		Kinder lernen grammatikalische Regeln, nachgewiesen durch Übergeneralisierung (man $\rightarrow$ mans, statt man  $\rightarrow$  men), gelingt es kindern nicht korrekte unregelmäßige Form abzurufen wird die Standardform verwendet
	\item
		Rückmeldung der Eltern (Korrektur von Fehlern) spielt nur sehr gerine Rolle
\end{itemize}

\subsubsection{Gesprächsfähigkeit}
\begin{itemize}
	\item
		Kinder sprechen zu sich selbst um Handlungen zu organisieren, später dann als Denken internalisiert
	\item
		Neigen dazu egozentrisch zu sprechen, kollektive Monologe entstehen, keine Gespräche, von 21 - 36 Monaten steigt Anteil von sinnvollen Antworte nvon 20 auf mehr als 40 prozent
	\item
		3jährige kaum über Vergangenes, fünfjährige schon.  
\end{itemize}

Zusammenfassung Siegler et.al. 351 ff.

\section{Moralentwicklung}
\subsection{Piagets Theorie des moralichen Urteils}
\begin{itemize}
	\item
		Moralisches Denken in Kindern wandelt sich vom starren übernehmen der gebote und regeln von Autoritätspersonen zu moralische Regeln sind ein Produkt sozialer Interaktion und veränderbar
\end{itemize}
\subsubsection{Heteronome Moral}
\begin{itemize}
	\item
		Vor konkret Operationalen Phase, also ca. unter 7 oder 8 Jahren
	\item
		Autoritäten geben vor was gut und böse ist, Folgen wichtiger als Motive
	\item
		Regeln sind absolut
\end{itemize}
\subsubsection{Übergangsphase}
\begin{itemize}
	\item
		Zwischen 7/8 und 10 Jahren, Interaktion mit Peers
	\item
		Lernen dass Regeln von Gruppen aufstellt werden und verändert werden können
\end{itemize}
\subsubsection{Stadium der Autononem Moral}
\begin{itemize}
	\item
		Gerechtigkeit und Gleichberechtigung wichtig, Bestrafung angemessen, Motive müssen Berücksichtigt werden, nicht nur Ergebnis
\end{itemize}
\subsubsection{Bewertung}
\begin{itemize}
	\item
		Interaktion mit Peers regt Moralentwicklung von sich aus nicht an, Qualität der Interaktion wichtig
	\item
		Kinder erkennen schon im Kindergartenalter, dass schlimme Absichten schlimmer sind als gutartige Absichte
\end{itemize}

\subsection{Kohlbergs Theorie des moralischen Urteils}
\begin{itemize}
	\item
		Stufenmodell, siehe Siegler et.al. S 762
	\item
		Drei Ebenen des moralischen Urteils:
		\begin{enumerate}
			\item
				Präkonventionelles moralisches Denken: selbstbezogen, Belohnung bekommen - Strafe vermeiden
				\begin{enumerate}
					\item
						Orientierung an Strafe und Gehorsam
					\item
						Orientierung an Kostne-Nutzen und Reziprozität
				\end{enumerate}
			\item
				Konventionelles moralisches Denken: Übereinstimmung mit sozialen Pflichten und Gesetzen
				\begin{enumerate}
					\item
						Orientierung an wechselseitigen zwischenmenschlichen Erwartungen, Beziehungen und zwischenmenschlicher Übereinstimmung (\enquote{gutes Mädchen, guter Junge})
					\item
						Orientierung am sozialan System und am Gewissen (\enquote{Recht und Ordnung})
			\end{enumerate}
			\item
				Postkonventionelles moralisches Denken: auf Ideale ausgerichtet, moralische Prinzipien
				\begin{enumerate}
					\item
						Orientierung am sozialen Vertrag oder an individuellen Rechten
					\item
						Orientierung an universellen ethischen Prinzipien
			\end{enumerate}
		\end{enumerate}
	\item
		Jede Ebene in 2 Stufen unterteilt
	\item
		Fähigkeit zur Perspektivübernahme wichtig
	\item
		Moral Kombination aus kognitiver, sozialer und emotionaler Entwicklung
	\item
		Kritik an Kohlberg:
		\begin{itemize}
			\item
				Westlich-zentrierte Moralvorstellung (andere Gesellschaften weniger Individuelle Freiheiten)
			\item
				Entgegen Kohlberg moralische Entwicklung kontinuierlich, auch Rückschritte sind möglich, auch mehrere Stufen gleichzeitig
		\end{itemize}
\end{itemize}



\subsection{Prosoziales moralisches Urteilsvermögen}
\begin{itemize}
	\item
		Prosoziale moralische Dilemmata: jemandem helfen oder eigenen Bedürfnissen nachgehen?
	\item
		Eisenbergs fünf Stufen des prosozialen moralischen Denkens:
		\begin{enumerate}
			\item
				Hedonistische, selbstbezogene Orientierung
			\item
				Orientierung an Bedürfnissen
			\item
				Orientierung an Anerkennung und/oder Stereotyp
			\item
				\begin{enumerate}
					\item
						Selbstreflexive empathische Orientierung
					\item
						Übergangsniveau
			\end{enumerate}
		\item
			Stark internalisiertes Stadium

	\end{enumerate}
\end{itemize}

\subsection{Bereiche sozialer Urteile}
\begin{itemize}
	\item
		Moralische Urteile betreffen Fragen von Richtig und Falsch, Fairness und Gerechtigkeit
	\item
		Sozial-konventionale Urteile beziehen sich auf Sitten und Regeln zur sozialen Koordination und Organisation
	\item
		Persönliche Urteile Entscheidungen über Handlungen bei denen persönliche Präferenz berücksichtigt werden
	\item
		Kinder unterscheiden zwischen moralische Verfehlungen schlimmer sind als gegen soziale Konventionen zu verstoßen
	\item
		Zusammenfassung Siegler et.al. 773
\end{itemize}

\section{Entwicklung von Prosozialem und Aggressiven Verhalten}
\begin{itemize}
	\item
		Gewissen: Ein innerer Regulationsmechanismus, der die Fähigkeit eines Individuums erhöht Verhaltensstandards seiner Kultur zu entsprechen
	\item

\end{itemize}
\subsection{Faktoren, welche die Gewissensentwicklung beeinflussen}
\begin{itemize}
	\item
		Ab zwei Jahren fangen Kinder an Verständnis f"ur NOrmen und Regeln zu zeigen, erste Anzeichen von Schuldgef"uhlen
	\item
		"Ubernehmen vermutlich Werte der Eltern
	\item
		Bei ängstlichen Kindern \enquote{behutsame} Disziplinierungspraktiken für Gewissensentwicklung
	\item
		Bei furchtlosen Kindern besser positive Eltern-Kind-Beziehung, Kinder wollen mehr der Mutter gefallen als sich vor ihr f"urchten
\end{itemize}

\section{Prosoziales Verhalten}
\begin{itemize}
	\item
		Gibt Entwicklungskonsistenz von Kindern, die sich auf prosoziale Verhaltensweisen einlassen
\end{itemize}
\subsection{Die Entwicklung des prosozialen Verhaltens}
\begin{itemize}
	\item
		Urspr"unge f"ur altruistisches prosoziales Verhalten liegen in Empathie und Mitleid begr"undet. (Empathie = F"ahigkeit sich in andere hineinzuversetzen, emotionale Reaktion auf Zustand des anderen, die Gef"uhlslage des anderen widerspiegelt
	\item
		Unterschied Mitleid Empathie: Mitleid zus"atzlich Sorge und Anteilnahme, helfen wollen
	\item
		Wichtiger Faktor f"ur Mitleid: F"ahigkeit Perspektive des anderen einzunehmen (entgegen Piaget ("nicht vor 6 oder 7 Jahren") schon viel fr"uher)
		Im zweiten und 3. Lebensjahr erh"ohen sich H"aufigkeit und Vielfalt prosozialer Verhaltensweisen: Nicht nur tr"osten, sondern auch helfen
	\item
		Aber auch aggresives Verhalten, Sticheln oder Gleichgültigkeit gegen"uber Unbehagen und Bed"urfnisse anderer
	\item
		Prosoziales Verhalten wird mit Alter h"aufiger
\end{itemize}
\subsection{Urspr"unge individueller Unterschiede beim prosozialen Verhalten}
\subsubsection{Biologische Faktoren}
\begin{itemize}
	\item
		Genetische Faktoren scheinen moderate Rolle zu spielen (kleinder bei Kindern als bei Erwachsenen): Eineiige Zweillinge im Hinblick auf Empathie und prosoziales Verhalten "ahnlicher als Zweieiige. 
	\item
		Temperamentsunterschiede spielen Rolle: Kinder, die Gef"uhle bewusst erleben k"onnen ohne "uberw"altigt zu sein besonders h"aufig empathisch
\end{itemize}
\subsubsection{Die Sozialisation prosozialen Verhaltens}
\begin{itemize}
	\item
		3 Erziehungsarten mit denen Eltern soziales Verhalten von Kindern fördern:
		\begin{enumerate}
			\item
				Vorbild sein und prosoziales Verhalten beibringen
				\begin{itemize}
					\item
						Kinder und Eltern häufig ähnliches Niveau an dAnteilnahme und prosozialem Verhalten
					\item
						Studie mit Rettern von Juden und Zuschauern Oliner \& Oliner 1988: Fairness/Gerechtigkeit bei beidne Gruppen gleich: Sorge für andere (/als univeselles Prinzip) bei Retter häufiger
					\item
						Wirksames Mittel: an Mitgefühl appelieren (\enquote{die armen Kinder...}), danach Kinder eher bereit zu spenden als wenn einfach nur \enquote{Helfer sind gut}
				\end{itemize}

			\item
				arrangieren Gelegenheiten, bei denen sich kinder prosozial verhalten können
				\begin{itemize}
					\item
						Z.B. Haushaltspflicht, freiwillige soziales Dienste 
					\item
						Gelegenheit für Hilfeleistunhgen emotionale Belohnung zu erfahren, lernen sich in andere hineinzuversetzen und Vertrauen in eigene Fähigkeit zu Hlefen zu steigern
				\end{itemize}
			\item
				Erziehung und Disziplinierung zu prosozialem Verhalten
				\begin{itemize}
					\item
						autoritäterer Erziehungsansatz (Bestrafungen, Drohungen) häufig Kinder mit Mangel an Mitgefühl und prosozialem Verhalten
					\item
						Wenn Kinder bestraft/materiell belohnt werden anderen zu helfen, nehmen sie an, dass sie nur um der Belohnung/Bestrafung willen helfen. Ist das nicht mehr gegeben fällt Anreiz zu helfen weg
					\item
						Rationale, vernünftige Argumente helfen Kindern Folgen ihres Verhaltens zu verstehen und geben Gründe mit, an denen Verhalten orientiert werden kann (Schon bei 1/2-jährigen)
					\item
						Kinder meistens sozialer, wenn Eltern nicht nur Wärme und Unterstützung bieten sondern bei Erziehung auch prosoziales Verhalten vorleben
				\end{itemize}
	\end{enumerate}
\item
	Fernsehen zeigt antisoziales Verhalten (Gewalt etc.), Gefahr?
	\begin{itemize}
		\item
			Manche Inhalte zeigen prosoziales Verhalten, Kinder die so etwas (Sesamstraße...) sehen, neigen sofort danach und auch manchmal noch später zu prosozialem Verhalten, meistens jedoch kein lang anhaltender Effekt
	\end{itemize}
\item
	Zusammenfassung Siegler et.al. 786
\end{itemize}

\section{Antisoziales Verhalten}
\begin{itemize}
	\item
		Aggression: Verhalten das darauf abzielt andere zu schädigen oder zu verletzen
\end{itemize}
\subsection{Die Entwicklung von Aggression und anderer anti-sozialer Verhaltensweisen}
\begin{itemize}
	\item
		Schon zwischen 12 und 18 Monaten Konflikte, meistens ohne Aggression
	\item
		Ab 18 Monaten körperliche Aggression wie Schlagen oder Stoßen, steigen bis 2 Jahren
	\item
		Danach sinkt die Häufigkeit körperlicher Aggression, mit den sprachlichen Fähigkeiten kommt verbale Aggression
	\item
		Viel Streit um Eigentumsverhältnisse, dies ist Beispiel für instrumentelle Aggression: Aggression mit Wunsch etwas zu erreichen
	\item
		Manchmal auch Beziehungsaggression um Peers zu beherrschen oder verletzen (Peer-Beziehungen schädigen: aus Gruppe ausschließen, negative Gerüchte)
	\item
		Aggressives Verhalten bei jüngeren Kindern normalerweise durch wunsch motiviert, bei Grundschulkindern häufig durch Feindschaft, Wunsch anderen zu verletzen oder Bedürfnis sich gegen wahrgenommene Bedrohung des Selbstwerts zu schützen
	\item
		In Adoleszenz sinkt Häufigkeit offener Aggressionen, schwere Gewaltanwendung steigt aber
\end{itemize}
\subsection{Die Beständigkeit aggressiven und antisozialen Verhaltens}
\begin{itemize}
	\item
		Kinder die mit 8 Jahren von Peers als aggressiv eingeschätzt wurden mit 30 Jahren mehr kriminelle Vorstrafen und häufiger Aggressiv als andere (Eron et.al. 1987)
	\item
		Größtes Risiko für Kinder die sowohl aggressiv sind, als auch anderes antisoziales Verhalten (lügen, stehlen) zeigen, Aggression aber kein notwendiger Bestandteil zukünftiger Verhaltensprobleme
	\item
		Viele Kinder die bereits früh Aggressionen zeigen haben neurologische Defizite
\end{itemize}


\subsection{Kennzeichen aggressiver und/oder antisozialer Kinder und Jugendlicher}
\subsubsection{Temperament und persönlichkeit}
\begin{itemize}
	\item
		Intensive negative Emotionen im Kleinkindalter neigen später zu mehr Problemverhalten (z.B. Aggression) (Bates et.al 1991)
	\item
		Vor Schuleintritt wenig Selbstkontrolle, Impulsivität und hohes Aktivierunsniveau und reizbar und ablenkbar im Alter zwischen 9 und 15 Jahren mehr Schlägereien, Kriminalität etc.
	\item
		Kombination aus Impulsivität, Aufmerksamkeitsprobleme und Verlogenheit in Kindheit gute Vorhersage für antisoziales Verhalten  in Adoleszenz
\end{itemize}
\subsubsection{Soziale Kognition}
\begin{itemize}
	\item
		Auch soziale Kognition ist wichtig, weil sie sich darauf auswirkt, wie Kinder ihre Interaktionen mit anderen interpretieren und wie sie auf diese reagieren
	\item
		Aggressive Kinder betrachten Welt durch \enquote{aggressive Linse}
	\item
		Aggressive Kinder häufig negative, feindselige Ziele (Bedrohung von Peers, anderen heimzahlen...)
	\item
		Aggressive Kinder bewerten aggressive Reaktionen positiver als andere Kinder, prosoziale Reaktionen weniger günstig
	\item
		Haben mehr Vertrauen in eigene Fähigkeiten körperlich und verbale Aggressionen auszuführen
	\item
		Kinder mit reaktiver Aggression (emotionsgesteuerte, feindselige Aggression) nehmen Motive anderer häufig als feindselig war, reagieren auf Provokationen aggressiv
	\item
		Kinder mit proaktiver Aggression (nicht emotional begründet, geht darum Bedürfnissen nachzukommen) erwarten positive soziale Folgen von Aggression
\end{itemize}


\subsection{Ursprünge der Aggression}
\subsubsection{Biologische Faktoren}
\begin{itemize}
	\item
		Genetik spielt Rolle, vor allem wenn Verhalten schon in Kindheit und nicht erst in Adoleszenz
	\item
		Testosteronspiegel wird manchmal mit aggressivem Verhalten in Verbindung gebracht
	\item
		Genetische, neurologische oder hormonelle Eigenschaften sind Risikofaktoren, entscheiden aber nicht, ob Kind tatsächlich aggressiv wird
\end{itemize}

\subsubsection{Die Sozialisation von Aggression und antisozialem Verhalten}
\begin{itemize}
	\item
		Ausmaß in dem schlechte Erziehung für antisoziales Verhalten verantwortlich ist nicht bekannt
	\item
		Elterliche Bestrafung
		\begin{itemize}
			\item
				Kinder neigen zu Problemverhalten wenn Eltern generell kaltherzig und strafend erziehen, streng aber nicht misshandelnd körperlich bestrafen
			\item
				Hohe Wahrscheinlichkeit, dass antisoziale Verhaltensweisen auf misshandelnde Bestrafung folgen
			\item
				Kinder mit hoher Ausprägung von antisozialem Verhalten und wenig Selbstregulation häufig mit strengen Erziehungsmaßnahmen konfrontiert $\Rightarrow$ Teufelskreis
			\item
				Genetik und Sozialisation schwer zu trennen bei diesen Untersuchungen (aggressive gene bei eltern $\Rightarrow$ auch bei Kindern, die damit sowohl aggressives verhalten zeigen, als auch streng/strafend erzogen werden
		\end{itemize}
	\item
		Unwirksame Erziehungsmaßnahmen
		\begin{itemize}
			\item
				Inkonsequente Bestrafungen führen häufig zu aggressiven und kriminellen Kindern
			\item
				Wenn Eltern Kinder überwachen/beaufsichtigen generell weniger Verhaltensprobleme. Möglicher Grund: Kinder hängen dann nicht mit antisozialen Peers rum
			\item
				Wenn Wutausbrüchen und Wünschen von Jungen nachgegeben wird, wird Aggression des Kindes verstärkt. Gibt Gründe zur Annahme, dass Mädchen anders reagieren
		\end{itemize}
	\item
		Konflikte zwischen den Eltern
		\begin{itemize}
			\item
				Kinder, die Zeuge verbaler und physischer Gewalt zwischen Eltern werden neigen dazu aggressiver und antisozialer zu sein. Grund: streitende Eltern als Modelle für aggressives Verhalten
			\item
				Scheidung/Wiederheirat führen tendenziell auch zu antisozialem Verhalten
			\item
				Mütter unterstützen Kinder nach Scheidung erst mal weniger und sind weniger konsequent bei Kontrolle und Beaufsichtigung
		\end{itemize}
	\item
		Kinder aus einkommensschwachen Familien in der Regel antisozialer und aggressiver
		\begin{itemize}
			\item
				Mehr Stressoren (Krankeit/Gewalt/Scheidung/Kriminalität in der Familie, Gewalt in der Wohngegend) als reiche Kinder
			\item
				Möglicherweise alleinerziehendes Elternteil oder sehr frühe Eltern (beides mit aggressivem Verhalten verknüpft)
			\item
				Von Armut gestresste Eltern häúfig Vorbilder für Aggression und \enquote{schlechte Erziehung}
		\end{itemize}
\end{itemize}


\subsubsection{Der Einfluss der Peers}
\begin{itemize}
	\item
		Aggressive Kinder tun sich gern zusasmmen
	\item
		Mäßig aggressive Kinder werden aggressiver, wenn sie mit aggressiven Freunden rumhängen
	\item
		Kriminelle Peers die verdeckte(Diebstahl, Drogenhandel)/offene(Gewalt, Waffengebrauch) antisoziale Verhaltensweisen zeigen erhöhen Wahrscheinlichkeit für krminelles Verhalten um 400/300 Prozent
	\item
		Gangs sind schlecht für Kinder
	\item
		Gewalt im Fernsehen wirkt sich auf Kinder aus
		\begin{itemize}
			\item
				Gewalt im Fernsehen kausaler Faktor für spätere Aggression
		\end{itemize}
	\item
		Zusammenfassung Siegler 804 ff.
\end{itemize}


\end{document}
